			\begin{tikzpicture}[%
	%%%% Ausrichtung
	%every child/.style={anchor=west,grow=down},
	%%%% kleinere Schriftgröße
	%node font=\small,
	%%%% definiert von wo (Parent) nach wo (Child) die Linien gehen sollen, wenn Pfeil zu Child gewünscht, dann "[->]" wie folgt einsetzen "...path={[->](tikzparentnode.south)..."
	%edge from parent path={([xshift=0em]\tikzparentnode.east) |- (\tikzchildnode.west)},
	%%%% horizontale Abstände
	edge from parent/.style={very thick,draw=black!70,-latex},
	level a/.style={xshift=2em},
	level b1/.style={xshift=9em},
	level b2/.style={xshift=5em},
	level b3/.style={xshift=2em}
	]
	%%%% Beginn tikz-Baumdiagramm
	\node [sumGraph](1) {}
	child { node [sumGraph](2){}
		child {node [sumGraph](3){}}};
	\node [output]	at (1.east)	[label=0:in1F1]	{};
	\node [output]	at (2.east)	[label=0:tankL1]	{};
	\node [output]	at (3.east)	[label=0:C4\_1]	{};
%	child[level distance=0cm,level a] { node [box] {anschauungsraumbezogene Verfahren}
%		child[level distance=0cm,level b1] { node [box] {Guyan}
%			child[level distance=0cm,level b2] {node [box] {RM a}}}
%		child[level distance=1cm,level b1] { node [box] {Dyn. Kondensation}
%			child[level distance=0cm,level b2] {node [box] {RM b}}}
%		child[level distance=2cm,level b1] { node [box] {IRS}
%			child[level distance=0cm,level b2] {node [box] {RM c}}}
%	}
%	child[level distance=3cm,level a] { node [box] {Verfahren im modalen Raum}
%		child[level distance=0cm,level b1] { node [box] {SOMT}
%			child[level distance=0cm,level b2] {node [box] {RM d}}}
%		child[level distance=1cm,level b1] { node [box] {SEREP}
%			child[level distance=0cm,level b2] {node [box] {RM e}}}
%	}
%	child[level distance=5cm,level a] { node [box] {Verfahren im allg. Vektorraum}
%		child[level distance=0cm,level b1] { node [box] {KSM}
%			child[level distance=0cm,level b2] {node [box] {RM f}}}
%		child[level distance=1cm,level b1] { node [box] {SOBT}
%			child[level distance=0cm,level b2] {node [box] {RM g}}}
%		child[level distance=2cm,level b1] { node [box] {POD}
%			child[level distance=0cm,level b2] {node [box] {RM h}}}
%	}
%	};
	\end{tikzpicture}