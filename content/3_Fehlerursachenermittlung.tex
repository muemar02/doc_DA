\chapter{Analyse von Fehlern in prozessleittechnischen Anlagen} \label{ch:fehlerfortpfl}
\section{Einleitung}
Die Durchf\"uhrung einer \ac{hazop} dient der Bewertung des Risikos, welches von einer verfahrenstechnischen Anlage ausgeht. Ein bedeutender Teil bei der Durchf\"uhrung einer \ac{hazop} besteht in Erkennung und Bewertung m\"oglicher Fehlerfortpflanzungen. \newline
Es existieren zahlreiche Systeme, welche die Durchf\"uhrung einer \ac{hazop} unterst\"utzen oder sogar teilweise automatisieren sollen. Solche Systeme sind besonders dann kostensparend einsetzbar, wenn sie einen hohen Grad an Wiederverwendbarkeit von bereits geleisteten Engineeringleistungen aufweisen und den Nutzer durch geeignete Algorithmen bei der Auswirkungsanalyse von in einem System m\"oglicherweise auftretenden Fehlern unterst\"utzen. Die Auswirkungsanalyse sollte dabei die Fortpflanzung von Fehlern \"uber das gesamte System hinweg ermitteln k\"onnen. Die Anwendbarkeit dieser Algorithmen f\"ur modulare Anlagen soll \"uberpr\"uft werden, weswegen Systeme zur Automatisierung von \acp{hazop} im Abschnitt \ref{sec:fAna_automatHazop} untersucht werden. \newline
Ein weiterer Forschungsschwerpunkt, bei welchem die Fortpflanzung von Fehlern untersucht wird, ist die \ac{fdi}. Die zur \ac{fdi} verwendeten Methoden \"uberschneiden sich zum Gro\ss{}teil mit denen, welche im Rahmen von automatisierten \acp{hazop} zur Beschreibung von Fehlerfortpflanzungen verwendet werden. Im Abschnitt \ref{sec:fAna_fdi} wird die Motivation f\"ur das Themengebiet der \ac{fdi} dargelegt, die verwendeten Methoden in drei Klassen eingeteilt und anschlie\ss{}end auf Ihre Verwendbarkeit im Rahmen modularer Analgen untersucht. 

\section{Automatisierung einer \ac{hazop}} \label{sec:fAna_automatHazop}
Die Durchf\"uhrung einer \ac{hazop} ist kompliziert, zeitaufwendig und arbeitsintensiv. Au\ss{}erdem ist die Verl\"asslichkeit der erarbeiteten Ergebnisse einer \ac{hazop} ma\ss{}geblich von der Erfahrung der durchf\"uhrenden Experten abh\"angig. Durch den hohen Zeitaufwand und die wiederkehrende Anwendung der gleichen Leitworte stellt sich schnell eine erm\"udende Routine eine, welche das \"Ubersehen wichtiger Zusammenh\"ange beg\"unstigt. In Folge dieser Tatsachen wurden zahlreiche computergest\"utzte Systeme entwickelt, um die Durchf\"uhrung einer \ac{hazop} zu unterst\"utzen. Diese Expertensysteme sollen die Wiederverwendbarkeit von gewonnenem Expertenwissen erm\"oglichen, die Anwendung der Leitworte vereinfachen und die Wahrscheinlichkeit von \"ubersehenen Zusammenh\"angen reduzieren. Insgesamt sollen also die Kosten der Durchf\"uhrung einer \ac{hazop} gesenkt und die Qualit\"at der Ergebnisse gesteigert werden. 

Eine erste Arbeit zur Automatisierung von \acp{hazop} wurde von \citeauthor{Parmar_1987} ver\"offentlicht \cite{Parmar_1987,Parmar_1987a}. Die untersuchte Anlage zur Trennung von Wasser und Kohlenwasserstoffen wird basierend auf dem \ac{pid} in Einheiten unterteilt. Durch ausl\"osende und terminierende Ereignisse wird die Fortpflanzung von Fehlern beschrieben. Mit Hilfe eines regelbasierten Algorithmus k\"onnen die Ursachen und Auswirkungen der Abweichung von Prozessvariablen ermittelt werden. Die Analyse ist dabei auf jeweils benachbarte Einheiten begrenzt und eine anlagenweite Untersuchung ist nicht m\"oglich, obwohl eine solche Betrachtung  ein wichtiger Teil einer \ac{hazop} ist. Das System ist in den Programmiersprachen Fortran 77 and Prolog umgesetzt. Das implementierte Expertenwissen ist nur ungen\"ugend abstrahiert, wodurch eine Wiederverwendbarkeit erschwert wird. Die Anwendbarkeit f\"ur Anlagen im gro\ss{}industriellen Ma\ss{}stab konnte bisher nicht gezeigt werden. Eine Weiterentwicklung dieses Ansatzes wird daher als nicht sinnvoll erachtet.

\citeauthor{Catino_1995} stellen in \citetitle{Catino_1995} ein System mit dem Namen \glqq Qualitative Hazard Identification (QHI)\grqq { }vor \cite{Catino_1995}. Die Autoren definieren in einer Bibliothek allgemeine Fehler wie Lecks oder verstopfte Filter und je nach Struktur der untersuchten Anlage werden diese untersucht. Das Vorliegen eines Fehlers wird mit Hilfe automatisch generierter Prozessmodelle simuliert und die Auswirkung der Fehler bewertet. Der dazu notwendige Code ist unabh\"angig von einer konkreten Anlage formuliert, ben\"otigt zur Simulation aber eine weitere Bibliothek mit Prozessmodellen. QHI ist flexibel einsetzbar, ben\"otigt aber aktualisierte Bibliotheken. In Folge der sehr rechenintensiven Algorithmen konnte sich dieses System nicht in der Industrie etablieren. Im Rahmen modularer Anlagen m\"ussten die Bibliotheken aufwendig aktualisiert werden, weswegen die M\"oglichkeit einer Weiterentwicklung von QHI f\"ur die Sicherheitsuntersuchung modularer Anlagen nicht weiter analysiert wird. 

\citeauthor{Khan_1997} pr\"asentieren in \citetitle{Khan_1997} eine Adaption des konkreten Vorgehens bei der Durchf\"uhrung einer \ac{hazop} \cite{Khan_1997}. Durch Verwendung einer Wissensbasis wollen die Autoren die notwendige Arbeitszeit einer \ac{hazop} in etwa halbieren. Die konkrete Implementierung der Wissensbasis bezeichnen die Autoren als TOPHAZOP, welche sie in \citetitle{Khan_1997a} vorstellen \cite{Khan_1997a}. Als Implementierungsumgebung nutzen \citeauthor{Khan_1997a} die Software \glqq G2\grqq { }der Firma \glqq gensym\grqq { }. Die Wissensbasis ist in einen allgemeinen Teil und den Prozess spezifischen Teil untergliedert. Der allgemeine Teil enth\"alt generische Fehlerursachen und Fehlerauswirkungen, der Prozess spezifische Teil umfasst f\"ur bestimmte Prozesse spezifische Kombinationen aus Ursache und Wirkung, sowie die Fortpflanzung von Auswirkungen auf andere Prozesseinheiten. Die Software wurde durch \citeauthor{Khan_2000} weiterentwickelt, zu EXPERTOP umbenannt und in Visual C++ neu implementiert \cite{Khan_2000}. Die Anwendung von EXPERTOP kann die Durchf\"uhrung einer \ac{hazop} erfolgreich beschleunigen, wenn die zu untersuchende Anlage durch die vorhandene Wissensbasis ad\"aquat beschrieben werden kann. Das Werkzeug ist jedoch nicht in der Lage abstrakt definiertes Wissen auf neue Anlagen zu \"ubertragen. Aus diesem Grund wird EXPERTOP im Rahmen der vorliegenden Arbeit als nicht geeignet bewertet. \newline
Eine erneute Weiterentwicklung von EXPERTOP wird von \citeauthor{Rahman_2009} unter dem Namen $\text{ExpHAZOP}^{\mathplus}$ vorgestellt \cite{Rahman_2009}. EXPERTOP umfasst einen Algorithmus zur Analyse von Fehlerfortpflanzungen \"uber die gesamte Anlage. Der Nutzer interagiert mit dem Programm \"uber eine grafische Oberfl\"ache, in welcher der Nutzer gezielt einzelne Abschnitte der Anlage sicherheitstechnisch ausw\"ahlen und untersuchen lassen kann. Die vorhandene Wissensbasis basiert auf Erfahrungen des Betriebs von Offshore"=Bohranlagen, wodurch das m\"ogliche Anwendungsfeld eingeschr\"ankt wird. Als Nutzereingabe erfordert $\text{ExpHAZOP}^{\mathplus}$ ein \ac{pid}, welches von Hand in $\text{ExpHAZOP}^{\mathplus}$ zu zeichnen ist, da eine Anbindung an andere Softwaresysteme fehlt. Verbesserungen bei der Erweiterbarkeit der Wissensbasis durch den Nutzer im Vergleich zu EXPERTOP werden von \citeauthor{Rahman_2009} nicht benannt, daher ist auch dieses Werkzeug f\"ur modulare Anlagen als ungeeignet einzustufen.  
 
Die besonders umfangreiche Ver\"offentlichung von \citeauthor{McCoy_1999} beschreibt das computergest\"utzte Werkzeug HAZID \cite{McCoy_1999, McCoy_1999a, McCoy_1999b, McCoy_2000, McCoy_2000a}. HAZID besteht aus einer Reihe von Werkzeugen, welche auf Basis von Wissensdatenbanken durch qualitative Fehlerfortpflanzung die Ursachen und Auswirkungen von Prozessfehlern bestimmen. Als Eingangsdaten wird ein \ac{pid} ben\"otigt, welches in der Software \glqq SmartPlant \ac{pid}\grqq { }entweder entworfen oder in diese importiert werden muss. Der \glqq Hazard Import Wizard\grqq { }erstellt auf Basis des \ac{pid} ein Modell der Anlage. Das Modell wird mit Hilfe von zwei Bibliotheken erweitert, welche Stoffeigenschaften, m\"ogliche Fehlerursachen und Fehlerauswirkungen sowie Informationen zur Fortpflanzung von Fehlern enthalten. Die \glqq HAZID Analysis Engine\grqq { }f\"uhrt dann eine hochgradig automatisierte Analyse der Anlage durch und liefert einen umfassenden Bericht im XML Format. Dieser kann mit Hilfe des \glqq HViewer Results Browser\grqq { }betrachtet werden, wobei der Nutzer vielf\"altige Einstellm\"oglichkeit hat, um die Anzeige anzupassen und den XML Bericht gezielt zu durchsuchen. Zus\"atzlich wird der Bericht in einem Format generiert, welcher auf die Empfehlungen zur Notation eines \ac{hazop}--Berichts {(vgl. \cite{Crawley_2015})} angepasst ist. \newline
Die erste Bibliothek beschreibt \glqq Unit Models\grqq { }, welche in einer hierarchischen Struktur gespeichert werden, wodurch das Vererben von Eigenschaften unterst\"utzt wird. Unit Models sind wohl unterscheidbare Einheiten, welche Prozessfunktionen erf\"ullen. Je nach Detaillierungsgrad dieser Einheiten sind die erf\"ullbaren Funktionen mehr oder weniger umfassend. Die konkrete Definition eines Unit Models kann durch den Nutzer mit Hilfe des \glqq Model Generation Tool\grqq { }erfolgen. Ein Unit Model umfasst unter anderem Informationen zu den {Ein-- und Aus}g\"angen und zu m\"oglichen Fehlerfortpflanzungen, welche mit Hilfe von \ac{sdg} modelliert werden. {(vlg. \cite{McCoy_1999a})} \newline
Die zweite Bibliothek umfasst Eigenschaften von Fl\"ussigkeiten. Deren Analyse zeigt beispielsweise eine Brandgefahr oder Toxische Wirkung auf, welche durch das ungewollte vermischen zweier Substanzen entsteht. {(vgl. \cite{McCoy_1999b})} \newline
Die HAZID Analysis Engine f\"uhrt auf Basis der Unit Models und der Eigenschaften der verwendeten Stoffe eine umfassende Fehleranalyse durch. Dazu wird f\"ur eine gegebene Prozessabweichung ein \ac{sdg} konstruiert, welcher die Abweichung mit allen m\"oglichen Auswirkungen verbindet. Die Pfade zu jeder m\"oglichen Auswirkung werden dann mit Hilfe der zuvor definierten Eigenschaften in den Unit Models und den Stoffeigenschaften auf G\"ultigkeit gepr\"uft und die ung\"ultigen Pfade werden entfernt. Die verbleibenden Pfade werden abgespeichert und in den Ergebnisreport \"ubernommen. {(vgl. \cite{McCoy_1999b, McCoy_2000a})}\newline
HAZID wurde seit seiner Ver\"offentlichung erfolgreich weiter entwickelt und ist kommerziell \"uber die Website \url{https://www.hazid.com/} zu erwerben. \citeauthor{Palmer_2009} erweitern das Anwendungsfeld von HAZID erfolgreich von kontinuierlichen Prozessen auf Batchprozesse \cite{Palmer_2009}. \newline
Die der Fehlerfortpflanzung zugrunde liegenden Algorithmen werden in den genannten Ver\"offentlichungen oberfl\"achlich beschrieben. Weiterhin gehen die Autoren davon aus, dass die Bibliothek der Model Units durch einzeln durchgef\"uhrte \acp{hazop} um wichtige Details erg\"anzt werden kann. Bei der Sicherheitsuntersuchung modularer Anlagen ist davon auszugehen, dass die Durchf\"uhrung einer \ac{hazop} f\"ur einzelne Module sinnvoll ist. Wenn es gelingt einzelne Module in Form von Model Units zu beschreiben, so ist HAZID nach dem aktuellen Forschungsstand ein vielversprechendes Mittel, um die Sicherheitsuntersuchung einer aus Modulen bestehenden Anlage zu beschleunigen. Da der verwendete Code jedoch nicht frei verf\"ugbar ist, wird HAZID im Rahmen dieser Arbeit nicht weiter im Detail betrachtet. \newline

Ein weiteres Expertensysteme wird von \citeauthor{Venkatasubramanian1994} in \citetitle{Venkatasubramanian1994} \cite{Venkatasubramanian1994} vorgestellt. Das wissensbasierte System HAZOPExpert ist auf Basis der Software \glqq G2\grqq { }der Firma \glqq gensym\grqq { }implementiert. HAZOPExpert sieht eine Aufteilung des Expertenwissens in Prozess spezifisches und allgemein g\"ultiges Wissen vor. Das allgemein g\"ultige Wissen wird in einer Datenbank hinterlegt und kann bei jeder neuen \ac{hazop} wiederverwendet werden, wobei die Darstellung nach einem objektorientierten Ansatz erfolgt. Dadurch wird eine Erweiterung der Datenbank vereinfacht. Es werden \ac{hazop} Modelle f\"ur verschiedene Prozesseinheiten entworfen, sowie Methoden zur Fehlererkennung,  Auswirkungsanalyse und Fehlerfortpflanzung implementiert. Mit Hilfe des Prozess spezifischen Wissens wie konkreten Stoffeigenschaften und einem \ac{pid} wird das allgemeine Wissen konkretisiert und spezifiziert und die Durchf\"uhrung einer \ac{hazop} damit teilweise automatisiert. Insbesondere k\"onnen Ursachen und Auswirkungen einer betrachteten Abweichung einer Prozessvariablen anlagenweit analysiert werden  \newline
Die Fehleranalyse in HAZOPExpert wurde durch den Einsatz von erweiterten \acp{sdg} verbessert \cite{Vaidhyanathan_1995}, die Bewertung von Fehlerauswirkungen durch Wichtungsmethoden erg\"anzt \cite{Vaidhyanathan_1996}, die Anwendbarkeit von kontinuierlichen Prozessen auf Batchprozesse erweitert \cite{Venkatasubramanian_2000} und die industrielle Nutzbarkeit des Systems anhand von Fallbeispielen verifiziert \cite{Venkatasubramanian1994,Vaidhyanathan1996,Venkatasubramanian_2000}. Die Nutzung von erweiterten \acp{sdg} zur Analyse von Fehlerfortpflanzungen in modularen Anlagen ist eine Methode, welche im Abschnitt \ref{sec:fAna_modQual} gesondert untersucht wird.

Die meisten Systeme zur Automatisierung von \acp{hazop} haben den Nachteil, dass sie nicht an andere Computer gest\"utzte Werkzeuge gekoppelt werden k\"onnen. Die Ursache daf\"ur ist die jeweils Werkzeug spezifische Art der Informationsdarstellung. Der Einsatz einer Ontologie bietet eine formale Darstellung von Begrifflichkeiten und Beziehungen zwischen diesen und kann daher dieses Problem beheben. \citeauthor{Batres_2004} stellt in \citetitle{Batres_2004} eine Ontologie vor, welche f\"ur die Modellierung der Informationen, welche im Rahmen einer \ac{hazop} ben\"otigt werden, genutzt werden kann. \newline
\citeauthor{Zhao_2005} zeigen in \citetitle{Zhao_2005} und \citetitle{Zhao_2005a}, wie die Nutzung einer eigens entwickelten Ontologie im Rahmen des Werkzeugs PHASuite zur automatisierten Durchf\"uhrung von \acp{hazop} erfolgen kann \cite{Zhao_2005,Zhao_2005a}. Die Verwendung der Ontologie erm\"oglicht beispielsweise einen Datenaustausch mit der Software \glqq Batch Plus\grqq { }der Firma \glqq aspentech\grqq { }, welche der Modellierung von Batchprozessen dient und auf der Website \url{http://www.aspentech.com/products/engineering/aspen-batch-process-developer/} kommerziell sowie als Testversion erh\"altlich ist. Weiterhin k\"onnen die Ergebnisse der Sicherheitsuntersuchung beispielsweise mit dem kommerziellen Werkzeug \glqq PHAPro\textregistered\grqq { }ausgetauscht werden. In PHASuite wird die Beschreibung einzelner Prozesseinheiten und Prozessparameter in Form von Tabellen vorgenommen, deren Beschreibungsmittel auf der Ontologie der Autoren basieren. Die Wechselwirkung zwischen Prozesseinheiten wird wie in HAZOPExpert und HAZID durch \acp{sdg} beschrieben. PHASuite enth\"alt ein als Prototyp implementiertes Werkzeug mit dem Namen \glqq Knowledge Builder\grqq { }, mit welchem der Nutzer \"uber eine grafische Oberfl\"ache die Wissensbasis direkt ver\"andern und erweitern kann. Auf Basis der Datenbank, einem durch den Nutzer vorgegebenen \ac{pid} und den durch \acp{sdg} beschriebenen Wechselwirkungen zwischen Komponenten wird der Gesamtprozess in Form von automatisch generierten \glqq Gef\"arbten Petri Netzen \grqq { }(siehe bspw. \cite{Jensen_1997}) modelliert und analysiert. Die von \citeauthor{Zhao_2005} entwickelte Datenbank basiert auf Microsoft Access, die Funktionen von PHASuite sind in Visual $\text{C}\mathplus \mathplus$ umgesetzt. \newline
Die Verwendung von PHASuite f\"ur modulare Anlagen bedarf weitreichender Erweiterungen der Schnittstellen. Das Einlesen umfangreich modellierter Module ist derzeit noch nicht m\"oglich. Eine Anpassung der verwendeten Ontologie w\"are ein geeigneter Weg, um die Implementierung von neuen Schnittstellen zu vereinfachen. Wenn PHASuite und modulare Anlagen die gleichen oder ineinander \"uberf\"uhrbare Ontologien verwenden, so k\"onnte PHASuite f\"ur die Sicherheitsuntersuchung modularer Anlagen erfolgreich eingesetzt werden. Die Verwendung einer objektorientierten Programmierung in Kombination mit der detailliert geplanten Softwarearchitektur von PHASuite und der Verwendung einer Datenbank auf Basis einer Ontologie {(vgl. \cite{Zhao_2005a})} vereinfachen derartige Erweiterungen ma\ss{}geblich. Die vorhandenen Algorithmen zur Fehlerfortpflanzung k\"onnen nicht ohne gro\ss{}en Aufwand vom Rest der Software getrennt eingesetzt werden. Daher wird PHASuite im Rahmen dieser Arbeit als nicht geeignet bewertet, um die Fehlerfortpflanzung in modularen Anlagen zu beschreiben.

In \citetitle{Rossing_2010} zeigen \citeauthor{Rossing_2010} wie durch Einsatz von \ac{mfm} ein Prozess modelliert und eine \ac{hazop} durchgef\"uhrt werden k\"onnen \cite{Rossing_2010}. \citeauthor{Rossing_2010} pr\"asentieren eine Anpassung der Durchf\"uhrung einer \ac{hazop}. Ihr Ansatz gliedert die zu untersuchende Anlage Software gest\"utzt derart, das funktional gleiche Komponenten nicht mehrfach untersucht werden m\"ussen. Ein \ac{mfm} Modell kann nur in Kombination mit zus\"atzlich definierten Kausalzusammenh\"angen die Fehlerfortpflanzung in einem System beschreiben. Zur Analyse von Fehlerfortpflanzungen ist diese Art der Modellierung daher im Vergleich zu Petri Netzen oder \ac{sdg} weniger geeignet.

\citeauthor{Wang_2012} stellen mit HELPHAZOP ein weiteres Expertensystem vor, mit dessen Hilfe die Durchf\"uhrung einer \ac{hazop} verbessert und beschleunigt wird. HELPHAZOP dient dabei als Wissensspeicher vor, w\"ahrend und nach einer \ac{hazop}. Die Autoren adressieren damit das Problem der Informations\"ubergabe von der Prozessplanung zu dem Team, welches die Sicherheitsuntersuchung durchf\"uhren soll. HELPHAZOP vereinfacht durch die Sammlung aller vorhandenen Informationen die konkrete Durchf\"uhrung einer \ac{hazop}, liefert aber selbstst\"andig keine umfangreichen Analysen. Daher kann diese System nicht f\"ur die Analyse von Fehlerfortpflanzung in modularen Anlagen verwendet werden. 

\citeauthor{Boonthum_2014} pr\"asentieren in \citetitle{Boonthum_2014} eine Variante zur Automatisierung von \acp{hazop}, welche durch Einsatz eines Zustandsraummodells den Prozess beschreibt \cite{Boonthum_2014}. Auf Basis dieses Modells werden die Beziehungen zwischen Prozessvariablen mit \acp{sdg} modelliert und anschlie\ss{}ssend in eine Ursache"=Wirkungsmatrix \"uberf\"uhrt. Der implementierte Algorithmus erwartet als Eingangsgr\"o\ss{}en die Systemmatrizen der Zustandsraumbeschreibung und liefert nach automatischer Analyse die Ursache"=Wirkungsmatrix. Der vorgestellte Algorithmus ist in der Lage die Fortpflanzung von Fehlern System weit zu beschreiben, die notwendige Beschreibung des gesamten Prozesses im Zustandsraum erfordert jedoch hohen Modellierungsaufwand. Somit bietet dieser Algorithmus gegen\"uber den bereits vorgestellten Verfahren hinsichtlich der Anwendbarkeit f\"ur modulare Anlagen keine offensichtlichen Vorteile und wird daher als ungeeignet bewertet. 

Die aktuelle Arbeit von \citeauthor{Mechhoud_2016} beschreibt das Werkzeug TORANAS, mit Hilfe dessen eine automatisierte Analyse von petrochemischen Anlagen m\"oglich ist, wobei eine Kombination von \ac{hazop} und \ac{fmea} verwendet wird \cite{Mechhoud_2016}. Die Software ist in Matlab umgesetzt und enth\"alt eine Nutzeroberfl\"ache, Algorithmen zur automatisierten Durchf\"uhrung von \acp{hazop} und \ac{fmea} sowie Simulationen der Ausbreitung von Explosionen. Die Autoren gehen in \citetitle{Mechhoud_2016} in erster Linie auf die Nutzeroberfl\"ache ein und beschreiben, wie der Anwender die Ergebnisse der Sicherheitsuntersuchung visualisieren kann, die implementierten Algorithmen werden nicht erl\"autert. In Folge dessen kann die Anwendbarkeit von TORANAS f\"ur modulare Anlagen nicht fundiert bewertet werden.

Zusammenfassend l\"asst sich feststellen, dass zahlreiche Arbeiten zur Automatisierung von \acp{hazop} existieren, wobei nur wenige der vorgestellten Systeme eine industrielle Reife erlangt haben. Nach derzeitigem Wissensstand existiert kein System, welches ohne Anpassungen f\"ur die Sicherheitsuntersuchung einer modularen Anlagen eingesetzt werden kann, wenn das detaillierte Wissen aus einzelnen Modul genutzt werden soll. \newline
Die verwendeten Verfahren sind gr\"o\ss{}tenteils in der Lage die Fortpflanzung von Fehlern zu analysieren. Die dazu verwendeten Methoden umfassen vor allem \ac{sdg} und Petri Netze, weiterhin kommen Fehlerb\"aume und Zustandsgraphen zum Einsatz {(vgl. \cite[S. 3]{Palmer_2009} )}. Diese Methoden werden in die Klasse der modellbasierten qualitativen Methoden der Fehlerfortpflanzungsanalyse eingeordnet und daher im Abschnitt \ref{sec:fAna_modQual} weiter betrachtet. 

\section{Fehlererkennung und Fehleridentifikation}\label{sec:fAna_fdi}
\textcolor{red}{Definitionen: DIN EN 61511-1: Risiko(risk), Schaden(harm), Gef\"ahrdung(hazard), St\"orung, fault, basic event(s)/root cause(s)/malfunction/failure, sicherheit(safety), DIN EN 61511 }

In Verfahrenstechnischen Anlagen kommt es immer wieder zu Abweichungen des Prozesses vom Sollzustand. Es ist Aufgabe der Anlagenfahrer auf diese Abweichungen geeignet zu reagieren, um den Prozess wieder in den sicheren Sollzustand zu \"uberf\"uhren. Diesen Vorgang k\"onnen Menschen nicht allein bew\"altigen, da sie in Folge des Umfangs moderner Anlagen nicht mehr in der Lage sind, den kompletten Zustand einer Anlage zu erfassen. Zu dem enormen Umfang an Prozessvariablen kommt erschwerend hinzu, dass Abweichungen in Folge von Sensorfehlern und Messungenauigkeiten teilweise erst sp\"at erkannt werden. Dies erschwert das rechtzeitige Initiieren notwendiger Prozesskorrekturen. Durch versp\"atetes ergreifen notwendiger Ma\ss{}nahmen kommt es immer wieder zu St\"orungen und kleinen Unf\"allen. Der dadurch entstehende Schaden betr\"agt j\"ahrlich mehrere Milliarden Euro. Es wurden umfangreiche Forschungen durchgef\"uhrt, um zu ermitteln wie Abweichungen fr\"uhzeitig erkannt und zugrunde liegende Ursachen identifiziert werden k\"onnen. Dieses Vorgehen nennt man \acf{fdi}. Zur Ermittlung der Ursachen einer Prozessabweichung werden erfolgreich Methoden der Fehlerfortpflanzung (engl. fault propagation methods) eingesetzt.  \cite[S. 2]{Venkatasubramanian_2003} \newline
Eine kompakte Einf\"uhrung in die \ac{fdi} findet sich im Abschnitt \glqq Fault Detection and Diagnosis\grqq { }des Buches \citetitle{Baillieul_2015} \cite[S. 417 ff.]{Baillieul_2015}.

Die Fortpflanzung von St\"orungen beim Betrieb chemischer Anlagen ist ein intensiv erforschtes Problem. Von Interesse sind Abweichungen der Prozessvariablen von einem definierten Sollwert und Abweichungen des Betriebszustandes von Ger\"aten, Instrumenten und Gewerken vom Optimalzustand und daraus resultierende Auswirkungen auf den Prozess.

Die Abweichung einer Prozessvariable vom Sollwert kann vielf\"altige, verkettete Ursachen haben. Am Beispiel einer stark exothermen Reaktion zweier fl\"ussiger Stoffe A und B soll dies verdeutlicht werden. Die Reaktion sei dabei noch in der Erprobungsphase, weswegen keine Erfahrungswerte in der gegebenen Anlage bestehen. \newline
Ein Reaktionsbeh\"alter mit R\"uhrer habe zwei \"uber Pumpen gesteuerte Zufl\"usse. Um die gew\"unschte Reaktion zu starten, wird der Stoff A eindosiert. Durch langsame Zugabe von Stoff B soll unter Vermischung der Reaktanten die Reaktion gestartet werden. Sei nun der R\"uhrer durch Alterung deutlich langsamer als im optimalen Zustand und au\ss{}erdem der im Reaktor befindliche Temperatursensor defekt. Trotz Zugabe von Stoff B und Einschalten des R\"uhrers findet die Reaktion dann nicht zu dem erwarteten Zeitpunkt statt, da keine ausreichende Vermischung der Stoffe A und B zustande kommt. Eine Ursachenanalyse ist in Folge fehlender Erfahrungswerte sehr kompliziert. Ein m\"ogliches Vorgehen besteht in der vermehrten Zugabe von Stoff B in der Annahme so die Reaktion starten zu k\"onnen. Durch die vermehrte Zugabe und das langsame Vermischen der Reaktanten beginnt die Reaktion. Dies wird durch den defekten Temperatursensor jedoch nicht bemerkt. Der Stoff B wird daher weiter zudosiert. Die stark exotherme Reaktion geht in Folge dessen durch. Dies wird jedoch erst durch einen Druckanstieg im Beh\"alter erkannt, welcher durch Verdampfen der Reaktanten zustande kommt. Aus Sicht der Anlagenfahrer hat die Reaktion jedoch noch immer nicht begonnen, da die Temperatur im Reaktor nicht gestiegen ist. Der Druckanstieg k\"onnte also einem fehlerhaften Drucksensor oder einer nicht erkannten Reaktion geschuldet sein. Die Ursachenanalyse f\"ur den erh\"ohten Druck ist zu diesem Zeitpunkt in Folge einer m\"oglichen Verkettung von Fehlfunktionen sehr kompliziert. Eine durchgehende Reaktion ist hochgradig gef\"ahrlich. Wird die Gefahr nicht unmittelbar erkannt, so kann dies verheerende Folgen f\"ur die Anlage, die Betreiber und die Umwelt haben.

Das Auffinden m\"oglicher Ursachen eines vorliegenden Fehlers ist eine Aufgabe, welche mit Hilfe der Analyse von Fehlerfortpflanzungen gel\"ost werden soll. Weitere Aufgaben sind die Bewertung und Auslegung von Systemen, welche das Betriebsrisiko einer Anlage auf ein gew\"unschtes Level bringen sollen. Die Planung optimaler Wartungsintervalle ist eine Aufgabe, welche direkt auf deren Ergebnissen aufbauen kann. \newline
Im Rahmen einer Fehlerfortpflanzungsanalyse wird je nach Verfahren der Einfluss von Prozessgr\"o\ss{}en, die r\"aumliche Positionierung von Anlagenteilen, die Alterung von Anlagenkomponenten oder eine Kombination dieser Faktoren betrachtet. Je nach Art der verwendeten Informationen unterscheidet man in \begin{itemize}
\item modellbasierte qualitative Verfahren,
\item modellbasierte quantitative Verfahren und 
\item auf historischen Messdaten basierende Verfahren. 
\end{itemize} 
Die modellbasierten Verfahren sind solche, welche Wissen \"uber die Struktur und Funktion einer Anlage auswerten. Zur Durchf\"uhrung eines solchen Verfahrens werden meist Experten eingesetzt. Die datenbasierten Verfahren analysieren hingegen Messwerte, welche durch den Betrieb der Anlage oder Simulation der Anlage verf\"ugbar werden. Die Auswertung findet dann beispielsweise mit Hilfe statistischer Methoden oder Verfahren zur Mustererkennung statt. \newline
In der dreiteiligen Ver\"offentlichung von \citeauthor{Venkatasubramanian_2003} (\cite{Venkatasubramanian_2003, Venkatasubramanian_2003a,Venkatasubramanian_2003b}) wird eine umfassende Einordnung der bis zum Jahr 2002 ver\"offentlichen Methoden zur Analyse von Fehlerfortpflanzungen in diese drei Kategorien und eine Bewertung der Eignung der Methoden vorgenommen. F\"ur zahlreiche Anwendungsf\"alle er\"ortern die Autoren geeignete Methoden, wodurch sie {Nicht--Ex}perten der Fehlerfortpflanzungsanalyse ein Hilfsmittel zur Evaluierung der Anwendbarkeit bestimmter Methoden f\"ur weitere F\"alle bieten. Diese Ver\"offentlichung ist daher bei der Suche nach einer anwendbaren Methode zur Fehlerfortpflanzungsanalyse ein besonders zweckm\"a\ss{}iger Startpunkt. Es gibt neben den Arbeiten von \citeauthor{Venkatasubramanian_2003} weitere Literaturanalysen, diese haben aber einen geringeren Umfang und konzentrieren sich zumeist auf eine der drei genannten Kategorien. Eine hervorzuhebende Ausnahme bildet die zweiteilige Arbeit von \citeauthor{Gao_2015}, in welcher ein umfangreicher, aktueller Literatur\"uberblick zum Thema \ac{fdi} pr\"asentiert wird \cite{Gao_2015,Gao_2015a}. \newline
In den folgenden Abschnitten \ref{sec:fAna_dat}, \ref{sec:fAna_modQuant} und \ref{sec:fAna_modQual} werden die einzelnen Kategorien detailliert betrachtet, weiter untergliedert und anhand von Beispielmethoden wird die Einsetzbarkeit einiger Methoden f\"ur modulare Anlagen bewertet. 

Als Alternative zu den drei Kategorien nach \citeauthor{Venkatasubramanian_2003}\cite{Venkatasubramanian_2003} ist eine Einteilung in off-line und on-line Verfahren m\"oglich \cite{Kavcic_2001}. Ersteres sind Verfahren, die losgel\"ost vom Betrieb einer Anlage durchgef\"uhrt werden. Ein solches Verfahren kann beispielsweise vor der Erstinbetriebnahme auf Basis von Expertenwissen durchgef\"uhrt werden. Dann ist ein solches Verfahren gleichzeitig ein modellbasiertes Verfahren. Off-line Verfahren k\"onnen aber auch datenbasierte Verfahren sein. Dies ist dann der Fall, wenn Messwerte durch zeitintensive Rechenoperationen ausgewertet werden. Ist dies nicht mehr in Echtzeit m\"oglich, so kann nur eine off-line Analyse durchgef\"uhrt werden. \newline 
Ist ein Verfahren in Echtzeit berechenbar und basiert es auf aktuellen Messdaten einer Anlage, so wird es als on-line Verfahren kategorisiert. Ein solches Verfahren ist zwangsl\"aufig fr\"uhestens nach der Erstinbetriebnahme einer Anlage durchf\"uhrbar. 

Eine klare Abgrenzung der Einteilung in off-line und on-line Verfahren beziehungsweise in modell- und datenbasierte Verfahren ist offensichtlich kompliziert. Im Rahmen dieser Arbeit liegt der Fokus auf den notwendigen Daten, welche zur Durchf\"uhrung eines Verfahrens zur Analyse von Fehlerfortpflanzungen notwendig sind. Die Einteilung, ob es sich um on-line oder off-line Verfahren handelt, ist hingegen nebens\"achlich. Daher wird im Folgenden nur noch in modellbasierte qualitative, modellbasierte quantitative, auf historischen Messdaten basierende  und Hybride dieser Verfahren unterschieden. 
  
\section{Modellbasierte Quantitative Fehlerfortpflanzungsmethoden}\label{sec:fAna_modQuant}
Als modellbasierte quantitative Verfahren zur \ac{fdi} werden im Rahmen dieser Arbeit Verfahren bezeichnet, welche auf Basis von \glqq analytischer Redundanz\grqq { }Residuen generieren, die wiederum zur Fehleridentifikation und Isolation genutzt werden. Die Ausf\"uhrungen im Abschnitt \ref{sec:fAna_modQuant} basieren in weiten Teilen auf der Arbeit von \citeauthor{Venkatasubramanian_2003} \cite{Venkatasubramanian_2003}. 

Verfahren dieser Art bestehen aus zwei grundlegenden Schritten. Im ersten Schritt werden mit Hilfe eines analytischen Prozessmodells und gemessenen Gr\"o\ss{}en durch Einsatz mathematischer Verfahren Residuen generiert. Diese werden im zweiten Schritt analysiert, um das Vorliegen eines Fehlers zu erkennen und diesen eindeutig zu identifizieren.

Ein analytisches Prozessmodell kann entweder durch einen Satz an Gleichungen oder in Form einer Black Box beschrieben werden. Analysiert man die physikalischen und chemischen Gesetze welchen ein Prozess unterliegt, so kann man Massen-, Energie-, Impuls- und Reaktionsbilanzen formulieren. Diese lassen sich allgemein als \acp{dae} darstellen. Betrachtet man hingegen Messwerte der Ein- und Ausgangsgr\"o\ss{}en eines Prozesses, so kann man beispielsweise durch den Einsatz stochastischer Methoden ein Ein"=Ausgangsmodell f\"ur den Prozess erstellen. Diese Art der Modellbildung bezeichnet man als \glqq Systemidentifikation\grqq. Zahlreiche Ver\"offentlichungen der Regelungstechnik behandeln dieses Thema. \citeauthor{Unbehauen_2010} bietet in \citetitle{Unbehauen_2010} \cite{Unbehauen_2010} einen Einstieg in die Systemidentifikation. Fortgeschrittene Verfahren zur Identifikation nicht--linearer Systeme werden beispielsweise von \citeauthor{Schroeder_2010} in \citetitle{Schroeder_2010} \cite{Schroeder_2010} vorgestellt. \newline
Physikalische Modelle zeichnen sich durch die Interpretierbarkeit der Prozessvariablen und eine  genau absch\"atzbare G\"ute aus, haben aber einen hohen Entwurfs-- und Berechnungsaufwand zur Folge.\newline
Es existieren zahlreiche computergest\"utzte Hilfsmittel, welche die Entwicklung eines Black Box Modells unterst\"utzen und damit stark beschleunigen. Ein Beispiel daf\"ur ist die \glqq System Identification Toolbox\texttrademark\grqq { }der Firma \glqq Mathworks\grqq. Black Box Modelle sind daher weniger aufwendig in der Formulierung und der Berechnung als die analytische Beschreibung eines Prozesses. Die Erstellung eines solchen Modells ben\"otigt aber Messdaten, welche alle m\"oglichen Betriebsbedingungen abdecken. Nur so kann ein genaues Modell erstellt werden. Die Erzeugung dieser Daten ist besonders an den Auslegungsgrenzen einer Anlage und f\"ur transiente Bedingungen kompliziert und kostspielig eventuell sogar \"uberhaupt nicht m\"oglich. \newline
Ein analytisches Modell l\"asst sich prinzipiell durch
\begin{align}
y\of{t}&= f\of{u\of{t},w\of{t},x\of{t},\theta\of{t}} \label{gl:fAna_sysIdent}
\end{align}
formulieren, wobei die Systemausgangsgr\"o\ss{}en $y\of{t}$ in Abh\"angigkeit von den Systemeingangsgr\"o\ss{}en $u\of{t}$, den auf das System wirkenden St\"orungen $w\of{t}$ , den Zustandsvariablen des Systems $x\of{t}$ und den Prozessparameter $\theta\of{t}$ berechnet werden.

Ein vorhandener Prozessfehler wirkt sich bei einer Systembeschreibung nach \eqnref{gl:fAna_sysIdent} direkt auf die Zustandsvariablen und beziehungsweise oder auf die Prozessparameter aus. Diese k\"onnen aber h\"aufig nicht direkt gemessen werden. Die Systemein- und Systemausgangsgr\"o\ss{}en k\"onnen hingegen in der Regel entweder durch Sensoren direkt erfasst oder mit Hilfe mathematischer Modelle beobachtet werden. Die Zustandsvariablen und Prozessparameter k\"onnen daher mit Hilfe geeigneter mathematischer Methoden aus Messwerten von $u\of{t}$ und $y\of{t}$ gesch\"atzt werden. Typische Methoden der Zustands- beziehungsweise Parametersch\"atzung sind die \ac{lqm}, die Verwendung von Kalman Filtern, die Formulierung von geeigneten Beobachterstrukturen oder der Einsatz von Parit\"atsgleichungen.   

Die zur Berechnung der Residuen notwendigen Redundanzbeziehungen basieren auf Ein- und Ausgangsgr\"o\ss{}en, welche voneinander nicht unabh\"angig sind. Die Abh\"angigkeiten k\"onnen durch zus\"atzliche Hardware oder analytische Beziehungen erzeugt werden. L\"asst sich die Redundanz als Gleichung formulieren und werden die Redundanzbeziehungen mit den Modellgleichungen zu einem Gleichungssystem kombiniert, so ist dieses Gleichungssystem \"uberbestimmt. Der entstehende Freiheitsgrad der L\"osung wird zur Entwicklung der Residuen genutzt. \newline
Redundanz durch Hardware entsteht beispielsweise durch mehrere Sensoren, welche die gleiche Gr\"o\ss{}e erfassen. Bei sicherheitstechnisch besonders anspruchsvollen Anwendungen wie der Luft- und Raumfahrt ist dieses Vorgehen trotz der damit verbunden gesteigerten Kosten und dem erh\"ohten Raumbedarf \"ublich. Analytische Redundanz wird erreicht, wenn sich bestimmte Sensorwerte algebraisch aus anderen Sensorwerten berechnen lassen, oder wenn es einen zeitlichen Zusammenhang zwischen der \"Anderung von Messwerten gibt, welcher sich analytisch beschreiben l\"asst. Ein Beispiel f\"ur eine solche Gr\"o\ss{}e ist der F\"ullstand in einem Tank. Werden der Zufluss und der Abfluss durch Sensoren erfasst, so kann der F\"ullstand direkt berechnet werden. Wird trotzdem ein F\"ullstandssensor verbaut, so f\"uhrt dies zu nutzbarer Redundanz, da die zeitlichen \"Anderungen der drei Messwerte zueinander vertr\"aglich sein m\"ussen. Ist dies nicht der Fall, so kann auf einen Sensordefekt, ein Leck oder auf einen anderen Fehler geschlossen werden.

Die auf Basis der analytischen Redundanzen ermittelten Residuen sollen zur Fehlerdiagnose eingesetzt werden k\"onnen. Es ist daher zweckm\"a\ss{}ig, wenn die Residuen bei Vorliegen einer Abweichung vom Sollverhalten des Prozesses signifikante Werte annehmen. Liegt keine St\"orung vor, so sollten die Residuen Werte nahe Null annehmen. Weiterhin ist es g\"unstig, wenn die Residuen robust gegen zuf\"allige Fehler wie Sensorrauschen und systematische Fehler wie Modellungenauigkeiten sind.

Als ersten Verfahrenstyp zur Residuenberechnung stellen \citeauthor{Venkatasubramanian_2003}  die Diagnose mit Beobachtern\footnote{im englischen spricht man von \glqq diagnostic observer\grqq { }oder \glqq unknown input observer\grqq { }(UIO)} vor {(\cite[S. 11 ff.]{Venkatasubramanian_2003})}. Methoden dieser Art entwickeln eine bestimmte Menge an Beobachtern, welche Residuen generieren. Jeder dieser Beobachter wird so definiert, dass er bez\"uglich einer definierten Menge an Fehlern sensitiv und bez\"uglich den restlichen Fehlern und unbekannten Gr\"o\ss{}en unempfindlich ist. Die Menge der Beobachter ist derart zu strukturieren, dass jeder Fehler ein eindeutiges Muster an Residuen zur Folge hat. Wird dies erreicht, so kann das Vorliegen eines Fehlers durch stark von null abweichende Werte der Residuen erkannt und mit Hilfe der bekannten Residuenmuster identifiziert werden. Eine wichtige Besonderheit dieses Verfahren ist es, dass die Sch\"atzung der Zustandsvariablen $x\of{t}$ nicht notwendig ist, statt dessen muss nur der Systemausgang durch Messung oder Sch\"atzung ermittelt werden.

Die Formulierung von Parit\"atsgleichungen ist ein alternatives Vorgehen zur Generierung von Residuen {(vgl. \cite[S. 13 f.]{Venkatasubramanian_2003})}. Bei diesem Vorgehen werden die Modellgleichungen geeignet umgestellt, sodass Residuenvektoren entstehen, die orthogonal zueinander sind. Die Residuenvektoren sind dann linear unabh\"angig und das Auftreten jedes betrachteten Fehlers wird durch genau einen Residuenvektor beschrieben. Voraussetzung zur Einsetzbarkeit dieser Methode ist, dass die Anzahl der Ausgangsgr\"o\ss{}en gr\"o\ss{}er als die der Zustandsgr\"o\ss{}en ist. Dieser Zusammenhang wird in Definition \ref{def:fAna_sysRedundanz} verdeutlicht. 
\begin{defn}[Systemredundanz]\label{def:fAna_sysRedundanz}
Sei ein System nach \eqnref{gl:fAna_sysIdent} beschrieben und gelten die Eigenschaften
\begin{align}
\dim\of{y\of{t}}&= n, &\dim\of{x\of{t}}&= m, & n&>m, \label{gl:fAna_sysRedundanzBdg}
\end{align}
dann ist das System redundant mit dem Freiheitsgrad
\begin{align}
f&= n-m, \label{gl:fAna_sysDimRedundanz}
\end{align}
da das System mehr erfassbare Ausgangsgr\"o\ss{}en als Zust\"ande umfasst.
\end{defn}
Mit Hilfe des Freiheitsgrades $f$ kann dann eine Projektionsmatrix $\matr{V}$ derart entworfen werden, dass f\"ur Abweichungen von jedem redundant vorhandenen Ausgangswert ein Vektor berechenbar ist, der zu den anderen Vektoren dieser Art orthogonal ist. \newline
Parit\"atsgleichungen und die Verwendung von Beobachtern zur Residuenerzeugung \"ahneln sich sehr stark. Beide Verfahren sind ohne eine Sch\"atzung von $x\of{t}$ anwendbar. Man kann sogar zeigen, dass beide Verfahren unter Verwendung der gleichen Designziele zu \"aquivalenten Residuen f\"ur ein fehlerbehaftetes System f\"uhren. Die Methoden der Auswertung von Residuen zur Diagnose von Fehlern sind f\"ur diese beiden Verfahren ebenfalls gleich. \"Ublich ist die Definition von Schwellwerten f\"ur die Residuen, bei deren \"Uberschreiten ein Fehler als vorliegend erkannt wird. 

Es gibt weitere Methoden, welche Residuen auf Basis quantitativer Modelle berechnen um so Fehler zu diagnostizieren und zu isolieren. Dazu z\"ahlen Methoden, welche die Zustandsvariablen oder Prozessparameter sch\"atzen, um auf Basis derer Residuen zu generieren. Dies sind beispielsweise Kalman Filter und \ac{lqm} {(vgl. \cite[S. 14 f.]{Venkatasubramanian_2003})}. Au\ss{}erdem gibt es fortgeschrittene Methoden zur Residuenberechnung wie der Entwurf von gerichteten oder strukturierten Residuen\footnote{engl. directional residuals and structured residuals} {(vgl. \cite[S. 15 f.]{Venkatasubramanian_2003})}. 

\subsection{Bewertung modellbasierter quantitativer Methoden der \ac{fdi} f\"ur modulare Anlagen}
Modellbasierte quantitative Methoden bieten den gro\ss{}en Vorteil, dass der Anwender bei der Wahl eines Verfahrens zur Residuengenerierung viele Freiheiten hat. Auch die Verfahren selbst bieten M\"oglichkeiten, um sie hinsichtlich der Erkennung bestimmter Fehler gezielt zu entwerfen. Werden entkoppelte Beobachterstrukturen geeignet entworfen, so kann jeder betrachte Fehler durch einen gesonderten Beobachter gezielt diagnostiziert werden. Dem gegen\"uber steht der gro\ss{}e Nachteil der Notwendigkeit von m\"oglichst genauen Prozessmodellen. Der Entwurf dieser Modelle ist aufwendig und h\"aufig mit Ungenauigkeiten verbunden. Dies gilt f\"ur analytische Modelle und Black Box Modelle gleicherma\ss{}en. Weiterhin sind Analysemethoden, welche auf quantitativen Modellen basieren, in aller Regel auf die Erkennung von Fehlern, welche additiv auftreten, beschr\"ankt. Die Erkennung von multiplikativ auftretenden Fehlern wie einem Drift von Prozessparametern ist nur in Sonderf\"allen m\"oglich. Dar\"uber hinaus m\"ussen die Residuen zur Erkennung von Fehlern vorab definiert werden. Das Auftreten von vorab  unbekannten Fehlern ist dadurch nur stark eingeschr\"ankt m\"oglich. Auch die Ursachenanalyse ist zumeist nicht m\"oglich -- nur das Vorliegen eines Fehlers wird diagnostiziert und der konkrete Fehler ermittelt.  {\cite[S. 17 f.]{Venkatasubramanian_2003}}

In Hinblick auf modular konstruierte Anlagen l\"asst sich feststellen, dass Methoden dieser Kategorie nicht geeignet sind, um die zur Genehmigung einer aus Modulen bestehenden Anlage notwendige Sicherheitsuntersuchung zu beschleunigen oder anderweitig zu vereinfachen. \newline
F\"ur die Analyse eines einzelnen Moduls k\"onnten jedoch solche Verfahren zum Einsatz kommen. Module sollen entsprechend ihrer Definition einzeln komplett testbar sein. Daher ist die Erstellung von Ein"=/Ausgangsdaten und darauf aufbauend die Entwicklung eines Black Box Modells prinzipiell m\"oglich. Die Erstellung eines analytischen Modells durch den Modullieferanten ist ebenfalls m\"oglich und sollte ohnehin ein Ziel dessen sein, denn auf Basis eines analytischen Modells k\"onnen die im Abschnitt \ref{sec:sdt_modularisierung} geforderten Simulationsmodelle geeignet entworfen werden. Die notwendige Grundlage f\"ur modellbasierte quantitative Verfahren w\"are damit zumindest auf Modulebene gegeben, jedoch verbleiben zwei bedeutende Probleme: \begin{itemize}
\item der aufwendige Entwurf eines analytischen Modells der Gesamtanlage ist notwendig und
\item nicht erkannte Fehler k\"onnen nicht verl\"asslich identifiziert werden.
\end{itemize}
Zum einen ist zu erwarten, dass die Modelle der einzelnen Module noch keine ausreichenden Informationen \"uber die m\"oglichen Wechselwirkungen, welche im Rahmen der Gesamtanlage auftreten k\"onnen, enthalten. Ein Modell der Gesamtanlage m\"usste daher vor dem Einsatz von Verfahren der betrachteten Kategorie noch erstellt werden. Dies w\"are nur durch analytische Ans\"atze m\"oglich. Die zur Generierung von Black Box Modellen notwendigen Ein"=/Ausgangsdaten m\"ussten von der Gesamtanlage stammen, diese ist zum Zeitpunkt der durchzuf\"uhrenden Sicherheitsbetrachtung aber noch gar nicht betriebsf\"ahig. Die Durchf\"uhrung praktischer Tests und die Erstellung von Messdaten ist daher keine zur Verf\"ugung stehende Option und die Erstellung von Black Box Modellen nicht m\"oglich. Die Erstellung von analytischen Modellen ist mit den bereits genannten Problemen des hohen Aufwands und der entstehenden Ungenauigkeiten verbunden. Die notwendige Entwicklungsleistung eines analytischen Modells reduziert damit m\"ogliche Zeiteinsparungen bei der Sicherheitsbetrachtung ma\ss{}geblich. \newline
Zum anderen werden vorab unbekannte Fehler durch modellbasierte quantitative Verfahren nicht verl\"asslich identifiziert. Das Auffinden von bisher nicht betrachteten Fehlern wird also bereits auf der Betrachtungsebene einzelner Module durch Methoden dieser Kategorie nicht erm\"oglicht. Durch das Verbinden von Modulen zu einer Gesamtanlage ist mit neuen Fehlerquellen und f\"ur die Anlage spezifischen m\"oglichen Auswirkungen zu rechnen. Die Erkennung dieser neuen Fehler ist nicht m\"oglich. Die durch Kopplung der Module potentiellen neuen Fehler sind aber genau die Fehler, welche durch die Sicherheitsuntersuchung der Gesamtanlage aufgedeckt werden m\"ussen. Eine Vereinfachung dieser Aufgabe durch die Verwendung von modellbasierten quantitativen Verfahren zur Fehlerdiagnose ist damit nicht zu erwarten.

Im Rahmen der vorliegenden Arbeit soll davon ausgegangen werden, dass die vorhandene Datenbasis aus f\"ur die einzelnen Module durchgef\"uhrten \acp{hazop}, Beschreibungen der Module und einer Beschreibung der Gesamtanlage besteht. Auf dieser Basis l\"asst sich nicht ohne gro\ss{}en Aufwand ein analytisches Modell der Gesamtanlage formulieren. Selbst wenn quantitative modellbasierte Verfahren der \ac{fdi} potentielle Einsparungen bei der Sicherheitsbetrachtung der Gesamtanlage bieten w\"urden, was wie dargelegt wird, nicht der Fall ist, so w\"are die notwendige Datenbasis in keiner weise vorhanden. Im Rahmen dieser Arbeit ergibt sich daher  zwangsl\"aufig, dass der Einsatz von modellbasierten quantitativen Verfahren nicht geeignet ist, um die Fehlerfortpflanzung innerhalb einer aus Modulen bestehenden Anlage zu untersuchen.
 
\section{Modellbasierte Qualitative  Fehlerfortpflanzungsmethoden}\label{sec:fAna_modQual}
Modellbasierte qualitative Verfahren unterscheiden sich von den modellbasierten quantitativen dadurch, wie das vorab vorhandene Wissen \"uber den betrachteten Prozess formuliert wird. Im Fall der quantitativen Verfahren dienen dazu mathematische Gleichungen, bei den qualitativen Methoden werden die bekannten Beziehungen zwischen Prozessvariablen als relative Aussagen formuliert. Diese relativen Aussagen beschreiben zumeist eine Abh\"angigkeit der Wertentwicklung von Prozessvariablen zueinander. Ein solche Beziehung besteht beispielsweise zwischen dem Druck und der Temperatur in einem geschlossenen Beh\"alter. Als qualitative Aussage kann formuliert werden, dass ein Ansteigen der Temperatur einen Druckanstieg zur Folge hat. Die Aussage l\"asst jedoch keinen Schluss \"uber das Ausma\ss{} der \"Anderung zu und wird daher als qualitativ bezeichnet. \newline
Aufbauend auf qualitativen Aussagen kann ein System entworfen werden, welches zur \ac{fdi} genutzt werden kann. \newline
Die Formulierung eines qualitativen Modells wird vorwiegend durch Experten vorgenommen. Soll eine Methode zur \ac{fdi} auf Basis eines qualitativen Modells angewandt werden, so ist eine Untersuchung des Modellverhaltens im Sollzustand und im fehlerbehafteten Zustand notwendig. Die Beschreibung des Sollzustandes wird im Rahmen der Prozessentwicklung durchgef\"uhrt. Eine geeignete qualitative Beschreibung des Prozesses bei Vorliegen von Fehlern wird in Folge der im Abschnitt \ref{sec:sdt_hazop} beschriebenen \ac{hazop} erlangt. Es sind aber auch andere Methoden zur qualitativen Beschreibung von Fehlzust\"anden m\"oglich. \textcolor{red}{Verweis auf Abschnitt, in welchem andere Methoden genannt werden.}

Zur Darstellung einer qualitativen Systembeschreibung gibt es mehrere M\"oglichkeiten. Ein umfassend untersuchter Ansatz in die Verwendung von wissensbasierten Expertensystemen \footnote{engl. knowledge-based expert systems}. Im Abschnitt \ref{sec:fAna_automatHazop} werden einige Systeme vorgestellt, welche als wissensbasierte Expertensysteme aufgebaut sind. Ein solches System definiert in einer f\"ur einen PC analysierbaren Weise eine vorgegebene Menge von Aussagen, welche das Systemverhalten beschreiben. Dies geschieht zumeist durch den Einsatz von verschachtelten wenn--dann--sonst Formulierungen. Das Expertensystem kann auf Basis von vorgegebenen Zust\"anden die G\"ultigkeit der zuvor definierten Aussagen pr\"ufen und dadurch Schlussfolgerungen ziehen. Als vorgegebener Zustand kann beispielsweise die Abweichung einer Prozessvariable vom Sollverhalten definiert werden, deren Auswirkung zu untersuchen ist. Das Expertensystem imitiert auf Basis der wenn--dann--sonst Zusammenh\"ange die Gedankeng\"ange eines Menschen und ermittelt die m\"oglichen Auswirkungen. Ein gro\ss{}er Vorteil des Expertensystems ist es, dass keine zuvor definierten Auswirkungen \"ubersehen werden k\"onnen. Jedoch ist das Expertensystem stets auf die zuvor definierten Systemeigenschaften beschr\"ankt und kann keine komplett neuen Erkenntnisse entwickeln, da es auf Deduktion basiert.

Das Ziehen von Schlussfolgerungen basiert in aller Regel auf der Auswertung von vorhandenem Wissen. Dieses Vorgehen geschieht unter der Anwendung von \glqq Inferenzmechanismen\grqq { }. Diese schlie\ss{}en das Pr\"ufen von bekannten Aussagen auf Erf\"ulltheit durch Analyse eines aktuellen Zustands ebenso wie das Herleiten von neuen Aussagen (Konklusionen) auf Basis von bekannten, wahren Aussagen (Pr\"amissen) ein. Man unterscheidet dabei die drei Inferenzmechanismen \begin{enumerate}
\item deduktives Schlie\ss{}en,
\item abduktives Schlie\ss{}en und
\item induktives Schlie\ss{}en.
\end{enumerate} 
Als deduktives Schlie\ss{}en bezeichnet man die Formulierung einer logischen Konsequenz, welche auf Basis von mehreren festgelegten Pr\"amissen formuliert wird. Dabei findet zumeist eine Ableitung vom Allgemeinen auf einen Einzelfall statt. Als Beispiel dienen die Pr\"amissen ein Zufluss erh\"oht das Volumen im Lagertank und Fl\"ussigkeit A str\"omt in den Lagertank. Als Konklusion erh\"alt man die Aussage, dass die Fl\"ussigkeit A das Tankvolumen erh\"oht. \newline
Beim abduktiven Schlie\ss{}en folgert man aus einer Pr\"amisse und einem beobachteten Resultat die G\"ultigkeit einer Voraussetzung. Eine Abduktion liefert bei Betrachtung gewisser Effekte eine plausible Ursache f\"ur deren Eintreten, die durchgef\"uhrte Schlussfolgerung muss jedoch nicht notwendigerweise korrekt sein. Trotzdem ist dieses Vorgehen bei der Suche nach m\"oglichen Erkl\"arungen f\"ur einen beobachteten Effekt sehr hilfreich. Der Zusammenhang \glqq das \"Offnen des Abflussventils senkt das Tankvolumen\grqq { }in Verbindung mit der Beobachtung, dass das Tankvolumen sinkt, l\"asst den abduktiven Schluss zu, dass das Abflussventil ge\"offnet ist. Eine alternative Begr\"undung ist aber auch, dass der Tank ein Leck aufweist. Die abduktive Schlussfolgerung ist demnach zwingend auf G\"ultigkeit zu pr\"ufen. \newline
Die Induktion ist in ihrem Vorgehen in etwa das Gegenteil der Deduktion. Im Rahmen induktiver Schlussfolgerungen wird aus mehreren Beobachtungen eine allgemeine Regel abstrahiert. Deren G\"ultigkeit ist aber nicht zwangsl\"aufig gegeben. Werden in einem Reaktor zwei Stoffe mit hoher Temperatur zugef\"uhrt und zu einer endothermen Reaktion gebracht so k\"onnte man schlie\ss{}en, dass das Zusammenf\"uhren von zwei Reaktanten prinzipiell ein Absinken der Temperatur zur Folge hat. Diese Aussage ist f\"ur exotherm reagierende Stoffe aber offensichtlich falsch und mit einem hohem Risiko verbunden. \newline
Im Gegensatz zu Induktion und Abduktion ist die Deduktion wahrheitserhaltend und daher die zu bevorzugende Inferenzmethode. Ist Deduktion jedoch nicht m\"oglich oder zu aufwendig, so ist der Einsatz von Induktion oder Abduktion aber m\"oglich, um neue Aussagen zum Systemverhalten zu generieren. {(vgl. \cite[S. 28 ff.]{Dengel_2012})}



\subsection{Verwendung von \ac{sdg} zur Fehlerfortpflanzung}
siehe markierte Quellen in \cite{Boonthum_2014}.

\section{auf historischen Messdaten basierende Fehlerfortpflanzungsmethoden}\label{sec:fAna_dat}
Diese Verfahren beruhen auf der Verwendung von historischen Messdaten konkreter Anlagen. Je nach Verfahren werden Messdaten zum Normalbetrieb oder beziehungsweise und Daten zum St\"orbetrieb ben\"otigt. Manche Methoden ben\"otigen weiterhin ein \ac{pid}. Ziel der Verfahren ist es zum einen St\"orungen der Anlage fr\"uhzeitig zu erkennen und zum anderen deren Ursache oder Ursachen zu ermitteln. Dazu werden Ursache--Effekt Beziehungen zwischen untersuchten Parametern ermittelt.  

Die Auswertung historischer Messdaten basiert zumeist auf statistischen Methoden. Die kausalen Zusammenh\"ange, welche mit Hilfe dieser Methoden ermittelt werden sollen, k\"onnen dann geeignet als Graphen dargestellt werden. Wie man aus statistischen Gr\"o\ss{}en kausale Zusammenh\"ange ermitteln kann wird in den fr\"uhen Werken \textcite{Holland_1986} und  \textcite{Pearl_1995} aufgezeigt. Ein umfassendes Lehrbuch zu dieser Thematik wurde von \citeauthor{Pearl_2009} ver\"offentlicht, welches mittlerweile in der zweiten Auflage verf\"ugbar ist \cite{Pearl_2009}.

Datenbasierte Methoden sind hervorragend f\"ur die Erstellung von quantitativen Modellen geeignet. Eine Erstellung von qualitativen Modellen ist ebenfalls m\"oglich. \textcolor{red}{Beispiele}
\paragraph*{Nennung von Verfahren}
\cite{Zhang_2017}, \cite{Thornhill_2006}



\section{Vorstellung ausgew\"ahlter Algorithmen}
\section{Bewertung der Verwendbarkeit f\"ur modulare Anlagen}
Wichtige Gesichtspunkte:
\begin{itemize}
\item Welche Daten sind notwendig
\item Automatisierbarkeit der Berechnung
\item Dokumentationsf\"ahigkeit der Ergebnisse
\item Sind die Ergebnisse f\"ur eine \ac{hazop} nutzbar
\end{itemize}