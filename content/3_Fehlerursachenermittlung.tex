\chapter{Fehlerursachenermittlung} \label{ch:fehlerfortpfl}
\textcolor{red}{Definitionen: DIN EN 61511-1: Risiko(risk), Schaden(harm), Gef\"ahrdung(hazard), St\"orung, fault, basic event(s)/root cause(s)/malfunction/failure, sicherheit(safety), DIN EN 61511 }

In Verfahrenstechnischen Anlagen kommt es immer wieder zu Abweichungen des Prozesses vom Sollzustand. Es ist Aufgabe der Anlagenfahrer auf diese Abweichungen geeignet zu reagieren, um den Prozess wieder in den sicheren Sollzustand zu \"uberf\"uhren. Diesen Vorgang k\"onnen Menschen nicht allein bew\"altigen, da sie in Folge des Umfangs moderner Anlagen nicht mehr in der Lage sind, den kompletten Zustand einer Anlage zu erfassen. Zu dem enormen Umfang an Prozessvariablen kommt erschwerend hinzu, dass Abweichungen in Folge von Sensorfehlern und Messungenauigkeiten teilweise erst sp\"at erkannt werden. Dies erschwert das rechtzeitige Initiieren notwendiger Prozesskorrekturen. Durch versp\"atetes ergreifen notwendiger Ma\ss{}nahmen kommt es immer wieder zu St\"orungen und kleinen Unf\"allen. Der dadurch entstehende Schaden betr\"agt j\"ahrlich mehrere Milliarden Euro. Es wurden umfangreiche Forschungen durchgef\"uhrt, um zu ermitteln wie Abweichungen fr\"uhzeitig erkannt und zugrunde liegende Ursachen identifiziert werden k\"onnen. Dieses Vorgehen nennt man \ac{fdi}. Zur Ermittlung der Ursachen einer Prozessabweichung werden erfolgreich Methoden der Fehlerfortpflanzung (engl. fault propagation methods) eingesetzt.  \cite[S. 2]{Venkatasubramanian_2003} \newline
Eine kompakte Einf\"uhrung in die \ac{fdi} findet sich im Abschnitt \glqq Fault Detection and Diagnosis\grqq { }des Buches \citetitle{Baillieul_2015} \cite[S. 417 ff.]{Baillieul_2015}.

Die Fortpflanzung von St\"orungen beim Betrieb chemischer Anlagen ist ein intensiv erforschtes Problem. Von Interesse sind Abweichungen der Prozessvariablen von einem definierten Sollwert und Abweichungen des Betriebszustandes von Ger\"aten, Instrumenten und Gewerken vom Optimalzustand und daraus resultierende Auswirkungen auf den Prozess.

Die Abweichung einer Prozessvariable vom Sollwert kann vielf\"altige, verkettete Ursachen haben. Am Beispiel einer stark exothermen Reaktion zweier fl\"ussiger Stoffe A und B soll dies verdeutlicht werden. Die Reaktion sei dabei noch in der Erprobungsphase, weswegen keine Erfahrungswerte in der gegebenen Anlage bestehen. \newline
Ein Reaktionsbeh\"alter mit R\"uhrer habe zwei \"uber Pumpen gesteuerte Zufl\"usse. Um die gew\"unschte Reaktion zu starten, wird der Stoff A eindosiert. Durch langsame Zugabe von Stoff B soll unter Vermischung der Reaktanten die Reaktion gestartet werden. Sei nun der R\"uhrer durch Alterung deutlich langsamer als im optimalen Zustand und au\ss{}erdem der im Reaktor befindliche Temperatursensor defekt. Trotz Zugabe von Stoff B und Einschalten des R\"uhrers findet die Reaktion dann nicht zu dem erwarteten Zeitpunkt statt, da keine ausreichende Vermischung der Stoffe A und B zustande kommt. Eine Ursachenanalyse ist in Folge fehlender Erfahrungswerte sehr kompliziert. Ein m\"ogliches Vorgehen besteht in der vermehrten Zugabe von Stoff B in der Annahme so die Reaktion starten zu k\"onnen. Durch die vermehrte Zugabe und das langsame Vermischen der Reaktanten beginnt die Reaktion. Dies wird durch den defekten Temperatursensor jedoch nicht bemerkt. Der Stoff B wird daher weiter zudosiert. Die stark exotherme Reaktion geht in Folge dessen durch. Dies wird jedoch erst durch einen Druckanstieg im Beh\"alter erkannt, welcher durch Verdampfen der Reaktanten zustande kommt. Aus Sicht der Anlagenfahrer hat die Reaktion jedoch noch immer nicht begonnen, da die Temperatur im Reaktor nicht gestiegen ist. Der Druckanstieg k\"onnte also einem fehlerhaften Drucksensor oder einer nicht erkannten Reaktion geschuldet sein. Die Ursachenanalyse f\"ur den erh\"ohten Druck ist zu diesem Zeitpunkt in Folge einer m\"oglichen Verkettung von Fehlfunktionen sehr kompliziert. Eine durchgehende Reaktion ist hochgradig gef\"ahrlich. Wird die Gefahr nicht unmittelbar erkannt, so kann dies verheerende Folgen f\"ur die Anlage, die Betreiber und die Umwelt haben.

Das Auffinden m\"oglicher Ursachen eines vorliegenden Fehlers ist eine Aufgabe, welche mit Hilfe der Analyse von Fehlerfortpflanzungen gel\"ost werden soll. Weitere Aufgaben sind die Bewertung und Auslegung von Systemen, welche das Betriebsrisiko einer Anlage auf ein gew\"unschtes Level bringen sollen. Die Planung optimaler Wartungsintervalle ist eine Aufgabe, welche direkt auf deren Ergebnissen aufbauen kann. \newline
Im Rahmen einer Fehlerfortpflanzungsanalyse wird je nach Verfahren der Einfluss von Prozessgr\"o\ss{}en, die r\"aumliche Positionierung von Anlagenteilen, die Alterung von Anlagenkomponenten oder eine Kombination dieser Faktoren betrachtet. Je nach Art der verwendeten Informationen unterscheidet man in \begin{itemize}
\item modellbasierte qualitative Verfahren,
\item modellbasierte quantitative Verfahren und 
\item auf historischen Messdaten basierende Verfahren. 
\end{itemize} 
Die modellbasierten Verfahren sind solche, welche Wissen \"uber die Struktur und Funktion einer Anlage auswerten. Zur Durchf\"uhrung eines solchen Verfahrens werden meist Experten eingesetzt. Die datenbasierten Verfahren analysieren hingegen Messwerte, welche durch den Betrieb der Anlage oder Simulation der Anlage verf\"ugbar werden. Die Auswertung findet dann beispielsweise mit Hilfe statistischer Methoden oder Verfahren zur Mustererkennung statt. \newline
In der dreiteiligen Ver\"offentlichung von \citeauthor{Venkatasubramanian_2003} (\cite{Venkatasubramanian_2003, Venkatasubramanian_2003a,Venkatasubramanian_2003b}) wird eine umfassende Einordnung der bis zum Jahr 2002 ver\"offentlichen Methoden zur Analyse von Fehlerfortpflanzungen in diese drei Kategorien und eine Bewertung der Eignung der Methoden vorgenommen. F\"ur zahlreiche Anwendungsf\"alle er\"ortern die Autoren geeignete Methoden, wodurch sie {Nicht--Ex}perten der Fehlerfortpflanzungsanalyse ein Hilfsmittel zur Evaluierung der Anwendbarkeit bestimmter Methoden f\"ur weitere F\"alle bieten. Diese Ver\"offentlichung ist daher bei der Suche nach einer anwendbaren Methode zur Fehlerfortpflanzungsanalyse ein besonders zweckm\"a\ss{}iger Startpunkt. Es gibt neben den Arbeiten von \citeauthor{Venkatasubramanian_2003} weitere Literaturanalysen, diese haben aber einen geringeren Umfang und konzentrieren sich zumeist auf eine der drei genannten Kategorien. Eine hervorzuhebende Ausnahme bildet die zweiteilige Arbeit von \citeauthor{Gao_2015}, in welcher ein umfangreicher, aktueller Literatur\"uberblick zum Thema \ac{fdi} pr\"asentiert wird \cite{Gao_2015,Gao_2015a}. \newline
In den folgenden Abschnitten \ref{sec:fAna_dat}, \ref{sec:fAna_modQuant} und \ref{sec:fAna_modQual} werden die einzelnen Kategorien detailliert betrachtet, weiter untergliedert und anhand von Beispielmethoden wird die Einsetzbarkeit einiger Methoden f\"ur modulare Anlagen bewertet. 

Als Alternative zu den drei Kategorien nach \citeauthor{Venkatasubramanian_2003}\cite{Venkatasubramanian_2003} ist eine Einteilung in off-line und on-line Verfahren m\"oglich \cite{Kavcic_2001}. Ersteres sind Verfahren, die losgel\"ost vom Betrieb einer Anlage durchgef\"uhrt werden. Ein solches Verfahren kann beispielsweise vor der Erstinbetriebnahme auf Basis von Expertenwissen durchgef\"uhrt werden. Dann ist ein solches Verfahren gleichzeitig ein modellbasiertes Verfahren. Off-line Verfahren k\"onnen aber auch datenbasierte Verfahren sein. Dies ist dann der Fall, wenn Messwerte durch zeitintensive Rechenoperationen ausgewertet werden. Ist dies nicht mehr in Echtzeit m\"oglich, so kann nur eine off-line Analyse durchgef\"uhrt werden. \newline 
Ist ein Verfahren in Echtzeit berechenbar und basiert es auf aktuellen Messdaten einer Anlage, so wird es als on-line Verfahren kategorisiert. Ein solches Verfahren ist zwangsl\"aufig fr\"uhestens nach der Erstinbetriebnahme einer Anlage durchf\"uhrbar. 

Eine klare Abgrenzung der Einteilung in off-line und on-line Verfahren beziehungsweise in modell- und datenbasierte Verfahren ist offensichtlich kompliziert. Im Rahmen dieser Arbeit liegt der Fokus auf den notwendigen Daten, welche zur Durchf\"uhrung eines Verfahrens zur Analyse von Fehlerfortpflanzungen notwendig sind. Die Einteilung, ob es sich um on-line oder off-line Verfahren handelt, ist hingegen nebens\"achlich. Daher wird im Folgenden nur noch in modellbasierte qualitative, modellbasierte quantitative, auf historischen Messdaten basierende  und Hybride dieser Verfahren unterschieden. 
  
\section{Modellbasierte Quantitative Fehlerfortpflanzungsmethoden}\label{sec:fAna_modQuant}
Als modellbasierte quantitative Verfahren zur \ac{fdi} werden im Rahmen dieser Arbeit Verfahren bezeichnet, welche auf Basis von \glqq analytischer Redundanz\grqq { }Residuen generieren, die wiederum zur Fehleridentifikation und Isolation genutzt werden. Die Ausf\"uhrungen im Abschnitt \ref{sec:fAna_modQuant} basieren in weiten Teilen auf der Arbeit von \citeauthor{Venkatasubramanian_2003} \cite{Venkatasubramanian_2003}. 

Verfahren dieser Art bestehen aus zwei grundlegenden Schritten. Im ersten Schritt werden mit Hilfe eines analytischen Prozessmodells und gemessenen Gr\"o\ss{}en durch Einsatz mathematischer Verfahren Residuen generiert. Diese werden im zweiten Schritt analysiert, um das Vorliegen eines Fehlers zu erkennen und diesen eindeutig zu identifizieren.

Ein analytisches Prozessmodell kann entweder durch einen Satz an Gleichungen oder in Form einer Black Box beschrieben werden. Analysiert man die physikalischen und chemischen Gesetze welchen ein Prozess unterliegt, so kann man Massen-, Energie-, Impuls- und Reaktionsbilanzen formulieren. Diese lassen sich allgemein als \acp{dae} darstellen. Betrachtet man hingegen Messwerte der Ein- und Ausgangsgr\"o\ss{}en eines Prozesses, so kann man beispielsweise durch den Einsatz stochastischer Methoden ein Ein"=Ausgangsmodell f\"ur den Prozess erstellen. Diese Art der Modellbildung bezeichnet man als \glqq Systemidentifikation\grqq. Zahlreiche Ver\"offentlichungen der Regelungstechnik behandeln dieses Thema. \citeauthor{Unbehauen_2010} bietet in \citetitle{Unbehauen_2010} \cite{Unbehauen_2010} einen Einstieg in die Systemidentifikation. Fortgeschrittene Verfahren zur Identifikation nicht--linearer Systeme werden beispielsweise von \citeauthor{Schroeder_2010} in \citetitle{Schroeder_2010} \cite{Schroeder_2010} vorgestellt. \newline
Physikalische Modelle zeichnen sich durch die Interpretierbarkeit der Prozessvariablen und eine  genau absch\"atzbare G\"ute aus, haben aber einen hohen Entwurfs-- und Berechnungsaufwand zur Folge.\newline
Es existieren zahlreiche computergest\"utzte Hilfsmittel, welche die Entwicklung eines Black Box Modells unterst\"utzen und damit stark beschleunigen. Ein Beispiel daf\"ur ist die \glqq System Identification Toolbox\texttrademark\grqq { }der Firma \glqq Mathworks\grqq. Black Box Modelle sind daher weniger aufwendig in der Formulierung und der Berechnung als die analytische Beschreibung eines Prozesses. Die Erstellung eines solchen Modells ben\"otigt aber Messdaten, welche alle m\"oglichen Betriebsbedingungen abdecken. Nur so kann ein genaues Modell erstellt werden. Die Erzeugung dieser Daten ist besonders an den Auslegungsgrenzen einer Anlage und f\"ur transiente Bedingungen kompliziert und kostspielig eventuell sogar \"uberhaupt nicht m\"oglich. \newline
Ein analytisches Modell l\"asst sich prinzipiell durch
\begin{align}
y\of{t}&= f\of{u\of{t},w\of{t},x\of{t},\theta\of{t}} \label{gl:fAna_sysIdent}
\end{align}
formulieren, wobei die Systemausgangsgr\"o\ss{}en $y\of{t}$ in Abh\"angigkeit von den Systemeingangsgr\"o\ss{}en $u\of{t}$, den auf das System wirkenden St\"orungen $w\of{t}$ , den Zustandsvariablen des Systems $x\of{t}$ und den Prozessparameter $\theta\of{t}$ berechnet werden.

Ein vorhandener Prozessfehler wirkt sich bei einer Systembeschreibung nach \eqnref{gl:fAna_sysIdent} direkt auf die Zustandsvariablen und beziehungsweise oder auf die Prozessparameter aus. Diese k\"onnen aber h\"aufig nicht direkt gemessen werden. Die Systemein- und Systemausgangsgr\"o\ss{}en k\"onnen hingegen in der Regel entweder durch Sensoren direkt erfasst oder mit Hilfe mathematischer Modelle beobachtet werden. Die Zustandsvariablen und Prozessparameter k\"onnen daher mit Hilfe geeigneter mathematischer Methoden aus Messwerten von $u\of{t}$ und $y\of{t}$ gesch\"atzt werden. Typische Methoden der Zustands- beziehungsweise Parametersch\"atzung sind die \ac{lqm}, die Verwendung von Kalman Filtern, die Formulierung von geeigneten Beobachterstrukturen oder der Einsatz von Parit\"atsgleichungen.   

Die zur Berechnung der Residuen notwendigen Redundanzbeziehungen basieren auf Ein- und Ausgangsgr\"o\ss{}en, welche voneinander nicht unabh\"angig sind. Die Abh\"angigkeiten k\"onnen durch zus\"atzliche Hardware oder analytische Beziehungen erzeugt werden. L\"asst sich die Redundanz als Gleichung formulieren und werden die Redundanzbeziehungen mit den Modellgleichungen zu einem Gleichungssystem kombiniert, so ist dieses Gleichungssystem \"uberbestimmt. Der entstehende Freiheitsgrad der L\"osung wird zur Entwicklung der Residuen genutzt. \newline
Redundanz durch Hardware entsteht beispielsweise durch mehrere Sensoren, welche die gleiche Gr\"o\ss{}e erfassen. Bei sicherheitstechnisch besonders anspruchsvollen Anwendungen wie der Luft- und Raumfahrt ist dieses Vorgehen trotz der damit verbunden gesteigerten Kosten und dem erh\"ohten Raumbedarf \"ublich. Analytische Redundanz wird erreicht, wenn sich bestimmte Sensorwerte algebraisch aus anderen Sensorwerten berechnen lassen, oder wenn es einen zeitlichen Zusammenhang zwischen der \"Anderung von Messwerten gibt, welcher sich analytisch beschreiben l\"asst. Ein Beispiel f\"ur eine solche Gr\"o\ss{}e ist der F\"ullstand in einem Tank. Werden der Zufluss und der Abfluss durch Sensoren erfasst, so kann der F\"ullstand direkt berechnet werden. Wird trotzdem ein F\"ullstandssensor verbaut, so f\"uhrt dies zu nutzbarer Redundanz, da die zeitlichen \"Anderungen der drei Messwerte zueinander vertr\"aglich sein m\"ussen. Ist dies nicht der Fall, so kann auf einen Sensordefekt, ein Leck oder auf einen anderen Fehler geschlossen werden.

Die auf Basis der analytischen Redundanzen ermittelten Residuen sollen zur Fehlerdiagnose eingesetzt werden k\"onnen. Es ist daher zweckm\"a\ss{}ig, wenn die Residuen bei Vorliegen einer Abweichung vom Sollverhalten des Prozesses signifikante Werte annehmen. Liegt keine St\"orung vor, so sollten die Residuen Werte nahe Null annehmen. Weiterhin ist es g\"unstig, wenn die Residuen robust gegen zuf\"allige Fehler wie Sensorrauschen und systematische Fehler wie Modellungenauigkeiten sind.

Als ersten Verfahrenstyp zur Residuenberechnung stellen \citeauthor{Venkatasubramanian_2003}  die Diagnose mit Beobachtern\footnote{im englischen spricht man von \glqq diagnostic observer\grqq { }oder \glqq unknown input observer\grqq { }(UIO)} vor {(\cite[S. 11 ff.]{Venkatasubramanian_2003})}. Methoden dieser Art entwickeln eine bestimmte Menge an Beobachtern, welche Residuen generieren. Jeder dieser Beobachter wird so definiert, dass er bez\"uglich einer definierten Menge an Fehlern sensitiv und bez\"uglich den restlichen Fehlern und unbekannten Gr\"o\ss{}en unempfindlich ist. Die Menge der Beobachter ist derart zu strukturieren, dass jeder Fehler ein eindeutiges Muster an Residuen zur Folge hat. Wird dies erreicht, so kann das Vorliegen eines Fehlers durch stark von null abweichende Werte der Residuen erkannt und mit Hilfe der bekannten Residuenmuster identifiziert werden. Eine wichtige Besonderheit dieses Verfahren ist es, dass die Sch\"atzung der Zustandsvariablen $x\of{t}$ nicht notwendig ist, statt dessen muss nur der Systemausgang durch Messung oder Sch\"atzung ermittelt werden.

Die Formulierung von Parit\"atsgleichungen ist ein alternatives Vorgehen zur Generierung von Residuen {(vgl. \cite[S. 13 f.]{Venkatasubramanian_2003})}. Bei diesem Vorgehen werden die Modellgleichungen geeignet umgestellt, sodass Residuenvektoren entstehen, die orthogonal zueinander sind. Die Residuenvektoren sind dann linear unabh\"angig und das Auftreten jedes betrachteten Fehlers wird durch genau einen Residuenvektor beschrieben. Voraussetzung zur Einsetzbarkeit dieser Methode ist, dass die Anzahl der Ausgangsgr\"o\ss{}en gr\"o\ss{}er als die der Zustandsgr\"o\ss{}en ist. Dieser Zusammenhang wird in Definition \ref{def:fAna_sysRedundanz} verdeutlicht. 
\begin{defn}[Systemredundanz]\label{def:fAna_sysRedundanz}
Sei ein System nach \eqnref{gl:fAna_sysIdent} beschrieben und gelten die Eigenschaften
\begin{align}
\dim\of{y\of{t}}&= n, &\dim\of{x\of{t}}&= m, & n&>m, \label{gl:fAna_sysRedundanzBdg}
\end{align}
dann ist das System redundant mit dem Freiheitsgrad
\begin{align}
f&= n-m, \label{gl:fAna_sysDimRedundanz}
\end{align}
da das System mehr erfassbare Ausgangsgr\"o\ss{}en als Zust\"ande umfasst.
\end{defn}
Mit Hilfe des Freiheitsgrades $f$ kann dann eine Projektionsmatrix $\matr{V}$ derart entworfen werden, dass f\"ur Abweichungen von jedem redundant vorhandenen Ausgangswert ein Vektor berechenbar ist, der zu den anderen Vektoren dieser Art orthogonal ist. \newline
Parit\"atsgleichungen und die Verwendung von Beobachtern zur Residuenerzeugung \"ahneln sich sehr stark. Beide Verfahren sind ohne eine Sch\"atzung von $x\of{t}$ anwendbar. Man kann sogar zeigen, dass beide Verfahren unter Verwendung der gleichen Designziele zu \"aquivalenten Residuen f\"ur ein fehlerbehaftetes System f\"uhren. Die Methoden der Auswertung von Residuen zur Diagnose von Fehlern sind f\"ur diese beiden Verfahren ebenfalls gleich. \"Ublich ist die Definition von Schwellwerten f\"ur die Residuen, bei deren \"Uberschreiten ein Fehler als vorliegend erkannt wird. 

Es gibt weitere Methoden, welche Residuen auf Basis quantitativer Modelle berechnen um so Fehler zu diagnostizieren und zu isolieren. Dazu z\"ahlen Methoden, welche die Zustandsvariablen oder Prozessparameter sch\"atzen, um auf Basis derer Residuen zu generieren. Dies sind beispielsweise Kalman Filter und \ac{lqm} {(vgl. \cite[S. 14 f.]{Venkatasubramanian_2003})}. Au\ss{}erdem gibt es fortgeschrittene Methoden zur Residuenberechnung wie der Entwurf von gerichteten oder strukturierten Residuen\footnote{engl. directional residuals and structured residuals} {(vgl. \cite[S. 15 f.]{Venkatasubramanian_2003})}. 

\subsection{Bewertung modellbasierter quantitativer Methoden der \ac{fdi} f\"ur modulare Anlagen}
Modellbasierte quantitative Methoden bieten den gro\ss{}en Vorteil, dass der Anwender bei der Wahl eines Verfahrens zur Residuengenerierung viele Freiheiten hat. Auch die Verfahren selbst bieten M\"oglichkeiten, um sie hinsichtlich der Erkennung bestimmter Fehler gezielt zu entwerfen. Werden entkoppelte Beobachterstrukturen geeignet entworfen, so kann jeder betrachte Fehler durch einen gesonderten Beobachter gezielt diagnostiziert werden. Dem gegen\"uber steht der gro\ss{}e Nachteil der Notwendigkeit von m\"oglichst genauen Prozessmodellen. Der Entwurf dieser Modelle ist aufwendig und h\"aufig mit Ungenauigkeiten verbunden. Dies gilt f\"ur analytische Modelle und Black Box Modelle gleicherma\ss{}en. Weiterhin sind Analysemethoden, welche auf quantitativen Modellen basieren, in aller Regel auf die Erkennung von Fehlern, welche additiv auftreten, beschr\"ankt. Die Erkennung von multiplikativ auftretenden Fehlern wie einem Drift von Prozessparametern ist nur in Sonderf\"allen m\"oglich. Dar\"uber hinaus m\"ussen die Residuen zur Erkennung von Fehlern vorab definiert werden. Das Auftreten von vorab  unbekannten Fehlern ist dadurch nur stark eingeschr\"ankt m\"oglich. Auch die Ursachenanalyse ist zumeist nicht m\"oglich -- nur das Vorliegen eines Fehlers wird diagnostiziert und der konkrete Fehler ermittelt.  {\cite[S. 17 f.]{Venkatasubramanian_2003}}

In Hinblick auf modular konstruierte Anlagen l\"asst sich feststellen, dass Methoden dieser Kategorie nicht geeignet sind, um die zur Genehmigung einer aus Modulen bestehenden Anlage notwendige Sicherheitsuntersuchung zu beschleunigen oder anderweitig zu vereinfachen. \newline
F\"ur die Analyse eines einzelnen Moduls k\"onnten jedoch solche Verfahren zum Einsatz kommen. Module sollen entsprechend ihrer Definition einzeln komplett testbar sein. Daher ist die Erstellung von Ein"=/Ausgangsdaten und darauf aufbauend die Entwicklung eines Black Box Modells prinzipiell m\"oglich. Die Erstellung eines analytischen Modells durch den Modullieferanten ist ebenfalls m\"oglich und sollte ohnehin ein Ziel dessen sein, denn auf Basis eines analytischen Modells k\"onnen die im Abschnitt \ref{sec:sdt_modularisierung} geforderten Simulationsmodelle geeignet entworfen werden. Die notwendige Grundlage f\"ur modellbasierte quantitative Verfahren w\"are damit zumindest auf Modulebene gegeben, jedoch verbleiben zwei bedeutende Probleme: \begin{itemize}
\item der aufwendige Entwurf eines analytischen Modells der Gesamtanlage ist notwendig und
\item nicht erkannte Fehler k\"onnen nicht verl\"asslich identifiziert werden.
\end{itemize}
Zum einen ist zu erwarten, dass die Modelle der einzelnen Module noch keine ausreichenden Informationen \"uber die m\"oglichen Wechselwirkungen, welche im Rahmen der Gesamtanlage auftreten k\"onnen, enthalten. Ein Modell der Gesamtanlage m\"usste daher vor dem Einsatz von Verfahren der betrachteten Kategorie noch erstellt werden. Dies w\"are nur durch analytische Ans\"atze m\"oglich. Die zur Generierung von Black Box Modellen notwendigen Ein"=/Ausgangsdaten m\"ussten von der Gesamtanlage stammen, diese ist zum Zeitpunkt der durchzuf\"uhrenden Sicherheitsbetrachtung aber noch gar nicht betriebsf\"ahig. Die Durchf\"uhrung praktischer Tests und die Erstellung von Messdaten ist daher keine zur Verf\"ugung stehende Option und die Erstellung von Black Box Modellen nicht m\"oglich. Die Erstellung von analytischen Modellen ist mit den bereits genannten Problemen des hohen Aufwands und der entstehenden Ungenauigkeiten verbunden. Die notwendige Entwicklungsleistung eines analytischen Modells reduziert damit m\"ogliche Zeiteinsparungen bei der Sicherheitsbetrachtung ma\ss{}geblich. \newline
Zum anderen werden vorab unbekannte Fehler durch modellbasierte quantitative Verfahren nicht verl\"asslich identifiziert. Das Auffinden von bisher nicht betrachteten Fehlern wird also bereits auf der Betrachtungsebene einzelner Module durch Methoden dieser Kategorie nicht erm\"oglicht. Durch das Verbinden von Modulen zu einer Gesamtanlage ist mit neuen Fehlerquellen und f\"ur die Anlage spezifischen m\"oglichen Auswirkungen zu rechnen. Die Erkennung dieser neuen Fehler ist nicht m\"oglich. Die durch Kopplung der Module potentiellen neuen Fehler sind aber genau die Fehler, welche durch die Sicherheitsuntersuchung der Gesamtanlage aufgedeckt werden m\"ussen. Eine Vereinfachung dieser Aufgabe durch die Verwendung von modellbasierten quantitativen Verfahren zur Fehlerdiagnose ist damit nicht zu erwarten. \newline

Im Rahmen der vorliegenden Arbeit soll davon ausgegangen werden, dass die vorhandene Datenbasis aus f\"ur die einzelnen Module durchgef\"uhrten \acp{hazop}, Beschreibungen der Module und einer Beschreibung der Gesamtanlage besteht. Auf dieser Basis l\"asst sich nicht ohne gro\ss{}en Aufwand ein analytisches Modell der Gesamtanlage formulieren. Selbst wenn quantitative modellbasierte Verfahren der \ac{fdi} potentielle Einsparungen bei der Sicherheitsbetrachtung der Gesamtanlage bieten w\"urden, was wie dargelegt wird, nicht der Fall ist, so w\"are die notwendige Datenbasis in keiner weise vorhanden. Im Rahmen dieser Arbeit ergibt sich daher  zwangsl\"aufig, dass der Einsatz von modellbasierten quantitativen Verfahren nicht geeignet ist, um die Fehlerfortpflanzung innerhalb einer aus Modulen bestehenden Anlage zu untersuchen. 
\section{Modellbasierte Qualitative  Fehlerfortpflanzungsmethoden}\label{sec:fAna_modQual}


\section{auf historischen Messdaten basierende Fehlerfortpflanzungsmethoden}\label{sec:fAna_dat}
Diese Verfahren beruhen auf der Verwendung von historischen Messdaten konkreter Anlagen. Je nach Verfahren werden Messdaten zum Normalbetrieb oder beziehungsweise und Daten zum St\"orbetrieb ben\"otigt. Manche Methoden ben\"otigen weiterhin ein \ac{pid}. Ziel der Verfahren ist es zum einen St\"orungen der Anlage fr\"uhzeitig zu erkennen und zum anderen deren Ursache oder Ursachen zu ermitteln. Dazu werden Ursache--Effekt Beziehungen zwischen untersuchten Parametern ermittelt.  

Die Auswertung historischer Messdaten basiert zumeist auf statistischen Methoden. Die kausalen Zusammenh\"ange, welche mit Hilfe dieser Methoden ermittelt werden sollen, k\"onnen dann geeignet als Graphen dargestellt werden. Wie man aus statistischen Gr\"o\ss{}en kausale Zusammenh\"ange ermitteln kann wird in den fr\"uhen Werken \textcite{Holland_1986} und  \textcite{Pearl_1995} aufgezeigt. Ein umfassendes Lehrbuch zu dieser Thematik wurde von \citeauthor{Pearl_2009} ver\"offentlicht, welches mittlerweile in der zweiten Auflage verf\"ugbar ist \cite{Pearl_2009}.

Datenbasierte Methoden sind hervorragend f\"ur die Erstellung von quantitativen Modellen geeignet. Eine Erstellung von qualitativen Modellen ist ebenfalls m\"oglich. \textcolor{red}{Beispiele}
\paragraph*{Nennung von Verfahren}
\cite{Zhang_2017}, \cite{Thornhill_2006}



\section{Vorstellung ausgew\"ahlter Algorithmen}
\section{Bewertung der Verwendbarkeit f\"ur modulare Anlagen}
Wichtige Gesichtspunkte:
\begin{itemize}
\item Welche Daten sind notwendig
\item Automatisierbarkeit der Berechnung
\item Dokumentationsf\"ahigkeit der Ergebnisse
\item Sind die Ergebnisse f\"ur eine \ac{hazop} nutzbar
\end{itemize}