\chapter{Einleitung}\label{ch:einleitung}
\section{Herausforderungen der deutschen Chemie- und Pharmaindustrie}\label{sec:einltg_chemPharmaIndustrie}
Das reale Bruttoinlandsprodukt Deutschlands wuchs von 1995 bis 2013 im Mittel weniger als $2\%$ und wurde damit von den teils zweistelligen Wachstumsraten der Schwellenl\"ander und insbesondere China deutlich \"ubertroffen. Um international erfolgreich zu bleiben, sind die schnell wachsende Pharmazeutische Industrie und die exportlastige Chemische Industrie f\"ur die deutsche Wirtschaft von besonderer Bedeutung. \hfill \newline

Die Chemische Industrie und die Pharmazeutische Industrie sind Schl\"usselbranchen der deutschen Wirtschaft. Sie exportierten im Jahr 2013 Waren im Wert von \"uber 150 Milliarden \euro. Dies entspricht $15 \%$ der deutschen Gesamtexporte des verarbeitenden Gewerbes. Der Export von pharmazeutischen Erzeugnissen wuchs in den Jahren 1995 bis 2013 j\"ahrlich im Durchschnitt $11,3 \%$ und damit schneller, als der jeder anderen Branche. Im gleichen Zeitraum entwickelte sich die Chemische Industrie nur unterdurchschnittlich -- im globalen Wettbewerb verlor sie sogar Marktanteile. Als Ursache hierf\"ur wird die besonders hohe Abh\"angigkeit der Branche von den in diesem Zeitraum in Deutschland stark gestiegenen Energiepreisen angesehen. Es m\"ussen geeignete Ma\ss{}nahmen entwickelt und angewandt werden, um die Standortnachteile auszugleichen. Nur so kann die Pharmazeutische Industrie ihre Wachstumsdynamik beibehalten und die Chemische Industrie ihre Entwicklungschancen realisieren. \hfill \newline
Eine wichtige Grundlage f\"ur den Erfolg der Chemischen und Pharmazeutischen Industrie ist die best\"andige Weiterentwicklung und Erschaffung innovativer Produkte. Die allein im Jahr 2013 \"uber 7500 in Deutschland neu angemeldeten Patente belegen die bereits aufgebrachte Innovationskraft. Die gr\"o\ss{}ten deutschen Industriezweige Maschinen- und Fahrzeugbau meldeten im gleichen Zeitraum in Summe nur circa Dreihundert Patente mehr an. Die Entwicklungsleistung im Chemie- und Pharmabereich ist in Relation zu den \"ubrigen Gewerbezweigen offensichtlich bereits \"uberdurschnittlich hoch. Es erscheint daher sinnvoll andere Faktoren zu untersuchen, welche die Entwicklung der betrachteten Industriezweige ma\ss{}geblich beeinflussen. \cite{PerspektiveD_2016} \hfill \newline

Im aktuellen Bericht des Verbands der Chemischen Industrie  untersucht Jan Limbers die Lage und Entwicklungsm\"oglichkeiten der chemisch-pharmazeutischen Industrie und prognostiziert die Entwicklung bis zum Jahr 2030. Durch die schnellere Verbreitung von Technologie und Wissen und den damit verbundenen gesteigerten globalen Wettbewerb wird ein weiter ansteigender Innovationsdruck erwartet. Wird die umfangreiche Forschungsarbeit auf die Bereiche Spezialchemikalien und Pharmazeutika fokussiert, so k\"onnen die Standortnachteile, welche durch hohe Energiekosten entstehen, ausgeglichen werden und ein \"uberdurchschnittliches Wachstum ist m\"oglich. Dies erfordert jedoch insbesondere ein insgesamt h\"oheres Innovationstempo. Der Entwicklungsfaktor Innovationstempo soll daher im Folgenden weiter betrachtet werden. \cite{PerspektiveC_2016} \hfill \newline

Das Innovationstempo ist mit der ben\"otigten \glqq Time to market\grqq { }eines Produktes gleichzusetzen. Darunter versteht man in diesem Zusammenhang den Zeitraum von der ersten Idee f\"ur ein neues Produkt bis zum Zeitpunkt der Inbetriebnahme der Produktionsanlagen im marktangepassten Ma\ss{}stab. Die dazu erforderlichen Schritte umfassen die notwendige Forschungsarbeit zur Produkt- und Prozessentwicklung, die Planung und den Bau der Produktionsanlagen. Der Zeitraum nach erfolgter Produktentwicklung bis zum Produktionsbeginn umfasst in etwa 5 -- 10 Jahre, wobei davon circa die H\"alte der Zeit auf Anlagenplanung und Konstruktion entfallen. \cite{Schembecker_2009}

Dieser Zeitraum muss reduziert werden, um die von J. Limbers prognostizierte Entwicklung der chemisch-pharmazeutischen Industrie zu erm\"oglichen. \hfill \newline
Die prinzipielle Notwendigkeit einer schnelleren und vor allem auch flexibleren Produktentwicklung beziehungsweise Produktion ist seit langem bekannt. Die Arbeit von I. E. Grossmann zu den Herausforderungen f\"ur die Forschung im Bereich der Verfahrens- und Anlagentechnik aus dem Jahr 2000 weist beispielsweise auf diese Herausforderungen hin. \cite{Grossmann_2000} \hfill \newline

Es wurden bereits zahlreiche Untersuchungen unternommen, wie das Ziel einer flexibleren und  beschleunigten Produktentwicklung erreicht werden kann. Auf einige dieser Ans\"atze wird im folgenden Abschnitt eingegangen. 

\section{Beschleunigung des Innovationstempo}\label{sec:einltg_beschlngInnovationstempo}
Im Abschnitt \ref{sec:einltg_chemPharmaIndustrie} wurde die Relevanz der chemisch-pharmazeutischen Industrie f\"ur die deutsche Wirtschaft dargelegt und die Notwendigkeit eines erh\"ohten Innovationstempo begr\"undet. In diesem Abschnitt werden Methoden vorgestellt, welche dieses Ziel realisieren sollen. \hfill \newline

Ein Ansatz zur Verbesserung des Produktentwicklungsprozesses wurde \"uber ein Jahrzehnt hinweg an der Universit\"at Clausthal untersucht. Anhand eines neu entworfenen Apparates zur Herstellung von Chlorsilanen aus Ferrosilicium und Chlorwasserstoff wurde von P. Dietz and U.  Neumann in \cite{Dietz_2000} gezeigt, wie durch eine fr\"uhzeitige Parallelisierung von Prozessplanung und dem Entwurf der notwendigen Maschinen die Entwicklungszeit verk\"urzt werden kann. Die Parallelisierung wird erreicht, indem die zu realisierenden Prozessschritte in Teilsysteme geringer Komplexit\"at so weit zerlegt werden, dass sich deren Funktion durch naturwissenschaftliche Grundoperationen darstellen l\"asst. Durch die so erhaltene Darstellung wird ein Blick f\"ur die m\"ogliche Zusammenfassung von mehreren Teilsystemen in einer einzigen Maschine erm\"oglicht. Eine derart entworfene Maschine kann auf innovative Weise einen Prozess optimal erf\"ullen. Prozessschritte wie Zerkleinern, Reagieren und Mischen k\"onnen beispielsweise in einem Apparat vereint werden. Es wird bei diesem Entwicklungsprozess bewusst auf Standardl\"osungen verzichtet, was die Wiederverwendbarkeit der erhaltenen L\"osungen zumindest erschwert. Das Innovationstempo kann jedoch erfolgreich gesteigert werden und es wird eine hocheffiziente Umsetzung f\"ur einen  Produktionsprozess gefunden.  \hfill \newline

Neben der Parallelisierung von Prozessplanung und Anlagenentwicklung gibt es weitere Methoden zur Verk\"urzung der Entwicklungszeit. Dazu z\"ahlen unter anderem der verst\"arkte Einsatz von mathematischen Modellen beispielsweise in Simulationen, das Verwenden von Mini- und Mikroplants und der Gebrauch von standardisierten Modulen. \hfill \newline

\textcolor{red}{Hier ist noch ein Abschnitt zum Einsatz von Millireaktoren m\"oglich.} \hfill \newline

Zahlreiche Vertreter aus Wissenschaft und Wirtschaft haben sich 2009 zum Tutzing Symposium getroffen. Diskussionsschwerpunkt war die \glqq 50\% -- Idee Vom Produkt zur Produktionsanlage in der halben Zeit\grqq { }. Es sollte analysiert werden, welche Methoden besonders dazu geeignet sind, um die \glqq Time to Market\grqq { }auf die H\"alfte zu reduzieren. Als Ergebnis wurden unter anderem die Thesen von Tutzing \cite{Processnet_2009}, ein Positionspapier \cite{Processnet_2010} und ein \"Ubersichtsvortrag \cite{Schembecker_2009} ver\"offentlicht. Es wurden die notwendigen Forschungsschwerpunkte herausgearbeitet, um das Ziel eines signifikant erh\"ohten Innovationstempos zu erreichen. \hfill \newline

In diesen Arbeiten wurde als Kernthema die Modularisierung von Anlagen und deren Komponenten identifiziert. Durch die Verwendung von Standardl\"osungen sollen umfangreiche Detailarbeiten entfallen und Anlagenkomponenten durch erneuten Einsatz perfektioniert werden. Dies bedeutet eine bewusste Abkehr von dem in \cite{Dietz_2000} vorgestellten Vorgehen einer parallelisierten Prozess- und Anlagenentwicklung in Verbindung mit einer optimal ausgelegten Anlage. Statt dessen werden wiederverwendbare Module in diskreten Gr\"o\ss{}en erzeugt, welche skalierbar und vielseitig einsetzbar sein sollen. Die Skalierbarkeit erm\"oglicht dabei eine flexible Ver\"anderung von Produktionsvolumina und damit eine schnelle Anpassbarkeit an Marktver\"anderungen. Module k\"onnen dezentral vorgefertigt und am Standort der Gesamtanlage schnell und montiert werden. Dies beschleunigt die Konstruktion der Gesamtanlage. \hfill \newline

Die Verwendung von Modulen bedeutet eine signifikante \"Anderung im Entwicklungsprozess. Es wurden mehrere Themenschwerpunkte identifiziert, welche die notwendigen Anpassungen beschreiben sollen. Die hier aufgef\"uhrten Schwerpunkte sind dem bereits aufgef\"uhrten Positionspapier zur 50 \% -- Idee entnommen \cite{Processnet_2010}.  Die Reduktion der Entwicklungszeit auf die H\"alfte durch den Einsatz von Modulen ist nur m\"oglich, wenn die genannten Themen erfolgreich bearbeitet werden. \hfill \newline
Ein Themenschwerpunkt ist die Entwicklung von Modellen zur Beschreibung von Modulen. Module sollen abgeschlossene Funktionseinheiten bilden. F\"ur einzelne Prozessschritte sind Apparate zu entwickeln, welche diese realisieren k\"onnen. Sie sollen skalierbar und getrennt von anderen Anlagenteilen testbar sein. \hfill \newline
Die Funktion eines Moduls und die Dokumentation in verschiedenen Detaillierungsgraden sowie entwickelte Skalierungsvarianten und alle weiteren relevanten Informationen sollen f\"ur Anlagenbauer, Zulieferer, Prozessplaner und alle \"ubrigen am Produktentwicklungszyklus Beteiligten abrufbar sein. Dazu sind geeignete Informationsmodelle notwendig. Diese sollten in Verbindung mit bestehenden Softwarel\"osungen verwendet werden k\"onnen. \hfill \newline
Das Konzept der Modularisierung soll in allen Phasen eines Projektes zur Entwicklung neuer Produkte eingesetzt werden. Die Projektplanung muss dazu umstrukturiert werden. Die notwendigen Anpassungen der etablierten Projektabl\"aufe sind zu erarbeiten und zu testen. \hfill \newline
Um ein durchg\"angiges Modulkonzept zu etablieren ist die Definition von Standards und Schnittstellen zwischen Modulen unumg\"anglich. Dazu ist eine firmen\"ubergreifende Kooperation und die Zusammenarbeit mit der Wissenschaft notwendig. \hfill \newline
Weiterhin muss die Automatisierungstechnik an die modulare Bauweise angepasst werden. Insbesondere ist zu er\"ortern, wie autonom einzelne Module gesteuert werden sollen und wie die Kommunikation zwischen Modulen im Rahmen einer Gesamtanlage konzipiert werden kann. \hfill \newline

Diese Themen wurden weitreichend untersucht. Die prinzipielle Anwendbarkeit der modularen Anlagenbauweise konnte anhand mehrerer Fallstudien im Rahmen des Projektes $\text{F}^{3}$--Factory gezeigt werden. Die Ergebnisse wurden im Abschlussbericht dieses Projektes \cite{f3_2014} ver\"offentlicht. \hfill \newline
Weitere Untersuchungen wurden beispielsweise im Projekt CoPIRIDE \cite{copiride_2014} durchgef\"uhrt. \hfill \newline

Die Anwendbarkeit von Modulen konnte bereits erfolgreich gezeigt werden. Es sind aber noch einige Forschungsschwerpunkte offen. Insbesondere ein wichtiger Gesichtspunkt wurde bisher noch nicht umfassend betrachtet: die Auswirkung der Modularisierung auf notwendige Sicherheitsbetrachtungen. \hfill \newline
Bei der Planung und Inbetriebnahme einer neuen prozessleittechnischen Anlage ist die Gew\"ahrleistung des sicheren Betriebs von h\"ohster Wichtigkeit. Dazu sind geeignete Sicherheitsuntersuchungen durchzuf\"uhren. Die Auswirkung der Modularisierung von Anlagen auf Sicherheitsuntersuchungen findet in den direkten Ver\"offentlichungen zum Tutzing Symposium keine gesonderte Beachtung. Im Abschnitt \ref{sec:einltg_sicherheitstechnik} wird auf die prinzipielle Problemstellung von Sicherheitsuntersuchungen eingegangen und im daran anschlie\ss{}enden Abschnitt \ref{sec:einltg_aufgabe} das Thema dieser Arbeit herausgearbeitet. 

\section{Notwendigkeit von Sicherheitstechnik} \label{sec:einltg_sicherheitstechnik}
Das Bed\"urfnis nach Sicherheit ist ein menschliches Grundbed\"urfnis. Wird dieses Bed\"urfnis nicht in ausreichendem Ma\ss{}e erf\"ullt, so hat dies gravierende Auswirkungen. Die massenweise Flucht aus Kriegsgebieten ist beispielsweise ein solcher Extremfall. \hfill \newline
Unter normalen Umst\"anden k\"onnen Risiken durch geeignete Mittel reduziert werden. Dazu kann entweder die Eintrittswahrscheinlichkeit eines Schadensfalles oder dessen sch\"adliche Auswirkung gemindert werden. Um eine dieser Methoden anzuwenden ist jedoch entweder ein Kostenaufwand oder die Einschr\"ankung von m\"oglichem Nutzungsumfang notwendig. Typische Beispiele des Alltags sind die Verwendung von Versicherungen um die Auswirkungen eines Schadens zu reduzieren und die Einschr\"ankung der erlaubten Fahrtgeschwindigkeit in St\"adten zur Reduktion von Schadenseintrittswahrscheinlichkeiten. Es gibt offensichtlich einen Interessenkonflikt zwischen Risikominimierung und der Aufwendung von Kapital oder der Einschr\"ankung von Nutzungsumfang. \hfill \newline
Es ist eine gesellschaftliche Aufgabe, einen Kompromiss zwischen dem Wunsch nach hoher Sicherheit und den notwendigen Ma\ss{}nahmen zu finden. Diese Aufgabe soll in erster Linie von der Politik gel\"ost werden. \hfill \newline

Ein geeignetes Mittel dieses Ziel zu erreichen ist die Verwendung einer Risikoanalyse. Diese soll Aufschluss \"uber m\"ogliche schadhafte Ereignisse, deren Auswirkungen und m\"ogliche Pr\"aventionsmethoden liefern. Die Verwendung von Risikoanalysen ist kein Konstrukt der Neuzeit, sondern existiert bereits seit mehreren hundert Jahren. Das Buch von Peter L. Bernstein \glqq Against the Gods: The Remarkable Story of Risk\grqq { } \cite{Bernstein_1998} zeigt die geschichtliche Entwicklung von Risikobetrachtungen auf. Die zweite Auflage des Werkes von Bilal M. Ayyub ist eine aktuelle umfassende Referenz zum Thema Risikoanalyse \cite{Ayyub_2014}. \hfill \newline

Im Bereich der chemischen Industrie gab es lange Zeit keine verbindlichen Richtlinien, wie die Sicherheit von Anlagen zu bewerten ist und welches Sicherheitslevel als von der Gesellschaft akzeptiert angesehen werden kann. In Folge einer Reihe schwerer Chemieunf\"alle wurden die aktuellen Gesetze zum Betrieb sicherheitsrelevanter Anlagen entworfen und weiterentwickelt.\hfill \newline
Ma\ss{}geblich f\"ur die Forderung und Entwicklung von einheitlichen Regeln waren insbesondere die Unf\"alle in Seveso -- Italien im Jahr 1976, Bhopal -- Indien im Jahr 1984, Enschede -- Niederlande und Baia Mare -- Rum\"anien im Jahr 2000 sowie Kolont\'ar -- Ungarn im Jahr 2010. Diese Unf\"alle waren allesamt mit gravierenden Humansch\"aden verbunden. Sie f\"uhrten vom ersten europaweiten Regelwerk -- der Seveso I Richtlinie 1982 -- bis hin zur aktuellen Seveso III Richtlinie 2012/18/EU. Diese europ\"aische Richtlinie wurde in den Mitgliedsstaaten der \ac{eu} in Form nationaler Gesetzte, Verordnungen und Richtlinien umgesetzt. In Deutschland dient dazu das \ac{bimschg} in der Fassung vom 05.04.2017. \hfill \newline

Entsprechend der aktuellen deutschen Gesetze sind geeignete Methoden anzuwenden, um den sicheren Betrieb von sicherheitstechnisch relevanten Anlagen sicherzustellen. Dazu geh\"oren also chemische Anlagen, welche potentiell gef\"ahrliche Stoffe verarbeiten und in Modulbauweise entsprechend den im Abschnitt \ref{sec:einltg_beschlngInnovationstempo} dargelegten Konzepten konzipiert werden. Modulare Anlagen m\"ussen also in geeigneter Weise auf die Erf\"ullung von Sicherheitsanforderungen untersucht werden. Dieser gesonderten Problematik wurde bisher wenig Beachtung geschenkt. Die Arbeit von Fleischer et. al. \cite{Fleischer_2015} setzt sich als eine der wenigen Ver\"offentlichungen mit dieser Problematik auseinander. Die Arbeit von Fleischer et. al. konzentriert sich auf die Sicherheitsbetrachtung von Modulen in Containerbauweise, wie sie im Projekt $\text{F}^{3}$ erfolgreich eingesetzt wurden. Sie weist auf die prinzipiellen Probleme einer Sicherheitsbetrachtung von modularen Anlagen hin. Zum einen ist mit den aktuellen Methoden eine Wiederverwendung von bereits durchgef\"uhrten Sicherheitsanalysen beispielsweise derer von einzelnen Modulen nicht m\"oglich. Weiterhin wird die Flexibilit\"at beim Einsatz von Modulen, welche einen der gr\"o\ss{}ten Vorteile dieses Konzeptes bildet, stark eingeschr\"ankt. Die Ursache davon liegt darin, dass bei \"Anderungen an einer genehmigungspflichtigen Anlage eine erneute Sicherheits\"uberpr\"ufung der gesamten Anlage durchzuf\"uhren ist. Dies ist sehr Kosten- und Zeitintensiv und daher ein Problem, welches gel\"ost werden sollte. \hfill \newline
Der Ansatz von Fleischer et. al. sieht eine Zweiteilung der Sicherheitsanalyse vor. Die Wechselwirkung zwischen Modulen soll mit Hilfe einer \ac{hazop} analysiert werden. Dies ist aber erst m\"oglich, wenn die Anlagenplanung weit fortgeschritten ist, und eine konkrete Auswahl der einzubindenden Module stattgefunden hat. Die Untersuchung einzelner Module soll unter Verwendung von Checklisten und Heuristiken durchgef\"uhrt werden. Die Untersuchung einzelner Module liefert dann Aufschluss \"uber deren Verwendbarkeit f\"ur einen bestimmten Prozess. Um die Einsetzbarkeit prinzipiell bewerten zu k\"onnen, wird die Definition von Stoffklassen, Reaktionsklassen und zul\"assigen Betriebsfenstern vorgeschlagen. Einem Modul wird dann anhand seiner Eigenschaften jeweils eine diskrete Stufe dieser Kategorien zugeordnet und durch diese Zuordnung kann die Einsetzbarkeit eines Moduls f\"ur einen Prozess schnell und fr\"uhzeitig bewertet werden. Auf die konkrete Verwendbarkeit dieser intramodularen Sicherheitsanalyse f\"ur eine nachfolgende \ac{hazop} wird nicht im Detail eingegangen. Das Problem der Wiederverwendbarkeit von durchgef\"uhrten Sicherheitsbetrachtungen von einzelnen Modulen f\"ur eine anschlie\ss{}ende Analyse der Gesamtanlage ist Motivation f\"ur die vorliegende Arbeit. Das Thema der Arbeit wird im folgenden Abschnitt \ref{sec:einltg_aufgabe} detailliert formuliert.

\section{Problemstellung dieser Arbeit}\label{sec:einltg_aufgabe}

Es ist zu beachten, dass eine zeitaufwendige Sicherheitsbetrachtung nach erfolgter Auswahl von Modulen die Zeit bis zur Erteilung der Betriebserlaubnis und damit der Entwicklungszeit ma\ss{}geblich verl\"angern kann. Es ist daher w\"unschenswert, die Sicherheitsanalyse der geplanten Gesamtanlage so z\"ugig wie m\"oglich durchzuf\"uhren. \hfill \newline
Ein einzelnes Modul sollte bereits einer Sicherheitsuntersuchung unterzogen werden. Dazu ist die in Abschnitt \ref{sec:einltg_sicherheitstechnik} vorgestellte Methode von Fleischer et. al. , welche auf dem Einsatz von Checklisten und Heuristiken basiert, geeignet.  Es stellt sich daher die Frage, inwiefern die Erkenntnisse aus der Sicherheitsbetrachtung eines einzelnen Moduls f\"ur die Sicherheitsbetrachtung der Gesamtanlage verwendet werden k\"onnen. \hfill \newline
Ein geeignetes Mittel zur Sicherheitsuntersuchung von Gesamtanlagen ist die Durchf\"uhrung einer \ac{hazop}. Im Rahmen dieser Analyse wird gepr\"uft, wodurch Anlagenparameter vom Normbetrieb abweichen k\"onnen und mit welchem Risiko eine solche Abweichung verbunden ist. Dazu werden Fehler identifiziert, welche ungewollte Schwankungen von Prozessparametern zur Folgen haben k\"onnen. Zur Bewertung des Risikos muss die Auswirkung des Fehlers auf die Gesamtanlage betrachtet werden. Dazu ist eine Analyse der Fehlerfortpflanzung notwendig. \hfill \newline

Das Ziel der vorliegenden Arbeit ist es zu untersuchen, welche Algorithmen geeignet sind, um eine automatisierte Untersuchung der Fehlerfortpflanzung in modularen Anlagen durchzuf\"uhren. Als Basis f\"ur die Algorithmen sollen die Beschreibung der Module, die \ac{hazop}--Studien der Module und die Beschreibung der modularen Gesamtanlage dienen. \hfill \newline
Anhand eines geeigneten Beispiels soll \"uberpr\"uft werden, welche Auswirkungen von Fehlern, die in einem Modul auftreten, mit der vorgegebenen Instrumentierung in den anderen Modulen der Anlage erkannt und beherrscht werden k\"onnen. \hfill \newline
Ein solche automatisierte Bewertung der Fehlerfortpflanzung auf Basis der Sicherheitsuntersuchung von Anlagenmodulen kann die \ac{hazop} einer Gesamtanlage ma\ss{}geblich beschleunigen und damit das Innovationstempo erh\"ohen.