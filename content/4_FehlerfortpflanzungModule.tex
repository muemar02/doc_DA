\chapter{Fehlerfortpflanzung in modularen Anlagen durch Einsatz von Graphen}\label{ch:ffpflMod}
In Kapitel \ref{ch:fehlerfortpfl} werden verschiedene Methoden der \acf{fdi} hinsichtlich einer Verwendbarkeit f\"ur die Untersuchung von modularen Anlagen bewertet. Die vorgestellten Methoden werden dazu in drei Kategorien eingeteilt, welche sich ma\ss{}geblich hinsichtlich der Informationen unterscheiden, die zur Anwendung der Methoden notwendig sind. Im Kontext der vorliegenden Arbeit wird davon ausgegangen, dass die Beschreibung einzelner Module, die \ac{hazop}"=Studien der Module und eine Beschreibung der modularen Gesamtanlage vorliegt. Das vorhandene Wissen besteht also aus Expertenwissen und Informationen \"uber die Kopplung der Module. Es umfasst keinerlei Messdaten, weswegen die im Abschnitt \ref{sec:fAna_dat} vorgestellten datenbasierten Verfahren der \ac{fdi} als ungeeignet bewertet werden. \newline
Die im Abschnitt \ref{sec:fAna_modQuant} vorgestellten quantitativen modellbasierten Verfahren ben\"otigen ein geeignetes quantitatives Modell der Gesamtanlage. Die Erstellung dieser Modelle ist aufwendig und m\"ogliche Zeiteinsparungen daher gering, die Wiederverwendbarkeit limitiert und die Nutzung von vorhandenem Wissen \"uber die einzelnen Module kompliziert. Daher werden auch diese Methoden als nicht geeignet eingestuft. \newline
Im Abschnitt \ref{sec:fAna_modQual} werden die modellbasierten qualitativen Verfahren vorgestellt. Diese nutzen a priori vorhandenes Wissen \"uber den betrachteten Prozess und entwickeln darauf aufbauend ein Modell des Systemverhaltens. Diese Art von Wissen wird im Rahmen der vorliegenden Arbeit als vorhanden angenommen. Die erfolgreiche Nutzung von Expertenwissen zur Analyse von Fehlerfortpflanzungen wird im Abschnitt \ref{sec:fAna_automatHazop} anhand zahlreicher Verfahren zur Automatisierung von \acp{hazop} dargelegt. Die vorgestellten Methoden der qualitativen modellbasierten \ac{fdi} werden als prinzipiell geeignet bewertet, um die Fehlerfortpflanzung von Prozessabweichungen innerhalb modularer Anlagen zu untersuchen, da diese Methoden im industriellen Ma\ss{}stab durch Einsatz von wiederverwendbarem Detailwissen erfolgreich f\"ur Fehlerfortpflanzungsbetrachtungen eingesetzt wurden. \newline
Derzeit existieren noch keine bekannten Verfahren, welche direkt f\"ur aus Modulen bestehende Anlagen eingesetzt werden k\"onnen. Problematisch ist die Beschreibung der Auswirkungen einer Kopplung der Module und die konkrete Verwendung des \"uber die einzelnen Module vorhandenen Wissens. In diesem Kapitel \ref{ch:ffpflMod} wird die Anwendung von gerichteten Graphen skizziert, um dieses Problem zu l\"osen. Die Anwendung anderer modellbasierter qualitativer Verfahren ist m\"oglich und sollte gezielt untersucht werden. Gerichtete Graphen sind die allgemeine Form von \ac{sdg}. Deren Verwendung ist besonders weit verbreitet und es existieren zahlreiche Softwares zur Computer"=gest\"utzten Implementierung. Ein kostenloses Werkzeug ist auf der Website \url{http://www.graphviz.org/} zu beziehen. In dieser Arbeit wird jedoch eine Implementierung mit Hilfe von Matlab durchgef\"uhrt, da dieses Werkzeug besonders weit in der Industrie verbreitet ist. \newline
Die Verwendung von Graphen bietet die M\"oglichkeit einer teilautomatisierten Erstellung, allerdings wird das Ergebnis weder plausibilisiert, noch bildet es zeitliche Zusammenh\"ange ab {(vgl. \cite{Yang_2010})}. Der Einsatz von Gewichten macht eine Bewertung der Wahrscheinlichkeiten von Fehlerfortpflanzungen m\"oglich. Diese M\"oglichkeit bleibt in der vorliegenden Arbeit aber ungenutzt, da kein Wissen \"uber konkrete Wahrscheinlichkeiten vorhanden ist. Statt dessen wird im folgenden Abschnit \ref{sec:graphHazop} gezeigt, wie auf Basis von \ac{hazop}"=Studien einzelner Modelle Graphen erzeugt werden k\"onnen, welche die bereits analysierten Wechselwirkungen abbilden. Anschlie\ss{}end wird im Abschnitt \ref{sec:kopplung} dargelegt, wie die bekannte Kopplung der Module zu einer Gesamtanlage zum Verbinden dieser Graphen genutzt werden kann. Der so entstehende Gesamtgraph wird dann auf die Beschreibung von modul\"ubergreifenden Wechselwirkungen hin untersucht. 

\section{Algorithmus zur Erstellung von gerichteten Graphen auf Basis von HAZOP"=Studien}\label{sec:graphHazop}
Die Auswertung von textbasierten Informationen durch Software ist besonders gut geeignet, wenn diese Informationen strukturiert sind. Dazu ist die Verwendung einer Ontologie sinnvoll. In der vorliegenden Arbeit wird die Analyse einer modularisierten Ethylacetatanlage nach \cite{Pfeffer_2016, Pfeffer_2017} genutzt, welche nicht derart strukturiert ist. Im Folgenden wird ein Modul bestehend aus Zulauf, Tank mit R\"uhrer und Pumpe im Ablauf betrachtet, welches in \figref{fig:PIDMod1} abgebildet ist. Die dazu vorliegende \ac{hazop} ist in \tabref{tab:hazopBsp_M1} zu finden. Betrachtet werden die physikalischen Gr\"o\ss{}en \glqq Fluss\grqq { }und \glqq F\"ullstand\grqq . Deren Wechselwirkung soll durch einen Graphen basierend auf der vorhandenen \ac{hazop} dargestellt werden.  

Da der zu konstruierende Graph auf der \ac{hazop} basieren soll, m\"ussen die vorhandenen Informationen geeignet verarbeitet werden. Dazu dient die entwickelte Funktion \glqq buildGraphForModule\grqq . Sie liest aus einem bestimmten Bereich eines w\"ahlbaren Arbeitsblattes einer spezifizierten Excel Datei die Daten aus. Die Daten werden anschlie\ss{}end analysiert. \newline
Im ersten Schritt werden die Variablen F\"ullstand und Fluss durch die Buchstaben L und F ersetzt und mit einer Zahl kombiniert, welche die Nummer des analysierten Moduls widerspiegelt. Weiterhin wird diese Abk\"urzung mit dem in der Spalte \glqq unit\grqq { }definierten Text kombiniert. Das Ergebnis wird im Graphen als Knoten dargestellt. Der Eingangsstrom wird durch den Knoten \textit{in1F1}, der Ausgangsstrom durch den Knoten \textit{out1F1} und der F\"ullstand im Tank mit Hilfe des Knotens \textit{tankL1} abgebildet. \newline
Im zweiten Schritt werden die Auswirkungen von Abweichungen betrachtet. Diese k\"onnen entweder auf Prozessvariablen im Modul, auf externe Variablen oder auf Equipment wirken. Da in der Implementierung nur Durchfluss und F\"ullstand betrachtet werden, ist eine Wirkung auf andere als diese Gr\"o\ss{}en in der Verarbeitung gleichbedeutend mit einer Wirkung auf das Equipment. Solche Auswirkungen werden durch spezielle Knoten dargestellt, welche durch den Buchstaben \textit{C} und eine Zahlenkette identifiziert werden. In \figref{fig:graph_mod1} wird das Ergebnis der Graphenbildung f\"ur das Tankmodul 1 dargestellt.

\begin{figure}[h!tb]
\centering
			\begin{tikzpicture}[%
			edge from parent/.style={very thick,draw=black!70,-latex},
	level a/.style={xshift=2em},
	level b1/.style={xshift=9em},
	level b2/.style={xshift=5em},
	level b3/.style={xshift=2em}
	]
	%%%% Beginn tikz-Baumdiagramm
	\node [sumGraph](1) {}
	child { node [sumGraph](2){}
		child {node [sumGraph](3){}}
		child {node [sumGraph](4){}
			child {node [sumGraph](5){}}
			child {node [sumGraph](6){}}
			child {node [sumGraph](7){}}}};
	\node [output]	at (1.east)	[label=0:in1F1]	{};
	\node [output]	at (2.east)	[label=0:tankL1]	{};
	\node [output]	at (3.west)	[label={[xshift=-0.60cm, yshift=-0.4cm]C4\_1}] {};
%	[label=-10:C4\_1]	{};%
	\node [output]	at (4.east)	[label={[xshift=0.9cm, yshift=-0.3cm]out1F1}]	{};
	\node [output]	at (5.west)	[label={[xshift=0.2cm, yshift=-0.9cm]C6\_1}]	{};
	\node [output]	at (6.west)	[label={[xshift=0.2cm, yshift=-0.9cm]C7\_1}]	{};
	\node [output]	at (7.west)	[label={[xshift=0.2cm, yshift=-0.9cm]C8\_1}]	{};
	\end{tikzpicture}
\caption[Digraph von Modul 1]{Digraph von Tankmodul 1 auf Basis der HAZOP}
\label{fig:graph_mod1}
\end{figure}

In \figref{fig:graph_mod1} ist beispielsweise die Auswirkung vom Eingangsstrom auf das Tankvolumen und dessen Wirkung auf den Rest des Moduls dargestellt. Entsprechend der in \tabref{tab:hazopBsp_M1} dargestellten \ac{hazop} hat der Stoffstrom vom Eingang, je nach betrachteter Abweichung, verschiedene Auswirkungen auf das Tankvolumen. Diese m\"oglichen Auswirkungen k\"onnen mit Hilfe der Funktion \glqq findConsequenceOfDeviation\grqq { }ermittelt werden. Diese Funktion liefert alle Auswirkungen auf einen betrachteten Knoten, welche durch Abweichungen von benachbarten Knoten entstehen k\"onnen. F\"ur den Knoten \textit{tankL1} liefert die Funktion daher die m\"oglichen Auswirkungen \textit{less{\_}tankL1, more{\_}tankL1}, und \textit{less{\_}tankL1}. Dies entspricht den in \tabref{tab:hazopBsp_M1} untersuchten Fehlerf\"allen mit den IDs $1-3$. \newline
Der Knoten \textit{C4{\_}1} repr\"asentiert die Auswirkung einer Abweichung der Variable \textit{tankL1} auf ein Equipment. Die Anwendung der Funktion \glqq findConsequenceOfDeviation\grqq { }auf diesen Knoten liefert die Information, dass es sich um die Auswirkung \textit{more pressure} handelt. Durch Anwendung der Funktion \glqq findDeviationforConsequence\grqq { }ergibt sich, dass die Abweichung, welche zu dieser Auswirkung gef\"uhrt hat, \textit{more tankL1} ist. Als Ursache f\"ur diese Abweichung kann mit Hilfe von \glqq findCauseForDeviation\grqq { }die Information \textit{$ID=2$} ermittelt werden. Die M\"oglichkeit eines Sensorfehlers als Ursache f\"ur die Abweichung \textit{more tankL1}, welche die Auswirkung \textit{more pressure} erzeugt, wird nicht angegeben. Der Grund daf\"ur ist der limitierte Algorithmus, welcher die \ac{hazop}"=Studie automatisiert in einen gerichteten Graphen \"ubersetzt. 

In \figref{fig:graph_mod2Error} ist dargestellt, was passiert, wenn die Syntax in der verarbeiteten \ac{hazop} nicht der entspricht, welche von der Funktion \glqq buildGraphForModule\grqq { }erwarteten wird. Das in \figref{fig:PIDMod2} dargestellte Modul 2 sollte erwartungsgem\"a\ss{} das gleiche Verhalten wie Modul 1 aufweisen, wenn der zweite Eingang von Modul 2 nicht betrachtet wird. Die verwendete \ac{hazop} f\"ur Modul 2 ist in \tabref{tab:hazopBsp_M2Fehler} zu finden. Im Vergleich zu \tabref{tab:hazopBsp_M1} zeigen sich vor allem Unterschiede in der Beschreibung der Auswirkungen von Abweichungen. Statt \textit{less tanklevel} ist der Fehlerfall 1 in \tabref{tab:hazopBsp_M2Fehler} durch \textit{less level} beschrieben. Durch diese syntaktische Abweichung scheitert der Algorithmus bei der Zuordnung dieser Auswirkung zur Prozessvariable \textit{tankL2}. In Folge dessen wird ein neuer Knoten \textit{L2} generiert, welcher die betrachtete Auswirkung eines sinkenden F\"ullstandes symbolisiert. Die Information der in der \ac{hazop} notierten Wechselwirkung geht dabei verloren. 
\begin{figure}[h!tb]
\centering
\begin{tikzpicture}[%
			edge from parent/.style={very thick,draw=black!70,-latex},
	level a/.style={xshift=2em},
	level b1/.style={xshift=9em},
	level b2/.style={xshift=5em},
	level b3/.style={xshift=2em}
	]
	%%%% Beginn tikz-Baumdiagramm
	\node [sumGraph](1) {}
	child { node [sumGraph](2){}};
	\node [output]	at (1.east)	[label=0:in1F2]	{};
	\node [output]	at (2.west)	[label={[xshift=0.2cm, yshift=-0.9cm]L2}]	{};
	
	\node [sumGraph, right of=1, node distance=4cm](3) {}
	child { node [sumGraph](4){}}
	child { node [sumGraph](5){}};
	\node [output]	at (3.east)	[label=0:tankL2]	{};
	\node [output]	at (4.west)	[label={[xshift=0.2cm, yshift=-0.9cm]C4\_2}]	{};
	\node [output]	at (5.west)	[label={[xshift=0.2cm, yshift=-0.9cm]C5\_2}]	{};
	
	\node [sumGraph, right of=3, node distance=6cm](6) {}
	child { node [sumGraph](7){}}
	child { node [sumGraph](8){}}
	child { node [sumGraph](9){}};
	\node [output]	at (6.east)	[label=0:out1F2]	{};
	\node [output]	at (7.west)	[label={[xshift=0.2cm, yshift=-0.9cm]C6\_2}]	{};
	\node [output]	at (8.west)	[label={[xshift=0.2cm, yshift=-0.9cm]C7\_2}]	{};
	\node [output]	at (9.west)	[label={[xshift=0.2cm, yshift=-0.9cm]C8\_2}]	{};
	
	\end{tikzpicture}
\caption[fehlerhafter Digraph von Modul 2]{Digraph von Modul 2 auf Basis der HAZOP mit syntaktischen Abweichungen}
\label{fig:graph_mod2Error}
\end{figure}

Entspricht die Syntax aller drei Module den Erwartungen, so kann ein Graph mit drei Subgraphen geniert werden, welcher in \figref{fig:graph_sysGes} abgebildet ist. Die Module 1 und 2 weisen in diesem Fall jeweils eine Kopplung vom Eingangsstrom \"uber das Tankvolumen zum Ausgangsstrom auf. Das in \figref{fig:PIDMod3} dargestellte dritte Modul wird durch drei Teilgraphen modelliert. Es weist entsprechend der in \tabref{tab:hazopBsp_M3} dargestellten \ac{hazop} keine Wechselwirkung zwischen den betrachteten Prozessgr\"o\ss{}en auf. Die betrachteten Auswirkungen von Abweichungen der Prozessvariablen \textit{in1F3, in2F3} und \textit{coloumnL3}  sind daher alle vom Typ Equipment und werden als spezielle Knoten repr\"asentiert. Die Kopplung der drei Module wird im folgenden Abschnitt \ref{sec:kopplung} erl\"autert.

\begin{figure}[h!tb]
\centering
\begin{tikzpicture}[%
			edge from parent/.style={very thick,draw=black!70,-latex},
	level a/.style={xshift=2em},
	level b1/.style={xshift=9em},
	level b2/.style={xshift=5em},
	level b3/.style={xshift=2em}
	]
	%%%% Beginn tikz-Baumdiagramm
	\node [sum](1) {}
	child { node [sum](2){}
		child {node [sum](3){}}
		child {node [sum](4){}
				child {node [sum](5){}}
				child {node [sum](6){}}
				child {node [sum](7){}}
				child {node [sum](8){}
					child {node [sum](9){}
						child {node [sum](10){}}
						child {node [sum](11){}}
						child {node [sum](12){}}}}}};
	\node [output]	at (1.east)	[label=0:in1F1]	{};
	\node [output]	at (2.east)	[label=0:tankL1]	{};
	\node [output]	at (3.west)	[label={[xshift=0cm, yshift=-0.8cm]C4\_1}]	{};
	\node [output]	at (4.east)	[label=0:out1F1]	{};
	\node [output]	at (5.west)	[label={[xshift=0.2cm, yshift=-0.9cm]C6\_1}]	{};
	\node [output]	at (6.west)	[label={[xshift=0.2cm, yshift=-0.9cm]C7\_1}]	{};
	\node [output]	at (7.west)	[label={[xshift=0.2cm, yshift=-0.9cm]C8\_1}]	{};
	\node [output]	at (8.east)	[label={[xshift=0.9cm, yshift=0.4cm]F\_out1F1\_in1F3}]	{};
	\node [output]	at (9.east)	[label=0:in1F3]	{};
	\node [output]	at (10.west)	[label={[xshift=0.2cm, yshift=-0.9cm]C1\_3}]	{};
	\node [output]	at (11.west)	[label={[xshift=0.2cm, yshift=-0.9cm]C2\_3}]	{};
	\node [output]	at (12.west)	[label={[xshift=0.2cm, yshift=-0.9cm]C3\_3}]	{};
	
	\node [sum, right of=1, node distance=7.5cm](14) {}
	child { node [sum](15){}
		child {node [sum](16){}}
		child {node [sum](17){}
				child {node [sum](19){}}
				child {node [sum](20){}}
				child {node [sum](21){}}
				child {node [sum](22){}
					child {node [sum](23){}
						child {node [sum](24){}}
						child {node [sum](25){}}
						child {node [sum](26){}}}}}};
	\node [output]	at (14.east)	[label=0:in1F2]	{};
	\node [output]	at (15.east)	[label=0:tankL2]	{};
	\node [output]	at (16.west)	[label={[xshift=0cm, yshift=-0.8cm]C4\_2}]	{};
	\node [output]	at (17.east)	[label=0:out1F2]	{};
	\node [output]	at (19.west)	[label={[xshift=0.2cm, yshift=-0.9cm]C6\_2}]	{};
	\node [output]	at (20.west)	[label={[xshift=0.2cm, yshift=-0.9cm]C7\_2}]	{};
	\node [output]	at (21.west)	[label={[xshift=0.2cm, yshift=-0.9cm]C8\_2}]	{};
	\node [output]	at (22.east)	[label={[xshift=0.9cm, yshift=0.4cm]F\_outF2\_in2F3}]	{};
	\node [output]	at (23.east)	[label=0:in2F3]	{};
	\node [output]	at (24.west)	[label={[xshift=0.2cm, yshift=-0.9cm]C4\_3}]	{};
	\node [output]	at (25.west)	[label={[xshift=0.2cm, yshift=-0.9cm]C5\_3}]	{};
	\node [output]	at (26.west)	[label={[xshift=0.2cm, yshift=-0.9cm]C6\_3}]	{};
	
	\node [sum, right of=1, node distance=4cm](27) {}
	child { node [sum](28){}}
	child { node [sum](29){}};
	\node [output]	at (27.east)	[label=0:coloumnL3]	{};
	\node [output]	at (28.west)	[label={[xshift=0.2cm, yshift=-0.9cm]C7\_3}]	{};
	\node [output]	at (29.west)	[label={[xshift=0.2cm, yshift=-0.9cm]C8\_3}]	{};
	
	\end{tikzpicture}
%\begin{tikzpicture}[%
  edge from parent/.style={very thick,draw=black!70,-latex},
	level a/.style={xshift=2em},
	level b1/.style={xshift=9em},
	level b2/.style={xshift=5em},
	level b3/.style={xshift=2em}
	]
	\node [sumGraph, right of=8, node distance=4cm](15) {}
	child { node [sumGraph](16){}}
	child { node [sumGraph](17){}}
	child { node [sumGraph](18){}};
	\node [output]	at (15.east)	[label=0:in1F3]	{};
	\node [output]	at (16.west)	[label=-10:C1\_3]	{};
	\node [output]	at (17.west)	[label=-10:C2\_3]	{};
	\node [output]	at (18.west)	[label=-10:C3\_3]	{};
	
	\node [sumGraph, right of=15, node distance=4.5cm](19) {}
	child { node [sumGraph](20){}}
	child { node [sumGraph](21){}}
	child { node [sumGraph](22){}};
	\node [output]	at (19.east)	[label=0:in2F3]	{};
	\node [output]	at (20.west)	[label=-10:C4\_3]	{};
	\node [output]	at (21.west)	[label=-10:C5\_3]	{};
	\node [output]	at (22.west)	[label=-10:C6\_3]	{};
	
	\node [sumGraph, right of=11, node distance=6cm](23) {}
	child { node [sumGraph](24){}}
	child { node [sumGraph](25){}};
	\node [output]	at (23.east)	[label=0:coloumnL3]	{};
	\node [output]	at (24.west)	[label=-10:C7\_3]	{};
	\node [output]	at (25.west)	[label=-10:C8\_3]	{};
	
	\end{tikzpicture}
\caption[Digraphen aller Module]{Digraph aller drei Module bei korrekter Syntax der HAZOP}
\label{fig:graph_sysGes}
\end{figure}

\section{Kopplung von Graphen und Fehlerfortpflanzung}\label{sec:kopplung}
Anhand der \ac{hazop} von einzelnen Modulen k\"onnen durch Einsatz der in Abschnitt \ref{sec:graphHazop} beschriebenen Funktionen gerichtete Graphen erzeugt und analysiert werden. Die erzeugten Graphen spiegeln die in der \ac{hazop} enthaltenen Informationen wider. Wird eine Anlage aus Modulen zu einer Gesamtanlage kombiniert, so enth\"alt das \ac{pid} der Gesamtanlage die notwendigen Informationen, um die Kopplung der Module zu beschreiben. Eine schematische Abbildung der betrachteten Gesamtanlage ist \figref{fig:PIDGes} zu entnehmen. \newline
Die physische Kopplung von Modulen erfolgt durch Einsatz von Rohren. Die Funktion \glqq connectPipe\grqq { }ist daher der Funktionsweise eines Rohres nachempfunden. Innerhalb eines Rohres kann es bez\"uglich des Durchflusses zu Abweichungen kommen. M\"ogliche Ursachen sind Lecks und Hindernisse im Rohr. Weiterhin kann ein Rohr als passive Komponente keine Abweichungen kompensieren. Es leitet vorhandene Abweichungen daher an angrenzende Module weiter. Ist beispielsweise die Temperatur am Auslass von Modul 1 zu hoch, so muss das angeschlossene Rohr diese Auswirkung auf Modul 3 weiterleiten. Da in dieser Arbeit nur Stoffstr\"ome und F\"ullst\"ande beachtet werden, wird eine solche Abweichung vom entwickelten Algorithmus nicht modul\"ubergreifend fortgepflanzt. \newline
Die Verbindung von zwei Modulen wird im Graphen durch die Einf\"uhrung eines neuen Knotens symbolisiert, welcher Abweichungen bez\"uglich eines Stoffstromes vom Vorg\"angermodul an das Nachfolgemodul weiterleitet. Die Module 1 und 2 werden durch Anwendung der beschriebenen Funktion \glqq connectPipe\grqq { }unter expliziter Angabe von Ursprungs- und Zielknoten mit dem Modul 3 verbunden. In \figref{fig:graph_sysGesVerb} ist der entstehende Graph der Gesamtanlage dargestellt.

\begin{figure}[h!tb]
\centering
%\begin{tikzpicture}[%
%			edge from parent/.style={very thick,draw=black!70,-latex},
%	level a/.style={xshift=2em},
%	level b1/.style={xshift=9em},
%	level b2/.style={xshift=5em},
%	level b3/.style={xshift=2em}
%	]
%	%%%% Beginn tikz-Baumdiagramm
%	\node [sumGraph](1) {}
%	child { node [sumGraph](2){}
%		child {node [sumGraph](3){}}
%		child {node [sumGraph](4){}
%				child {node [sumGraph](5){}}
%				child {node [sumGraph](6){}}
%				child {node [sumGraph](7){}}
%				child {node [sumGraph](8){}
%					child {node [sumGraph](9){}
%						child {node [sumGraph](10){}}
%						child {node [sumGraph](11){}}
%						child {node [sumGraph](12){}}}}}};
%	\node [output]	at (1.east)	[label=0:in1F1]	{};
%	\node [output]	at (2.east)	[label=0:tankL1]	{};
%	\node [output]	at (3.west)	[label=-10:C4\_1]	{};
%	\node [output]	at (4.east)	[label=0:out1F1]	{};
%	\node [output]	at (5.west)	[label=-10:C6\_1]	{};
%	\node [output]	at (6.west)	[label=-10:C7\_1]	{};
%	\node [output]	at (7.west)	[label=-10:C8\_1]	{};
%	\node [output]	at (8.east)	[label=0:F\_out1F1\_in1F3]	{};
%	\node [output]	at (9.east)	[label=0:in1F3]	{};
%	\node [output]	at (10.west)	[label=-10:C1\_3]	{};
%	\node [output]	at (11.west)	[label=-10:C2\_3]	{};
%	\node [output]	at (12.west)	[label=-10:C3\_3]	{};
%		
%	\node [sumGraph, right of=1, node distance=4cm](27) {}
%	child { node [sumGraph](28){}}
%	child { node [sumGraph](29){}};
%	\node [output]	at (27.east)	[label=0:coloumnL3]	{};
%	\node [output]	at (28.west)	[label=-10:C7\_3]	{};
%	\node [output]	at (29.west)	[label=-10:C8\_3]	{};
%	\end{tikzpicture}

\begin{tikzpicture}[%
			edge from parent/.style={very thick,draw=black!70,-latex},
	level a/.style={xshift=2em},
	level b1/.style={xshift=9em},
	level b2/.style={xshift=5em},
	level b3/.style={xshift=2em}
	]
	%%%% Beginn tikz-Baumdiagramm
	\node [sumGraph](1) {}
	child { node [sumGraph](2){}
		child {node [sumGraph](3){}}
		child {node [sumGraph](4){}
				child {node [sumGraph](5){}}
				child {node [sumGraph](6){}}
				child {node [sumGraph](7){}}
				child {node [sumGraph](8){}
					child {node [sumGraph](9){}
						child {node [sumGraph](10){}}
						child {node [sumGraph](11){}}
						child {node [sumGraph](12){}}}}}};
	\node [output]	at (1.east)	[label=0:in1F1]	{};
	\node [output]	at (2.east)	[label=0:tankL1]	{};
	\node [output]	at (3.west)	[label={[xshift=0cm, yshift=-0.9cm]C4\_1}]	{};
	\node [output]	at (4.east)	[label=0:out1F1]	{};
	\node [output]	at (5.west)	[label={[xshift=0.2cm, yshift=-0.9cm]C6\_1}]	{};
	\node [output]	at (6.west)	[label={[xshift=0.2cm, yshift=-0.9cm]C7\_1}]	{};
	\node [output]	at (7.west)	[label={[xshift=0.2cm, yshift=-0.9cm]C8\_1}]	{};
	\node [output]	at (8.east)	[label=0:F\_out1F1\_in1F3]	{};
	\node [output]	at (9.east)	[label=0:in1F3]	{};
	\node [output]	at (10.west)	[label={[xshift=0.2cm, yshift=-0.9cm]C1\_3}]	{};
	\node [output]	at (11.west)	[label={[xshift=0.2cm, yshift=-0.9cm]C2\_3}]	{};
	\node [output]	at (12.west)	[label={[xshift=0.2cm, yshift=-0.9cm]C3\_3}]	{};
		
	\node [sumGraph, right of=1, node distance=4cm](27) {}
	child { node [sumGraph](28){}}
	child { node [sumGraph](29){}};
	\node [output]	at (27.east)	[label=0:coloumnL3]	{};
	\node [output]	at (28.west)	[label={[xshift=0.2cm, yshift=-0.9cm]C7\_3}]	{};
	\node [output]	at (29.west)	[label={[xshift=0.2cm, yshift=-0.9cm]C8\_3}]	{};
	
	
	\node [sumGraph, right of=27, node distance=4cm](15) {}
	child { node [sumGraph](16){}}
	child { node [sumGraph](17){}}
	child { node [sumGraph](18){}};
	\node [output]	at (15.east)	[label=0:in1F3]	{};
	\node [output]	at (16.west)	[label={[xshift=0.2cm, yshift=-0.9cm]C1\_3}]	{};
	\node [output]	at (17.west)	[label={[xshift=0.2cm, yshift=-0.9cm]C2\_3}]	{};
	\node [output]	at (18.west)	[label={[xshift=0.2cm, yshift=-0.9cm]C3\_3}]	{};
	
	\node [sumGraph, below of=18, node distance=1.5cm](23) {}
	child { node [sumGraph](24){}}
	child { node [sumGraph](25){}};
	\node [output]	at (23.east)	[label=0:coloumnL3]	{};
	\node [output]	at (24.west)	[label={[xshift=0.2cm, yshift=-0.9cm]C7\_3}]	{};
	\node [output]	at (25.west)	[label={[xshift=0.2cm, yshift=-0.9cm]C8\_3}]	{};
	
		\node [sumGraph, below of=24, node distance=1.5cm](19) {}
	child { node [sumGraph](20){}}
	child { node [sumGraph](21){}}
	child { node [sumGraph](22){}};
	\node [output]	at (19.east)	[label=0:in2F3]	{};
	\node [output]	at (20.west)	[label={[xshift=0.2cm, yshift=-0.9cm]C4\_3}]	{};
	\node [output]	at (21.west)	[label={[xshift=0.2cm, yshift=-0.9cm]C5\_3}]	{};
	\node [output]	at (22.west)	[label={[xshift=0.2cm, yshift=-0.9cm]C6\_3}]	{};

	\end{tikzpicture}
%\begin{tikzpicture}[%
			edge from parent/.style={very thick,draw=black!70,-latex},
	level a/.style={xshift=2em},
	level b1/.style={xshift=9em},
	level b2/.style={xshift=5em},
	level b3/.style={xshift=2em}
	]
		\node [sumGraph](14) {}
	  child {node [sumGraph](15){}
		  child {node [sumGraph](16){}}
		  child {node [sumGraph](17){}
				child {node [sumGraph](19){}}
				child {node [sumGraph](20){}}
				child {node [sumGraph](21){}}
				child {node [sumGraph](22){}
					child {node [sumGraph](23){}
						child {node [sumGraph](24){}}
						child {node [sumGraph](25){}}
						child {node [sumGraph](26){}}}}}};
	\node [output]	at (14.east)	[label=0:in1F2]	{};
	\node [output]	at (15.east)	[label=0:tankL2]	{};
	\node [output]	at (16.west)	[label=-10:C4\_2]	{};
	\node [output]	at (17.east)	[label=0:out1F2]	{};
	\node [output]	at (19.west)	[label=-10:C6\_2]	{};
	\node [output]	at (20.west)	[label=-10:C7\_2]	{};
	\node [output]	at (21.west)	[label=-10:C8\_2]	{};
	\node [output]	at (22.east)	[label=0:F\_outF2\_in2F3]	{};
	\node [output]	at (23.east)	[label=0:in2F3]	{};
	\node [output]	at (24.west)	[label=-10:C4\_3]	{};
	\node [output]	at (25.west)	[label=-10:C5\_3]	{};
	\node [output]	at (26.west)	[label=-10:C6\_3]	{};
	
	
	
		\end{tikzpicture}
\caption[Digraph gekoppeltes Gesamtsystems]{Digraph vom Gesamtsystem mit verbundenen Modulen}
\label{fig:graph_sysGesVerb}
\end{figure}

Dieser gerichtete Graph kann durch Anwendung geeigneter Algorithmen auf die Fortpflanzung von Fehlern untersucht werden. Dazu wird im ersten Schritt ein Knoten festgelegt, dessen Auswirkungen auf den Gesamtprozess durch Abweichung des symbolisierten Parameters ermittelt werden soll. Mit Hilfe einer Tiefenanalyse k\"onnen alle Knoten ermittelt werden, welche ohne mehrfache Schleifen unter Beachtung der Richtung der Kanten mit diesem Knoten verbunden sind. Die so erhaltenen Knoten k\"onnen als Untergraph dargestellt werden. 

\begin{figure}[h!tb]
\centering
%\begin{tikzpicture}[%
%			edge from parent/.style={very thick,draw=black!70,-latex},
%	level a/.style={xshift=2em},
%	level b1/.style={xshift=9em},
%	level b2/.style={xshift=5em},
%	level b3/.style={xshift=2em}
%	]
%	%%%% Beginn tikz-Baumdiagramm
%	\node [sumGraph](1) {}
%		child {node [sumGraph](3){}}
%		child {node [sumGraph](4){}
%				child {node [sumGraph](5){}}
%				child {node [sumGraph](6){}}
%				child {node [sumGraph](7){}}
%				child {node [sumGraph](8){}
%					child {node [sumGraph](9){}
%						child {node [sumGraph](10){}}
%						child {node [sumGraph](11){}}
%						child {node [sumGraph](12){}}}}};
%	\node [output]	at (1.east)	[label=0:tankL1]	{};
%	\node [output]	at (3.west)	[label=-30:C4\_1]	{};
%	\node [output]	at (4.east)	[label=0:out1F1]	{};
%	\node [output]	at (5.west)	[label=-10:C6\_1]	{};
%	\node [output]	at (6.west)	[label=-10:C7\_1]	{};
%	\node [output]	at (7.west)	[label=-10:C8\_1]	{};
%	\node [output]	at (8.east)	[label=0:F\_out1F1\_in1F3]	{};
%	\node [output]	at (9.east)	[label=0:in1F3]	{};
%	\node [output]	at (10.west)	[label=-10:C1\_3]	{};
%	\node [output]	at (11.west)	[label=-10:C2\_3]	{};
%	\node [output]	at (12.west)	[label=-10:C3\_3]	{};
%	\end{tikzpicture}
\begin{tikzpicture}[%
			edge from parent/.style={very thick,draw=black!70,-latex},
	level a/.style={xshift=2em},
	level b1/.style={xshift=9em},
	level b2/.style={xshift=5em},
	level b3/.style={xshift=2em}
	]
	%%%% Beginn tikz-Baumdiagramm
	\node [sumGraph](1) {}
		child {node [sumGraph](3){}}
		child {node [sumGraph](4){}
				child {node [sumGraph](5){}}
				child {node [sumGraph](6){}}
				child {node [sumGraph](7){}}
				child {node [sumGraph](8){}
					child {node [sumGraph](9){}
						child {node [sumGraph](10){}}
						child {node [sumGraph](11){}}
						child {node [sumGraph](12){}}}}};
	\node [output]	at (1.east)	[label=0:tankL1]	{};
	\node [output]	at (3.west)	[label={[xshift=0cm, yshift=-0.9cm]C4\_1}]	{};
	\node [output]	at (4.east)	[label=0:out1F1]	{};
	\node [output]	at (5.west)	[label={[xshift=0.2cm, yshift=-0.9cm]C6\_1}]	{};
	\node [output]	at (6.west)	[label={[xshift=0.2cm, yshift=-0.9cm]C7\_1}]	{};
	\node [output]	at (7.west)	[label={[xshift=0.2cm, yshift=-0.9cm]C8\_1}]	{};
	\node [output]	at (8.east)	[label=0:F\_out1F1\_in1F3]	{};
	\node [output]	at (9.east)	[label=0:in1F3]	{};
	\node [output]	at (10.west)	[label={[xshift=0.2cm, yshift=-0.9cm]C1\_3}]	{};
	\node [output]	at (11.west)	[label={[xshift=0.2cm, yshift=-0.9cm]C2\_3}]	{};
	\node [output]	at (12.west)	[label={[xshift=0.2cm, yshift=-0.9cm]C3\_3}]	{};
	\end{tikzpicture}
\caption[Fehlerfortpflanzung im Gesamtsystem]{Auswirkung einer Abweichung des F\"ullstandes im Tank von Modul 1 auf das Gesamtsystem}
\label{fig:graph_sysGesFehlerfort}
\end{figure}

In \figref{fig:graph_sysGesFehlerfort} ist das Ergebnis aller m\"oglichen Abweichungen von \textit{tankL1} auf den restlichen Prozess dargestellt. Dabei ist aber noch nicht untersucht, ob diese Fortpflanzung sinnvoll ist. F\"ur den Untergraphen muss an jedem Knoten untersucht werden, ob die am Vorg\"angerknoten erzeugte Auswirkung einer g\"ultigen Abweichung des betrachteten Knotens entspricht und ob die Abweichungen des aktuell betrachteten Knotens durch Auswirkungen des Vorg\"angerknotens entstehen k\"onnen. Betrachtet man den Knoten \textit{out1F1} so stellt man fest, dass die Abweichungen \textit{more flow} und \textit{reverse flow} nicht als Auswirkung einer Abweichung von \textit{tankL1} entstehen k\"onnen. Die Verbindungen zu den entsprechenden Auswirkungsknoten \textit{C7{\_}1} und \textit{C8{\_}1} sind daher im Rahmen der Fehlerfortpflanzungsanalyse ung\"ultig. Die Abweichung \textit{no level} von \textit{tankL1} kann sich jedoch in das Modul 3 fortpflanzen und dort die Auswirkung mit der $\text{ID} = 3$ \textit{changed reaction balance} erzeugen. 

\begin{figure}[h!tb]
\centering
%\begin{tikzpicture}[%
%			edge from parent/.style={very thick,draw=black!70,-latex},
%	level a/.style={xshift=2em},
%	level b1/.style={xshift=9em},
%	level b2/.style={xshift=5em},
%	level b3/.style={xshift=2em}
%	]
%	%%%% Beginn tikz-Baumdiagramm
%	\node [sumGraph](1) {}
%	child { node [sumGraph](2){}
%		child {node [sumGraph](3){}}
%		child {node [sumGraph](4){}
%			child {node [sumGraph](5){}
%				child {node [sumGraph](6){}}
%				child {node [sumGraph](7){}}}}};
%	\node [output]	at (1.east)	[label=0:tankL1]	{};
%	\node [output]	at (2.east)	[label=0:out1F1]	{};
%	\node [output]	at (3.west)	[label=-10:C6\_1]	{};
%	\node [output]	at (4.east)	[label=0:F\_out1F1\_in1F3]	{};
%	\node [output]	at (5.east)	[label=0:in1F3]	{};
%	\node [output]	at (6.west)	[label=-10:C2\_3]	{};
%	\node [output]	at (7.west)	[label=-10:C3\_3]	{};
%	\end{tikzpicture}
\begin{tikzpicture}[%
			edge from parent/.style={very thick,draw=black!70,-latex},
	level a/.style={xshift=2em},
	level b1/.style={xshift=9em},
	level b2/.style={xshift=5em},
	level b3/.style={xshift=2em}
	]
	%%%% Beginn tikz-Baumdiagramm
	\node [sumGraph](1) {}
	child { node [sumGraph](2){}
		child {node [sumGraph](3){}}
		child {node [sumGraph](4){}
			child {node [sumGraph](5){}
				child {node [sumGraph](6){}}
				child {node [sumGraph](7){}}}}};
	\node [output]	at (1.east)	[label=0:tankL1]	{};
	\node [output]	at (2.east)	[label=0:out1F1]	{};
	\node [output]	at (3.west)	[label={[xshift=0.2cm, yshift=-0.9cm]C6\_1}]	{};
	\node [output]	at (4.east)	[label=0:F\_out1F1\_in1F3]	{};
	\node [output]	at (5.east)	[label=0:in1F3]	{};
	\node [output]	at (6.west)	[label={[xshift=0.2cm, yshift=-0.9cm]C2\_3}]	{};
	\node [output]	at (7.west)	[label={[xshift=0.2cm, yshift=-0.9cm]C3\_3}]	{};
	\end{tikzpicture}
\caption[Fehlerfortpflanzung im Gesamtsystem gefiltert]{Auswirkung der Abweichung \textit{no level} des F\"ullstandes im Tank von Modul 1 auf das Gesamtsystem unter}
\label{fig:graph_sysGesFehlerfortNoLevel}
\end{figure}

In \figref{fig:graph_sysGesFehlerfortNoLevel} ist der Graph dargestellt, welcher durch Analyse einer Fortpflanzung der Abweichung \textit{no level} entsteht. Es ist festzustellen, dass der Knoten \textit{in1F3} entgegen den Erwartungen mit zwei Folgeknoten verbunden ist. Der Knoten \textit{C3\_3} entspricht der $\text{ID} = 3$ aus \tabref{tab:hazopBsp_M3} und ist korrekt. Der Knoten \textit{C2\_3} ist fehlerhaft. Durch Anwendung der Funktion \glqq findDeviationforConsequence\grqq { }auf \textit{C2\_3} kann die zum Entstehen der von \textit{C2\_3} repr\"asentierten Auswirkung notwendige Abweichung von \textit{in1F3} ermittelt werden. Die erforderliche Abweichung lautet \textit{more{\_}in1F3}, welche entsprechend \tabref{tab:hazopBsp_M3} dem Fehlerfall $\text{ID} = 2$ zugeordnet werden kann. Der implementierte Algorithmus hat also f\"alschlicherweise analysiert, dass die Abweichung \textit{more{\_}in1F3} der Variable \textit{in1F3} eine m\"ogliche Auswirkung der Abweichung \textit{no level} der Variable \textit{tankL1} in Modul 1 ist. Die Ursache daf\"ur ist die rudiment\"are Implementierung des Rohres. Das Rohr leitet einerseits vorhandene Abweichungen weiter, es erzeugt aber auch neue Abweichungen. Da nicht unterschieden wird, ob die fortgepflanzte Abweichung im Rohr selbst oder davor entstanden ist, l\"ost das Rohr die Auswirkungen \textit{more{\_}in1F3} und \textit{less{\_}in1F3} aus, welche wiederum den Abweichungen $\text{ID} = 2,3$ aus \tabref{tab:hazopBsp_M3} entsprechen. 
  
\begin{figure}[h!tb]
\centering
			\begin{tikzpicture}[%
	%%%% Ausrichtung
	%every child/.style={anchor=west,grow=down},
	%%%% kleinere Schriftgröße
	%node font=\small,
	%%%% definiert von wo (Parent) nach wo (Child) die Linien gehen sollen, wenn Pfeil zu Child gewünscht, dann "[->]" wie folgt einsetzen "...path={[->](tikzparentnode.south)..."
	%edge from parent path={([xshift=0em]\tikzparentnode.east) |- (\tikzchildnode.west)},
	%%%% horizontale Abstände
	edge from parent/.style={very thick,draw=black!70,-latex},
	level a/.style={xshift=2em},
	level b1/.style={xshift=9em},
	level b2/.style={xshift=5em},
	level b3/.style={xshift=2em}
	]
	%%%% Beginn tikz-Baumdiagramm
	\node [sumGraph](1) {}
	child { node [sumGraph](2){}
		child {node [sumGraph](3){}}};
	\node [output]	at (1.east)	[label=0:in1F1]	{};
	\node [output]	at (2.east)	[label=0:tankL1]	{};
	\node [output]	at (3.east)	[label=0:C4\_1]	{};
%	child[level distance=0cm,level a] { node [box] {anschauungsraumbezogene Verfahren}
%		child[level distance=0cm,level b1] { node [box] {Guyan}
%			child[level distance=0cm,level b2] {node [box] {RM a}}}
%		child[level distance=1cm,level b1] { node [box] {Dyn. Kondensation}
%			child[level distance=0cm,level b2] {node [box] {RM b}}}
%		child[level distance=2cm,level b1] { node [box] {IRS}
%			child[level distance=0cm,level b2] {node [box] {RM c}}}
%	}
%	child[level distance=3cm,level a] { node [box] {Verfahren im modalen Raum}
%		child[level distance=0cm,level b1] { node [box] {SOMT}
%			child[level distance=0cm,level b2] {node [box] {RM d}}}
%		child[level distance=1cm,level b1] { node [box] {SEREP}
%			child[level distance=0cm,level b2] {node [box] {RM e}}}
%	}
%	child[level distance=5cm,level a] { node [box] {Verfahren im allg. Vektorraum}
%		child[level distance=0cm,level b1] { node [box] {KSM}
%			child[level distance=0cm,level b2] {node [box] {RM f}}}
%		child[level distance=1cm,level b1] { node [box] {SOBT}
%			child[level distance=0cm,level b2] {node [box] {RM g}}}
%		child[level distance=2cm,level b1] { node [box] {POD}
%			child[level distance=0cm,level b2] {node [box] {RM h}}}
%	}
%	};
	\end{tikzpicture}
\caption[Fehlerfortpflanzung im Gesamtsystem korrekt analysiert]{Auswirkung der Abweichung \textit{more flow} des Zuflusses von Modul 1 auf das Gesamtsystem}
\label{fig:graph_sysGesFehlerfortMoreFlow}
\end{figure}

Wird alternativ die Abweichung \textit{more{\_}flowIn1} betrachtet, so ergibt sich ein Graph nach \figref{fig:graph_sysGesFehlerfortMoreFlow}. Anhand von \tabref{tab:hazopBsp_M1} kann der entstehende Graph \"uberpr\"uft werden. Ein Anstieg des Zuflusses bewirkt ein Ansteigen des Tankf\"ullstandes. Dies f\"uhrt wiederum zu einem Druckanstieg, wie in \tabref{tab:hazopBsp_M1} als $\text{ID} = 4$ dargestellt ist. Ein Anstieg des F\"ullstandes hat aber keine Auswirkungen auf den Stoffstrom im Ausgang von Modul 1. Daher pflanzt sich die Abweichung \textit{more{\_}flowIn1} nicht auf andere Anlagenteil fort. 

Auf Basis der \acp{hazop} einzelner Module und einem \ac{pid} der Gesamtanlage kann demnach teilautomatisiert ein Graph erzeugt werden, anhand dessen man die Fortpflanzung von Fehlern untersuchen kann. Aussagen dar\"uber, ob die derart ermittelten Fehler in einem Modul erkannt und beherrscht werden k\"onnen, sind derzeit nicht m\"oglich, da zum Entwickeln eines geeigneten Algorithmus die Festlegung einer Ontologie erforderlich ist. Ohne Festlegung einer Ontologie m\"usste eine Syntax definiert werden, mit Hilfe derer die notwendigen Informationen dargestellt und ausgewertet werden. Wie im Abschnitt \ref{sec:graphHazop} gezeigt wird ist dieses Vorgehen bei komplexen, textuell basierten Informationen jedoch sehr fehleranf\"allig. Weiterhin ist die vorhandene Datenbasis, um derartige komplexe Untersuchungen durchzuf\"uhren, derzeit zu gering.