\chapter{Fehlerfortpflanzung in modularen Anlagen durch Einsatz von Graphen}\label{ch:ffpflMod}

Betrachtung der Vor- und Nachteile von \ac{sdg} wird in \cite{Yang_2010} behandelt: 
\begin{itemize}
\item Vorteile einer wissensbasierten Erstellung von \acp{sdg}: 
  \begin{itemize}
  \item teilautomatisiert durch Softwareunterst\"utzung m\"oglich
  \end{itemize}
\item Nachteile: 
  \begin{itemize}
  \item Graph wird nicht plausibilisiert, kein Test auf Kausalit\"at
  \item Kanten werden nicht gewichtet -> extrem unwahrscheinliche Fehlerfortpflanzungen sind nicht von wahrscheinlichen zu unterscheiden
  \item Wechselwirkungen zwischen mehreren Eingangsparametern werden nicht abgebildet
  \item Zeithorizont wird nicht abgebildet, in Folge dessen sind transiente Vorg\"ange nicht von Station\"aren zu unterscheiden
  \end{itemize}   
\end{itemize}
L\"osung: \"Ubergang zu datenbasierter Graphenentwicklung. Durch die Verwendung von Messdaten und den Einsatz gezielter Verz\"ogerungen kann die Korrelation zwischen Gr\"o\ss{}en bestimmt und der Graph konstruiert werden. Die Korrelation muss dabei mit Hilfe geeigneter Verfahren auf Signifikanz untersucht werden. Der datenbasierter Ansatz verfeinert den wissensbasierten Graphen dann erfolgreich. 