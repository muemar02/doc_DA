\chapter{Literatursichtung}\label{ch:lit}
\paragraph*{\cite{Behr_2012}} Relevanz der Miniplant-Technik an einem konkreten Beispiel.

\paragraph*{\cite{Bramsiepe_2012}} Anforderungen an Module. Grundlegende Gedanken und notwendige Schritte zur Verwendung von Modulen. \textcolor{red}{Arbeiten, welche diesen Artikel zitieren, sind wahrscheinlich hilfreich}.

\paragraph*{\cite{Brodhagen_2012}} Kostenvorteil durch Modularisierung von Anlange, insbesondere bei Verwendung von Mikroreaktoren. Ob damit auch Module im Sinne meiner Arbeit gemeint sind, ist noch zu kl\"aren. \textcolor{red}{Kosteneinsparung durch Modularisierung}

\paragraph*{\cite{Dietz2000}} Alternatives Vorgehen bei der Entwicklung einer Anlage, welches eine optimale L\"osung erm\"oglichen soll. Die Verwendung von Modulen im Sinne meiner Arbeit wird dadurch wahrscheinlich erheblich erschwert. \hfill \newline

\textit{Das habe ich hier schon selbst so aus der Arbeit zusammen gefasst.} \hfill \newline

Es ist ein w\"unschenswertes Ziel, bei der Entwicklung von \textbf{neuen Verfahren bzw. Prozessen} die Prozessplanung (was wird mit dem Stoff gemacht: Zerkleinern, Reagieren, Mischen, Trennen usw. damit z. B. ein neuer chem. Stoff gewonnen wird $\mapsto$ R u. I Flie\ss{}bild) und die konkrete Planung und den Entwurf der notwendigen Maschinen zu parallelisieren, um so innovative L\"osungen zu finden (Beispiel innovativer Lsg: die Ausf\"uhrung mehrerer Prozessschritte wie Zerkleinern und Reagieren von Stoffen in einem einzigen, neu entworfenen Apparat bei der Herstellung von Chlorsilanen aus Ferrosilicium und Chlorwasserstoff), welche einen Prozess \textbf{optimal} realisieren. Die Richtlinien VDI 2221 und VDI 2222 reichen dazu nicht aus, da sie zu branchenspezifisch ausgelegt sind (der Wunsch ist ja eine branchen\"ubergreifende, parallele Entwicklung). Eine Prozessentwicklung mittels Fli\ss{}bildern wird als L\"osung vorgeschlagen. Dazu wird eine zu l\"osende Aufgabenstellung in Teilsysteme geringer Komplexit\"at so weit zerlegt, dass sich deren Funktion durch naturwissenschaftliche Grundoperationen darstellen l\"asst. Die gesamte L\"osung der Aufgabe wird dann als Flie\ss{}bild solcher Teilsysteme dargestellt. Die Formulierung einer Funktion  ist dabei losgel\"ost von einer konkreten technischen Umsetzung durch bereits existierende Maschinen bzw. Apparate. Durch diese Darstellung wird ein Blick f\"ur die m\"ogliche Zusammenfassung von Teilsystemen in einer einzigen Maschine erm\"oglicht. Diese muss dann aber neu konstruiert werden; der Prozess wird aber optimal realisiert. Weiterhin kann die  Notwendigkeit jedes Prozessschrittes besser beurteilt werden. Die Wiederverwendbarkeit einer so entwickelten Maschine ist eher gering, da sie ja als optimale L\"osung von genau diesem einen Prozess entwickelt wurde. Es wird bewusst eine Abkehr von vorfabrizierten L\"osungen gefordert (und damit die Verwendung vorgefertigter Module zumindest erschwert, wenn nicht gar unterbunden) um zu innovativen L\"osungen zu gelangen.\textcolor{red}{Alternative zu modularisiertem Ansatz}.

\paragraph*{\cite{Dunjo2010}} HAZOP Literatur\"ubersicht

\paragraph*{\cite{Graf2000}}

\paragraph*{\cite{Grundemann_2012}} Relevanz von Mikroreaktoren und deren Beziehung zur 50 Prozent These anhand zweier Beispiele.

\paragraph*{\cite{Hady_2012}}
Das vorgestellte Konzept umfasst die Definition und Identifizierung von Modulen, deren dreidimensionales Design, die Ablage und Know-how-Sicherung zwecks der Wiederverwendung des Engineering und der Ausrüstungen sowie die Planung und Kostenschätzung mit wiederverwendbaren Modulen.\textcolor{red}{Arbeiten, welche diesen Artikel zitieren, sind wahrscheinlich hilfreich}.

\paragraph*{\cite{Hessel_2012}}Potenzialanalyse von Milli- und Mikroprozesstechniken

\paragraph*{\cite{Hugo2009}}
\textit{Alles nur zitiert!} \hfill \newline
Reaktions- und sicherheitstechnische Aspekte der Umwandlung eines diskontinuierlichen
chemischen Prozesses in ein kontinuierliches Verfahren unter Verwendung von Mikroreaktoren werden untersucht. Betrachtet man Mikroreaktoren als Module, so k\"onnte diese Arbeit interessant werden $\mapsto$ \textcolor{red}{das sollte ich daher fragen}. 

\paragraph*{\cite{Kampczyk_2003}}
Vorstellung eines Rechnerwerkzeugs zur Anlagenplanung \hfill \newline
Modul $=$ Ausr\"ustungen, welche Teil des gleichen Prozessschrittes sind. \textcolor{red}{Diese Arbeit hilft die verschiedenen Verst\"andnisse vom Modulbegriff darzulegen}.

\paragraph*{\cite{Kockmann_2012}} Hochskalieren von Anlagen mit Hilfe modularer Konzepte und Mikroreaktoren. \hfill \newline
Basierend auf Kennzahlen für die Reaktionskinetik und - enthalpie wird eine \textcolor{red}{Korrelation} abgeleitet und mit der konvektiven Wärmeübertragung als ein Kriterium des \textcolor{red}{sicheren Reaktorbetriebs} gekoppelt.

\paragraph*{\cite{Lier2016}} Relevanz modularisierter Anlagen

\paragraph*{\cite{Obst2013a}} Relevanz modularisierter Anlagen

\paragraph*{\cite{Obst2014}}
Inhalt dieser Arbeit

\paragraph*{\cite{Obst2013}}
Integration einer Package Unit in bestehende Anlage. Vergleich von bekannten Technologien und deren Bewertung.

\paragraph*{\cite{Ohle2014}} Konkretes Beispiel einer Modularen Anlage

\paragraph*{\cite{Oppelt2015}}
Notwendigkeit von Simulation w\"ahrend dem Lebenszyklus einer Anlage

\paragraph*{\cite{Rath2009}} Erkl\"arungen zu \textbf{quantitativen} Risikoanalysen anhand zweier Beispiele. \hfill \newline

\textit{Alles nur zitiert!} \hfill \newline

Im internationalen Anlagenbau wird in zunehmendem Ma\ss{}e die Durchf\"uhrung einer quantitativen Risikoanalyse gefordert. Die Methodik kann nicht nur zum Nachweis der Einhaltung \"ubergeordneter Akzeptanzkriterien dienen, sondern auch als eine qualifizierte Entscheidungsgrundlage z. B. zu Sicherheitsabst\"anden und -barrieren verwendet werden. Dies kann von nicht probabilistischen quantitativen Verfahren (z. B. HAZOP)) nicht geleistet werden. Durch die Identifizierung der Hauptrisikoquellen in der Anlage erm\"oglicht eine quantitative Risikoanalyse (QRA) zudem die Ableitung von Risikominderungsma\ss{}nahmen, deren
Wirksamkeit sich mit Hilfe von Sensitivit\"atsberechnungen analysieren und bewerten l\"asst. \textcolor{red}{Quantitative Sicherheitsanalyse}

\paragraph*{\cite{Sell_2013}}
\textit{Alles nur zitiert!} \hfill \newline
Die Ergebnisse bisheriger vergleichender Kostenanalysen zwischen diskontinuierlich betriebenen chemischen Verfahren und deren mikroverfahrenstechnischem Pendant haben gezeigt, dass letztere nicht pauschal als kosteneffizientere Alternative gesehen werden können. Es bedarf, wie auch bei Batch-Prozessen, einer umfassenden Prozessoptimierung und Effizienzmaximierung. Dann können mikroverfahrenstechnische Prozesse jedoch trotz teilweise höherer Investitionskosten aufgrund geringerer laufender Kosten und einem schnelleren Zugang zum Markt die ökonomisch günstigere Alternative darstellen. \textcolor{red}{Relevanz von Mikroanlagen}

\paragraph*{\cite{Urbas_2012}} \textcolor{red}{\"Ubersichtsbeitrag Modularisierung, offene Forschungsfragen}

\paragraph*{\cite{Vaidhyanathan1995}} Spezielles Verfahren zur HAZOP

\paragraph*{\cite{Wachsen_2015}} Relevanz der Spezialchemie und deren erwartetes Wachstum. Abschnitt 5 fokussiert auf die Rolle modularer Anlagen. 

\paragraph*{\cite{Wassilew2017}}

\paragraph*{\cite{Wassilew2016}}
Zur Darstellung von Modulen wird ein MTP erstellt. Dies entspricht dem in NAMUR vereinbarten Anforderungen? Der Informationsgehalt des MTP kann automatisiert in das Format von OPC UA transformiert werden. Wie das geht, steht in dieser Arbeit. 