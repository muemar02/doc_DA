\chapter{Zusammenfassung und Ausblick}\label{ch:zf}
Das Ziel dieser Arbeit ist es Algorithmen zu untersuchen, zu bewerten und gegebenenfalls zu implementieren, die zur automatisierten Untersuchung der Fehlerfortpflanzung in modularen Anlagen genutzt werden k\"onnen. \newline
Zur Motivation dieser Aufgabe wird im Kapitel \ref{ch:einleitung} die Rolle der Chemischen und Pharmazeutischen Industrie f\"ur die deutsche Wirtschaft dargestellt. Ein schnelleres Wachstum dieser bedeutenden Industrien erfordert eine schnellere Produkt- und Anlagenentwicklung. Um dieses Ziel zu erreichen wird der Einsatz modularer Anlagenkonzepte als ein besonders geeignetes Mittel identifiziert. \newline
Im Kapitel \ref{ch:sdt} werden wichtige Begriffe definiert und ein \"Uberblick zum Forschungsstand modularer Anlagen gegeben. Weiterhin werden die gesetzlichen Rahmenbedingungen f\"ur die Inbetriebnahme verfahrenstechnischer Anlagen dargestellt und die Notwendigkeit von Sicherheitsuntersuchungen f\"ur modulare Anlagen erl\"autert. Als eine konkrete Methode zum Nachweis der Sicherheit einer Anlage wird das Verfahren \ac{hazop} vorgestellt und an einem Beispiel erl\"autert. \newline
Im darauffolgenden Kapitel \ref{ch:fehlerfortpfl} werden Methoden der Fehleridentifikation und Diagnose vorgestellt. Die Verfahren werden in drei Kategorien unterteilt, welche sich vor allem hinsichtlich der ben\"otigten Daten unterscheiden. \newline
Als Datenbasis f\"ur die vorliegende Arbeit sollen die Beschreibungen der Module, die \ac{hazop}"=Studien der Module und die Beschreibung der modularen Gesamtanlage dienen. Die im Abschnitt \ref{sec:fAna_modQual} vorgestellten modellbasierten qualitativen Verfahren nutzen diese Form von Daten, daher werden Methoden dieser Kategorie als geeignet betrachtet, um Fehlerfortpflanzungen in modularen Gesamtanlagen zu beschreiben. Im Abschnitt \ref{sec:fAna_automatHazop} wird dargelegt, wie einige dieser Methoden in Form von automatisierten \acp{hazop} erfolgreich eingesetzt werden, um automatische Fehlerfortpflanzungsanalysen von konventionellen Anlagen durchzuf\"uhren. F\"ur einen Einsatz im Kontext modularer Anlagen sind aber keine Algorithmen bekannt, welche diese Aufgabe erf\"ullen. \newline
Im Kapitel \ref{ch:ffpflMod} wird die Nutzbarkeit modellbasierter qualitativer Methoden gezeigt. Dazu werden die Beschreibung und die \acp{hazop} einer modularen Anlage automatisiert ausgewertet und analysiert. Die in Matlab implementierten Funktionen konstruieren auf Basis der eingelesenen Daten gerichtete Graphen f\"ur die Module und verbinden diese entsprechend dem vorliegenden \ac{pid} zu einem Gesamtgraphen. Auf Basis dieses Gesamtgraphen kann die Fortpflanzung von Abweichungen untersucht und visualisiert werden. Die Fehlerfortpflanzungsanalyse ist auf die Prozessgr\"o\ss{}en \glqq F\"ullstand\grqq { }und \glqq Durchfluss\grqq { }beschr\"ankt. Innerhalb eines Moduls werden Wechselwirkungen korrekt erkannt, modul\"ubergreifende Wechselwirkungen werden mit Einschr\"ankungen ebenfalls erkannt. Die Erkennung und Beherrschung von auftretenden Fehler wird nicht untersucht, da die dazu notwendige Datenbasis nicht vorhanden ist. \newline
Die prinzipielle Verwendbarkeit von qualitativen modellbasierten Verfahren zur automatisierten Untersuchung der Fehlerfortpflanzung in modularen Anlagen wird damit erfolgreich gezeigt. Es verbleiben die \begin{itemize}
\item Wahl eines geeigneten Verfahrens zur Modellierung von Fehlerfortpflanzungen auf Basis einer \ac{hazop},
\item die Festlegung oder Entwicklung eines Informationsmodells und
\item die Weiterentwicklung eines Koppelmoduls f\"ur modulare Anlagen
\end{itemize} als zu l\"osende Aufgaben.

Im Kapitel \ref{ch:fehlerfortpfl} werden mehrere qualitative Verfahren benannt, die aufgef\"uhrte Liste ist aber nicht vollst\"andig. Es ist zu untersuchen, welche Verfahren besonders gut geeignet sind, um die in einer \ac{hazop} verf\"ugbaren Informationen zu modellieren. Die Wechselwirkungen zwischen Prozessgr\"o\ss{}en k\"onnen zwar durch gerichtete Graphen dargestellt werden, die notwendigen Informationen zu Abweichungen, Ursachen, Auswirkungen, Fehlererkennung und Fehlerbeherrschung k\"onnen aber nicht direkt modelliert werden. Ein alternativer Ansatz ist die Verwendung von Zustandsgraphen, welcher in \cite{Graf_2000, Graf_2000a} von \citeauthor{Graf_2000} untersucht wird. \citeauthor{Graf_2000} zeigt die Verwendbarkeit dieser Methode, benennt aber den gro\ss{}en Modellierungsaufwand als ma\ss{}geblichen Nachteil dieser Methode. Die Verwendbarkeit alternativer Methoden ist zu untersuchen. \newline
Die Darstellung der Daten, welche automatisiert zur Fehlerfortpflanzungsanalyse ausgewertet werden sollen, ist von hoher Relevanz f\"ur die Implementierung einer geeigneten Algorithmus. Die Verwendung einer Ontologie ist empfehlenswert, um dieses Problem zu l\"osen. Weiterhin w\"urde der Einsatz einer Ontologie eine Kopplung mit existierenden Softwares vereinfachen. Die Verwendbarkeit existierender Ontologien f\"ur diese Problemstellung ist daher zu bewerten. \newline
Die Modellierung der Modulkopplung ist ein besonders wichtiger Schritt, wenn die Abweichungen von Prozessgr\"o\ss{}en \"uber Modulgrenzen hinweg untersucht werden sollen. Es ist daher notwendig die rudiment\"are Modellierung der Kopplung als eine Rohrleitung, welche ausschlie\ss{}lich den Durchfluss beeinflusst, zu erweitern.

Zusammengefasst lautet das Ergebnis dieser Arbeit, dass keine Algorithmen existieren, um eine automatisierte Untersuchung der Fehlerfortpflanzung in modularen Anlagen durchzuf\"uhren. Die Anwendung von gerichteten Graphen als qualitative Methode der Informationsdarstellung und Fehleranalyse wird als m\"ogliche L\"osung skizziert. Es besteht aber Forschungsbedarf, um diese Methode weiterzuentwickeln oder durch eine besser geeignete zu ersetzen. 