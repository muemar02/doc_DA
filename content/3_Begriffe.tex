\chapter{Wichtige Begriffe}
\paragraph*{Package Unit}
aus Wikipedia: \hfill \newline
Eine Package Unit (aus dem Englischen package und unit entlehnt; wörtlich Paketeinheit[1][2] oder [ab]gepackte sinngemäß auch abgegrenzte Einheit ist eine von einem Fremdunternehmen geplante und gefertigte Anlage. Die Anforderungen und Voraussetzungen für eine Package Unit sind in einem Lastenheft genannt. Spezielle Anforderungen an eine Package unit sind z. B. Leistungsparameter, Abmessungen und der Steuerungsumfang.
\paragraph*{SIF}
Safety Integrated Function: Ein Zusammenschluss von Komponenten um das Risiko durch eine bestimmte Gefahrenquelle (Hazard) zu reduzieren. 
\paragraph*{SIL}
Der Safety Integrity Level bzw. Sicherheitsintegritätslevel, kurz SIL, ist eine Ma\ss{}einheit zur Quantifizierung von Risikoreduzierung im Bereich von 1 bis 4. Je gr\"o\ss{}er die Zahl ist, desto mehr muss ein erkanntes Risiko reduziert werden. 
\paragraph*{IPL}
An independent protection layer (IPL) is a device, system, or action that is capable
of preventing a scenario from proceeding to its undesired consequence independent
of the initiating event or the action of any other layer of protection associated with
the scenario.
\paragraph*{QRA} Quantitative Risk Analysis: 
\paragraph*{IEs} initiating events: Zu einem risikobehafteten Zustand f\"uhrende Ursachen/ Ereignisse

\paragraph*{PHA} Process Hazard Analysis: Untersuchung von Prozessrisiken. 