\chapter{Stand der Technik} \label{ch:sdt}
\section{Gesetzliche Rahmenbedingungen zur Genehmigung von Chemischen Anlagen}
\subsection{Weltweite Regeln}
\begin{itemize}
\item \ac{ghs} \textcolor{green}{Einstufung von gef\"ahrlichen Stoffen} -> in Europa durch \ac{clp} umgesetzt, Bezug zur \ac{bimschg}
\end{itemize}
\subsection{Europ\"aische Richtlinien und Gesetze}
\begin{itemize}
\item \ac{seveso3}
\item \ac{mrl}
\item \ac{ied}
\item \ac{dgrl}
\end{itemize}
\subsection{Deutsche Gesetze, Verordnungen, Technische Regeln und Richtlinien}
\begin{itemize}
\item Arbeitsschutzgesetz
\item Produktsicherheitsgesetz
\item \ac{bimschg}
\item \ac{bimschv}
\item \ac{bimschv4}
\item \ac{bimschv12}
\item \ac{uvpg} -> Pr\"ufung, ob Anlage gef\"ahrlich f\"ur Umwelt sein k\"onnte
\item \ac{betrsichv}
\item \ac{gefstoffv} -> Explosionsschutz, Gefahrenpotential von Arbeiten, 
\item \ac{tras}
\item \ac{trbs}
\item \ac{trgs}
\item \ac{trws}
\item \ac{trba}
\end{itemize}
\section{Risikoanalysemethoden}
%ganz wichtig: siehe \cite[S. 4]{Fleischer_2015}: Aus dem Stand der Technik sind zahlreiche industriell etablierten Methoden der Risikoanalyse f\"ur die Anlagen der
%Prozessindustrie bekannt. Eine Auff\"uhrung unterschiedlicher Methoden findet sich im Buch von Preiss [14], im DGQ-Band 17-10-Zuverl\"assigkeitsmanagement [15] oder
%in den Anh\"angen der DIN EN 60300-3-1 [16] sowie der
%VDI 4003 Zuverl\"assigkeitsmanagement [17].
\subsection{Quantitative Methoden}
\begin{itemize}
\item Was macht quantitative Methode aus
\item Welche Methoden gibt es
\item Vor/ Nachteile der Methoden
\item Verweis auf Forschung zur Methode, Verbreitung usw.
\end{itemize}
\subsection{Qualitative Methoden}
\begin{itemize}
\item Was macht qualitative Methode aus
\item Welche Methoden gibt es
\item Vor/ Nachteile der Methoden
\item Verweis auf Forschung zur Methode, Verbreitung usw.
\end{itemize}
  \subsubsection{\ac{hazop}}
  \begin{itemize}
  \item notwendige Schritte
  \item Ans\"atze der Automatisierung
  \end{itemize}