\chapter{Stand der Technik} \label{ch:sdt}
\section{Definition wichtiger Begriffe}
Da Sicherheit ein menschliches Grundbed\"urfnis ist, wird dieses Thema im Alltag vielfach diskutiert. Um Missverst\"andnisse zu vermeiden, werden im Folgenden einige Begriffe definiert, welche in diesem Zusammenhang h\"aufig nicht entsprechend ihrer genormten Bedeutung verwendet werden. 
\begin{defn}[Risiko] Als Risiko bezeichnet man die \glqq Kombination der Wahrscheinlichkeit des Auftretens eines Schadens und des Schweregrades dieses Schadens\grqq { }{(vgl. \cite[Abs. 3.2.64]{din61511_1} }).
\end{defn}
\begin{defn}[Schaden] Als Schaden bezeichnet man eine \glqq physische Verletzung oder Sch\"adigung der Gesundheit von Menschen, entweder direkt oder indirekt als ein Ergebnis von Sch\"aden an Eigentum oder an der Umwelt\grqq { }{(vgl. \cite[Abs. 3.2.20]{din61511_1} }).
\end{defn}
\begin{defn}[Gef\"ahrdung] Eine Gef\"ahrdung ist eine \glqq potentielle Schadensquelle\grqq { }{(vgl. \cite[Abs. 3.2.21]{din61511_1} }).
\end{defn}
\begin{defn}[Sicherheit] Sicherheit liegt vor, wenn die \glqq Freiheit von unvertretbaren Risiken\grqq { }gesichert ist {(vgl. \cite[Abs. 3.2.67]{din61511_1} }).
\end{defn}
\begin{defn}[Fehler] Ein Fehler ist ein \glqq anormaler Zustand, der eine Verminderung oder den Verlust der F\"ahigkeit einer Funktionseinheit verursachen kann, eine geforderte Funktion auszuf\"uhren\grqq { } {(vgl. \cite[Abs. 3.2.21]{din61511_1} }).
\end{defn}
\begin{defn}[Ausfall] Ein Ausfall ist die \glqq Beendigung der F\"ahigkeit einer Funktionseinheit, eine geforderte Funktion auszuf\"uhren\grqq { }{(vgl. \cite[Abs. 3.2.20]{din61511_1} }).
\end{defn}

\section{Modularisierung}\label{sec:sdt_modularisierung}
Im Kapitel \ref{ch:einleitung} wird erl\"autert, dass die Zukunftsf\"ahigkeit der Chemischen und Pharmazeutischen Industrie von Flexibilit\"at und Geschwindigkeit der Entwicklung und Herstellung von Produkten ma\ss{}geblich abh\"angig ist. Als geeignete Mittel um diese Ziele zu erreichen gelten der Einsatz von Mikro- und Millianlagen, sowie die Verwendung von modularen Komponenten. Diese beiden Planungsans\"atze sind eng miteinander verbunden und werden im folgenden Abschnitt beschrieben.  

Die Entwicklung eines neuen Produktes besteht \"ublicherweise aus vier Phasen. Zu Beginn wird die Herstellbarkeit eines neuen Produktes in einer Laboranlage in Form eines Batchprozesses untersucht. Darauf erfolgt der Entwurf einer Minianlage. Die Produktionsmengen werden dann durch Konstruktion einer Pilotanlage erh\"oht und der Produktionsprozess umfassend getestet und verfeinert. Auf Basis der Pilotanlage wird zuletzt die Produktionsanlage geplant und das neue Produkt in industriellen Mengen gefertigt. Je nach prognostiziertem Verkaufsvolumen des entwickelten Produktes geschieht die industrielle Produktion als kontinuierlicher oder Batchprozess. \cite{Grundemann_2012}

Dieser  Entwicklungsprozess ist stark beschleunigbar, wenn die Produktionsmengen der Laboranlage ohne gro\ss{}en Konstruktionsaufwand hoch skaliert werden k\"onnen. Die Konstruktion einer Pilotanlage ist dann nicht notwendig.

Das Hochskalieren eines entwickelten Batchprozesses ist kompliziert und nicht uneingeschr\"ankt m\"oglich. Dies betrifft insbesondere stark exotherme Reaktionen, da die W\"armeabfuhr in einem Reaktor begrenzt ist. \cite{Brodhagen_2012} \newline
Durch den Einsatz von Mikro- und Millireaktoren wird das Hochskalieren von Prozessen jedoch erm\"oglicht und der Entwicklungsprozess somit stark beschleunigt. \cite{Grundemann_2012, Brodhagen_2012, Helling_2012, Kockmann_2012, Hessel_2012}

Solche Reaktoren sind durch einen kontinuierlichen Betrieb, Str\"omungskan\"ale mit Durchmessern im Mikro- bis Millimeterbereich und W\"armeaustauschfl\"achen pro Volumeneinheit, welche im Vergleich zu klassischen Anlagen etwa um den Faktor 100 h\"oher sind, gekennzeichnet. Dies erm\"oglicht eine hohe Energieableitung, was inh\"arente Sicherheit zur Folge hat und Scale-Up Faktoren von zehntausend und dar\"uber erm\"oglicht. \cite{Brodhagen_2012, Kockmann_2012a,Behr_2012}

Der Einsatz dieser Reaktoren zur Produktentwicklung erfordert die Umwandlung des Batchprozesses der Laboranlage in einen kontinuierlichen Prozess. Diese Problematik ist seit dem Durchbruch der Mikroprozesse zu Beginn der neunziger Jahre bekannt und wurde weitreichend untersucht. \cite{Helling_2012} Die Arbeit \citetitle{Hugo_2009} von \citeauthor{Hugo_2009} widmet sich dieser Problematik. \citeauthor{Hugo_2009} f\"uhren darin aus, dass langsame Reaktionen schlechter als schnelle Reaktionen f\"ur die \"Uberf\"uhrung in einen kontinuierlichen Prozess geeignet sind. Als Grund daf\"ur wird vor allem die lange Verweilzeit angegeben, welche bei langsamen Reaktionen in kontinuierlicher Fahrweise notwendig ist. Eine Erh\"ohung der Prozesstemperatur kann die Verweildauer reduzieren, sie mindert aber gegebenenfalls die Produktqualit\"at und ist daher nicht unbedingt geeignet. Die zu beachtenden Regeln bei der \"Uberf\"uhrung eines Semi-Batchprozesses in einen kontinuierlichen Prozess werden von den Autoren f\"ur  schnell ablaufende, stark exotherme Reaktionen allgemein dargelegt und anhand eines Beispiels konkret angewandt. Der Fokus liegt dabei auf der Dimensionierung der Mikroreaktoren mit dem Ziel einer sicheren Anlagenf\"uhrung.\cite{Hugo_2009}

Die wirtschaftlichen Vorteile und die m\"ogliche Zeitersparnis bei der Produktentwicklung wurden anhand zahlreicher Fallbeispiele erfolgreich nachgewiesen. \cite{Brodhagen_2012, Behr_2012, Grundemann_2012, Sell_2013} Neben Reaktoren sollen auch andere Prozessschritte mittels Mikro- und Millianlagen realisiert werden. Insbesondere die Trennung von Stoffen ist ein aktueller Forschungsschwerpunkt \cite{Helling_2012}. Die aktuelle Arbeit von \citeauthor{Yang_2017} gibt den Kenntnisstand zur Verwendung von Mikroanlagen zur Destillation wieder \cite{Yang_2017}. \citeauthor{Lier_2016} f\"uhren im \"Ubersichtsbeitrag weitere modulare Apparate auf. Diese realisieren die Prozessschritte W\"armeaustausch, Reaktion, Mischen von Stoffen und Stofftrennung. Die vorgestellten Apparate  sind dabei nicht auf Konstruktionen in Mikro- oder Millibauweise beschr\"ankt. \cite{Lier_2016}

Mikro- und Millireaktoren sind ein bew\"ahrtes Mittel der Chemischen und Pharmazeutischen Industrie, um Produkte schneller und kosteng\"unstiger zu entwickeln. Insbesondere die Skalierbarkeit durch Parallelisierung von vielen Reaktoren der gleichen Bauart ist ein gro\ss{}er Vorteil. Diese Wiederverwendbarkeit ist ein wichtiger Aspekt der Tutzing Thesen \cite{Processnet_2009}. Mikro- und Millireaktoren k\"onnen als eine Auspr\"agung modularer Anlagenkomponenten angesehen werden und sind daher geeignete Beispiele, um die erfolgreiche Verwendung von modularen Anlagen zu belegen. Weitere Beispiele stellen die Modularisierung eines  Gasw\"aschers \cite{Ohle_2014} und die Modularisierung einer Anlage zur Hochleistungsfl\"ussigkeitschromatographie \cite{Rottke_2012} dar. \newline
Da Mikro- und Millireaktoren einen Spezialfall modularer Komponenten bilden wird auf ihre Besonderheiten im weiteren Verlauf der Arbeit jedoch nicht weiter eingegangen. Statt dessen liegt der Fokus im weiteren Verlauf dieser Arbeit auf der allgemeinen Verwendung von modularen Anlagenkomponenten und deren Sicherheitsbetrachtungen. \newline
Im Folgenden werden der Forschungsstand zu modularen Anlagen grob wiedergegeben und die offenen Schwerpunkte benannt. Dazu werden \"Ubersichtsarbeiten zu diesem Thema ausgewertet und die Entwicklung der Forschung wiedergegeben, ohne jedoch eine zu ausgepr\"agte Detailtiefe zu erreichen.  

Das Gebiet der Modularisierung von Anlagenkomponenten ist ein verh\"altnism\"a\ss{}ig junges Forschungsgebiet. In Folge der in Kapitel \ref{ch:einleitung} dargelegten hohen Relevanz wird diese Thematik intensiv untersucht. \newline
Ein Nachweis der wirtschaftlichen Sinnhaftigkeit von kleinskaligen Anlagen wurde von \citeauthor{Seifert_2012} in \citetitle{Seifert_2012} erfolgreich erbracht. \cite{Seifert_2012} \newline
Drei Jahre nach Ver\"offentlichung der Tutzing Thesen \cite{Processnet_2009} wurde von zwei der Teilnehmern des 48. Tutzing Symposiums eine grundlegende Arbeit zur Verwendung von Modulen im Planungsprozess einer verfahrenstechnischen Anlage ver\"offentlicht. Die Autoren \citeauthor{Bramsiepe_2012} betrachten in ihrer Arbeit \citetitle{Bramsiepe_2012} \cite{Bramsiepe_2012} Sichtweisen auf die Modularisierung, definieren verschiedene Typen von Modulen, erl\"autern deren Einsatzzweck und -zeitpunkt im Planungsprozess und gehen weiterhin auf offene Forschungsfragen ein. \newline
Als besondere Vorteile der Modularisierung werden die hohe Flexibilit\"at und eine schnelle Anpassung der Produktionskapazit\"at an Marktver\"anderungen genannt. Weitere Vorteile sind die M\"oglichkeit einer r\"aumlichen Trennung der Produktion verschiedener  Zwischenprodukte zur Einsparung von Transportkosten und die M\"oglichkeit, einen Gro\ss{}teil der Anlagenmontage an einem beliebigen Ort unter optimalen Bedingungen vornehmen zu k\"onnen. Am Aufstellungsort der Anlage m\"ussen die Module dann nur noch verbunden werden, was insbesondere bei klimatisch anspruchsvollen Anlagenstandorten sehr vorteilhaft ist. \newline
\citeauthor{Bramsiepe_2012} fordern eine Moduldefinition derart, dass ein Modul einen hohen Grad an Wiederverwendbarkeit besitzt und losgel\"ost von einer Gesamtanlage getestet werden kann. Module sollten au\ss{}erdem nach ihrem Detaillierungsgrad unterschieden werden. Die Aufteilung in Planungsmodule und Variantenmodule wird daher als sinnvoll erachtet. Planungsmodule stellen in erster Linie einen Wissensspeicher dar und dienen der Darstellung der Vielfalt von Variantenmodulen. Sie bieten Ans\"atze zu Auswahl, Funktionsumfang, Auslegung und Dimensionierung von Variantenmodulen. Ein Variantenmodul soll als 2D und als 3D Version entwickelt werden. Ein 2D Variantenmodul soll Informationen enthalten, welche am Ende des Basic Engineering vorhanden sind. Dies umfasst alle Informationen, welche zum Entwurf eines R\&{}I Flie\ss{}bildes f\"ur ein Modul notwendig sind. Ein 3D Variantenmodul ist um Auslegungsgr\"o\ss{}en derart erweitert, dass die Modulfertigung m\"oglich ist, wobei eine genaue Definition der Schnittstellen notwendig ist. \newline
Im Planungsprozess hat der Detaillierungsgrad der verwendeten Variantenmodule ma\ss{}geblichen Einfluss. 2D Module erleichtern die Erzeugung von Flie\ss{}bildern einer Gesamtanlage. Insbesondere erm\"oglichen sie einen direkten Vergleich verschiedener Anlagenstrukturen. Mit Hilfe von Simulationen k\"onnen in Kombination mit Planungsmodulen geeignete 3D Module f\"ur einen Prozess ausgew\"ahlt und die Gesamtanlage entworfen werden. \citeauthor{Bramsiepe_2012} verweisen auf Literatur, in welcher die zur Erlangung von 2D und 3D Modulen notwendigen Arbeitsschritte dargelegt werden. \newline
Bei der Entwicklung von Regelungs- und Sicherheitskonzepten muss betrachtet werden, welche Aufgaben ein einzelnes Modul losgel\"ost vom Gesamtsystem erf\"ullen kann und welche Aufgaben nur im Zusammenspiel mehrerer Module gel\"ost werden k\"onnen. Die implementierten F\"ahigkeiten des Moduls bestimmen also ma\ss{}geblich den Entwicklungsaufwand neuer Sicherheitsfunktionen einer Gesamtanlage. \newline
Um Module verwenden zu k\"onnen, nennen \citeauthor{Bramsiepe_2012} die folgenden Forschungsschwerpunkte:
\begin{itemize}
\item Systematischer Entwurf von 2D, 3D Variantenmodulen und Planungsmodulen, wobei besonders eine Systematik des Modulentwurfs zu definieren ist.
\item Die Verbesserung von Ans\"atzen, wie Module konkret in den Planungsprozess integriert werden k\"onnen.
\item Entwicklung von Berechnungsmodellen zum Scale-Up von Modulen.
\item Erstellung von Simulationsmodellen von Modulen, um deren Variantenauswahl und konkrete Auslegung durchf\"uhren zu k\"onnen.
\item Entwicklung eines Datenmodells, um Datenanreichung und Datenaustausch zu erm\"oglichen.
\end{itemize}

Ein weiterer \"Ubersichtsbeitrag zum Thema Modularisierung stammt von \citeauthor{Urbas_2012} . Die Autoren legen ihren Fokus dabei auf die Prozessf\"uhrung mit Hilfe modularer Anlagenkomponenten. Ein besonderer Schwerpunkt stellt die Zusammenarbeit von internen Komponenten eines Moduls, wie der Automatisierung und implementierten Sicherheitsfunktionen, mit den externen Komponenten, wie dem \"ubergeordneten Prozessleitsystem, dar. \newline
Zum Zeitpunkt der Ver\"offentlichung der Arbeit \citetitle{Urbas_2012} von \citeauthor{Urbas_2012} gab es noch keine einheitliche Definition des Begriffs \glqq Modul\grqq { }. Die Autoren definieren den Begriff Modul in Anlehnung an die Konstruktionslehre als \glqq abgeschlossene und wiederverwendbare Einheiten zur Erf\"ullung einer oder mehrerer Prozessfunktionen, die im Prozessf\"uhrungskontext sinnvoll zusammengefasst werden k\"onnen.\grqq { }\cite[S. 2]{Urbas_2012} Die für die Implementierung der Prozessfunktion notwendigen Equipments, Instrumente und Automatisierungsfunktionen sollen im Modul enthalten sein. Weiterhin ben\"otigt ein Modul klar definierte Schnittstellen zu seiner Umgebung. \cite[S. 2]{Urbas_2012}. Als Umfang eines Moduls wird eine beliebige Ebene zwischen Teilanlage und Einzeloperation eines Prozesses vorgeschlagen. Der Mangel einer klaren Methodik zur Definition von Modulen hat Forschungsbedarf zur Folge. Als besonderer Schwerpunkt wird die von \citeauthor{Bramsiepe_2012} ebenfalls geforderte Entwicklung von gewerke\"ubergreifenden formalen Informationsmodellen empfohlen. \cite{Urbas_2012}

Als dritte \"Ubersichtsarbeit zum Thema Modularisierung soll auf die Arbeit \citetitle{Hady_2012} von \citeauthor{Hady_2012} \cite{Hady_2012} verwiesen werden. Diese Arbeit betrachtet zum einen allgemeine Fragestellungen, wie die Definition von Modulen und deren Abgrenzung zum Baukastenprinzip und zum anderen Aspekte der modularen Anlagenplanung. Weiterhin wird ein Modularisierungskonzept im Detail vorgestellt, welches die Beschreibung und Verwendung von Modulen erm\"oglichen soll. Die Autoren gehen ebenfalls auf die Anwendbarkeit der Modularisierung f\"ur die in diesem Kapitel bereits genannten Minianlagen ein und legen au\ss{}erdem die Auswirkungen der Modularisierung auf die Kostensch\"atzung eines Anlagenbaus dar.\newline
\citeauthor{Hady_2012} bekr\"aftigen die bereits ausgef\"uhrte Notwendigkeit und die Vorteile einer modularen Anlagenplanung. Im Rahmen einer Industriebefragung stellen sie jedoch fest, dass die modulare Anlagenplanung nur in geringem Ma\ss{}e eingesetzt wird. \citeauthor{Hady_2012} erl\"autern den Begriff der Modularisierung und ziehen Parallelen zur Automobilindustrie und den dort ebenfalls etablierten Baukastensystemen. Ein Modul wird in Abgrenzung zu Bausteinen als eine Einheit charakterisiert, welche eine definierte Funktionalit\"at allgemein abdecken soll. Ein Baustein deckt lediglich in Bezug auf das System, dessen Bestandteil er ist, die gew\"unschte Funktionalit\"at ab. \newline
Den Vorteil einer m\"oglichen Vormontage von modularen Anlagen belegen die Autoren anhand mehrere Quellen, weisen aber darauf hin, dass die in diesem Zusammenhang verwendeten Module  eher als Einzelst\"ucke anzusehen sind, da die konzipierten Anlagen zumeist nur in sehr wenige Einzelmodule zerlegt wurden. An dieser Stelle zeigt sich besonders deutlich, dass die Verwendung von Modulen Kosten reduzieren kann, ohne zwangsl\"aufig einen hohen Grad an Wiederverwendbarkeit zur Folge zu haben. Als besonders wichtig gilt daher die bereits genannte systematische Entwicklung von Modulen und deren Darstellung in einer Form, welche Weiterentwicklungen und Austausch beg\"unstigt. \newline
Zur Ablage von bereits entwickelten Modulen schlagen die Autoren die Verwendung einer Bibliothek von Modulen und Dokumenten vor, welche von allen am Anlagenplanungsprozess beteiligten Personen verwendet werden kann. Eine solche Datenbank wurde entwickelt und wird von \citeauthor{Hady_2012} entsprechend referenziert. Die entwickelte Datenbank wurde an der TU Berlin erfolgreich eingesetzt und liefert in Kombination mit dem vorgestellten Vorgehen zum Einsatz von Modulen in der Anlagenplanung einen detaillierten Ansatz f\"ur die Industrie und weitere Forschungsvorhaben. \cite{Hady_2012}

Die Auswahl eines geeigneten Moduls sollte besonders bei gro\ss{}en Datenbanken rechnergest\"utzt erfolgen. \citeauthor{Obst_2013} stellen einen Algorithmus vor, mit Hilfe dessen die Eignung eines Moduls f\"ur einen gegebenen Einsatzzweck bewertet werden kann. \cite{Obst_2013}

Ein vergleichbarer Ansatz zum von \citeauthor{Hady_2012} vorgestellten Vorgehen zur Verwendung von Modulen wurde von \citeauthor{Uzuner_2013} in \cite{Uzuner_2012, Uzuner_2013} erarbeitet. In \citetitle{Uzuner_2013} wird gezeigt, wie die Erstellung eines Rohrleitungs-- und Instrumentenflie\ss{}schemas (Piping and instrumentation diagram, \acused{pid}\ac{pid}) durch Unterteilung einer Gesamtanlage in wiederverwendbare Funktionsgruppen und die Verwendung einer wissensbasierten Software geeignet beschleunigt werden kann. Module sollen laut \citeauthor{Uzuner_2013} derart definiert werden, dass sie prozesstechnisch sinnvoll sind und einen m\"oglichst hohen Grad an Wiederverwendbarkeit aufweisen. Der Autor folgt damit dieser etablierten Anforderung an ein Modul. Ein Modul soll Standard-Prozesseinheiten wie Pumpen, Verdichter, W\"arme\"ubertrager, Beh\"alter, Reaktoren oder Kolonnen umfassen und weiterhin die notwendigen Elemente der Sicherheitstechnik, Regelungstechnik, Nahverrohrung und Instrumentierung enthalten. Die damit verbundene Vereinfachung und Beschleunigung der Planungsarbeit wird aufgezeigt und best"=practice L\"osungen pr\"asentiert.

Eine Weiterentwicklung und Konkretisierung der von \citeauthor{Bramsiepe_2012}, \citeauthor{Uzuner_2012} und \citeauthor{Hady_2012} \cite{Bramsiepe_2012, Uzuner_2012, Hady_2012} vorgestellten modularen Planungsans\"atze findet sich in der Arbeit \citetitle{Fleischer_2016} von \citeauthor{Fleischer_2016} \cite{Fleischer_2016}. Die Autoren st\"utzen sich dabei schwerpunktm\"a\ss{}ig auf das Projekt $\text{F}^{3}$--Factory \cite{f3_2014}. Sie legen ausf\"uhrlich dar, wie mit Hilfe einer Datenbank bestehend aus Modulen und zugeh\"origen Planungsdokumenten wie Berechnungen, Flie\ss{}bildern, Betriebsanleitungen und Apparatelisten w\"ahrend der Planung einer neuen Anlage geeignete Module ausgew\"ahlt werden k\"onnen. Module gelten als geeignet, wenn sie zuvor definierte Prozessparameter und Funktionen direkt erf\"ullen, oder wenn sie durch geringf\"ugige Modifikationen dazu in die Lage versetzt werden k\"onnen. Prozessparameter sind dabei beispielsweise geforderte Durchflussmengen, Dr\"ucke und Temperaturen; als Funktionen gelten Prozessschritte wie Pumpen oder R\"uhren. Die Datenbank dient als wachsender Speicher an Engineeringleistung und bietet f\"ur alle an der Planung beteiligten Personen eine Planungsgrundlage und Wissensablage. 

Der Forschungsschwerpunkt der Simulation ist ein weiterhin intensiv zu untersuchendes Feld. Die Arbeit \citetitle{Oppelt_2015} von \citeauthor{Oppelt_2015} betrachtet die bisherige Verwendung von Simulationen bezogen auf den gesamten Lebenszyklus einer prozessleittechnischen Anlage. Die Autoren kommen zu dem Schluss, dass Simulationen zwar bereits verwendet werden, die Leistungsf\"ahigkeit von bereits vorhandener Software aber nicht ausgenutzt wird. Die Integration in den Lebenszyklus einer Anlage bedarf weiterer Forschung, wobei besonders der Vereinheitlichung von Schnittstellen eine gro\ss{}e Bedeutung beigemessen wird. Eine Bereitstellung von Simulationsmodellen von Komponentenlieferanten wird als sehr n\"utzlich erachtet. \cite{Oppelt_2015} Im Rahmen der Modularisierung k\"onnte genau diese Aufgabe erfolgreich gel\"ost werden. Dazu sind die von \citeauthor{Bramsiepe_2012} in \cite{Bramsiepe_2012} formulierten Arbeiten zur Erstellung von Simulationsmodellen f\"ur Module durchzuf\"uhren. 

Im Bereich der Darstellung von Daten wurden bereits wichtige Fortschritte erzielt. 
F\"ur die Beschreibung von Modulen wurde das \ac{mtp} entwickelt, welches in \cite{Obst_2015} und \cite{Obst_2015a} vorgestellt wird. Es dient als Informationstr\"ager, welcher alle Modulinformationen beinhaltet, die zur Integration eines Moduls ben\"otigt werden \cite[S. 2]{Obst_2015}. Wird dieser Informationstr\"ager erfolgreich verwendet, so kann ein Modullieferant das gesamte Modulengineering durchf\"uhren und der Betreiber mit wenig Aufwand ein geliefertes Modul in seine Anlage integrieren, ohne das Modul selbst detailliert zu kennen. Ein wichtiger Schritt zur Integration eines Moduls ist die Transformation des \ac{mtp} auf ein Modell, welches von der Gesamtanlage genutzt werden kann. Ein solches Informationsmodell kann auf Basis von \ac{opcua} entworfen werden. In \cite{Wassilew_2016} beziehungsweise \cite{Wassilew_2017} zeigen \citeauthor{Wassilew_2016}, wie die in einem \ac{mtp} gespeicherten Modulinformationen in einem \ac{opcua} Gesamtmodell abgebildet werden k\"onnen. Auf einem \ac{opcua} Server k\"onnen die Modulinformation dadurch online durchsucht werden, was eine wichtige Grundlage f\"ur Plug-and-Produce L\"osungen darstellt. 

In der aktuellen Arbeit \citetitle{Lier_2016a} \cite{Lier_2016a} wird erneut die Notwendigkeit modularer Anlagen dargelegt und die bereits erprobten Konzepte der Modularisierung bewertet. Es werden die im Abschnitt \ref{sec:einltg_chemPharmaIndustrie} bereits benannten Projekte $\text{F}^{3}$--Factory \cite{f3_2014} und CoPIRIDE \cite{copiride_2014} sowie der daraus hervorgegangene \glqq Evotrainer\grqq { }beziehungsweise \glqq EcoTrainer\grqq { }\cite{Lang_2012} betrachet. Die Autoren stellen fest, dass mit Ausnahme von Modulen in Containerbauweise der gro\ss{}e Durchbruch der modularen Strategie noch immer nicht erfolgt ist. Weiterhin werden Arbeiten zur Wirtschaftlichkeit von modularen Ans\"atzen ausgewertet (unter anderem \cite{Seifert_2012, Brodhagen_2012}). \citeauthor{Lier_2016a} kommen zu dem Schluss, dass modulare Anlagen in Folge einer deutlich verk\"urzten Amortisationszeit f\"ur wechselnde Marktverh\"altnisse bestens geeignet sind. F\"ur l\"angere Produktlebenszyklen sind derzeit konventionell geplante Anlagen die wirtschaftlich bessere Wahl. Die Ursache daf\"ur sind geringere Betriebskosten und eine auf den Prozess genauer abgestimmte Anlage. Die in \cite{Bramsiepe_2012} geforderten Berechnungsmodelle f\"ur die Skalierung von modularen Anlagen wurden bisher noch nicht entwickelt. Die Arbeiten von \citeauthor{Brodhagen_2012} \cite{Brodhagen_2012} und \citeauthor{Grundemann_2012} \cite{Grundemann_2012} zeigen aber ein m\"ogliches Vorgehen auf. Die Verfasser best\"atigen den von \citeauthor{Urbas_2012} in \cite{Urbas_2012} genannten akuten Bedarf nach Forschungsarbeit zur Automatisierung von Modulen und deren Einbindung in ein \"ubergeordnetes Prozessleitsystem. Zus\"atzlich verweisen sie auf die Notwendigkeit von anpassbaren, modularen Logistikl\"osungen. 

Die ebenfalls aktuelle Arbeit \citetitle{Hohmann_2017} von \citeauthor{Hohmann_2017} \cite{Hohmann_2017} best\"atigt diese Aussagen. Die Autoren geben einen groben \"Uberblick zur Entwicklung der Modularisierung von Anlagen und nennen eine Reihe an Herstellern, welche bereits modulare Anlagen f\"ur die Chemische und Pharmazeutische Industrie  zum Kauf anbieten. \newline
Sie verdeutlichen die Unterschiede von konventioneller Anlagenplanung und modularen Ans\"atzen und werten neben den bereits aufgef\"uhrten Ans\"atzen (\cite{Bramsiepe_2012, Uzuner_2012, Hady_2012,Fleischer_2016}) noch weitere aus. Sie kommen zu dem Schluss, dass die bisher entwickelten Vorgehen jeweils nur einen Spezialfall oder einen Teil einer kompletten Anlagenplanung bezogen auf den gesamten Lebenszyklus betrachten. Die Konzepte sind laut Aussage von \citeauthor{Hohmann_2017} nicht zu einem Gesamtkonzept kombinierbar, da die Begrifflichkeit eines Moduls im Rahmen der Prozessleittechnik noch immer nicht standardisiert ist und weil einheitliche Arten der Informationsdarstellung nicht vorhanden sind oder zumindest nicht verwendet werden. \newline
Die organisatorischen Konsequenzen der Modularisierung auf den Lebenszyklus wurden bereits von \citeauthor{Obst_2013b} in \cite{Obst_2013b} betrachtet. Eine Weiterentwicklung dieser Arbeit wurde durch den Namur Arbeitskreis 1.12 in Form der \citetitle{ne_148} \cite{ne_148} ver\"offentlicht. Darin wird ein prinzipieller Leitfaden f\"ur Anlagenbauer, Lieferanten und Betreiber zu Entwicklung und Einsatz von modularen Anlagen pr\"asentiert. Der Fokus liegt dabei auf der Automatisierung von Modulen und deren Einbindung in \"ubergeordnete Systeme. \newline
\citeauthor{Hohmann_2017} pr\"asentieren einen konkreten Planungsansatz, welcher den gesamten Lebenszyklus einer modular aufgebauten Anlage abdeckt. Da f\"ur die Automatisierung von Modulen noch kein einheitlicher Standard gefunden wurde gehen sie jedoch nicht im Detail auf die in \cite{ne_148} vorgestellten Methoden ein. \newline
Ein Modul wird in diesem Ansatz definiert als \glqq ein w\"ahrend der Planung und Fertigung von modularen Anlagen unver\"anderbares Element, welches eine bestimmte Funktion f\"ur einen Prozess erf\"ullt und welches im Rahmen weiterer Entwicklungen im Rahmen der Prozessindustrie wiederverwendbar ist\grqq { }\cite[S. 2]{Hohmann_2017}. Durch Verwendung einer Blockdarstellung (\glqq block representation frame\grqq { }) werden die Entwicklungsstadien einer Anlage abgebildet. Ein Block enth\"alt dabei Felder f\"ur Informationen, deren Anzahl und Umfang bei Fortschreiten der Planung zunimmt und welche durch Engineeringleistungen gef\"ullt werden. Die Felder beschreiben beispielsweise Kostensch\"atzungen, Massenbilanzen, notwendige Dr\"ucke und Temperaturen, entwickelte Simulationsmodelle und konkrete 3D Layouts.

Zusammenfassend l\"asst sich feststellen, dass die Modularisierung von Anlagen ein intensiv erforschtes Themengebiet darstellt. Der Nachweis der Notwendigkeit von modularen Anlagen und deren erwartete Verwendungsm\"oglichkeiten ist erbracht. Trotz zahlreicher Studien, die gro\ss{}es Entwicklungspotenzial belegen, gibt es weiterhin gro\ss{}en Forschungs- und Entwicklungsbedarf, um eine Steigerung der Akzeptanz und eine umfangreiche Anwendung in der Industrie zu erreichen.

\section{Gesetzliche Rahmenbedingungen zur Genehmigung von Chemischen Anlagen}\label{sdt_gesetze}
Abschnitt \ref{sec:einltg_sicherheitstechnik} erl\"autert, dass in Folge schwerer Unf\"alle in Industrieanlagen eine Vereinheitlichung von Sicherheitsstandards in der \ac{eu} angestrebt wird. Dies geschieht durch europ\"aische Richtlinien, welche von den Mitgliedstaaten in nationales Recht umzusetzen sind.

In diesem Zusammenhang ist vor allem die \ac{seveso3} relevant. Diese wird in Deutschland in den Regelungen des Bundes"=Immissionsschutzgesetzes \acused{bimschg}(\ac{bimschg}), des Umweltvertr\"aglichkeitspr\"ufungsgesetzes \acused{uvpg}(\ac{uvpg}), und in der \acf{bimschv12} umgesetzt. Weitere europ\"aische Richtlinien, welche f\"ur verfahrenstechnische Anlagen zu beachten sind, sind die \ac{mrl}, die \ac{ied} und die \ac{dgrl}. \newline
Zus\"atzlich zu den Richtlinien der \ac{eu} haben sich zahlreiche Nationen auch auf globale Regeln geeinigt. Dazu z\"ahlt beispielsweise das \ac{ghs}, welches europaweit durch die \ac{clp} umgesetzt ist.

Das \ac{bimschg} wird durch zahlreiche Verordnungen erg\"anzt und pr\"azisiert. \newline
F\"ur den Betrieb von verfahrenstechnischen Anlagen ist vor allem die \ac{bimschv4} von Interesse, in der geregelt wird, welche Anlagen einer immissionsschutzrechtlichen Genehmigung bed\"urfen. \newline
Dar\"uber hinaus dient die \ac{bimschv12} der Verh\"utung schwerer Unf\"alle, die durch bestimmte Industriet\"atigkeiten hervorgerufen werden k\"onnten und der Begrenzung der Unfallfolgen für die menschliche Gesundheit und die Umwelt. Weiterhin sind die Bestimmungen der \ac{gefstoffv} sowie der \ac{betrsichv} bei Industrieanlagen einzuhalten.

Die genannten Richtlinien, Gesetze und Verordnungen werden durch Technische Regeln und Leitf\"aden erg\"anzt. Diese definieren den Stand der Technik und geben Empfehlungen f\"ur die Umsetzung von Gesetzen und Verordnungen. Wichtige technische Regeln sind beispielsweise die \ac{tras}, die \ac{trbs}, die \ac{trgs}, die \ac{trws} und die \ac{trba}.

Bei \"Uberschreitung von festgelegten Mengenschwellen f\"ur verschiedene gef\"ahrliche Stoffe sind die Bestimmungen der \ac{bimschv12} von Anlagenbetreibern einzuhalten. Es werden Anlagen mit Betriebsbereichen \glqq der unteren\grqq { }und \glqq der oberen Klasse\grqq { }{(\S 1 Abs. 1 S. 1 \ac{bimschv12})} definiert. Die Einordnung in eine Klasse basiert auf der vorhandenen Menge von gef\"ahrlichen Stoffen, wobei das {\"Uber- oder Unterschreiten} festgelegter Grenzwerte zu einer Klassifikation f\"uhrt {(\S 2 Nr. 1 f. \ac{bimschv12})}. Die gef\"ahrlichen Stoffe und deren Mengenschwellen sind im {Anhang I \ac{bimschv12}} aufgef\"uhrt. Entsprechend {\S 3 Abs. 1 HS. 1 \ac{bimschv12}} hat der Betreiber \glqq $\dots$ die nach Art und Ausma\ss{} der m\"oglichen Gefahren erforderlichen Vorkehrungen zu treffen, um St\"orf\"alle zu verhindern $\dots$\grqq { }. Als St\"orfall gilt nach {\S 2 Nr. 7 \ac{bimschv12}} \glqq ein Ereignis, das unmittelbar oder sp\"ater innerhalb oder au\ss{}erhalb des Betriebsbereichs zu einer ernsten Gefahr oder zu Sachsch\"aden $\dots$ f\"uhrt\grqq { }. Ein \glqq Ereignis\grqq { }ist eine \glqq St\"orung des bestimmungsgem\"a\ss{}en Betriebs in einem Betriebsbereich unter Beteiligung eines oder mehrerer gef\"ahrlicher Stoffe\grqq { }{(\S 2 Nr. 6 \ac{bimschv12})}. Eine \glqq ernste Gefahr\grqq { }ist nach {\S 2 Nr. 8 \ac{bimschv12}} definiert als \glqq eine Gefahr, bei der das Leben von Menschen bedroht wird oder schwerwiegende Gesundheitsbeeintr\"achtigungen von Menschen zu bef\"urchten sind, die Gesundheit einer gro\ss{}en Zahl von Menschen beeintr\"achtigt werden kann oder die Umwelt, insbesondere Tiere und Pflanzen, der Boden, das Wasser, die Atmosph\"are sowie Kultur oder sonstige Sachg\"uter gesch\"adigt werden k\"onnen, falls durch eine Ver\"anderung ihres Bestandes oder ihrer Nutzbarkeit das Gemeinwohl beeintr\"achtigt w\"urde.\grqq { }Der Anlagenbetreiber, dessen gef\"ahrliche Stoffe die Mengenschwellen der Spalte 5 des Anhangs I der \ac{bimschv12} \"uberschreiten und der damit eine Anlage der \glqq oberen Klasse\grqq { }betreibt, hat dar\"uber hinaus die erweiterten Pflichten der \ac{bimschv12} zu erf\"ullen. Dazu z\"ahlt unter anderem das Verfassen eines Sicherheitsberichtes {(\S 9 Abs. 1 HS. 1 \ac{bimschv12})}, in dem dargelegt wird, dass \glqq die Gefahren von St\"orf\"allen und m\"ogliche St\"orfallszenarien ermittelt, sowie alle erforderlichen Ma\ss{}nahmen zur Verhinderung derartiger St\"orf\"alle und zur Begrenzung ihrer Auswirkungen auf die menschliche Gesundheit und die Umwelt ergriffen wurden\grqq { }{(\S 9 Abs. 1 Nr. 2 \ac{bimschv12})}. Der erstellte Sicherheitsbericht muss eine \glqq Beschreibung der Szenarien m\"oglicher St\"orf\"alle nebst ihrer Wahrscheinlichkeit oder den
Bedingungen für ihr Eintreten $\dots$\grqq { }{(Anhang II Abschnitt IV S. 1 \ac{bimschv12})} und eine \glqq Absch\"atzung des Ausma\ss{}es und der Schwere der Folgen der ermittelten St\"orf\"alle $\dots$\grqq { }{(Anhang II Abschnitt IV Nr. 2 \ac{bimschv12})} enthalten. In Folge dieser Formulierung k\"onnen deterministische oder probabilistische Methoden zum Einsatz kommen. Eine Reihe akzeptierter Methoden wird in \citetitle{2009_bimschv12Hilfe} \cite[S. 20 f.]{2009_bimschv12Hilfe} aufgef\"uhrt. Akzeptierte Methoden sind unter anderem die \ac{fmea}, die \ac{fta} und die \ac{hazop}. Eine umfangreiche Literatur\"ubersicht zur \ac{fta} ist in \cite{Baig_2013} zu finden \cite{Baig_2013}. Die konkrete Anwendung zahlreicher Methoden wird in \citetitle{Nolan_2014} \cite{Nolan_2014} und den Richtlinien des \MakeUppercase{Center for Chemical Process Safety} \cite{ChemicalProcessSafety_2007, ChemicalProcessSafety_2007a, ChemicalProcessSafety_2008, ChemicalProcessSafety_2008a, ChemicalProcessSafety_2008b,ChemicalProcessSafety_2009, ChemicalProcessSafety_2009a, ChemicalProcessSafety_2010, ChemicalProcessSafety_2012, ChemicalProcessSafety_2013, ChemicalProcessSafety_2015} erl\"autert. Auf die besonders weit verbreitete Methode \ac{hazop} wird im folgenden Abschnitt \ref{sec:sdt_hazop} weiter eingegangen. 

\section{Sicherheitsuntersuchungen in Form einer HAZOP}\label{sec:sdt_hazop}
Die Anwendung eines Verfahrens zur Prognose, Auffinden der Ursache, Absch\"atzen der Auswirkungen, Gegenma\ss{}nahmen (PAAG) (engl. Hazard and Operability Analysis \acused{hazop}(\ac{hazop})) ist ein bew\"ahrtes Mittel, um die in der \ac{bimschv12} geforderte Gefahrenanalyse durchzuf\"uhren. Die Methode wurde 1973 erstmals von \citeauthor{Lawley_1974} auf dem AICHE Loss Prevention Symposium \"offentlich vorgestellt und ein Jahr sp\"ater publiziert \cite{Lawley_1974}. Seit dem wurde die Methode vielfach angewandt und weiterentwickelt. \newline
\citeauthor{Kletz_1997} gibt in \citetitle{Kletz_1997} einen kurzen historischen \"Uberblick \"uber die Entstehung und Entwicklung dieser Methode \cite{Kletz_1997}. \citeauthor{Dunjo_2010} arbeiten in \citetitle{Dunjo_2010} \cite{Dunjo_2010} 166 Ver\"offentlichungen zur \ac{hazop} aus dem Zeitraum 1974 bis 2009 auf, fassen die grundlegenden Gedanken der untersuchten Ver\"offentlichungen zusammen, teilen sie in verschiedene Gruppen ein und ermitteln den Stand der Technik. Sie stellen fest, dass die Anzahl der Ver\"offentlichungen zur \ac{hazop} vom Jahr ihrer Vorstellung bis zur zweiten H\"alfte der neunziger Jahre stark zugenommen hat. Der Themenschwerpunkt hat sich dabei stark verschoben. Die ersten Arbeiten erarbeiten m\"ogliche Anwendungsf\"alle der \ac{hazop}, analysieren den abgedeckten Anlagenumfang und versuchen diesen durch Anpassungen der Methode zu erweitern. Der gr\"o\ss{}te Teil der Arbeiten untersucht M\"oglichkeiten eine \ac{hazop} zu automatisieren. Die aktuellsten Arbeiten versuchen die \ac{hazop} mit anderen Methoden wie beispielsweise Simulationen zu kombinieren (siehe beispielsweise \cite{Li_2013}). \citeauthor{Dunjo_2010} kommen zu dem Schluss, dass \ac{hazop} zwar bereits die am meisten untersuchte Methode zur Analyse von Gefahren in Prozessen ist, dass aber weiterhin Verbesserungen notwendig sind. Der Mensch als Gefahrenquelle f\"ur eine Anlage wird noch nicht hinreichend im Rahmen einer \ac{hazop} untersucht, weiterhin wird eine \ac{hazop} in weiten Teilen von Menschen durchgef\"uhrt, was zu Ungenauigkeiten und Fehlern f\"uhren kann. Dar\"uber hinaus sind St\"orungen und Ausf\"alle von speicherprogrammierbaren Steuerungen, welche eine wichtige Rolle bei der Steuerung und Regelung von Anlagen haben, derzeit nur ungen\"ugend durch eine \ac{hazop} abgebildet. Diese Punkte erfordern weitere Forschung.

\section{Durchf\"uhrung einer HAZOP}
Eine ausf\"uhrliche Anleitung zur konkreten Durchf\"uhrung einer \ac{hazop} findet sich beispielsweise in \citetitle{Crawley_2015} \cite{Crawley_2015} und in der Norm \citetitle{din61882} \cite{din61882}. \newline
Eine \ac{hazop} wird von einem interdisziplin\"aren Team in Form einer kreativen Analyse der Anlage durchgef\"uhrt. Das Ziel ist es, m\"ogliche Abweichungen eines Prozesses vom Sollverhalten zu untersuchen. Dazu wird die Gesamtanlage zuerst in kleinere Funktionsgruppen die sogenannten \glqq nodes\grqq { }untergliedert. Die \glqq nodes\grqq { }werden dann nacheinander untersucht. Dazu wird die Sollfunktion beziehungsweise der Sollwert einer betrachteten Variable oder eines Prozesses innerhalb der \glqq node\grqq { }definiert. Anschlie\ss{}end wird daf\"ur eine Reihe an Leitworten wie \glqq {kein/} nicht\grqq { },\glqq mehr\grqq { },\glqq weniger\grqq { },\glqq teilweise\grqq { },\glqq Umkehrung\grqq { },\glqq anders als\grqq { }oder \glqq sowohl als auch\grqq { }ausgew\"ahlt, mit Hilfe derer eine physikalisch sinnvolle Abweichung vom Sollverhalten beschrieben wird. Darauf folgend werden m\"ogliche Ursachen und Konsequenzen der betrachteten Abweichung abgesch\"atzt. Ursachen und Auswirkungen k\"onnen dabei sowohl innerhalb als auch au\ss{}erhalb der betrachteten \glqq node\grqq { }entstehen beziehungsweise wirksam werden. Danach wird das Risiko der ermittelten Auswirkungen unter der Annahme, dass keine Gegenma\ss{}nahmen bestehen, abgesch\"atzt. Im Anschluss werden die vorhandenen Gegenma\ss{}nahmen ermittelt und bewertet. Dazu wird das sich ergebende Restrisiko ermittelt, welches bei Vorhandensein der Gegenma\ss{}nahmen zu erwarten ist. Sind ungen\"ugende Schutzeinrichtungen zur Einhaltung des tolerierten Risikos vorgesehen, so soll durch das \ac{hazop}--Team eine zur Senkung des Risikos geeignete Ma\ss{}nahme vorgeschlagen werden. \newline
Die Ergebnisse einer \ac{hazop} k\"onnen in Form von Tabellen dargestellt werden. In \tabref{tab:hazopBsp} ist ein Auszug der Analyse eines Moduls zur Vorlage eines Stoffes mit Zwischenspeicher dargestellt. Das \ac{pid} dieses Moduls ist im Anhang in \figureref{fig:PIDMod1} dargestellt. Dargestellt wird eine Betrachtung der Prozessvariablen Zufluss, Abfluss und F\"ullstand. F\"ur jede Variable wird ein Leitwort angewendet, mit Hilfe dessen die Auswirkung der Abweichung auf den \"ubrigen Prozess und das Modul selbst ermittelt wird. Weiterhin werden m\"ogliche Ursachen f\"ur die betrachten Abweichungen identifiziert. Die Ursache \glqq externe Ursache\grqq { }stellt eine Besonderheit dar. Sie ist bei konventionellen \acp{hazop} un\"ublich, bei Betrachtung von modularen Anlagen wird damit eine m\"ogliche Wechselwirkung mit angeschlossenen Modulen betrachtet. Da die gekoppelten Module zum Durchf\"uhrungszeitpunkt der \ac{hazop} des einzelnen Moduls nicht bekannt sind, kann die Ursache nicht im Detail beschrieben werden. Statt dessen wird die M\"oglichkeit einer Wechselwirkung besonders hervorgehoben. Bei der Sicherheitsbetrachtung der Gesamtanlage sind solche Ursachen erneut zu analysieren. 

\begin{table}[htb]
\tablestyle
\caption[Beispiel f\"ur eine HAZOP]{Auszug der Ergebnisse einer \ac{hazop} f\"ur ein Vorlagemodul mit Zwischenspeicher nach \cite{Pfeffer_2017}}
\begin{tabularx}{\textwidth}{cccCC}
\tableheadcolor
   {\tablehead ID} &
   {\tablehead Gr\"o\ss{}e} &
   {\tablehead Leitwort} &
   {\tablehead Ursache}&
   {\tablehead Auswirkung}
   \tabularnewline
%
\tablebody
  1 & Zufluss & weniger & Ventil defekt ODER Sensor defekt ODER Ansteuerung defekt & F\"ullstand sinkt \\ \hline 
  2 & F\"ullstand & mehr & F\"ullstandssensor defekt ODER ID=3 & Druckanstieg im Tank \\ 
  \hline 
  3 & Abfluss & kein & Fehlfunktion der Pumpe & F\"ullstand steigt 
   \tabularnewline
%
\tableend
\end{tabularx}
\label{tab:hazopBsp}
\end{table}

%evtl wichtig: siehe \cite[S. 4]{Fleischer_2015}: Aus dem Stand der Technik sind zahlreiche industriell etablierten Methoden der Risikoanalyse f\"ur die Anlagen der
%Prozessindustrie bekannt. Eine Auff\"uhrung unterschiedlicher Methoden findet sich im Buch von Preiss [14], im DGQ-Band 17-10-Zuverl\"assigkeitsmanagement [15] oder
%in den Anh\"angen der DIN EN 60300-3-1 [16] sowie der
%VDI 4003 Zuverl\"assigkeitsmanagement [17].

\section{Sicherheitsbetrachtungen modularer Anlagen}
Modulare Anlagen unterliegen ebenso wie konventionelle Anlagen den im Abschnitt \ref{sdt_gesetze} dargelegten Gesetzen. Daraus ergibt sich die Notwendigkeit einer Sicherheitsuntersuchung, wenn die im {Anhang I \ac{bimschv12}} definierten Mengenschwellen an gef\"ahrlichen Stoffen \"uberschritten werden und eine modulare Anlage damit als eine Anlage \glqq der oberen Klasse\grqq { }nach {\S 1 Abs. 1 S. 1 \ac{bimschv12}} einzustufen ist. \newline
Die Notwendigkeit der Anwendung der \ac{mrl} und damit die Erstellung einer \ac{eg}"=Konformit\"atserkl\"arung f\"ur eine verfahrenstechnische Anlage ist allerdings nicht offensichtlich. Die \ac{mrl} ist unter anderem f\"ur \glqq Maschinen\grqq { }{(Art. 1 Nr. 1 lit. a \ac{mrl})} und \glqq unvollst\"andige Maschinen\grqq { }{(Art. 1 Nr. 1 lit. g \ac{mrl})} anzuwenden. Eine geeignete Interpretation dieser in Art. 2 lit. a,g \ac{mrl} definierten Begriffe wird von \citeauthor{Weber_2016} in \citetitle{Weber_2016} gegeben {(vgl. \cite[S. 589 ff.]{Weber_2016})}. \newline
\citeauthor{Weber_2016} beantwortet die Frage, ob eine verfahrenstechnische Anlage eine \glqq Gesamtheit von Maschinen\grqq { }darstellt und damit der \ac{mrl} unterliegt, verneinend. Als Begr\"undung wird angegeben, dass es in der Regel keinen sicherheitstechnischen Zusammenhang entsprechend des \textit{Interpretationspapieres zum Thema \glqq Gesamtheit von Maschinen\grqq { }des Bundesministeriums f\"ur Arbeit und Soziales} gibt, da bei einer sicherheitsgerichteten Abschaltung eines gef\"ahrdeten Anlagenteils h\"aufig nicht zeitgleich andere Anlagenteile abgeschaltet werden {(vgl. \cite[S. 591]{Weber_2016})}. Damit ist in der Regel keine Erstellung einer \ac{eg}"=Konformit\"atserkl\"arung f\"ur eine verfahrenstechnische Anlag notwendig. \newline
F\"ur modulare Anlagen gilt dies nach Meinung von \citeauthor{Weber_2016} jedoch nicht. Der Autor stellt fest, dass Package-unit-Anlagen als eine Gesamtheit von Maschinen anzusehen sind {(vgl. \cite[S. 592]{Weber_2016})}. Dieser Anlagentyp ist eine besondere Auspr\"agung eines Moduls, weswegen davon auszugehen ist, dass diese Aussage auf s\"amtliche modularen Anlagen anzuwenden ist. Dabei wird jedoch nicht deutlich, ob Package-unit-Anlagen den Zusammenschluss mehrerer Module zu einer Gesamtanlage oder einzelne Module umfassen. \citeauthor{Kockmann_2017} benennen in \citetitle{Kockmann_2017} einige Rechtsvorschriften, welche f\"ur modulare Anlagen einzuhalten sind. Die Anwendung der \ac{mrl} bringen sie explizit nur mit einzelnen Equipments und nicht mit der Gesamtanlage in Verbindung {(vgl. \cite[S. 18]{Kockmann_2017})}. Die Notwendigkeit eines Konformit\"atsverfahrens f\"ur modulare Anlagen kann daher nicht abschlie\ss{}end gekl\"art werden. Der Gesetzgeber sollte hier beispielsweise durch Ver\"offentlichung eines aktualisierten Positionspapieres zur Anwendbarkeit der \ac{mrl} dringend Klarheit schaffen. F\"ur einzelne Module ist die Durchf\"uhrung eines Konformit\"atsverfahrens aber empfehlenswert, da dies bei Einhaltung der relevanten Rechtsvorschriften problemlos m\"oglich ist. 

Im Rahmen modularer Konzepte wird teilweise davon ausgegangen, dass Module funktional eigensicher sind und modulinterne Ma\ss{}nahmen eine Fehlerfortpflanzung verhindern. \cite[S. 4]{Urbas_2012a} Ob dieser Zustand erfolgreich erreicht wird, kann aber erst gepr\"uft werden, wenn s\"amtliche Stoff- und Prozessdaten bekannt sind und die Gesamtanlage geplant ist. Daher kann erst nach dem Detailengineering eine Sicherheitsuntersuchung der Gesamtanlage durchgef\"uhrt werden. Im Rahmen dieser ist der Nachweis zu erbringen, dass die Module eigensicher sind. Ist dies nicht der Fall, so sind geeignete Sicherheitsma\ss{}nahmen zu entwickeln, um einen sicheren Betrieb zu gew\"ahrleisten.  

Die Anzahl ver\"offentlichter Forschungsarbeiten, welche sich speziell der Sicherheitsuntersuchung von aus Modulen bestehenden Anlagen widmen, ist sehr gering. Im Abschnitt \ref{sec:einltg_sicherheitstechnik} wird die Arbeit von \citeauthor{Fleischer_2015} \cite{Fleischer_2015} zusammengefasst. Die Autoren schlagen darin vor, dass die Sicherheit einzelner Module durch Anwendung von Checklisten und Heuristiken durchgef\"uhrt wird. F\"ur die Untersuchung der modularen Anlage wird die Durchf\"uhrung einer \ac{hazop} empfohlen. Wie das bei der Sicherheitsuntersuchung der Module gewonnene Wissen bei der Durchf\"uhrung der \ac{hazop} genutzt werden kann, bleibt jedoch offen. \newline
Die aktuelle Arbeit \citetitle{Kockmann_2017} von \citeauthor{Kockmann_2017} \cite{Kockmann_2017} widmet sich einer \"ahnlichen Fragestellung und stellt eine Erweiterung der Arbeit \citetitle{Fleischer_2015} \cite{Fleischer_2015} dar. Der Fokus von \citeauthor{Kockmann_2017} liegt  speziell auf der Sicherheitsuntersuchung von Mikroreaktoren, welche als Module eingesetzt werden. Die Sicherheitsuntersuchung dieser Module anhand von Checklisten und Heuristiken wird in \cite{Kockmann_2017} detaillierter als in \cite{Fleischer_2015} dargestellt. \citeauthor{Kockmann_2017} kommen ebenfalls zu dem Ergebnis, dass die Durchf\"uhrung einer \ac{hazop} aus den Untersuchungen der Module profitieren kann. Dieses Wissen soll aber lediglich als kreative Anregung bei den durchzuf\"uhrenden Teamsitzungen eingesetzt werden {(vgl. \cite[S. 20]{Kockmann_2017})}. Die M\"oglichkeit einer automatischen Verwendung des bereits gewonnen Wissens wird nicht betrachtet. \newline
Eine teilweise automatisierte Wiederverwendung dieses Wissens ist aber \"au\ss{}erst erstrebenswert, um die Durchf\"uhrung der \ac{hazop} zu beschleunigen und die Fehleranzahl in Folge von \"ubersehenen Wechselwirkungen von Prozessvariablen zu reduzieren. Einen wichtigen Schritt stellt dabei die Analyse der Fortpflanzung von Fehlern in der modularen Anlage dar. Daher wird in der vorliegenden Arbeit \"uberpr\"uft, wie bereits vorhandenes Wissen einzelner Module geeignet bei der Untersuchung von Wechselwirkungen von Prozessvariablen in der modularen Gesamtanlage genutzt werden kann. Als Basis dient dabei die Beschreibung der Gesamtanlage in Form eines \ac{pid}, die Detailbeschreibung der Module und die Ergebnisse der \ac{hazop} jedes Moduls.