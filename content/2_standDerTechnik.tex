\chapter{Stand der Technik}
\section{\"Ubersicht Risikoanalysemethoden}
ganz wichtig: siehe \cite[S. 4]{Fleischer_2015}: Aus dem Stand der Technik sind zahlreiche industriell etablierten Methoden der Risikoanalyse f\"ur die Anlagen der
Prozessindustrie bekannt. Eine Auff\"uhrung unterschiedlicher Methoden findet sich im Buch von Preiss [14], im DGQ-Band 17-10-Zuverl\"assigkeitsmanagement [15] oder
in den Anh\"angen der DIN EN 60300-3-1 [16] sowie der
VDI 4003 Zuverl\"assigkeitsmanagement [17].
\section{Fehlerfortpflanzung in Verfahrenstechnischen Anlagen}
\section{Gesetze und Vorschriften in Deutschland}
Bei den MKPCs (Container - Modul - Anlagen) handelt es sich wie
bei klassischen Anlagen um ein
Arbeitsmittel, das im Sinne des
Arbeitsschutzgesetzes [10] oder des
Produktsicherheitsgesetzes [11] vor
dem Einsatz hinsichtlich der sicheren
Benutzung gepr\"uft werden muss. Sie
unterliegen der Betriebssicherheitsverordnung [12] und den darin hinterlegten Sicherheitsvorschriften und
gesetzlichen Regelungen.
