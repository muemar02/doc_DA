\chapter{Stand der Technik} \label{ch:sdt}
\section{Modularisierung}
Im Kapitel \ref{ch:einleitung} wurde erl\"autert, dass die Zukunftsf\"ahigkeit der Chemischen und Pharmazeutischen Industrie von Flexibilit\"at und Geschwindigkeit bei der Entwicklung und Herstellung von Produkten ma\ss{}geblich abh\"angig ist. Als geeignete Mittel um diese Ziele zu erreichen gelten der Einsatz von Mikro- und Millianlagen sowie die Verwendung von modularen Komponenten. Diese beiden Produktions- beziehungsweise Planungsans\"atze sind eng miteinander verbunden und werden im folgenden Abschnitt beschrieben.  


Die Entwicklung eines neuen Produktes besteht \"ublicherweise aus vier Phasen. Zu Beginn wird die Herstellbarkeit eines neuen Produktes in einer Laboranlage in Form eines Batchprozesses untersucht. Darauf aufbauend wird eine Minianlage entworfen. Die Produktionsmengen werden dann durch Konstruktion einer Pilotanlage erh\"oht und der Produktionsprozess umfassend getestet und verfeinert. Auf Basis der Pilotanlage wird zuletzt die Produktionsanlage geplant und das neue Produkt in industriellen Mengen gefertigt. Je nach prognostiziertem Verkaufsvolumen des entwickelten Produktes geschieht die industrielle Produktion als Kontinuierlicher- oder Batchprozess. \cite{Grundemann_2012}

Der  Entwicklungsprozess kann stark beschleunigt werden, wenn die Produktionsmengen der Laboranlage ohne gro\ss{}en Konstruktionsaufwand hoch skaliert werden k\"onnen. Die Konstruktion einer Pilotanlage ist dann nicht notwendig.

Das Hochskalieren eines entwickelten Batchprozesses ist kompliziert und nicht uneingeschr\"ankt m\"oglich. Dies betrifft insbesondere stark exotherme Reaktionen, da die W\"armeabfuhr in einem Reaktor nicht beliebig gro\ss{} werden kann. \cite{Brodhagen_2012} \linebreak
Durch den Einsatz von Mikro- und Millireaktoren wird das Hochskalieren von Prozessen jedoch erm\"oglicht und der Entwicklungsprozess kann stark beschleunigt werden. \cite{Grundemann_2012}, \cite{Brodhagen_2012}, \cite{Helling_2012}, \cite{Kockmann_2012}, \cite{Hessel_2012}

Solche Reaktoren sind durch einen kontinuierlichen Betrieb, Str\"omungskan\"ale mit Durchmessern im Mikro- bis Millimeterbereich und W\"armeaustauschfl\"achen pro Volumeneinheit, welche circa um den Faktor 100 im Vergleich zu klassischen Anlagen h\"oher sind, gekennzeichnet. Dies erm\"oglicht eine hohe Energieableitung, was eine inh\"arente Sicherheit zur Folge hat. \cite{Brodhagen_2012}, \cite{Kockmann_2012a} In Folge dessen sind Scale-Up Faktoren von zehntausend und dar\"uber m\"oglich. \cite{Behr_2012}

Der Einsatz dieser Reaktoren zur Produktentwicklung erfordert die Umwandlung des Batchprozesses der Laboranlage in einen kontinuierlichen Prozess. Diese Problematik ist seit dem Durchbruch der Mikroprozesse zu Beginn der neunziger Jahre bekannt und wurde weitreichend untersucht. \cite{Helling_2012} Die Arbeit von \citeauthor{Hugo_2009} widmet sich dieser Problematik. \cite{Hugo_2009} 

Die wirtschaftlichen Vorteile und die m\"ogliche Zeitersparnis bei der Produktentwicklung wurden anhand zahlreicher Fallbeispiele erfolgreich nachgewiesen. \cite{Brodhagen_2012}, \cite{Behr_2012}, \cite{Grundemann_2012} Neben Reaktoren sollen auch andere Prozessschritte als Mikro- und Millanlagen realisiert werden. Insbesondere die Trennung von Stoffen ist ein aktueller Forschungsschwerpunkt \cite{Helling_2012}. Die aktuelle Arbeit von \citeauthor{Yang_2017} gibt den Kenntnisstand zur Verwendung von Mikroanlagen zur Destillation wieder \cite{Yang_2017}.

Mikro- und Millireaktoren sind ein bew\"ahrtes Mittel der Chemischen und Pharmazeutischen Industrie, um Produkte schneller und kosteng\"unstiger zu entwickeln. Insbesondere die Skalierbarkeit durch Parallelisierung von vielen Reaktoren der gleichen Bauart ist ein gro\ss{}er Vorteil. Diese Wiederverwendbarkeit ist ein wichtiger Aspekt der Tutzing Thesen \cite{Processnet_2009}. Mikro- und Millireaktoren k\"onnen als eine Auspr\"agung modularer Anlagenkomponenten angesehen werden. Da sie nur einen Spezialfall modularer Komponenten bilden wird auf ihre Besonderheiten im weiteren Verlauf der Arbeit jedoch nicht weiter eingegangen. Statt dessen liegt der Fokus im Folgenden auf der allgemeinen Verwendung von modularen Anlagenkomponenten und deren Sicherheitsbetrachtungen.   

\begin{itemize}
\item Arbeit von Bramsiepe/ Schembecker, \"Ubersichtsbeitr\"age, offene Fragen
\item Arbeiten zur Verwendung von Modulen im Planungsprozess -> notwendige klare Beschreibung von Modulen, automatische Verwendbarkeit
\item Arbeiten zur Beschreibung von Modulen (PU, MTP)
\item Arbeiten zur Verwendung von Modulen (Beispiele)
\item Arbeiten zur Wirtschaftlichkeitsrechnung von Modulen -- nicht Milli- Miniplants
\item fortlaufende Relevanz von Modulen anhand von Quellen
\end{itemize}
\textcolor{red}{Wichtige Papiere sind: \cite{ne_148}}

\section{Sicherheitsuntersuchung in Form einer \ac{hazop}}
\section{Gesetzliche Rahmenbedingungen zur Genehmigung von Chemischen Anlagen}
\subsection{Weltweite Regeln}
\begin{itemize}
\item \ac{ghs} \textcolor{green}{Einstufung von gef\"ahrlichen Stoffen} -> in Europa durch \ac{clp} umgesetzt, Bezug zur \ac{bimschg}
\end{itemize}
\subsection{Europ\"aische Richtlinien und Gesetze}
\begin{itemize}
\item \ac{seveso3}
\item \ac{mrl}
\item \ac{ied}
\item \ac{dgrl}
\end{itemize}
\subsection{Deutsche Gesetze, Verordnungen, Technische Regeln und Richtlinien}
\begin{itemize}
\item Arbeitsschutzgesetz
\item Produktsicherheitsgesetz
\item \ac{bimschg}
\item \ac{bimschv}
\item \ac{bimschv4}
\item \ac{bimschv12}
\item \ac{uvpg} -> Pr\"ufung, ob Anlage gef\"ahrlich f\"ur Umwelt sein k\"onnte
\item \ac{betrsichv}
\item \ac{gefstoffv} -> Explosionsschutz, Gefahrenpotential von Arbeiten, 
\item \ac{tras}
\item \ac{trbs}
\item \ac{trgs}
\item \ac{trws}
\item \ac{trba}
\end{itemize}
\section{Risikoanalysemethoden}
%ganz wichtig: siehe \cite[S. 4]{Fleischer_2015}: Aus dem Stand der Technik sind zahlreiche industriell etablierten Methoden der Risikoanalyse f\"ur die Anlagen der
%Prozessindustrie bekannt. Eine Auff\"uhrung unterschiedlicher Methoden findet sich im Buch von Preiss [14], im DGQ-Band 17-10-Zuverl\"assigkeitsmanagement [15] oder
%in den Anh\"angen der DIN EN 60300-3-1 [16] sowie der
%VDI 4003 Zuverl\"assigkeitsmanagement [17].
\subsection{Quantitative Methoden}
\begin{itemize}
\item Was macht quantitative Methode aus
\item Welche Methoden gibt es
\item Vor/ Nachteile der Methoden
\item Verweis auf Forschung zur Methode, Verbreitung usw.
\end{itemize}
\subsection{Qualitative Methoden}
\begin{itemize}
\item Was macht qualitative Methode aus
\item Welche Methoden gibt es
\item Vor/ Nachteile der Methoden
\item Verweis auf Forschung zur Methode, Verbreitung usw.
\end{itemize}
  \subsubsection{\ac{hazop}}
  \begin{itemize}
  \item notwendige Schritte
  \item Ans\"atze der Automatisierung
  \end{itemize}
