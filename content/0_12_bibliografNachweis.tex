\chapter*{Bibliografischer Nachweis}
\thispagestyle{empty}
%\pagestyle{empty}
Marius M\"uller\\[2ex]
\textbf{Fehlerfortpflanzung in modularen Anlagen} \newline
Diplomarbeit: 70 Seiten, 14 Abbildungen, 4 Tabellen, 122 Literaturangaben\\
Dresden, den 25.08.2017 \\
Technische Universit\"at Dresden \\
Fakult\"at Elektrotechnik und Informationstechnik \\
Professur für Prozessleittechnik\\[2ex]
Autorenreferat:\\
Die Verwendung von modularen Anlagen ist ein geeignetes Mittel, um reduzierte Produktentwicklungszeiten zu erreichen. Die entwickelten modularen Anlagen unterliegen den gleichen Regeln und Gesetzen, wie konventionelle Anlagen. Um die Entwicklungszeit weiter zu reduzieren sind Methoden erforderlich, welche die notwendigen Sicherheitsuntersuchungen modularer Anlagen beschleunigen. In der vorliegenden Diplomarbeit wird daher untersucht, ob es Algorithmen gibt, die zur automatisierten Untersuchung der Fehlerfortpflanzung in modularen Anlagen genutzt werden k\"onnen.

Abstract:\\
The usage of modular plants is a contemporary approach to reduce the necessary time to market. The newly developed modular plants are obligated to fulfill the same safety requirements as conventionally designed plants. To further reduce development periods it is essential to develop methods that support automated safety evaluations. Therefore the purpose of this thesis is to investigate applicable algorithms for automated fault propagation in modular plants. 

