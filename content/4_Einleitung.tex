\chapter{Einleitung}\label{ch:einleitung}
\section{Herausforderungen der deutschen Chemie- und Pharmaindustrie}
Das reale Bruttoinlandsprodukt Deutschlands wuchs von 1995 bis 2013 im Mittel weniger als $2\%$ und wurde damit von den teils zweistelligen Wachstumsraten der Schwellenl\"ander und insbesondere China deutlich \"ubertroffen. Um international erfolgreich zu bleiben, sind die schnell wachsende Pharmazeutische Industrie und die exportlastige Chemische Industrie von besonderer Bedeutung. \hfill \newline

Die Chemische Industrie und die Pharmazeutische Industrie sind Schl\"usselbranchen der deutschen Wirtschaft. Sie exportierten im Jahr 2013 Waren im Wert von \"uber 150 Milliarden \euro. Dies entspricht $15 \%$ der deutschen Gesamtexporte des verarbeitenden Gewerbes. Der Export von pharmazeutischen Erzeugnissen wuchs in den Jahren 1995 bis 2013 j\"ahrlich im Durchschnitt $11,3 \%$ und damit schneller als der jeder anderen Branche. Im gleichen Zeitraum entwickelte sich die Chemische Industrie nur unterdurchschnittlich, im globalen Wettbewerb verlor sie sogar Marktanteile. Als Ursache hierf\"ur wird die besonders hohe Abh\"angigkeit von den in diesem Zeitraum stark gestiegenen Energiepreisen angesehen. \hfill \newline
Eine wichtige Grundlage f\"ur den Erfolg der Chemischen und Pharmazeutischen Industrie ist die best\"andige Weiterentwicklung und Erschaffung innovativer Produkte. Die allein im Jahr 2013 \"uber 7500 in Deutschland neu angemeldeten Patente belegen die bereits aufgebrachte Innovationskraft. Die gr\"o\ss{}ten deutschen Industriezweige Maschinen- und Fahrzeugbau meldeten im gleichen Zeitraum nur circa Dreihundert Patente mehr an. \cite{PerspektiveD_2016} \hfill \newline

Im aktuellen Bericht des Verbands der Chemischen Industrie  untersucht Jan Limbers die Lage und Entwicklungsm\"oglichkeiten der chemisch-pharmazeutischen Industrie und prognostiziert die Entwicklung bis zum Jahr 2030. Durch die schnellere Verbreitung von Technologie und Wissen und den damit verbundenen gesteigerten globalen Wettbewerb wird ein weiter ansteigender Innovationsdruck erwartet. Gelingt es der Branche ihre Forschungsarbeit auf die Bereiche Spezialchemikalien und Pharmazeutika zu fokussieren, so k\"onnen die Standortnachteile, welche durch hohe Energiekosten entstehen, ausgeglichen werden und ein \"uberdurchschnittliches Wachstum ist m\"oglich. Dies erfordert jedoch insbesondere ein insgesamt h\"oheres Innovationstempo. \cite{PerspektiveC_2016} \hfill \newline

Das Innovationstempo ist mit der ben\"otigten \glqq Time to market\grqq { }eines Produktes gleichzusetzen. Darunter versteht man in diesem Zusammenhang den Zeitraum von der ersten Idee zu einem neuen Produkt bis hin zum Beginn der marktreifen Produktion. Die dazu erforderlichen Schritte umfassen die notwendige Forschungsarbeit zur Produkt- und Prozessentwicklung, die Planung und den Bau der Produktionsanlagen. Der Zeitraum nach erfolgter Produktentwicklung umfasst in etwa 5 -- 10 Jahre, wobei davon circa die H\"alte der Zeit auf Anlagenplanung und Konstruktion entfallen. \cite{Schembecker_2009}

Dieser Zeitraum muss reduziert werden, um die von J. Limbers prognostizierte Entwicklung der chemisch-pharmazeutischen Industrie zu erm\"oglichen. Die Notwendigkeit einer schnelleren und vor allem auch flexibleren Produktentwicklung beziehungsweise Produktion ist seit langem bekannt. Die Arbeit von I. E. Grossmann zu den Herausforderungen f\"ur die Forschung im Bereich der Verfahrens- und Anlagentechnik aus dem Jahr 2000 weist beispielsweise auf diese Herausforderungen hin. \cite{Grossmann_2000} \hfill \newline

In Folge dessen wurden bereits zahlreiche Untersuchungen unternommen, wie das Ziel einer beschleunigten Produktentwicklung erreicht werden kann. 

\section{Beschleunigung des Innovationstempo}
Um ein erh\"ohtes Innovationstempo zu erreichen gibt es verschiedene Strategien. Im Folgenden werden drei ver\"offentlichte Ans\"atze vorgestellt.\hfill \newline

Ein Ansatz zur Verbesserung des Produktentwicklungsprozesses wurde \"uber ein Jahrzehnt hinweg an der Universit\"at Clausthal untersucht. Anhand eines neu entworfenen Apparates zur Herstellung von Chlorsilanen aus Ferrosilicium und Chlorwasserstoff wird von P. Dietz and U.  Neumann in \cite{Dietz_2000} gezeigt, wie durch eine fr\"uhzeitige Parallelisierung von Prozessplanung und dem Entwurf der notwendigen Maschinen die Entwicklungszeit verk\"urzt werden kann. Dazu soll die zu l\"osende Aufgabenstellung in Teilsysteme geringer Komplexit\"at so weit zerlegt werden, dass sich deren Funktion durch naturwissenschaftliche Grundoperationen darstellen l\"asst. Durch die so erhaltene Darstellung wird ein Blick f\"ur die m\"ogliche Zusammenfassung von mehreren Teilsystemen in einer einzigen Maschine erm\"oglicht. Eine derart entworfene Maschine kann auf innovative Weise einen Prozess optimal erf\"ullen. Prozessschritte wie Zerkleinern, Reagieren und Mischen k\"onnten beispielsweise in einem Apparat vereint werden. Es wird bei diesem Entwicklungsprozess bewusst auf Standardl\"osungen verzichtet, was die Wiederverwendbarkeit der erhaltenen L\"osungen zumindest erschwert. Das Innovationstempo kann jedoch erfolgreich gesteigert werden und es wird eine hocheffiziente Umsetzung f\"ur einen  Produktionsprozess gefunden.  \hfill \newline

\subsection{Weitere Ans\"atze zum verbesserten Innovationstempo}
Dazu z\"ahlen der Einsatz von Mini- und Millireaktoren und neuerdings der Einsatz von modularen Anlagen.  

\hfill \newline
Um die Planung und den Bau zu Beschleunigen wurden verschiedene Konzepte untersucht. 
\begin{itemize}
\item Milli- und Minireaktoren
\item Modularisierung
\end{itemize}
Als eine M\"oglichkeit der Einsatz von modularen Anlagen identifiziert. Dies erm\"oglicht die Wiederverwendung geleisteter Entwicklungsarbeit und erh\"oht somit das Innovationstempo.    


\begin{itemize}
\item warum ist Modularisierung notwendig
  \begin{itemize}
  \item Ansatz von Mini- und Mikroreaktoren kurz benennen
  \item Ansatz der Modularisierung erl\"autern und auf Arbeiten zum Thema/ prinzpielle Schwerpunkte eingehen
  
  \end{itemize}
\item warum ist Sicherheitstechnik notwendig
\item Sicherheitstechnik als Teil der Modularisierung 
\end{itemize}
\section{Notwendigkeit von Sicherheitstechnik}
\section{Anteil von Sicherheisuntersuchungen am Gesamtplanungsprozess}

\section{Problemstellung dieser Arbeit}