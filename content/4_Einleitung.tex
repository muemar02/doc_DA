\chapter{Einleitung}\label{ch:einleitung}
\section{Problemstellung}
Das reale Bruttoinlandsprodukt Deutschlands wuchs von 1995 bis 2013 im Mittel weniger als $2\%$ und wurde damit von den teils zweistelligen Wachstumsraten der Schwellenl\"ander und insbesondere China deutlich \"ubertroffen. Um international erfolgreich zu bleiben, sind die schnell wachsende Pharmazeutische Industrie und die exportlastige Chemische Industrie von besonderer Bedeutung. \hfill \newline

Die Chemische Industrie und die Pharmazeutische Industrie sind Schl\"usselbranchen der deutschen Wirtschaft. Sie exportierten im Jahr 2013 Waren im Wert von \"uber 150 Milliarden \euro. Dies entspricht $15 \%$ der deutschen Gesamtexporte des verarbeitenden Gewerbes. Der Export von pharmazeutischen Erzeugnissen wuchs in den Jahren 1995 bis 2013 j\"ahrlich im Durchschnitt $11,3 \%$ und damit schneller als der jeder anderen Branche. Im gleichen Zeitraum entwickelte sich die Chemische Industrie nur unterdurchschnittlich, im globalen Wettbewerb verlor sie sogar Marktanteile. Als Ursache hierf\"ur wird die besonders hohe Abh\"angigkeit von den in diesem Zeitraum stark gestiegenen Energiepreisen angesehen. \hfill \newline
Eine wichtige Grundlage f\"ur den Erfolg der Chemischen und Pharmazeutischen Industrie ist die best\"andige Weiterentwicklung und Erschaffung innovativer Produkte. Die allein im Jahr 2013 \"uber 7500 in Deutschland neu angemeldeten Patente belegen die bereits aufgebrachte Innovationskraft. Die gr\"o\ss{}ten deutschen Industriezweige Maschinen- und Fahrzeugbau meldeten im gleichen Zeitraum nur circa Dreihundert Patente mehr an. \cite{PerspektiveD_2016} \hfill \newline

Im aktuellen Bericht des Verbands der Chemischen Industrie  untersucht Jan Limbers die Lage und Entwicklungsm\"oglichkeiten der chemisch-pharmazeutischen Industrie und prognostiziert die Entwicklung bis zum Jahr 2030. Durch die schnellere Verbreitung von Technologie und Wissen und den damit verbundenen gesteigerten globalen Wettbewerb wird ein weiter ansteigender Innovationsdruck erwartet. Gelingt es der Branche ihre Forschungsarbeit auf die Bereiche Spezialchemikalien und Pharmazeutika zu fokussieren, so k\"onnen die Standortnachteile, welche durch hohe Energiekosten entstehen, ausgeglichen werden und ein \"uberdurchschnittliches Wachstum ist m\"oglich. Dies erfordert jedoch insbesondere ein insgesamt h\"oheres Innovationstempo. \cite{PerspektiveC_2016} \hfill \newline

Unter Innovationstempo versteht man in diesem Zusammenhang den Zeitraum von der ersten Idee zu einem neuen Produkt bis hin zum Beginn der marktreifen Produktion. Die dazu erforderlichen Schritte umfassen die notwendige Forschungsarbeit, die Planung und den Bau der Produktionsanlagen. \textcolor{red}{Hier sollte noch ein Verweis darauf kommen, welchen Anteil diese Schritte jeweils haben.} \hfill \newline
Um die Planung und den Bau zu Beschleunigen wurden verschiedene Konzepte untersucht. 
\begin{itemize}
\item Milli- und Minireaktoren
\item \"Anpassungen im Planungsprozess -> Arbeit der Uni Clausthal
\item Modularisierung
\end{itemize}
Als eine M\"oglichkeit der Einsatz von modularen Anlagen identifiziert. Dies erm\"oglicht die Wiederverwendung geleisteter Entwicklungsarbeit und erh\"oht somit das Innovationstempo.    

\textcolor{red}{Hier komm jetzt der \"Ubergang zur Modularisierung.} Die Time-to market ist ca. 5 Jahre \cite{Schembecker_2009}. 

\begin{itemize}
\item Notwendigkeit von Schnellerer Entwicklung: \cite{Grossmann_2000}
\end{itemize}

\begin{itemize}
\item warum ist Modularisierung notwendig
  \begin{itemize}
  \item Ansatz von Mini- und Mikroreaktoren kurz benennen
  \item Ansatz der Modularisierung erl\"autern und auf Arbeiten zum Thema/ prinzpielle Schwerpunkte eingehen
  
  \end{itemize}
\item warum ist Sicherheitstechnik notwendig
\item Sicherheitstechnik als Teil der Modularisierung 
\end{itemize}