\chapter{Fehlerfortpflanzung} \label{ch:fehlerfortpfl}
Die Fortpflanzung von St\"orungen beim Betrieb chemischer Anlagen ist ein intensiv erforschtes Problem. Von Interesse sind Abweichungen von Prozessvariablen von einem definierten Sollwert und Abweichungen des Betriebszustandes von Ger\"aten, Instrumenten und Gewerken vom Optimalzustand und daraus resultierende Auswirkungen auf den Prozess.

Die Abweichung einer Prozessvariable vom Sollwert kann vielf\"altige, verkettete Ursachen haben. Am Beispiel einer stark exothermen Reaktion zweier fl\"ussiger Stoffe A und B soll dies verdeutlicht werden. Die Reaktion sei dabei noch in der Erprobungsphase, weswegen keine Erfahrungswerte zu der Reaktion in der gegebenen Anlage bestehen. \linebreak
Ein Reaktionsbeh\"alter mit R\"uhrer habe zwei \"uber Pumpen gesteuerte Zufl\"usse. Um die gew\"unschte Reaktion zu starten wird der Stoff A eindosiert. Durch langsame Zugabe von Stoff B soll unter Vermischung der Reaktanten die Reaktion gestartet werden. Sei nun der R\"uhrer durch Alterung deutlich langsamer als im optimalen Zustand und au\ss{}erdem der im Reaktor befindliche Temperatursensor defekt. Trotz Zugabe von Stoff B und Einschalten des R\"uhrers findet die Reaktion dann nicht zu dem erwarteten Zeitpunkt statt, da keine ausreichende Vermischung der Stoffe A und B zustande kommt. Eine Ursachenanalyse ist in Folge fehlender Erfahrungswerte sehr kompliziert. Ein m\"ogliches Vorgehen besteht in der vermehrten Zugabe von Stoff B in der Annahme so die Reaktion starten zu k\"onnen. Durch die vermehrte Zugabe und das langsame Vermischen der Reaktanten beginnt die Reaktion. Dies wird durch den defekten Temperatursensor jedoch nicht bemerkt. Der Stoff B wird daher weiter zudosiert. Die stark exotherme Reaktion geht in Folge dessen durch. Dies wird jedoch erst durch einen Druckanstieg im Beh\"alter erkannt, welcher durch Verdampfen der Reaktanten zustande kommt. Aus Sicht der Anlagenfahrer hat die Reaktion jedoch noch immer nicht begonnen, da die Temperatur im Reaktor nicht gestiegen ist. Der Druckanstieg k\"onnte also einem fehlerhaften Drucksensor oder einer nicht erkannten Reaktion geschuldet sein. Die Ursachenanalyse f\"ur den erh\"ohten Druck ist zu diesem Zeitpunkt in Folge einer m\"oglichen Verkettung von Fehlfunktionen sehr kompliziert. Eine durchgehende Reaktion ist hochgradig gef\"ahrlich. Wird die Gefahr nicht unmittelbar erkannt, so kann dies verheerende Folgen f\"ur die Anlage, die Betreiber und die Umwelt haben.

Das Auffinden m\"oglicher Ursachen eines vorliegenden Fehlers ist eine Aufgabe, welche mit Hilfe der Analyse von Fehlerfortpflanzungen gel\"ost werden soll. Weitere Aufgaben sind die Bewertung und Auslegung von Systemen, welche das Betriebsrisiko einer Anlage auf ein gew\"unschtes Level bringen sollen. Die Planung optimaler Wartungsintervalle ist eine Aufgabe, welche direkte auf deren Ergebnissen aufbauen kann. \linebreak
Im Rahmen einer Fehlerfortpflanzungsanalyse wird je nach Verfahren der Einfluss von Prozessgr\"o\ss{}en, die r\"aumliche Positionierung von Anlagenteilen, die Alterung von Anlagenkomponenten oder eine Kombination dieser Faktoren betrachtet. Je nach verwendeten Informationen und Ziel einer solchen Analyse werden die folgenden drei Arten von Fehlerfortpflanzungsanalysen unterschieden:
\begin{enumerate}
\item quantitative Verfahren
\item datenbasierte Verfahren
\item qualitative Verfahren
\end{enumerate}\cite{Zhang_2017}

Auf diese Analysemethoden wird im Folgenden eingegangen. Es werden wichtige Arbeiten zu den jeweiligen Methoden vorgestellt und es wird auf die Eignung f\"ur einen Einsatz in modularen Anlagen eingegangen.

\section{Quantitative Modellbasierte Fehlerfortpflanzungsmethoden}
\section{Datenbasierte Fehlerfortpflanzungsmethoden}
Dies Verfahren beruhen auf der Verwendung von historischen Messdaten konkreter Anlagen. Je nach Verfahren werden Messdaten zum Normalbetrieb oder beziehungsweise und Daten zum St\"orbetrieb ben\"otigt. Manche Methoden ben\"otigen weiterhin ein \ac{pid}. Ziel der Verfahren ist es zum einen St\"orungen der Anlage fr\"uhzeitig zu erkennen und zum anderen deren Ursache oder Ursachen zu ermitteln. Dazu werden Ursache--Effekt Beziehungen zwischen untersuchten Parametern ermittelt.  

Die Auswertung historischer Messdaten basiert zumeist auf statistischen Methoden. Die kausalen Zusammenh\"ange, welche mit Hilfe dieser Methoden ermittelt werden sollen, k\"onnen dann geeignet als Graphen dargestellt werden. Wie man aus statistischen Gr\"o\ss{}en kausale Zusammenh\"ange ermitteln kann wird in den fr\"uhen Werken \textcite{Holland_1986} und  \textcite{Pearl_1995} aufgezeigt. Ein umfassendes Lehrbuch zu dieser Thematik wurde von \citeauthor{Pearl_2009} ver\"offentlicht, welches mittlerweile in der zweiten Auflage verf\"ugbar ist \cite{Pearl_2009}. 
\paragraph*{Nennung von Verfahren}
\cite{Zhang_2017}, \cite{Thornhill_2006}
\section{Qualitative Modellbasierte Fehlerfortpflanzungsmethoden}

\section{\"Ubersicht Fehlerfortpflanzung in chemischen Anlagen}
\section{Vorstellung ausgew\"ahlter Algorithmen}
\section{Bewertung der Verwendbarkeit f\"ur modulare Anlagen}
Wichtige Gesichtspunkte:
\begin{itemize}
\item Welche Daten sind notwendig
\item Automatisierbarkeit der Berechnung
\item Dokumentationsf\"ahigkeit der Ergebnisse
\item Sind die Ergebnisse f\"ur eine \ac{hazop} nutzbar
\end{itemize}