%% Dokumentenklasse (Koma Script) -----------------------------------------
\documentclass[%
   %final,      % fertiges Dokument
	 % --- Paper Settings ---
   paper=a4,%
   paper=portrait, % landscape
   pagesize, % driver
   % --- Base Font Size ---
   fontsize=13bp,%
	 % --- Koma Script Version ---
   version=last, %
 ]{scrreprt} % Classes: scrartcl, scrreprt, scrbook

 %%%% Dokument/PDF Metadaten
\title{Studienarbeit}
\author{Marius M\"uller}

% Encoding der Dateien (sonst funktionieren Umlaute nicht)
% Fuer Linux -> utf8
% Fuer Windows, alte Linux Distributionen -> latin1

% Empfohlen latin1, da einige Pakete mit utf8 Zeichen nicht
% funktionieren, z.B: listings, soul.
%\usepackage[latin1]{inputenc}
%\usepackage[ansinew]{inputenc}
\usepackage[utf8]{inputenc}
\usepackage[T1]{fontenc}
%\usepackage{ucs}
%\usepackage[utf8x]{inputenc}

%%% Preambel
\input{latex_base/thesis/preambel/settings}
\input{latex_base/thesis/preambel/preambel}

%%Sonderzeichen
%\DeclareUnicodeCharacter{014C}{\={D}}
%\DeclareUnicodeCharacter{0304}{\={D}}
%%\usepackage{newunicodechar}
%%\newunicodechar{D̄}{\={D}}

\addbibresource{bib/quellen.bib}
\renewcommand*{\labelalphaothers}{}

\input{latex_base/general/preambel/Fonts}
%
%%%% Neue Befehle
\input{latex_base/thesis/macros/newcommands}
\input{latex_base/general/macros/TableCommands}



%%% zulaessige Worttrennung
\hyphenchar\font=\string"7F
%\hyphenation{Helmut-Schmidt-Uni-ver-si-t\"at}
%Di-graph-based
\hyphenation{Pro-zess-leit-tech-nik}

%% Dokument Beginn
%% Gliederung:
%%		Deckblatt
%%		Inhaltsverzeichnis..................I
%%		Abkürzungsverzeichnis...............II
%%		Verwendete Formelm..................III
%%		Verwendete Indizies.................IV
%%		Abbildungsverzichnis................V
%%		(Tabellenverzeichnis................VI)
%%		1. Einleitung.......................1
%%		2. .................................3
%%		.
%%		.
%%		.
%%		Literatur- und Quellenverzeichnis...56
%%		Erklärung...........................60
%%		(Danksagung.........................61)
%%		Anhang..............................A-1

\begin{document}
% Deckblatt
\pagenumbering{alph} %neue Seitenzahlen
\subject{Diplomarbeit}
% \subject{Diplomarbeit \\ Universität <einfügen>}
% \title{<Titel einfügen>}
% \author{<Autor einfügen>}
% \date{<Datum einfügen>}
% \maketitle

% \begin{titlepage}
% 	\mbox{}\vspace{5\baselineskip}\\
% 	\sffamily\huge
% 	\centering
% 	<Titel einfügen>
% 	\vspace{3\baselineskip}\\
% 	\rmfamily\Large
% 	Diplomarbeit \\ Universität <einfügen>
% 	\vspace{2\baselineskip}\\
% 	\rmfamily\Large
% 	<Autor einfügen>
% 	\vspace{1\baselineskip}\\
% 	<Datum einfügen>
% \end{titlepage}


% \begin{titlepage}
% 	\sffamily\huge
% 	\centering
%	Diplomarbeit
%  	\vspace{3\baselineskip}\\
% 	\rmfamily\huge\bfseries
%	\centering
% 	Hier steht der Titel
% 	\vspace{8\baselineskip}\\
% 	\rmfamily\small
% 	Marius M\"uller \\
% \end{titlepage}

% Name des Verfassers
\author{Marius M\"uller}
\matrikelNr{3661272}
% Geburtsort
\geburtsort{Dresden}
% Geburtsdatum
\geburtsdatum{29. September 1989}
% Titel der Arbeit
\title{Fehlerfortpflanzung in modularen Anlagen}
% Angabe der Betreuer
\betreuer{Dipl.-Ing. Annett Pfeffer (PLT/TUD)}
% Datum der Einreichung
\date{25.08.2017}
% Titelseite erstellen
\maketitlePLT

\chapter*{Bibliografischer Nachweis}
\thispagestyle{empty}
%\pagestyle{empty}
Marius M\"uller\\[2ex]
\textbf{Thema der Diplomarbeit} \newline
Diplomarbeit: Anzahl Seiten, Anzahl Abbildungen, Anzahl Literaturangaben\\
Datum \\
Technische Universit\"at Dresden \\
Fakult\"at Elektrotechnik und Informationstechnik \\
Professur für Prozessleittechnik\\[2ex]
Autorenreferat:\\
Zusammenfassung der Arbeit



\newpage
\thispagestyle{empty}
\null\vfill
\begin{center}
	Bitte ersetzen Sie diese Seite vor dem Binden mit der Aufgabenstellung.
\end{center}
\vfill

%% Selbstständigkeitserklärung
% Ort der Selbstständigkeitserklärung (Standard: Dresden)
\selbstort{Dresden}
% Datum der Selbstständigkeitserklärung (Standard: aktuelles Systemdatum)
\selbstdatum{25. August 2017}
\selbststaendigkeitserklaerung
%% Kurzfassung / Abstract
%\kurzfassung{Die Verwendung von modularen Anlagen ist ein geeignetes Mittel, um reduzierte Produktentwicklungszeiten zu erreichen. Die entwickelten modularen Anlagen unterliegen den gleichen Regeln und Gesetzen, wie konventionelle Anlagen. Um die Entwicklungszeit weiter zu reduzieren sind Methoden notwendig, welche die Sicherheitsuntersuchung modularer Anlagen beschleunigen. Daher wird in der vorliegenden Arbeit untersucht, ob es Algorithmen gibt, die zur automatisierten Untersuchung der Fehlerfortpflanzung in modularen Anlagen genutzt werden k\"onnen.}{English abstract here.}
%\thispagestyle{empty}
%\chapter*{Erklärung der Selbstständigkeit}
%\thispagestyle{empty}
%Hiermit versichere ich, die vorliegende Arbeit selbstständig verfasst und keine anderen als die angegebenen Quellen und Hilfsmittel benutzt sowie die Zitate deutlich kenntlich gemacht zu haben.
%\vspace{4\baselineskip}\\
%<Ort einfügen>, den <Datum einfügen> \hfill <Autor einfügen>
%\vspace{4\baselineskip}\\
%\clearpage
%\mbox{}\thispagestyle{empty}

\cleardoublepage

\frontmatter
\cleardoublepage

% Inhaltsverzeichnis in den PDF-Links eintragen
\pdfbookmark[0]{Inhaltsverzeichnis}{toc}
\tableofcontents 
\cleardoublepage

% Abkuerzungsverzeichnis
\pdfbookmark[0]{Abk\"urzungsverzeichnis}{abk}
\chapter*{Abk\"urzungsverzeichnis}
%\addcontentsline{toc}{chapter}{Abkürzungsverzeichnis}
\begin{acronym}[12. BImSchV] %<--in Klammern das laengste Wort
%\acro{Kuerzel}[Kurzform des Singulars]{Langform des Singulars}
%\acroplural{Kuerzel}[Kurzform des Plurals]{Langform des Plurals}
%\acro{<short>}{<long>\acroforeign{<foreign>}}
  %Bsp: \acro{ECU}{Steuerger\"at \acroforeign{Electonic Control Unit} }

 
%  \acro{opc}[OPC]{Open Platform Communications}
%  \acro{ua}[UA]{Unified Architecture}%
%\acro{hfpm}[HFPM]{hierarchical fault propagation model}
	%\acro{irml}[IRML]{Infrastructure Resilience-Oriented Modeling Language}
	%\acro{dtw}[DTW]{Dynamic time warping}
  \acro{bbn}[BBN]{Bayesian Belief Network}
  \acro{betrsichv}[BetrSichV]{Betriebssicherheitsverordnung}
  \acro{bimschg}[BImSchG]{Bundes--Immissionsschutzgesetz}
	\acro{bimschv}[BImSchV]{Bundes--Immissionsschutzverordnung}
  \acro{bimschv4}[4. BImSchV]{Verordnung \"uber genehmigungsbed\"urftige Anlagen}
	\acro{bimschv12}[12. BImSchV]{St\"orfall--Verordnung}
  \acro{clp}[CLP]{Verordnung Nr. 1272/2008/EG}
  \acro{dae}[DAE]{Differential--algebraische Gleichung \acroforeign{Differential Algebraic Equation}}
	\acroplural{dae}[DAEs]{Differential--algebraische Gleichungen \acroforeign{Differential Algebraic Equations}}
	\acro{dbn}[DBN]{Deep Belief Network}
  \acro{dgrl}[DGRL]{Druckger\"ate--Richtlinie 2014/68/EU}
	\acro{eg}[EG]{Europ\"aische Gemeinschaft} %die fruehere Europaeische Wirtschaftsgemeinschaft
	\acro{eu}[EU]{Europ\"aische Union} %Rechtsnachfolgerin der EG
	\acro{ewr}[EWR]{Europ\"aische Wirtschaftsraum }
	\acro{fdi}[FDI]{Fehlererkennung und Fehlerisolation \acroforeign{fault detection and isolation}}
  \acro{fmea}[FMEA]{Fehlerm\"oglichkeits- und -einflussanalyse \acroforeign{Failure Mode and Effects Analysis}}
	\acro{fta}[FTA]{Fehlerbaumanalyse \acroforeign{Fault Tree Analysis}}
	\acro{gefstoffv}[GefStoffV]{Verordnung zum Schutz vor Gefahrstoffen}
	\acro{ghs}[GHS]{global harmonisierte System zur Einstufung und Kennzeichnung von Chemikalien}
	\acro{hazop}[HAZOP]{Verfahren zur Prognose, Auffinden der Ursache, Absch\"atzen der Auswirkungen, Gegenma\ss{}nahmen (PAAG) \acroforeign{Hazard and Operability Analysis}}
	\acroplural{hazop}[HAZOPs]{Hazard and Operability Analyses}
	\acro{ied}[IED]{Industrieemissionsrichtlinie 2010/75/EU}
	\acro{lqm}[LQM]{Methode der kleinsten Fehlerquadrate \acroforeign{Least Squares Method}}
  \acro{mfm}[MFM]{Multilevel Flow Modeling}
  \acro{mrl}[MRL]{Maschinen--Richtlinie 2006/42/EG}	
  \acro{mtp}[MTP]{Module Type Package}	
  \acro{opcua}[OPC-UA]{Open Platform Communications Unified Architecture}
  \acro{pca}[PCA]{Singul\"arwertzerlegung \acroforeign{Principle component analysis}}
	\acro{pid}[P\&{}ID]{Rohrleitungs-- und Instrumentenflie\ss{}schema \acroforeign{Piping and instrumentation diagram}}	
	\acro{pls}[PLS]{Regression mit partiellen kleinsten Quadraten \acroforeign{Partial Least Squares}}	
	\acro{qsim}[QSIM]{Qualitative Simulation \acroforeign{qualitative simulation}}
	\acro{rbn}[RBN]{Restricted Boltzmann Machine}
	\acro{sdg}[SDG]{Signed Digraph}
	\acroplural{sdg}[SDGs]{Signed Digraphs}
	\acro{seveso3}[SEVESO III]{Richtlinie 2012/18/EU}
	\acro{teb}[TEB]{Tennessee Eastman Benchmark}
	\acro{tras}[TRAS]{Technische Regeln f\"ur Anlagensicherheit}
	\acro{trba}[TRBA]{Technische Regeln f\"ur biologische Arbeitsstoffe}
	\acro{trbs}[TRBS]{Technische Regeln f\"ur Betriebssicherheit}
	\acro{trgs}[TRGS]{Technische Regeln f\"ur Gefahrstoffe}
	\acro{trws}[TRwS]{Technische Regeln f\"ur wassergef\"ahrdende Stoffe}
	\acro{uvpg}[UVPG]{Umweltvertr\"aglichkeitspr\"ufungsgesetz}
\end{acronym}
\cleardoublepage

% Verzeichnis der Formelzeichen
\pdfbookmark[0]{Verzeichnis der verwendeten Formelzeichen}{for}
\chapter*{Verzeichnis der verwendeten Formelzeichen}
%\addcontentsline{toc}{chapter}{Verzeichnis der verwendeten Formelzeichen}
\begin{acronym}[LabVIEW] %<--in Klammern das laengste Wort
	\acro{alpha}[\ensuremath{\alpha}]{ \acrounit{\meter^2\per\second}Temperaturleitf\"ahigkeit}
%	\acro{rho}[\ensuremath{\rho}]{ \acrounit{\kilo\gram\per\meter^3}Dichte}
%	\acro{c}[\ensuremath{c}]{ \acrounit{\joule\per\kilo\gram\per\degreeCelsius}spezifische Wärmekapazität}	
%	\acro{k}[\ensuremath{k}]{ \acrounit{\watt\per\meter\per\degreeCelsius}thermische Leitfähigkeit}	
%	\acro{L}[\ensuremath{L}]{ \acrounit{\joule\per\kilo\gram}latente Wärme}	
\end{acronym}
\cleardoublepage

% Verzeichnis der verwendeten Indizes
\pdfbookmark[0]{Verzeichnis der verwendeten Indizes}{ind}
\chapter*{Verzeichnis der verwendeten Indizes}
%\addcontentsline{toc}{chapter}{Verzeichnis der verwendeten Indizes}
\begin{acronym}[LabVIEW] %<--in Klammern das laengste Wort
\acro{l}{liquid/fl\"ussig}
%	\acro{s}{solid/fest}
\end{acronym}
\cleardoublepage

% Verzeichnis der verwendeten Symbole
\pdfbookmark[0]{Symbolverzeichnis}{sym}
\input{content/0_04_symbole}
\cleardoublepage

% Abbildungsverzeichnis
\listoffigures
\cleardoublepage

% Tabellenverzeichnis
\listoftables
\cleardoublepage

\chapter*{Thesen der Diplomarbeit}
\begin{enumerate}
\item Erste These 
\end{enumerate}
\cleardoublepage

% Hauptteil
\mainmatter
\chapter{Einleitung}
Die Diplomarbeit hat ein Thema und eine Struktur.


\chapter{Stand der Technik} \label{ch:sdt}
\section{Definition wichtiger Begriffe}
Da Sicherheit ein menschliches Grundbed\"urfnis ist, wird dieses Thema im Alltag vielfach diskutiert. Um Missverst\"andnisse zu vermeiden, werden im Folgenden einige Begriffe definiert, welche in diesem Zusammenhang h\"aufig nicht entsprechend ihrer genormten Bedeutung verwendet werden. 
\begin{defn}[Risiko] Als Risiko bezeichnet man die \glqq Kombination der Wahrscheinlichkeit des Auftretens eines Schadens und des Schweregrades dieses Schadens\grqq { }{(vgl. \cite[Abs. 3.2.64]{din61511_1} }).
\end{defn}
\begin{defn}[Schaden] Als Schaden bezeichnet man eine \glqq physische Verletzung oder Sch\"adigung der Gesundheit von Menschen, entweder direkt oder indirekt als ein Ergebnis von Sch\"aden an Eigentum oder an der Umwelt\grqq { }{(vgl. \cite[Abs. 3.2.20]{din61511_1} }).
\end{defn}
\begin{defn}[Gef\"ahrdung] Eine Gef\"ahrdung ist eine \glqq potentielle Schadensquelle\grqq { }{(vgl. \cite[Abs. 3.2.21]{din61511_1} }).
\end{defn}
\begin{defn}[Sicherheit] Sicherheit liegt vor, wenn die \glqq Freiheit von unvertretbaren Risiken\grqq { }gesichert ist {(vgl. \cite[Abs. 3.2.67]{din61511_1} }).
\end{defn}
\begin{defn}[Fehler] Ein Fehler ist ein \glqq anormaler Zustand, der eine Verminderung oder den Verlust der F\"ahigkeit einer Funktionseinheit verursachen kann, eine geforderte Funktion auszuf\"uhren\grqq { } {(vgl. \cite[Abs. 3.2.21]{din61511_1} }).
\end{defn}
\begin{defn}[Ausfall] Ein Ausfall ist die \glqq Beendigung der F\"ahigkeit einer Funktionseinheit, eine geforderte Funktion auszuf\"uhren\grqq { }{(vgl. \cite[Abs. 3.2.20]{din61511_1} }).
\end{defn}

\section{Modularisierung}\label{sec:sdt_modularisierung}
Im Kapitel \ref{ch:einleitung} wird erl\"autert, dass die Zukunftsf\"ahigkeit der Chemischen und Pharmazeutischen Industrie von Flexibilit\"at und Geschwindigkeit der Entwicklung und Herstellung von Produkten ma\ss{}geblich abh\"angig ist. Als geeignete Mittel um diese Ziele zu erreichen gelten der Einsatz von Mikro- und Millianlagen, sowie die Verwendung von modularen Komponenten. Diese beiden Planungsans\"atze sind eng miteinander verbunden und werden im folgenden Abschnitt beschrieben.  

Die Entwicklung eines neuen Produktes besteht \"ublicherweise aus vier Phasen. Zu Beginn wird die Herstellbarkeit eines neuen Produktes in einer Laboranlage in Form eines Batchprozesses untersucht. Darauf erfolgt der Entwurf einer Minianlage. Die Produktionsmengen werden dann durch Konstruktion einer Pilotanlage erh\"oht und der Produktionsprozess umfassend getestet und verfeinert. Auf Basis der Pilotanlage wird zuletzt die Produktionsanlage geplant und das neue Produkt in industriellen Mengen gefertigt. Je nach prognostiziertem Verkaufsvolumen des entwickelten Produktes geschieht die industrielle Produktion als kontinuierlicher oder Batchprozess. \cite{Grundemann_2012}

Dieser  Entwicklungsprozess ist stark beschleunigbar, wenn die Produktionsmengen der Laboranlage ohne gro\ss{}en Konstruktionsaufwand hoch skaliert werden k\"onnen. Die Konstruktion einer Pilotanlage ist dann nicht notwendig.

Das Hochskalieren eines entwickelten Batchprozesses ist kompliziert und nicht uneingeschr\"ankt m\"oglich. Dies betrifft insbesondere stark exotherme Reaktionen, da die W\"armeabfuhr in einem Reaktor begrenzt ist. \cite{Brodhagen_2012} \newline
Durch den Einsatz von Mikro- und Millireaktoren wird das Hochskalieren von Prozessen jedoch erm\"oglicht und der Entwicklungsprozess somit stark beschleunigt. \cite{Grundemann_2012, Brodhagen_2012, Helling_2012, Kockmann_2012, Hessel_2012}

Solche Reaktoren sind durch einen kontinuierlichen Betrieb, Str\"omungskan\"ale mit Durchmessern im Mikro- bis Millimeterbereich und W\"armeaustauschfl\"achen pro Volumeneinheit, welche im Vergleich zu klassischen Anlagen etwa um den Faktor 100 h\"oher sind, gekennzeichnet. Dies erm\"oglicht eine hohe Energieableitung, was inh\"arente Sicherheit zur Folge hat und Scale-Up Faktoren von zehntausend und dar\"uber erm\"oglicht. \cite{Brodhagen_2012, Kockmann_2012a,Behr_2012}

Der Einsatz dieser Reaktoren zur Produktentwicklung erfordert die Umwandlung des Batchprozesses der Laboranlage in einen kontinuierlichen Prozess. Diese Problematik ist seit dem Durchbruch der Mikroprozesse zu Beginn der neunziger Jahre bekannt und wurde weitreichend untersucht. \cite{Helling_2012} Die Arbeit \citetitle{Hugo_2009} von \citeauthor{Hugo_2009} widmet sich dieser Problematik. \citeauthor{Hugo_2009} f\"uhren darin aus, dass langsame Reaktionen schlechter als schnelle Reaktionen f\"ur die \"Uberf\"uhrung in einen kontinuierlichen Prozess geeignet sind. Als Grund daf\"ur wird vor allem die lange Verweilzeit angegeben, welche bei langsamen Reaktionen in kontinuierlicher Fahrweise notwendig ist. Eine Erh\"ohung der Prozesstemperatur kann die Verweildauer reduzieren, sie mindert aber gegebenenfalls die Produktqualit\"at und ist daher nicht unbedingt geeignet. Die zu beachtenden Regeln bei der \"Uberf\"uhrung eines Semi-Batchprozesses in einen kontinuierlichen Prozess werden von den Autoren f\"ur  schnell ablaufende, stark exotherme Reaktionen allgemein dargelegt und anhand eines Beispiels konkret angewandt. Der Fokus liegt dabei auf der Dimensionierung der Mikroreaktoren mit dem Ziel einer sicheren Anlagenf\"uhrung.\cite{Hugo_2009}

Die wirtschaftlichen Vorteile und die m\"ogliche Zeitersparnis bei der Produktentwicklung wurden anhand zahlreicher Fallbeispiele erfolgreich nachgewiesen. \cite{Brodhagen_2012, Behr_2012, Grundemann_2012, Sell_2013} Neben Reaktoren sollen auch andere Prozessschritte mittels Mikro- und Millianlagen realisiert werden. Insbesondere die Trennung von Stoffen ist ein aktueller Forschungsschwerpunkt \cite{Helling_2012}. Die aktuelle Arbeit von \citeauthor{Yang_2017} gibt den Kenntnisstand zur Verwendung von Mikroanlagen zur Destillation wieder \cite{Yang_2017}. \citeauthor{Lier_2016} f\"uhren im \"Ubersichtsbeitrag weitere modulare Apparate auf. Diese realisieren die Prozessschritte W\"armeaustausch, Reaktion, Mischen von Stoffen und Stofftrennung. Die vorgestellten Apparate  sind dabei nicht auf Konstruktionen in Mikro- oder Millibauweise beschr\"ankt. \cite{Lier_2016}

Mikro- und Millireaktoren sind ein bew\"ahrtes Mittel der Chemischen und Pharmazeutischen Industrie, um Produkte schneller und kosteng\"unstiger zu entwickeln. Insbesondere die Skalierbarkeit durch Parallelisierung von vielen Reaktoren der gleichen Bauart ist ein gro\ss{}er Vorteil. Diese Wiederverwendbarkeit ist ein wichtiger Aspekt der Tutzing Thesen \cite{Processnet_2009}. Mikro- und Millireaktoren k\"onnen als eine Auspr\"agung modularer Anlagenkomponenten angesehen werden und sind daher geeignete Beispiele, um die erfolgreiche Verwendung von modularen Anlagen zu belegen. Weitere Beispiele stellen die Modularisierung eines  Gasw\"aschers \cite{Ohle_2014} und die Modularisierung einer Anlage zur Hochleistungsfl\"ussigkeitschromatographie \cite{Rottke_2012} dar. \newline
Da Mikro- und Millireaktoren einen Spezialfall modularer Komponenten bilden wird auf ihre Besonderheiten im weiteren Verlauf der Arbeit jedoch nicht weiter eingegangen. Statt dessen liegt der Fokus im weiteren Verlauf dieser Arbeit auf der allgemeinen Verwendung von modularen Anlagenkomponenten und deren Sicherheitsbetrachtungen. \newline
Im Folgenden werden der Forschungsstand zu modularen Anlagen grob wiedergegeben und die offenen Schwerpunkte benannt. Dazu werden \"Ubersichtsarbeiten zu diesem Thema ausgewertet und die Entwicklung der Forschung wiedergegeben, ohne jedoch eine zu ausgepr\"agte Detailtiefe zu erreichen.  

Das Gebiet der Modularisierung von Anlagenkomponenten ist ein verh\"altnism\"a\ss{}ig junges Forschungsgebiet. In Folge der in Kapitel \ref{ch:einleitung} dargelegten hohen Relevanz wird diese Thematik intensiv untersucht. \newline
Ein Nachweis der wirtschaftlichen Sinnhaftigkeit von kleinskaligen Anlagen wurde von \citeauthor{Seifert_2012} in \citetitle{Seifert_2012} erfolgreich erbracht. \cite{Seifert_2012} \newline
Drei Jahre nach Ver\"offentlichung der Tutzing Thesen \cite{Processnet_2009} wurde von zwei der Teilnehmern des 48. Tutzing Symposiums eine grundlegende Arbeit zur Verwendung von Modulen im Planungsprozess einer verfahrenstechnischen Anlage ver\"offentlicht. Die Autoren \citeauthor{Bramsiepe_2012} betrachten in ihrer Arbeit \citetitle{Bramsiepe_2012} \cite{Bramsiepe_2012} Sichtweisen auf die Modularisierung, definieren verschiedene Typen von Modulen, erl\"autern deren Einsatzzweck und -zeitpunkt im Planungsprozess und gehen weiterhin auf offene Forschungsfragen ein. \newline
Als besondere Vorteile der Modularisierung werden die hohe Flexibilit\"at und eine schnelle Anpassung der Produktionskapazit\"at an Marktver\"anderungen genannt. Weitere Vorteile sind die M\"oglichkeit einer r\"aumlichen Trennung der Produktion verschiedener  Zwischenprodukte zur Einsparung von Transportkosten und die M\"oglichkeit, einen Gro\ss{}teil der Anlagenmontage an einem beliebigen Ort unter optimalen Bedingungen vornehmen zu k\"onnen. Am Aufstellungsort der Anlage m\"ussen die Module dann nur noch verbunden werden, was insbesondere bei klimatisch anspruchsvollen Anlagenstandorten sehr vorteilhaft ist. \newline
\citeauthor{Bramsiepe_2012} fordern eine Moduldefinition derart, dass ein Modul einen hohen Grad an Wiederverwendbarkeit besitzt und losgel\"ost von einer Gesamtanlage getestet werden kann. Module sollten au\ss{}erdem nach ihrem Detaillierungsgrad unterschieden werden. Die Aufteilung in Planungsmodule und Variantenmodule wird daher als sinnvoll erachtet. Planungsmodule stellen in erster Linie einen Wissensspeicher dar und dienen der Darstellung der Vielfalt von Variantenmodulen. Sie bieten Ans\"atze zu Auswahl, Funktionsumfang, Auslegung und Dimensionierung von Variantenmodulen. Ein Variantenmodul soll als 2D und als 3D Version entwickelt werden. Ein 2D Variantenmodul soll Informationen enthalten, welche am Ende des Basic Engineering vorhanden sind. Dies umfasst alle Informationen, welche zum Entwurf eines R\&{}I Flie\ss{}bildes f\"ur ein Modul notwendig sind. Ein 3D Variantenmodul ist um Auslegungsgr\"o\ss{}en derart erweitert, dass die Modulfertigung m\"oglich ist, wobei eine genaue Definition der Schnittstellen notwendig ist. \newline
Im Planungsprozess hat der Detaillierungsgrad der verwendeten Variantenmodule ma\ss{}geblichen Einfluss. 2D Module erleichtern die Erzeugung von Flie\ss{}bildern einer Gesamtanlage. Insbesondere erm\"oglichen sie einen direkten Vergleich verschiedener Anlagenstrukturen. Mit Hilfe von Simulationen k\"onnen in Kombination mit Planungsmodulen geeignete 3D Module f\"ur einen Prozess ausgew\"ahlt und die Gesamtanlage entworfen werden. \citeauthor{Bramsiepe_2012} verweisen auf Literatur, in welcher die zur Erlangung von 2D und 3D Modulen notwendigen Arbeitsschritte dargelegt werden. \newline
Bei der Entwicklung von Regelungs- und Sicherheitskonzepten muss betrachtet werden, welche Aufgaben ein einzelnes Modul losgel\"ost vom Gesamtsystem erf\"ullen kann und welche Aufgaben nur im Zusammenspiel mehrerer Module gel\"ost werden k\"onnen. Die implementierten F\"ahigkeiten des Moduls bestimmen also ma\ss{}geblich den Entwicklungsaufwand neuer Sicherheitsfunktionen einer Gesamtanlage. \newline
Um Module verwenden zu k\"onnen, nennen \citeauthor{Bramsiepe_2012} die folgenden Forschungsschwerpunkte:
\begin{itemize}
\item Systematischer Entwurf von 2D, 3D Variantenmodulen und Planungsmodulen, wobei besonders eine Systematik des Modulentwurfs zu definieren ist.
\item Die Verbesserung von Ans\"atzen, wie Module konkret in den Planungsprozess integriert werden k\"onnen.
\item Entwicklung von Berechnungsmodellen zum Scale-Up von Modulen.
\item Erstellung von Simulationsmodellen von Modulen, um deren Variantenauswahl und konkrete Auslegung durchf\"uhren zu k\"onnen.
\item Entwicklung eines Datenmodells, um Datenanreichung und Datenaustausch zu erm\"oglichen.
\end{itemize}

Ein weiterer \"Ubersichtsbeitrag zum Thema Modularisierung stammt von \citeauthor{Urbas_2012} . Die Autoren legen ihren Fokus dabei auf die Prozessf\"uhrung mit Hilfe modularer Anlagenkomponenten. Ein besonderer Schwerpunkt stellt die Zusammenarbeit von internen Komponenten eines Moduls, wie der Automatisierung und implementierten Sicherheitsfunktionen, mit den externen Komponenten, wie dem \"ubergeordneten Prozessleitsystem, dar. \newline
Zum Zeitpunkt der Ver\"offentlichung der Arbeit \citetitle{Urbas_2012} von \citeauthor{Urbas_2012} gab es noch keine einheitliche Definition des Begriffs \glqq Modul\grqq { }. Die Autoren definieren den Begriff Modul in Anlehnung an die Konstruktionslehre als \glqq abgeschlossene und wiederverwendbare Einheiten zur Erf\"ullung einer oder mehrerer Prozessfunktionen, die im Prozessf\"uhrungskontext sinnvoll zusammengefasst werden k\"onnen.\grqq { }\cite[S. 2]{Urbas_2012} Die für die Implementierung der Prozessfunktion notwendigen Equipments, Instrumente und Automatisierungsfunktionen sollen im Modul enthalten sein. Weiterhin ben\"otigt ein Modul klar definierte Schnittstellen zu seiner Umgebung. \cite[S. 2]{Urbas_2012}. Als Umfang eines Moduls wird eine beliebige Ebene zwischen Teilanlage und Einzeloperation eines Prozesses vorgeschlagen. Der Mangel einer klaren Methodik zur Definition von Modulen hat Forschungsbedarf zur Folge. Als besonderer Schwerpunkt wird die von \citeauthor{Bramsiepe_2012} ebenfalls geforderte Entwicklung von gewerke\"ubergreifenden formalen Informationsmodellen empfohlen. \cite{Urbas_2012}

Als dritte \"Ubersichtsarbeit zum Thema Modularisierung soll auf die Arbeit \citetitle{Hady_2012} von \citeauthor{Hady_2012} \cite{Hady_2012} verwiesen werden. Diese Arbeit betrachtet zum einen allgemeine Fragestellungen, wie die Definition von Modulen und deren Abgrenzung zum Baukastenprinzip und zum anderen Aspekte der modularen Anlagenplanung. Weiterhin wird ein Modularisierungskonzept im Detail vorgestellt, welches die Beschreibung und Verwendung von Modulen erm\"oglichen soll. Die Autoren gehen ebenfalls auf die Anwendbarkeit der Modularisierung f\"ur die in diesem Kapitel bereits genannten Minianlagen ein und legen au\ss{}erdem die Auswirkungen der Modularisierung auf die Kostensch\"atzung eines Anlagenbaus dar.\newline
\citeauthor{Hady_2012} bekr\"aftigen die bereits ausgef\"uhrte Notwendigkeit und die Vorteile einer modularen Anlagenplanung. Im Rahmen einer Industriebefragung stellen sie jedoch fest, dass die modulare Anlagenplanung nur in geringem Ma\ss{}e eingesetzt wird. \citeauthor{Hady_2012} erl\"autern den Begriff der Modularisierung und ziehen Parallelen zur Automobilindustrie und den dort ebenfalls etablierten Baukastensystemen. Ein Modul wird in Abgrenzung zu Bausteinen als eine Einheit charakterisiert, welche eine definierte Funktionalit\"at allgemein abdecken soll. Ein Baustein deckt lediglich in Bezug auf das System, dessen Bestandteil er ist, die gew\"unschte Funktionalit\"at ab. \newline
Den Vorteil einer m\"oglichen Vormontage von modularen Anlagen belegen die Autoren anhand mehrere Quellen, weisen aber darauf hin, dass die in diesem Zusammenhang verwendeten Module  eher als Einzelst\"ucke anzusehen sind, da die konzipierten Anlagen zumeist nur in sehr wenige Einzelmodule zerlegt wurden. An dieser Stelle zeigt sich besonders deutlich, dass die Verwendung von Modulen Kosten reduzieren kann, ohne zwangsl\"aufig einen hohen Grad an Wiederverwendbarkeit zur Folge zu haben. Als besonders wichtig gilt daher die bereits genannte systematische Entwicklung von Modulen und deren Darstellung in einer Form, welche Weiterentwicklungen und Austausch beg\"unstigt. \newline
Zur Ablage von bereits entwickelten Modulen schlagen die Autoren die Verwendung einer Bibliothek von Modulen und Dokumenten vor, welche von allen am Anlagenplanungsprozess beteiligten Personen verwendet werden kann. Eine solche Datenbank wurde entwickelt und wird von \citeauthor{Hady_2012} entsprechend referenziert. Die entwickelte Datenbank wurde an der TU Berlin erfolgreich eingesetzt und liefert in Kombination mit dem vorgestellten Vorgehen zum Einsatz von Modulen in der Anlagenplanung einen detaillierten Ansatz f\"ur die Industrie und weitere Forschungsvorhaben. \cite{Hady_2012}

Die Auswahl eines geeigneten Moduls sollte besonders bei gro\ss{}en Datenbanken rechnergest\"utzt erfolgen. \citeauthor{Obst_2013} stellen einen Algorithmus vor, mit Hilfe dessen die Eignung eines Moduls f\"ur einen gegebenen Einsatzzweck bewertet werden kann. \cite{Obst_2013}

Ein vergleichbarer Ansatz zum von \citeauthor{Hady_2012} vorgestellten Vorgehen zur Verwendung von Modulen wurde von \citeauthor{Uzuner_2013} in \cite{Uzuner_2012, Uzuner_2013} erarbeitet. In \citetitle{Uzuner_2013} wird gezeigt, wie die Erstellung eines Rohrleitungs-- und Instrumentenflie\ss{}schemas (Piping and instrumentation diagram, \acused{pid}\ac{pid}) durch Unterteilung einer Gesamtanlage in wiederverwendbare Funktionsgruppen und die Verwendung einer wissensbasierten Software geeignet beschleunigt werden kann. Module sollen laut \citeauthor{Uzuner_2013} derart definiert werden, dass sie prozesstechnisch sinnvoll sind und einen m\"oglichst hohen Grad an Wiederverwendbarkeit aufweisen. Der Autor folgt damit dieser etablierten Anforderung an ein Modul. Ein Modul soll Standard-Prozesseinheiten wie Pumpen, Verdichter, W\"arme\"ubertrager, Beh\"alter, Reaktoren oder Kolonnen umfassen und weiterhin die notwendigen Elemente der Sicherheitstechnik, Regelungstechnik, Nahverrohrung und Instrumentierung enthalten. Die damit verbundene Vereinfachung und Beschleunigung der Planungsarbeit wird aufgezeigt und best"=practice L\"osungen pr\"asentiert.

Eine Weiterentwicklung und Konkretisierung der von \citeauthor{Bramsiepe_2012}, \citeauthor{Uzuner_2012} und \citeauthor{Hady_2012} \cite{Bramsiepe_2012, Uzuner_2012, Hady_2012} vorgestellten modularen Planungsans\"atze findet sich in der Arbeit \citetitle{Fleischer_2016} von \citeauthor{Fleischer_2016} \cite{Fleischer_2016}. Die Autoren st\"utzen sich dabei schwerpunktm\"a\ss{}ig auf das Projekt $\text{F}^{3}$--Factory \cite{f3_2014}. Sie legen ausf\"uhrlich dar, wie mit Hilfe einer Datenbank bestehend aus Modulen und zugeh\"origen Planungsdokumenten wie Berechnungen, Flie\ss{}bildern, Betriebsanleitungen und Apparatelisten w\"ahrend der Planung einer neuen Anlage geeignete Module ausgew\"ahlt werden k\"onnen. Module gelten als geeignet, wenn sie zuvor definierte Prozessparameter und Funktionen direkt erf\"ullen, oder wenn sie durch geringf\"ugige Modifikationen dazu in die Lage versetzt werden k\"onnen. Prozessparameter sind dabei beispielsweise geforderte Durchflussmengen, Dr\"ucke und Temperaturen; als Funktionen gelten Prozessschritte wie Pumpen oder R\"uhren. Die Datenbank dient als wachsender Speicher an Engineeringleistung und bietet f\"ur alle an der Planung beteiligten Personen eine Planungsgrundlage und Wissensablage. 

Der Forschungsschwerpunkt der Simulation ist ein weiterhin intensiv zu untersuchendes Feld. Die Arbeit \citetitle{Oppelt_2015} von \citeauthor{Oppelt_2015} betrachtet die bisherige Verwendung von Simulationen bezogen auf den gesamten Lebenszyklus einer prozessleittechnischen Anlage. Die Autoren kommen zu dem Schluss, dass Simulationen zwar bereits verwendet werden, die Leistungsf\"ahigkeit von bereits vorhandener Software aber nicht ausgenutzt wird. Die Integration in den Lebenszyklus einer Anlage bedarf weiterer Forschung, wobei besonders der Vereinheitlichung von Schnittstellen eine gro\ss{}e Bedeutung beigemessen wird. Eine Bereitstellung von Simulationsmodellen von Komponentenlieferanten wird als sehr n\"utzlich erachtet. \cite{Oppelt_2015} Im Rahmen der Modularisierung k\"onnte genau diese Aufgabe erfolgreich gel\"ost werden. Dazu sind die von \citeauthor{Bramsiepe_2012} in \cite{Bramsiepe_2012} formulierten Arbeiten zur Erstellung von Simulationsmodellen f\"ur Module durchzuf\"uhren. 

Im Bereich der Darstellung von Daten wurden bereits wichtige Fortschritte erzielt. 
F\"ur die Beschreibung von Modulen wurde das \ac{mtp} entwickelt, welches in \cite{Obst_2015} und \cite{Obst_2015a} vorgestellt wird. Es dient als Informationstr\"ager, welcher alle Modulinformationen beinhaltet, die zur Integration eines Moduls ben\"otigt werden \cite[S. 2]{Obst_2015}. Wird dieser Informationstr\"ager erfolgreich verwendet, so kann ein Modullieferant das gesamte Modulengineering durchf\"uhren und der Betreiber mit wenig Aufwand ein geliefertes Modul in seine Anlage integrieren, ohne das Modul selbst detailliert zu kennen. Ein wichtiger Schritt zur Integration eines Moduls ist die Transformation des \ac{mtp} auf ein Modell, welches von der Gesamtanlage genutzt werden kann. Ein solches Informationsmodell kann auf Basis von \ac{opcua} entworfen werden. In \cite{Wassilew_2016} beziehungsweise \cite{Wassilew_2017} zeigen \citeauthor{Wassilew_2016}, wie die in einem \ac{mtp} gespeicherten Modulinformationen in einem \ac{opcua} Gesamtmodell abgebildet werden k\"onnen. Auf einem \ac{opcua} Server k\"onnen die Modulinformation dadurch online durchsucht werden, was eine wichtige Grundlage f\"ur Plug-and-Produce L\"osungen darstellt. 

In der aktuellen Arbeit \citetitle{Lier_2016a} \cite{Lier_2016a} wird erneut die Notwendigkeit modularer Anlagen dargelegt und die bereits erprobten Konzepte der Modularisierung bewertet. Es werden die im Abschnitt \ref{sec:einltg_chemPharmaIndustrie} bereits benannten Projekte $\text{F}^{3}$--Factory \cite{f3_2014} und CoPIRIDE \cite{copiride_2014} sowie der daraus hervorgegangene \glqq Evotrainer\grqq { }beziehungsweise \glqq EcoTrainer\grqq { }\cite{Lang_2012} betrachet. Die Autoren stellen fest, dass mit Ausnahme von Modulen in Containerbauweise der gro\ss{}e Durchbruch der modularen Strategie noch immer nicht erfolgt ist. Weiterhin werden Arbeiten zur Wirtschaftlichkeit von modularen Ans\"atzen ausgewertet (unter anderem \cite{Seifert_2012, Brodhagen_2012}). \citeauthor{Lier_2016a} kommen zu dem Schluss, dass modulare Anlagen in Folge einer deutlich verk\"urzten Amortisationszeit f\"ur wechselnde Marktverh\"altnisse bestens geeignet sind. F\"ur l\"angere Produktlebenszyklen sind derzeit konventionell geplante Anlagen die wirtschaftlich bessere Wahl. Die Ursache daf\"ur sind geringere Betriebskosten und eine auf den Prozess genauer abgestimmte Anlage. Die in \cite{Bramsiepe_2012} geforderten Berechnungsmodelle f\"ur die Skalierung von modularen Anlagen wurden bisher noch nicht entwickelt. Die Arbeiten von \citeauthor{Brodhagen_2012} \cite{Brodhagen_2012} und \citeauthor{Grundemann_2012} \cite{Grundemann_2012} zeigen aber ein m\"ogliches Vorgehen auf. Die Verfasser best\"atigen den von \citeauthor{Urbas_2012} in \cite{Urbas_2012} genannten akuten Bedarf nach Forschungsarbeit zur Automatisierung von Modulen und deren Einbindung in ein \"ubergeordnetes Prozessleitsystem. Zus\"atzlich verweisen sie auf die Notwendigkeit von anpassbaren, modularen Logistikl\"osungen. 

Die ebenfalls aktuelle Arbeit \citetitle{Hohmann_2017} von \citeauthor{Hohmann_2017} \cite{Hohmann_2017} best\"atigt diese Aussagen. Die Autoren geben einen groben \"Uberblick zur Entwicklung der Modularisierung von Anlagen und nennen eine Reihe an Herstellern, welche bereits modulare Anlagen f\"ur die Chemische und Pharmazeutische Industrie  zum Kauf anbieten. \newline
Sie verdeutlichen die Unterschiede von konventioneller Anlagenplanung und modularen Ans\"atzen und werten neben den bereits aufgef\"uhrten Ans\"atzen (\cite{Bramsiepe_2012, Uzuner_2012, Hady_2012,Fleischer_2016}) noch weitere aus. Sie kommen zu dem Schluss, dass die bisher entwickelten Vorgehen jeweils nur einen Spezialfall oder einen Teil einer kompletten Anlagenplanung bezogen auf den gesamten Lebenszyklus betrachten. Die Konzepte sind laut Aussage von \citeauthor{Hohmann_2017} nicht zu einem Gesamtkonzept kombinierbar, da die Begrifflichkeit eines Moduls im Rahmen der Prozessleittechnik noch immer nicht standardisiert ist und weil einheitliche Arten der Informationsdarstellung nicht vorhanden sind oder zumindest nicht verwendet werden. \newline
Die organisatorischen Konsequenzen der Modularisierung auf den Lebenszyklus wurden bereits von \citeauthor{Obst_2013b} in \cite{Obst_2013b} betrachtet. Eine Weiterentwicklung dieser Arbeit wurde durch den Namur Arbeitskreis 1.12 in Form der \citetitle{ne_148} \cite{ne_148} ver\"offentlicht. Darin wird ein prinzipieller Leitfaden f\"ur Anlagenbauer, Lieferanten und Betreiber zu Entwicklung und Einsatz von modularen Anlagen pr\"asentiert. Der Fokus liegt dabei auf der Automatisierung von Modulen und deren Einbindung in \"ubergeordnete Systeme. \newline
\citeauthor{Hohmann_2017} pr\"asentieren einen konkreten Planungsansatz, welcher den gesamten Lebenszyklus einer modular aufgebauten Anlage abdeckt. Da f\"ur die Automatisierung von Modulen noch kein einheitlicher Standard gefunden wurde gehen sie jedoch nicht im Detail auf die in \cite{ne_148} vorgestellten Methoden ein. \newline
Ein Modul wird in diesem Ansatz definiert als \glqq ein w\"ahrend der Planung und Fertigung von modularen Anlagen unver\"anderbares Element, welches eine bestimmte Funktion f\"ur einen Prozess erf\"ullt und welches im Rahmen weiterer Entwicklungen im Rahmen der Prozessindustrie wiederverwendbar ist\grqq { }\cite[S. 2]{Hohmann_2017}. Durch Verwendung einer Blockdarstellung (\glqq block representation frame\grqq { }) werden die Entwicklungsstadien einer Anlage abgebildet. Ein Block enth\"alt dabei Felder f\"ur Informationen, deren Anzahl und Umfang bei Fortschreiten der Planung zunimmt und welche durch Engineeringleistungen gef\"ullt werden. Die Felder beschreiben beispielsweise Kostensch\"atzungen, Massenbilanzen, notwendige Dr\"ucke und Temperaturen, entwickelte Simulationsmodelle und konkrete 3D Layouts.

Zusammenfassend l\"asst sich feststellen, dass die Modularisierung von Anlagen ein intensiv erforschtes Themengebiet darstellt. Der Nachweis der Notwendigkeit von modularen Anlagen und deren erwartete Verwendungsm\"oglichkeiten ist erbracht. Trotz zahlreicher Studien, die gro\ss{}es Entwicklungspotenzial belegen, gibt es weiterhin gro\ss{}en Forschungs- und Entwicklungsbedarf, um eine Steigerung der Akzeptanz und eine umfangreiche Anwendung in der Industrie zu erreichen.

\section{Gesetzliche Rahmenbedingungen zur Genehmigung von Chemischen Anlagen}\label{sdt_gesetze}
Abschnitt \ref{sec:einltg_sicherheitstechnik} erl\"autert, dass in Folge schwerer Unf\"alle in Industrieanlagen eine Vereinheitlichung von Sicherheitsstandards in der \ac{eu} angestrebt wird. Dies geschieht durch europ\"aische Richtlinien, welche von den Mitgliedstaaten in nationales Recht umzusetzen sind.

In diesem Zusammenhang ist vor allem die \ac{seveso3} relevant. Diese wird in Deutschland in den Regelungen des Bundes"=Immissionsschutzgesetzes \acused{bimschg}(\ac{bimschg}), des Umweltvertr\"aglichkeitspr\"ufungsgesetzes \acused{uvpg}(\ac{uvpg}), und in der \acf{bimschv12} umgesetzt. Weitere europ\"aische Richtlinien, welche f\"ur verfahrenstechnische Anlagen zu beachten sind, sind die \ac{mrl}, die \ac{ied} und die \ac{dgrl}. \newline
Zus\"atzlich zu den Richtlinien der \ac{eu} haben sich zahlreiche Nationen auch auf globale Regeln geeinigt. Dazu z\"ahlt beispielsweise das \ac{ghs}, welches europaweit durch die \ac{clp} umgesetzt ist.

Das \ac{bimschg} wird durch zahlreiche Verordnungen erg\"anzt und pr\"azisiert. \newline
F\"ur den Betrieb von verfahrenstechnischen Anlagen ist vor allem die \ac{bimschv4} von Interesse, in der geregelt wird, welche Anlagen einer immissionsschutzrechtlichen Genehmigung bed\"urfen. \newline
Dar\"uber hinaus dient die \ac{bimschv12} der Verh\"utung schwerer Unf\"alle, die durch bestimmte Industriet\"atigkeiten hervorgerufen werden k\"onnten und der Begrenzung der Unfallfolgen für die menschliche Gesundheit und die Umwelt. Weiterhin sind die Bestimmungen der \ac{gefstoffv} sowie der \ac{betrsichv} bei Industrieanlagen einzuhalten.

Die genannten Richtlinien, Gesetze und Verordnungen werden durch Technische Regeln und Leitf\"aden erg\"anzt. Diese definieren den Stand der Technik und geben Empfehlungen f\"ur die Umsetzung von Gesetzen und Verordnungen. Wichtige technische Regeln sind beispielsweise die \ac{tras}, die \ac{trbs}, die \ac{trgs}, die \ac{trws} und die \ac{trba}.

Bei \"Uberschreitung von festgelegten Mengenschwellen f\"ur verschiedene gef\"ahrliche Stoffe sind die Bestimmungen der \ac{bimschv12} von Anlagenbetreibern einzuhalten. Es werden Anlagen mit Betriebsbereichen \glqq der unteren\grqq { }und \glqq der oberen Klasse\grqq { }{(\S 1 Abs. 1 S. 1 \ac{bimschv12})} definiert. Die Einordnung in eine Klasse basiert auf der vorhandenen Menge von gef\"ahrlichen Stoffen, wobei das {\"Uber- oder Unterschreiten} festgelegter Grenzwerte zu einer Klassifikation f\"uhrt {(\S 2 Nr. 1 f. \ac{bimschv12})}. Die gef\"ahrlichen Stoffe und deren Mengenschwellen sind im {Anhang I \ac{bimschv12}} aufgef\"uhrt. Entsprechend {\S 3 Abs. 1 HS. 1 \ac{bimschv12}} hat der Betreiber \glqq $\dots$ die nach Art und Ausma\ss{} der m\"oglichen Gefahren erforderlichen Vorkehrungen zu treffen, um St\"orf\"alle zu verhindern $\dots$\grqq { }. Als St\"orfall gilt nach {\S 2 Nr. 7 \ac{bimschv12}} \glqq ein Ereignis, das unmittelbar oder sp\"ater innerhalb oder au\ss{}erhalb des Betriebsbereichs zu einer ernsten Gefahr oder zu Sachsch\"aden $\dots$ f\"uhrt\grqq { }. Ein \glqq Ereignis\grqq { }ist eine \glqq St\"orung des bestimmungsgem\"a\ss{}en Betriebs in einem Betriebsbereich unter Beteiligung eines oder mehrerer gef\"ahrlicher Stoffe\grqq { }{(\S 2 Nr. 6 \ac{bimschv12})}. Eine \glqq ernste Gefahr\grqq { }ist nach {\S 2 Nr. 8 \ac{bimschv12}} definiert als \glqq eine Gefahr, bei der das Leben von Menschen bedroht wird oder schwerwiegende Gesundheitsbeeintr\"achtigungen von Menschen zu bef\"urchten sind, die Gesundheit einer gro\ss{}en Zahl von Menschen beeintr\"achtigt werden kann oder die Umwelt, insbesondere Tiere und Pflanzen, der Boden, das Wasser, die Atmosph\"are sowie Kultur oder sonstige Sachg\"uter gesch\"adigt werden k\"onnen, falls durch eine Ver\"anderung ihres Bestandes oder ihrer Nutzbarkeit das Gemeinwohl beeintr\"achtigt w\"urde.\grqq { }Der Anlagenbetreiber, dessen gef\"ahrliche Stoffe die Mengenschwellen der Spalte 5 des Anhangs I der \ac{bimschv12} \"uberschreiten und der damit eine Anlage der \glqq oberen Klasse\grqq { }betreibt, hat dar\"uber hinaus die erweiterten Pflichten der \ac{bimschv12} zu erf\"ullen. Dazu z\"ahlt unter anderem das Verfassen eines Sicherheitsberichtes {(\S 9 Abs. 1 HS. 1 \ac{bimschv12})}, in dem dargelegt wird, dass \glqq die Gefahren von St\"orf\"allen und m\"ogliche St\"orfallszenarien ermittelt, sowie alle erforderlichen Ma\ss{}nahmen zur Verhinderung derartiger St\"orf\"alle und zur Begrenzung ihrer Auswirkungen auf die menschliche Gesundheit und die Umwelt ergriffen wurden\grqq { }{(\S 9 Abs. 1 Nr. 2 \ac{bimschv12})}. Der erstellte Sicherheitsbericht muss eine \glqq Beschreibung der Szenarien m\"oglicher St\"orf\"alle nebst ihrer Wahrscheinlichkeit oder den
Bedingungen für ihr Eintreten $\dots$\grqq { }{(Anhang II Abschnitt IV S. 1 \ac{bimschv12})} und eine \glqq Absch\"atzung des Ausma\ss{}es und der Schwere der Folgen der ermittelten St\"orf\"alle $\dots$\grqq { }{(Anhang II Abschnitt IV Nr. 2 \ac{bimschv12})} enthalten. In Folge dieser Formulierung k\"onnen deterministische oder probabilistische Methoden zum Einsatz kommen. Eine Reihe akzeptierter Methoden wird in \citetitle{2009_bimschv12Hilfe} \cite[S. 20 f.]{2009_bimschv12Hilfe} aufgef\"uhrt. Akzeptierte Methoden sind unter anderem die \ac{fmea}, die \ac{fta} und die \ac{hazop}. Eine umfangreiche Literatur\"ubersicht zur \ac{fta} ist in \cite{Baig_2013} zu finden \cite{Baig_2013}. Die konkrete Anwendung zahlreicher Methoden wird in \citetitle{Nolan_2014} \cite{Nolan_2014} und den Richtlinien des \MakeUppercase{Center for Chemical Process Safety} \cite{ChemicalProcessSafety_2007, ChemicalProcessSafety_2007a, ChemicalProcessSafety_2008, ChemicalProcessSafety_2008a, ChemicalProcessSafety_2008b,ChemicalProcessSafety_2009, ChemicalProcessSafety_2009a, ChemicalProcessSafety_2010, ChemicalProcessSafety_2012, ChemicalProcessSafety_2013, ChemicalProcessSafety_2015} erl\"autert. Auf die besonders weit verbreitete Methode \ac{hazop} wird im folgenden Abschnitt \ref{sec:sdt_hazop} weiter eingegangen. 

\section{Sicherheitsuntersuchungen in Form einer HAZOP}\label{sec:sdt_hazop}
Die Anwendung eines Verfahrens zur Prognose, Auffinden der Ursache, Absch\"atzen der Auswirkungen, Gegenma\ss{}nahmen (PAAG) (engl. Hazard and Operability Analysis \acused{hazop}(\ac{hazop})) ist ein bew\"ahrtes Mittel, um die in der \ac{bimschv12} geforderte Gefahrenanalyse durchzuf\"uhren. Die Methode wurde 1973 erstmals von \citeauthor{Lawley_1974} auf dem AICHE Loss Prevention Symposium \"offentlich vorgestellt und ein Jahr sp\"ater publiziert \cite{Lawley_1974}. Seit dem wurde die Methode vielfach angewandt und weiterentwickelt. \newline
\citeauthor{Kletz_1997} gibt in \citetitle{Kletz_1997} einen kurzen historischen \"Uberblick \"uber die Entstehung und Entwicklung dieser Methode \cite{Kletz_1997}. \citeauthor{Dunjo_2010} arbeiten in \citetitle{Dunjo_2010} \cite{Dunjo_2010} 166 Ver\"offentlichungen zur \ac{hazop} aus dem Zeitraum 1974 bis 2009 auf, fassen die grundlegenden Gedanken der untersuchten Ver\"offentlichungen zusammen, teilen sie in verschiedene Gruppen ein und ermitteln den Stand der Technik. Sie stellen fest, dass die Anzahl der Ver\"offentlichungen zur \ac{hazop} vom Jahr ihrer Vorstellung bis zur zweiten H\"alfte der neunziger Jahre stark zugenommen hat. Der Themenschwerpunkt hat sich dabei stark verschoben. Die ersten Arbeiten erarbeiten m\"ogliche Anwendungsf\"alle der \ac{hazop}, analysieren den abgedeckten Anlagenumfang und versuchen diesen durch Anpassungen der Methode zu erweitern. Der gr\"o\ss{}te Teil der Arbeiten untersucht M\"oglichkeiten eine \ac{hazop} zu automatisieren. Die aktuellsten Arbeiten versuchen die \ac{hazop} mit anderen Methoden wie beispielsweise Simulationen zu kombinieren (siehe beispielsweise \cite{Li_2013}). \citeauthor{Dunjo_2010} kommen zu dem Schluss, dass \ac{hazop} zwar bereits die am meisten untersuchte Methode zur Analyse von Gefahren in Prozessen ist, dass aber weiterhin Verbesserungen notwendig sind. Der Mensch als Gefahrenquelle f\"ur eine Anlage wird noch nicht hinreichend im Rahmen einer \ac{hazop} untersucht, weiterhin wird eine \ac{hazop} in weiten Teilen von Menschen durchgef\"uhrt, was zu Ungenauigkeiten und Fehlern f\"uhren kann. Dar\"uber hinaus sind St\"orungen und Ausf\"alle von speicherprogrammierbaren Steuerungen, welche eine wichtige Rolle bei der Steuerung und Regelung von Anlagen haben, derzeit nur ungen\"ugend durch eine \ac{hazop} abgebildet. Diese Punkte erfordern weitere Forschung.

\section{Durchf\"uhrung einer HAZOP}
Eine ausf\"uhrliche Anleitung zur konkreten Durchf\"uhrung einer \ac{hazop} findet sich beispielsweise in \citetitle{Crawley_2015} \cite{Crawley_2015} und in der Norm \citetitle{din61882} \cite{din61882}. \newline
Eine \ac{hazop} wird von einem interdisziplin\"aren Team in Form einer kreativen Analyse der Anlage durchgef\"uhrt. Das Ziel ist es, m\"ogliche Abweichungen eines Prozesses vom Sollverhalten zu untersuchen. Dazu wird die Gesamtanlage zuerst in kleinere Funktionsgruppen die sogenannten \glqq nodes\grqq { }untergliedert. Die \glqq nodes\grqq { }werden dann nacheinander untersucht. Dazu wird die Sollfunktion beziehungsweise der Sollwert einer betrachteten Variable oder eines Prozesses innerhalb der \glqq node\grqq { }definiert. Anschlie\ss{}end wird daf\"ur eine Reihe an Leitworten wie \glqq {kein/} nicht\grqq { },\glqq mehr\grqq { },\glqq weniger\grqq { },\glqq teilweise\grqq { },\glqq Umkehrung\grqq { },\glqq anders als\grqq { }oder \glqq sowohl als auch\grqq { }ausgew\"ahlt, mit Hilfe derer eine physikalisch sinnvolle Abweichung vom Sollverhalten beschrieben wird. Darauf folgend werden m\"ogliche Ursachen und Konsequenzen der betrachteten Abweichung abgesch\"atzt. Ursachen und Auswirkungen k\"onnen dabei sowohl innerhalb als auch au\ss{}erhalb der betrachteten \glqq node\grqq { }entstehen beziehungsweise wirksam werden. Danach wird das Risiko der ermittelten Auswirkungen unter der Annahme, dass keine Gegenma\ss{}nahmen bestehen, abgesch\"atzt. Im Anschluss werden die vorhandenen Gegenma\ss{}nahmen ermittelt und bewertet. Dazu wird das sich ergebende Restrisiko ermittelt, welches bei Vorhandensein der Gegenma\ss{}nahmen zu erwarten ist. Sind ungen\"ugende Schutzeinrichtungen zur Einhaltung des tolerierten Risikos vorgesehen, so soll durch das \ac{hazop}--Team eine zur Senkung des Risikos geeignete Ma\ss{}nahme vorgeschlagen werden. \newline
Die Ergebnisse einer \ac{hazop} k\"onnen in Form von Tabellen dargestellt werden. In \tabref{tab:hazopBsp} ist ein Auszug der Analyse eines Moduls zur Vorlage eines Stoffes mit Zwischenspeicher dargestellt. Das \ac{pid} dieses Moduls ist im Anhang in \figureref{fig:PIDMod1} dargestellt. Dargestellt wird eine Betrachtung der Prozessvariablen Zufluss, Abfluss und F\"ullstand. F\"ur jede Variable wird ein Leitwort angewendet, mit Hilfe dessen die Auswirkung der Abweichung auf den \"ubrigen Prozess und das Modul selbst ermittelt wird. Weiterhin werden m\"ogliche Ursachen f\"ur die betrachten Abweichungen identifiziert. Die Ursache \glqq externe Ursache\grqq { }stellt eine Besonderheit dar. Sie ist bei konventionellen \acp{hazop} un\"ublich, bei Betrachtung von modularen Anlagen wird damit eine m\"ogliche Wechselwirkung mit angeschlossenen Modulen betrachtet. Da die gekoppelten Module zum Durchf\"uhrungszeitpunkt der \ac{hazop} des einzelnen Moduls nicht bekannt sind, kann die Ursache nicht im Detail beschrieben werden. Statt dessen wird die M\"oglichkeit einer Wechselwirkung besonders hervorgehoben. Bei der Sicherheitsbetrachtung der Gesamtanlage sind solche Ursachen erneut zu analysieren. 

\begin{table}[htb]
\tablestyle
\caption[Beispiel f\"ur eine HAZOP]{Auszug der Ergebnisse einer \ac{hazop} f\"ur ein Vorlagemodul mit Zwischenspeicher nach \cite{Pfeffer_2017}}
\begin{tabularx}{\textwidth}{cccCC}
\tableheadcolor
   {\tablehead ID} &
   {\tablehead Gr\"o\ss{}e} &
   {\tablehead Leitwort} &
   {\tablehead Ursache}&
   {\tablehead Auswirkung}
   \tabularnewline
%
\tablebody
  1 & Zufluss & weniger & Ventil defekt ODER Sensor defekt ODER Ansteuerung defekt & F\"ullstand sinkt \\ \hline 
  2 & F\"ullstand & mehr & F\"ullstandssensor defekt ODER ID=3 & Druckanstieg im Tank \\ 
  \hline 
  3 & Abfluss & kein & Fehlfunktion der Pumpe & F\"ullstand steigt 
   \tabularnewline
%
\tableend
\end{tabularx}
\label{tab:hazopBsp}
\end{table}

%evtl wichtig: siehe \cite[S. 4]{Fleischer_2015}: Aus dem Stand der Technik sind zahlreiche industriell etablierten Methoden der Risikoanalyse f\"ur die Anlagen der
%Prozessindustrie bekannt. Eine Auff\"uhrung unterschiedlicher Methoden findet sich im Buch von Preiss [14], im DGQ-Band 17-10-Zuverl\"assigkeitsmanagement [15] oder
%in den Anh\"angen der DIN EN 60300-3-1 [16] sowie der
%VDI 4003 Zuverl\"assigkeitsmanagement [17].

\section{Sicherheitsbetrachtungen modularer Anlagen}
Modulare Anlagen unterliegen ebenso wie konventionelle Anlagen den im Abschnitt \ref{sdt_gesetze} dargelegten Gesetzen. Daraus ergibt sich die Notwendigkeit einer Sicherheitsuntersuchung, wenn die im {Anhang I \ac{bimschv12}} definierten Mengenschwellen an gef\"ahrlichen Stoffen \"uberschritten werden und eine modulare Anlage damit als eine Anlage \glqq der oberen Klasse\grqq { }nach {\S 1 Abs. 1 S. 1 \ac{bimschv12}} einzustufen ist. \newline
Die Notwendigkeit der Anwendung der \ac{mrl} und damit die Erstellung einer \ac{eg}"=Konformit\"atserkl\"arung f\"ur eine verfahrenstechnische Anlage ist allerdings nicht offensichtlich. Die \ac{mrl} ist unter anderem f\"ur \glqq Maschinen\grqq { }{(Art. 1 Nr. 1 lit. a \ac{mrl})} und \glqq unvollst\"andige Maschinen\grqq { }{(Art. 1 Nr. 1 lit. g \ac{mrl})} anzuwenden. Eine geeignete Interpretation dieser in Art. 2 lit. a,g \ac{mrl} definierten Begriffe wird von \citeauthor{Weber_2016} in \citetitle{Weber_2016} gegeben {(vgl. \cite[S. 589 ff.]{Weber_2016})}. \newline
\citeauthor{Weber_2016} beantwortet die Frage, ob eine verfahrenstechnische Anlage eine \glqq Gesamtheit von Maschinen\grqq { }darstellt und damit der \ac{mrl} unterliegt, verneinend. Als Begr\"undung wird angegeben, dass es in der Regel keinen sicherheitstechnischen Zusammenhang entsprechend des \textit{Interpretationspapieres zum Thema \glqq Gesamtheit von Maschinen\grqq { }des Bundesministeriums f\"ur Arbeit und Soziales} gibt, da bei einer sicherheitsgerichteten Abschaltung eines gef\"ahrdeten Anlagenteils h\"aufig nicht zeitgleich andere Anlagenteile abgeschaltet werden {(vgl. \cite[S. 591]{Weber_2016})}. Damit ist in der Regel keine Erstellung einer \ac{eg}"=Konformit\"atserkl\"arung f\"ur eine verfahrenstechnische Anlag notwendig. \newline
F\"ur modulare Anlagen gilt dies nach Meinung von \citeauthor{Weber_2016} jedoch nicht. Der Autor stellt fest, dass Package-unit-Anlagen als eine Gesamtheit von Maschinen anzusehen sind {(vgl. \cite[S. 592]{Weber_2016})}. Dieser Anlagentyp ist eine besondere Auspr\"agung eines Moduls, weswegen davon auszugehen ist, dass diese Aussage auf s\"amtliche modularen Anlagen anzuwenden ist. Dabei wird jedoch nicht deutlich, ob Package-unit-Anlagen den Zusammenschluss mehrerer Module zu einer Gesamtanlage oder einzelne Module umfassen. \citeauthor{Kockmann_2017} benennen in \citetitle{Kockmann_2017} einige Rechtsvorschriften, welche f\"ur modulare Anlagen einzuhalten sind. Die Anwendung der \ac{mrl} bringen sie explizit nur mit einzelnen Equipments und nicht mit der Gesamtanlage in Verbindung {(vgl. \cite[S. 18]{Kockmann_2017})}. Die Notwendigkeit eines Konformit\"atsverfahrens f\"ur modulare Anlagen kann daher nicht abschlie\ss{}end gekl\"art werden. Der Gesetzgeber sollte hier beispielsweise durch Ver\"offentlichung eines aktualisierten Positionspapieres zur Anwendbarkeit der \ac{mrl} dringend Klarheit schaffen. F\"ur einzelne Module ist die Durchf\"uhrung eines Konformit\"atsverfahrens aber empfehlenswert, da dies bei Einhaltung der relevanten Rechtsvorschriften problemlos m\"oglich ist. 

Im Rahmen modularer Konzepte wird teilweise davon ausgegangen, dass Module funktional eigensicher sind und modulinterne Ma\ss{}nahmen eine Fehlerfortpflanzung verhindern. \cite[S. 4]{Urbas_2012a} Ob dieser Zustand erfolgreich erreicht wird, kann aber erst gepr\"uft werden, wenn s\"amtliche Stoff- und Prozessdaten bekannt sind und die Gesamtanlage geplant ist. Daher kann erst nach dem Detailengineering eine Sicherheitsuntersuchung der Gesamtanlage durchgef\"uhrt werden. Im Rahmen dieser ist der Nachweis zu erbringen, dass die Module eigensicher sind. Ist dies nicht der Fall, so sind geeignete Sicherheitsma\ss{}nahmen zu entwickeln, um einen sicheren Betrieb zu gew\"ahrleisten.  

Die Anzahl ver\"offentlichter Forschungsarbeiten, welche sich speziell der Sicherheitsuntersuchung von aus Modulen bestehenden Anlagen widmen, ist sehr gering. Im Abschnitt \ref{sec:einltg_sicherheitstechnik} wird die Arbeit von \citeauthor{Fleischer_2015} \cite{Fleischer_2015} zusammengefasst. Die Autoren schlagen darin vor, dass die Sicherheit einzelner Module durch Anwendung von Checklisten und Heuristiken durchgef\"uhrt wird. F\"ur die Untersuchung der modularen Anlage wird die Durchf\"uhrung einer \ac{hazop} empfohlen. Wie das bei der Sicherheitsuntersuchung der Module gewonnene Wissen bei der Durchf\"uhrung der \ac{hazop} genutzt werden kann, bleibt jedoch offen. \newline
Die aktuelle Arbeit \citetitle{Kockmann_2017} von \citeauthor{Kockmann_2017} \cite{Kockmann_2017} widmet sich einer \"ahnlichen Fragestellung und stellt eine Erweiterung der Arbeit \citetitle{Fleischer_2015} \cite{Fleischer_2015} dar. Der Fokus von \citeauthor{Kockmann_2017} liegt  speziell auf der Sicherheitsuntersuchung von Mikroreaktoren, welche als Module eingesetzt werden. Die Sicherheitsuntersuchung dieser Module anhand von Checklisten und Heuristiken wird in \cite{Kockmann_2017} detaillierter als in \cite{Fleischer_2015} dargestellt. \citeauthor{Kockmann_2017} kommen ebenfalls zu dem Ergebnis, dass die Durchf\"uhrung einer \ac{hazop} aus den Untersuchungen der Module profitieren kann. Dieses Wissen soll aber lediglich als kreative Anregung bei den durchzuf\"uhrenden Teamsitzungen eingesetzt werden {(vgl. \cite[S. 20]{Kockmann_2017})}. Die M\"oglichkeit einer automatischen Verwendung des bereits gewonnen Wissens wird nicht betrachtet. \newline
Eine teilweise automatisierte Wiederverwendung dieses Wissens ist aber \"au\ss{}erst erstrebenswert, um die Durchf\"uhrung der \ac{hazop} zu beschleunigen und die Fehleranzahl in Folge von \"ubersehenen Wechselwirkungen von Prozessvariablen zu reduzieren. Einen wichtigen Schritt stellt dabei die Analyse der Fortpflanzung von Fehlern in der modularen Anlage dar. Daher wird in der vorliegenden Arbeit \"uberpr\"uft, wie bereits vorhandenes Wissen einzelner Module geeignet bei der Untersuchung von Wechselwirkungen von Prozessvariablen in der modularen Gesamtanlage genutzt werden kann. Als Basis dient dabei die Beschreibung der Gesamtanlage in Form eines \ac{pid}, die Detailbeschreibung der Module und die Ergebnisse der \ac{hazop} jedes Moduls.
\chapter{Fehlerursachenermittlung} \label{ch:fehlerfortpfl}
\textcolor{red}{Definitionen: DIN EN 61511-1: Risiko(risk), Schaden(harm), Gef\"ahrdung(hazard), St\"orung, fault, basic event(s)/root cause(s)/malfunction/failure, sicherheit(safety), DIN EN 61511 }

In Verfahrenstechnischen Anlagen kommt es immer wieder zu Abweichungen des Prozesses vom Sollzustand. Es ist Aufgabe der Anlagenfahrer auf diese Abweichungen geeignet zu reagieren, um den Prozess wieder in den sicheren Sollzustand zu \"uberf\"uhren. Diesen Vorgang k\"onnen Menschen nicht allein bew\"altigen, da sie in Folge des Umfangs moderner Anlagen nicht mehr in der Lage sind, den kompletten Zustand einer Anlage zu erfassen. Zu dem enormen Umfang an Prozessvariablen kommt erschwerend hinzu, dass Abweichungen in Folge von Sensorfehlern und Messungenauigkeiten teilweise erst sp\"at erkannt werden. Dies erschwert das rechtzeitige Initiieren notwendiger Prozesskorrekturen. Durch versp\"atetes ergreifen notwendiger Ma\ss{}nahmen kommt es immer wieder zu St\"orungen und kleinen Unf\"allen. Der dadurch entstehende Schaden betr\"agt j\"ahrlich mehrere Milliarden Euro. Es wurden umfangreiche Forschungen durchgef\"uhrt, um zu ermitteln wie Abweichungen fr\"uhzeitig erkannt und zugrunde liegende Ursachen identifiziert werden k\"onnen. Dieses Vorgehen nennt man \ac{fdi}. Zur Ermittlung der Ursachen einer Prozessabweichung werden erfolgreich Methoden der Fehlerfortpflanzung (engl. fault propagation methods) eingesetzt.  \cite[S. 2]{Venkatasubramanian_2003} \newline
Eine kompakte Einf\"uhrung in die \ac{fdi} findet sich im Abschnitt \glqq Fault Detection and Diagnosis\grqq { }des Buches \citetitle{Baillieul_2015} \cite[S. 417 ff.]{Baillieul_2015}.

Die Fortpflanzung von St\"orungen beim Betrieb chemischer Anlagen ist ein intensiv erforschtes Problem. Von Interesse sind Abweichungen der Prozessvariablen von einem definierten Sollwert und Abweichungen des Betriebszustandes von Ger\"aten, Instrumenten und Gewerken vom Optimalzustand und daraus resultierende Auswirkungen auf den Prozess.

Die Abweichung einer Prozessvariable vom Sollwert kann vielf\"altige, verkettete Ursachen haben. Am Beispiel einer stark exothermen Reaktion zweier fl\"ussiger Stoffe A und B soll dies verdeutlicht werden. Die Reaktion sei dabei noch in der Erprobungsphase, weswegen keine Erfahrungswerte in der gegebenen Anlage bestehen. \newline
Ein Reaktionsbeh\"alter mit R\"uhrer habe zwei \"uber Pumpen gesteuerte Zufl\"usse. Um die gew\"unschte Reaktion zu starten, wird der Stoff A eindosiert. Durch langsame Zugabe von Stoff B soll unter Vermischung der Reaktanten die Reaktion gestartet werden. Sei nun der R\"uhrer durch Alterung deutlich langsamer als im optimalen Zustand und au\ss{}erdem der im Reaktor befindliche Temperatursensor defekt. Trotz Zugabe von Stoff B und Einschalten des R\"uhrers findet die Reaktion dann nicht zu dem erwarteten Zeitpunkt statt, da keine ausreichende Vermischung der Stoffe A und B zustande kommt. Eine Ursachenanalyse ist in Folge fehlender Erfahrungswerte sehr kompliziert. Ein m\"ogliches Vorgehen besteht in der vermehrten Zugabe von Stoff B in der Annahme so die Reaktion starten zu k\"onnen. Durch die vermehrte Zugabe und das langsame Vermischen der Reaktanten beginnt die Reaktion. Dies wird durch den defekten Temperatursensor jedoch nicht bemerkt. Der Stoff B wird daher weiter zudosiert. Die stark exotherme Reaktion geht in Folge dessen durch. Dies wird jedoch erst durch einen Druckanstieg im Beh\"alter erkannt, welcher durch Verdampfen der Reaktanten zustande kommt. Aus Sicht der Anlagenfahrer hat die Reaktion jedoch noch immer nicht begonnen, da die Temperatur im Reaktor nicht gestiegen ist. Der Druckanstieg k\"onnte also einem fehlerhaften Drucksensor oder einer nicht erkannten Reaktion geschuldet sein. Die Ursachenanalyse f\"ur den erh\"ohten Druck ist zu diesem Zeitpunkt in Folge einer m\"oglichen Verkettung von Fehlfunktionen sehr kompliziert. Eine durchgehende Reaktion ist hochgradig gef\"ahrlich. Wird die Gefahr nicht unmittelbar erkannt, so kann dies verheerende Folgen f\"ur die Anlage, die Betreiber und die Umwelt haben.

Das Auffinden m\"oglicher Ursachen eines vorliegenden Fehlers ist eine Aufgabe, welche mit Hilfe der Analyse von Fehlerfortpflanzungen gel\"ost werden soll. Weitere Aufgaben sind die Bewertung und Auslegung von Systemen, welche das Betriebsrisiko einer Anlage auf ein gew\"unschtes Level bringen sollen. Die Planung optimaler Wartungsintervalle ist eine Aufgabe, welche direkt auf deren Ergebnissen aufbauen kann. \newline
Im Rahmen einer Fehlerfortpflanzungsanalyse wird je nach Verfahren der Einfluss von Prozessgr\"o\ss{}en, die r\"aumliche Positionierung von Anlagenteilen, die Alterung von Anlagenkomponenten oder eine Kombination dieser Faktoren betrachtet. Je nach Art der verwendeten Informationen unterscheidet man in \begin{itemize}
\item modellbasierte qualitative Verfahren,
\item modellbasierte quantitative Verfahren und 
\item auf historischen Messdaten basierende Verfahren. 
\end{itemize} 
Die modellbasierten Verfahren sind solche, welche Wissen \"uber die Struktur und Funktion einer Anlage auswerten. Zur Durchf\"uhrung eines solchen Verfahrens werden meist Experten eingesetzt. Die datenbasierten Verfahren analysieren hingegen Messwerte, welche durch den Betrieb der Anlage oder Simulation der Anlage verf\"ugbar werden. Die Auswertung findet dann beispielsweise mit Hilfe statistischer Methoden oder Verfahren zur Mustererkennung statt. \newline
In der dreiteiligen Ver\"offentlichung von \citeauthor{Venkatasubramanian_2003} (\cite{Venkatasubramanian_2003, Venkatasubramanian_2003a,Venkatasubramanian_2003b}) wird eine umfassende Einordnung der bis zum Jahr 2002 ver\"offentlichen Methoden zur Analyse von Fehlerfortpflanzungen in diese drei Kategorien und eine Bewertung der Eignung der Methoden vorgenommen. F\"ur zahlreiche Anwendungsf\"alle er\"ortern die Autoren geeignete Methoden, wodurch sie {Nicht--Ex}perten der Fehlerfortpflanzungsanalyse ein Hilfsmittel zur Evaluierung der Anwendbarkeit bestimmter Methoden f\"ur weitere F\"alle bieten. Diese Ver\"offentlichung ist daher bei der Suche nach einer anwendbaren Methode zur Fehlerfortpflanzungsanalyse ein besonders zweckm\"a\ss{}iger Startpunkt. Es gibt neben den Arbeiten von \citeauthor{Venkatasubramanian_2003} weitere Literaturanalysen, diese haben aber einen geringeren Umfang und konzentrieren sich zumeist auf eine der drei genannten Kategorien. Eine hervorzuhebende Ausnahme bildet die zweiteilige Arbeit von \citeauthor{Gao_2015}, in welcher ein umfangreicher, aktueller Literatur\"uberblick zum Thema \ac{fdi} pr\"asentiert wird \cite{Gao_2015,Gao_2015a}. \newline
In den folgenden Abschnitten \ref{sec:fAna_dat}, \ref{sec:fAna_modQuant} und \ref{sec:fAna_modQual} werden die einzelnen Kategorien detailliert betrachtet, weiter untergliedert und anhand von Beispielmethoden wird die Einsetzbarkeit einiger Methoden f\"ur modulare Anlagen bewertet. 

Als Alternative zu den drei Kategorien nach \citeauthor{Venkatasubramanian_2003}\cite{Venkatasubramanian_2003} ist eine Einteilung in off-line und on-line Verfahren m\"oglich \cite{Kavcic_2001}. Ersteres sind Verfahren, die losgel\"ost vom Betrieb einer Anlage durchgef\"uhrt werden. Ein solches Verfahren kann beispielsweise vor der Erstinbetriebnahme auf Basis von Expertenwissen durchgef\"uhrt werden. Dann ist ein solches Verfahren gleichzeitig ein modellbasiertes Verfahren. Off-line Verfahren k\"onnen aber auch datenbasierte Verfahren sein. Dies ist dann der Fall, wenn Messwerte durch zeitintensive Rechenoperationen ausgewertet werden. Ist dies nicht mehr in Echtzeit m\"oglich, so kann nur eine off-line Analyse durchgef\"uhrt werden. \newline 
Ist ein Verfahren in Echtzeit berechenbar und basiert es auf aktuellen Messdaten einer Anlage, so wird es als on-line Verfahren kategorisiert. Ein solches Verfahren ist zwangsl\"aufig fr\"uhestens nach der Erstinbetriebnahme einer Anlage durchf\"uhrbar. 

Eine klare Abgrenzung der Einteilung in off-line und on-line Verfahren beziehungsweise in modell- und datenbasierte Verfahren ist offensichtlich kompliziert. Im Rahmen dieser Arbeit liegt der Fokus auf den notwendigen Daten, welche zur Durchf\"uhrung eines Verfahrens zur Analyse von Fehlerfortpflanzungen notwendig sind. Die Einteilung, ob es sich um on-line oder off-line Verfahren handelt, ist hingegen nebens\"achlich. Daher wird im Folgenden nur noch in modellbasierte qualitative, modellbasierte quantitative, auf historischen Messdaten basierende  und Hybride dieser Verfahren unterschieden. 
  
\section{Modellbasierte Quantitative Fehlerfortpflanzungsmethoden}\label{sec:fAna_modQuant}
Als modellbasierte quantitative Verfahren zur \ac{fdi} werden im Rahmen dieser Arbeit Verfahren bezeichnet, welche auf Basis von \glqq analytischer Redundanz\grqq { }Residuen generieren, die wiederum zur Fehleridentifikation und Isolation genutzt werden. Die Ausf\"uhrungen im Abschnitt \ref{sec:fAna_modQuant} basieren in weiten Teilen auf der Arbeit von \citeauthor{Venkatasubramanian_2003} \cite{Venkatasubramanian_2003}. 

Verfahren dieser Art bestehen aus zwei grundlegenden Schritten. Im ersten Schritt werden mit Hilfe eines analytischen Prozessmodells und gemessenen Gr\"o\ss{}en durch Einsatz mathematischer Verfahren Residuen generiert. Diese werden im zweiten Schritt analysiert, um das Vorliegen eines Fehlers zu erkennen und diesen eindeutig zu identifizieren.

Ein analytisches Prozessmodell kann entweder durch einen Satz an Gleichungen oder in Form einer Black Box beschrieben werden. Analysiert man die physikalischen und chemischen Gesetze welchen ein Prozess unterliegt, so kann man Massen-, Energie-, Impuls- und Reaktionsbilanzen formulieren. Diese lassen sich allgemein als \acp{dae} darstellen. Betrachtet man hingegen Messwerte der Ein- und Ausgangsgr\"o\ss{}en eines Prozesses, so kann man beispielsweise durch den Einsatz stochastischer Methoden ein Ein"=Ausgangsmodell f\"ur den Prozess erstellen. Diese Art der Modellbildung bezeichnet man als \glqq Systemidentifikation\grqq. Zahlreiche Ver\"offentlichungen der Regelungstechnik behandeln dieses Thema. \citeauthor{Unbehauen_2010} bietet in \citetitle{Unbehauen_2010} \cite{Unbehauen_2010} einen Einstieg in die Systemidentifikation. Fortgeschrittene Verfahren zur Identifikation nicht--linearer Systeme werden beispielsweise von \citeauthor{Schroeder_2010} in \citetitle{Schroeder_2010} \cite{Schroeder_2010} vorgestellt. \newline
Physikalische Modelle zeichnen sich durch die Interpretierbarkeit der Prozessvariablen und eine  genau absch\"atzbare G\"ute aus, haben aber einen hohen Entwurfs-- und Berechnungsaufwand zur Folge.\newline
Es existieren zahlreiche computergest\"utzte Hilfsmittel, welche die Entwicklung eines Black Box Modells unterst\"utzen und damit stark beschleunigen. Ein Beispiel daf\"ur ist die \glqq System Identification Toolbox\texttrademark\grqq { }der Firma \glqq Mathworks\grqq. Black Box Modelle sind daher weniger aufwendig in der Formulierung und der Berechnung als die analytische Beschreibung eines Prozesses. Die Erstellung eines solchen Modells ben\"otigt aber Messdaten, welche alle m\"oglichen Betriebsbedingungen abdecken. Nur so kann ein genaues Modell erstellt werden. Die Erzeugung dieser Daten ist besonders an den Auslegungsgrenzen einer Anlage und f\"ur transiente Bedingungen kompliziert und kostspielig eventuell sogar \"uberhaupt nicht m\"oglich. \newline
Ein analytisches Modell l\"asst sich prinzipiell durch
\begin{align}
y\of{t}&= f\of{u\of{t},w\of{t},x\of{t},\theta\of{t}} \label{gl:fAna_sysIdent}
\end{align}
formulieren, wobei die Systemausgangsgr\"o\ss{}en $y\of{t}$ in Abh\"angigkeit von den Systemeingangsgr\"o\ss{}en $u\of{t}$, den auf das System wirkenden St\"orungen $w\of{t}$ , den Zustandsvariablen des Systems $x\of{t}$ und den Prozessparameter $\theta\of{t}$ berechnet werden.

Ein vorhandener Prozessfehler wirkt sich bei einer Systembeschreibung nach \eqnref{gl:fAna_sysIdent} direkt auf die Zustandsvariablen und beziehungsweise oder auf die Prozessparameter aus. Diese k\"onnen aber h\"aufig nicht direkt gemessen werden. Die Systemein- und Systemausgangsgr\"o\ss{}en k\"onnen hingegen in der Regel entweder durch Sensoren direkt erfasst oder mit Hilfe mathematischer Modelle beobachtet werden. Die Zustandsvariablen und Prozessparameter k\"onnen daher mit Hilfe geeigneter mathematischer Methoden aus Messwerten von $u\of{t}$ und $y\of{t}$ gesch\"atzt werden. Typische Methoden der Zustands- beziehungsweise Parametersch\"atzung sind die \ac{lqm}, die Verwendung von Kalman Filtern, die Formulierung von geeigneten Beobachterstrukturen oder der Einsatz von Parit\"atsgleichungen.   

Die zur Berechnung der Residuen notwendigen Redundanzbeziehungen basieren auf Ein- und Ausgangsgr\"o\ss{}en, welche voneinander nicht unabh\"angig sind. Die Abh\"angigkeiten k\"onnen durch zus\"atzliche Hardware oder analytische Beziehungen erzeugt werden. L\"asst sich die Redundanz als Gleichung formulieren und werden die Redundanzbeziehungen mit den Modellgleichungen zu einem Gleichungssystem kombiniert, so ist dieses Gleichungssystem \"uberbestimmt. Der entstehende Freiheitsgrad der L\"osung wird zur Entwicklung der Residuen genutzt. \newline
Redundanz durch Hardware entsteht beispielsweise durch mehrere Sensoren, welche die gleiche Gr\"o\ss{}e erfassen. Bei sicherheitstechnisch besonders anspruchsvollen Anwendungen wie der Luft- und Raumfahrt ist dieses Vorgehen trotz der damit verbunden gesteigerten Kosten und dem erh\"ohten Raumbedarf \"ublich. Analytische Redundanz wird erreicht, wenn sich bestimmte Sensorwerte algebraisch aus anderen Sensorwerten berechnen lassen, oder wenn es einen zeitlichen Zusammenhang zwischen der \"Anderung von Messwerten gibt, welcher sich analytisch beschreiben l\"asst. Ein Beispiel f\"ur eine solche Gr\"o\ss{}e ist der F\"ullstand in einem Tank. Werden der Zufluss und der Abfluss durch Sensoren erfasst, so kann der F\"ullstand direkt berechnet werden. Wird trotzdem ein F\"ullstandssensor verbaut, so f\"uhrt dies zu nutzbarer Redundanz, da die zeitlichen \"Anderungen der drei Messwerte zueinander vertr\"aglich sein m\"ussen. Ist dies nicht der Fall, so kann auf einen Sensordefekt, ein Leck oder auf einen anderen Fehler geschlossen werden.

Die auf Basis der analytischen Redundanzen ermittelten Residuen sollen zur Fehlerdiagnose eingesetzt werden k\"onnen. Es ist daher zweckm\"a\ss{}ig, wenn die Residuen bei Vorliegen einer Abweichung vom Sollverhalten des Prozesses signifikante Werte annehmen. Liegt keine St\"orung vor, so sollten die Residuen Werte nahe Null annehmen. Weiterhin ist es g\"unstig, wenn die Residuen robust gegen zuf\"allige Fehler wie Sensorrauschen und systematische Fehler wie Modellungenauigkeiten sind.

Als ersten Verfahrenstyp zur Residuenberechnung stellen \citeauthor{Venkatasubramanian_2003}  die Diagnose mit Beobachtern\footnote{im englischen spricht man von \glqq diagnostic observer\grqq { }oder \glqq unknown input observer\grqq { }(UIO)} vor {(\cite[S. 11 ff.]{Venkatasubramanian_2003})}. Methoden dieser Art entwickeln eine bestimmte Menge an Beobachtern, welche Residuen generieren. Jeder dieser Beobachter wird so definiert, dass er bez\"uglich einer definierten Menge an Fehlern sensitiv und bez\"uglich den restlichen Fehlern und unbekannten Gr\"o\ss{}en unempfindlich ist. Die Menge der Beobachter ist derart zu strukturieren, dass jeder Fehler ein eindeutiges Muster an Residuen zur Folge hat. Wird dies erreicht, so kann das Vorliegen eines Fehlers durch stark von null abweichende Werte der Residuen erkannt und mit Hilfe der bekannten Residuenmuster identifiziert werden. Eine wichtige Besonderheit dieses Verfahren ist es, dass die Sch\"atzung der Zustandsvariablen $x\of{t}$ nicht notwendig ist, statt dessen muss nur der Systemausgang durch Messung oder Sch\"atzung ermittelt werden.

Die Formulierung von Parit\"atsgleichungen ist ein alternatives Vorgehen zur Generierung von Residuen {(vgl. \cite[S. 13 f.]{Venkatasubramanian_2003})}. Bei diesem Vorgehen werden die Modellgleichungen geeignet umgestellt, sodass Residuenvektoren entstehen, die orthogonal zueinander sind. Die Residuenvektoren sind dann linear unabh\"angig und das Auftreten jedes betrachteten Fehlers wird durch genau einen Residuenvektor beschrieben. Voraussetzung zur Einsetzbarkeit dieser Methode ist, dass die Anzahl der Ausgangsgr\"o\ss{}en gr\"o\ss{}er als die der Zustandsgr\"o\ss{}en ist. Dieser Zusammenhang wird in Definition \ref{def:fAna_sysRedundanz} verdeutlicht. 
\begin{defn}[Systemredundanz]\label{def:fAna_sysRedundanz}
Sei ein System nach \eqnref{gl:fAna_sysIdent} beschrieben und gelten die Eigenschaften
\begin{align}
\dim\of{y\of{t}}&= n, &\dim\of{x\of{t}}&= m, & n&>m, \label{gl:fAna_sysRedundanzBdg}
\end{align}
dann ist das System redundant mit dem Freiheitsgrad
\begin{align}
f&= n-m, \label{gl:fAna_sysDimRedundanz}
\end{align}
da das System mehr erfassbare Ausgangsgr\"o\ss{}en als Zust\"ande umfasst.
\end{defn}
Mit Hilfe des Freiheitsgrades $f$ kann dann eine Projektionsmatrix $\matr{V}$ derart entworfen werden, dass f\"ur Abweichungen von jedem redundant vorhandenen Ausgangswert ein Vektor berechenbar ist, der zu den anderen Vektoren dieser Art orthogonal ist. \newline
Parit\"atsgleichungen und die Verwendung von Beobachtern zur Residuenerzeugung \"ahneln sich sehr stark. Beide Verfahren sind ohne eine Sch\"atzung von $x\of{t}$ anwendbar. Man kann sogar zeigen, dass beide Verfahren unter Verwendung der gleichen Designziele zu \"aquivalenten Residuen f\"ur ein fehlerbehaftetes System f\"uhren. Die Methoden der Auswertung von Residuen zur Diagnose von Fehlern sind f\"ur diese beiden Verfahren ebenfalls gleich. \"Ublich ist die Definition von Schwellwerten f\"ur die Residuen, bei deren \"Uberschreiten ein Fehler als vorliegend erkannt wird. 

Es gibt weitere Methoden, welche Residuen auf Basis quantitativer Modelle berechnen um so Fehler zu diagnostizieren und zu isolieren. Dazu z\"ahlen Methoden, welche die Zustandsvariablen oder Prozessparameter sch\"atzen, um auf Basis derer Residuen zu generieren. Dies sind beispielsweise Kalman Filter und \ac{lqm} {(vgl. \cite[S. 14 f.]{Venkatasubramanian_2003})}. Au\ss{}erdem gibt es fortgeschrittene Methoden zur Residuenberechnung wie der Entwurf von gerichteten oder strukturierten Residuen\footnote{engl. directional residuals and structured residuals} {(vgl. \cite[S. 15 f.]{Venkatasubramanian_2003})}. 

\subsection{Bewertung modellbasierter quantitativer Methoden der \ac{fdi} f\"ur modulare Anlagen}
Modellbasierte quantitative Methoden bieten den gro\ss{}en Vorteil, dass der Anwender bei der Wahl eines Verfahrens zur Residuengenerierung viele Freiheiten hat. Auch die Verfahren selbst bieten M\"oglichkeiten, um sie hinsichtlich der Erkennung bestimmter Fehler gezielt zu entwerfen. Werden entkoppelte Beobachterstrukturen geeignet entworfen, so kann jeder betrachte Fehler durch einen gesonderten Beobachter gezielt diagnostiziert werden. Dem gegen\"uber steht der gro\ss{}e Nachteil der Notwendigkeit von m\"oglichst genauen Prozessmodellen. Der Entwurf dieser Modelle ist aufwendig und h\"aufig mit Ungenauigkeiten verbunden. Dies gilt f\"ur analytische Modelle und Black Box Modelle gleicherma\ss{}en. Weiterhin sind Analysemethoden, welche auf quantitativen Modellen basieren, in aller Regel auf die Erkennung von Fehlern, welche additiv auftreten, beschr\"ankt. Die Erkennung von multiplikativ auftretenden Fehlern wie einem Drift von Prozessparametern ist nur in Sonderf\"allen m\"oglich. Dar\"uber hinaus m\"ussen die Residuen zur Erkennung von Fehlern vorab definiert werden. Das Auftreten von vorab  unbekannten Fehlern ist dadurch nur stark eingeschr\"ankt m\"oglich. Auch die Ursachenanalyse ist zumeist nicht m\"oglich -- nur das Vorliegen eines Fehlers wird diagnostiziert und der konkrete Fehler ermittelt.  {\cite[S. 17 f.]{Venkatasubramanian_2003}}

In Hinblick auf modular konstruierte Anlagen l\"asst sich feststellen, dass Methoden dieser Kategorie nicht geeignet sind, um die zur Genehmigung einer aus Modulen bestehenden Anlage notwendige Sicherheitsuntersuchung zu beschleunigen oder anderweitig zu vereinfachen. \newline
F\"ur die Analyse eines einzelnen Moduls k\"onnten jedoch solche Verfahren zum Einsatz kommen. Module sollen entsprechend ihrer Definition einzeln komplett testbar sein. Daher ist die Erstellung von Ein"=/Ausgangsdaten und darauf aufbauend die Entwicklung eines Black Box Modells prinzipiell m\"oglich. Die Erstellung eines analytischen Modells durch den Modullieferanten ist ebenfalls m\"oglich und sollte ohnehin ein Ziel dessen sein, denn auf Basis eines analytischen Modells k\"onnen die im Abschnitt \ref{sec:sdt_modularisierung} geforderten Simulationsmodelle geeignet entworfen werden. Die notwendige Grundlage f\"ur modellbasierte quantitative Verfahren w\"are damit zumindest auf Modulebene gegeben, jedoch verbleiben zwei bedeutende Probleme: \begin{itemize}
\item der aufwendige Entwurf eines analytischen Modells der Gesamtanlage ist notwendig und
\item nicht erkannte Fehler k\"onnen nicht verl\"asslich identifiziert werden.
\end{itemize}
Zum einen ist zu erwarten, dass die Modelle der einzelnen Module noch keine ausreichenden Informationen \"uber die m\"oglichen Wechselwirkungen, welche im Rahmen der Gesamtanlage auftreten k\"onnen, enthalten. Ein Modell der Gesamtanlage m\"usste daher vor dem Einsatz von Verfahren der betrachteten Kategorie noch erstellt werden. Dies w\"are nur durch analytische Ans\"atze m\"oglich. Die zur Generierung von Black Box Modellen notwendigen Ein"=/Ausgangsdaten m\"ussten von der Gesamtanlage stammen, diese ist zum Zeitpunkt der durchzuf\"uhrenden Sicherheitsbetrachtung aber noch gar nicht betriebsf\"ahig. Die Durchf\"uhrung praktischer Tests und die Erstellung von Messdaten ist daher keine zur Verf\"ugung stehende Option und die Erstellung von Black Box Modellen nicht m\"oglich. Die Erstellung von analytischen Modellen ist mit den bereits genannten Problemen des hohen Aufwands und der entstehenden Ungenauigkeiten verbunden. Die notwendige Entwicklungsleistung eines analytischen Modells reduziert damit m\"ogliche Zeiteinsparungen bei der Sicherheitsbetrachtung ma\ss{}geblich. \newline
Zum anderen werden vorab unbekannte Fehler durch modellbasierte quantitative Verfahren nicht verl\"asslich identifiziert. Das Auffinden von bisher nicht betrachteten Fehlern wird also bereits auf der Betrachtungsebene einzelner Module durch Methoden dieser Kategorie nicht erm\"oglicht. Durch das Verbinden von Modulen zu einer Gesamtanlage ist mit neuen Fehlerquellen und f\"ur die Anlage spezifischen m\"oglichen Auswirkungen zu rechnen. Die Erkennung dieser neuen Fehler ist nicht m\"oglich. Die durch Kopplung der Module potentiellen neuen Fehler sind aber genau die Fehler, welche durch die Sicherheitsuntersuchung der Gesamtanlage aufgedeckt werden m\"ussen. Eine Vereinfachung dieser Aufgabe durch die Verwendung von modellbasierten quantitativen Verfahren zur Fehlerdiagnose ist damit nicht zu erwarten. \newline

Im Rahmen der vorliegenden Arbeit soll davon ausgegangen werden, dass die vorhandene Datenbasis aus f\"ur die einzelnen Module durchgef\"uhrten \acp{hazop}, Beschreibungen der Module und einer Beschreibung der Gesamtanlage besteht. Auf dieser Basis l\"asst sich nicht ohne gro\ss{}en Aufwand ein analytisches Modell der Gesamtanlage formulieren. Selbst wenn quantitative modellbasierte Verfahren der \ac{fdi} potentielle Einsparungen bei der Sicherheitsbetrachtung der Gesamtanlage bieten w\"urden, was wie dargelegt wird, nicht der Fall ist, so w\"are die notwendige Datenbasis in keiner weise vorhanden. Im Rahmen dieser Arbeit ergibt sich daher  zwangsl\"aufig, dass der Einsatz von modellbasierten quantitativen Verfahren nicht geeignet ist, um die Fehlerfortpflanzung innerhalb einer aus Modulen bestehenden Anlage zu untersuchen. 
\section{Modellbasierte Qualitative  Fehlerfortpflanzungsmethoden}\label{sec:fAna_modQual}


\section{auf historischen Messdaten basierende Fehlerfortpflanzungsmethoden}\label{sec:fAna_dat}
Diese Verfahren beruhen auf der Verwendung von historischen Messdaten konkreter Anlagen. Je nach Verfahren werden Messdaten zum Normalbetrieb oder beziehungsweise und Daten zum St\"orbetrieb ben\"otigt. Manche Methoden ben\"otigen weiterhin ein \ac{pid}. Ziel der Verfahren ist es zum einen St\"orungen der Anlage fr\"uhzeitig zu erkennen und zum anderen deren Ursache oder Ursachen zu ermitteln. Dazu werden Ursache--Effekt Beziehungen zwischen untersuchten Parametern ermittelt.  

Die Auswertung historischer Messdaten basiert zumeist auf statistischen Methoden. Die kausalen Zusammenh\"ange, welche mit Hilfe dieser Methoden ermittelt werden sollen, k\"onnen dann geeignet als Graphen dargestellt werden. Wie man aus statistischen Gr\"o\ss{}en kausale Zusammenh\"ange ermitteln kann wird in den fr\"uhen Werken \textcite{Holland_1986} und  \textcite{Pearl_1995} aufgezeigt. Ein umfassendes Lehrbuch zu dieser Thematik wurde von \citeauthor{Pearl_2009} ver\"offentlicht, welches mittlerweile in der zweiten Auflage verf\"ugbar ist \cite{Pearl_2009}.

Datenbasierte Methoden sind hervorragend f\"ur die Erstellung von quantitativen Modellen geeignet. Eine Erstellung von qualitativen Modellen ist ebenfalls m\"oglich. \textcolor{red}{Beispiele}
\paragraph*{Nennung von Verfahren}
\cite{Zhang_2017}, \cite{Thornhill_2006}



\section{Vorstellung ausgew\"ahlter Algorithmen}
\section{Bewertung der Verwendbarkeit f\"ur modulare Anlagen}
Wichtige Gesichtspunkte:
\begin{itemize}
\item Welche Daten sind notwendig
\item Automatisierbarkeit der Berechnung
\item Dokumentationsf\"ahigkeit der Ergebnisse
\item Sind die Ergebnisse f\"ur eine \ac{hazop} nutzbar
\end{itemize}
\chapter{Literatursichtung}\label{ch:lit}
%\section{Einleitungsliteratur}
%\paragraph*{\cite{Wachsen_2015}} Relevanz der Spezialchemie und deren erwartetes Wachstum. Abschnitt 5 fokussiert auf die Rolle modularer Anlagen.

%\paragraph*{\cite{Dietz_2000}} Diese Arbeit der Universit\"at Clausthal zeigt ein alternatives Vorgehen bei der Entwicklung verfahrenstechnischer Prozesse. Der Fokus liegt dabei auf die Einbindung der Maschinenkonstruktion in die Verfahrensplanung. Als Ziel sollen innovative Maschinen entstehen, welche Prozesse optimal erf\"ullen k\"onnen. Dieser Ansatz ist das Gegenteil der Verwendung von Modulen im Sinne meiner Arbeit. Hier werden optimierte Sondermaschinen entwickelt und bewusst auf Standardsl\"osungen verzichtet.\hfill \newline
%
%Es ist ein w\"unschenswertes Ziel, bei der Entwicklung von \textbf{neuen Verfahren bzw. Prozessen} die Prozessplanung (was wird mit dem Stoff gemacht: Zerkleinern, Reagieren, Mischen, Trennen usw. damit z. B. ein neuer chem. Stoff gewonnen wird $\mapsto$ R u. I Flie\ss{}bild) und die konkrete Planung und den Entwurf der notwendigen Maschinen (die konkrete Konstruktion der Maschine) zu parallelisieren, um so innovative L\"osungen zu finden (Beispiel innovativer Lsg: die Ausf\"uhrung mehrerer Prozessschritte wie Zerkleinern und Reagieren von Stoffen in einem einzigen, neu entworfenen Apparat bei der Herstellung von Chlorsilanen aus Ferrosilicium und Chlorwasserstoff), welche einen Prozess \textbf{optimal} realisieren. Die Richtlinien VDI 2221 und VDI 2222 reichen dazu nicht aus, da sie zu branchenspezifisch ausgelegt sind (der Wunsch ist ja eine branchen\"ubergreifende, parallele Entwicklung). Eine Prozessentwicklung mittels Fli\ss{}bildern wird als L\"osung vorgeschlagen. Dazu wird eine zu l\"osende Aufgabenstellung in Teilsysteme geringer Komplexit\"at so weit zerlegt, dass sich deren Funktion durch naturwissenschaftliche Grundoperationen darstellen l\"asst. Die gesamte L\"osung der Aufgabe wird dann als Flie\ss{}bild solcher Teilsysteme dargestellt. Die Formulierung einer Funktion  ist dabei losgel\"ost von einer konkreten technischen Umsetzung durch bereits existierende Maschinen bzw. Apparate. Durch diese Darstellung wird ein Blick f\"ur die m\"ogliche Zusammenfassung von mehreren Teilsystemen in einer einzigen Maschine erm\"oglicht. Diese muss dann aber neu konstruiert werden; der Prozess wird aber optimal realisiert. Weiterhin kann die  Notwendigkeit jedes Prozessschrittes besser beurteilt werden. Die Wiederverwendbarkeit einer so entwickelten Maschine ist eher gering, da sie ja als optimale L\"osung von genau diesem einen Prozess entwickelt wurde. Es wird bewusst eine Abkehr von vorfabrizierten L\"osungen gefordert (und damit die Verwendung vorgefertigter Module zumindest erschwert, wenn nicht gar unterbunden) um zu innovativen L\"osungen zu gelangen.
%\textcolor{green}{Fertig. Alternative zu modularisierter Verfahrensplanung}.

%\paragraph*{\cite{Grossmann_2000}} Research challenges in process systems engineering

%\paragraph*{\cite{PerspektiveD_2016}} Lage und Zukunft der deutschen Industrie (Perspektive 2030)

%\paragraph*{\cite{PerspektiveC_2016}} AKTUALISIERUNG: DIE DEUTSCHE CHEMISCHE INDUSTRIE 2030

%\paragraph*{\cite{Schembecker_2009}} 50 \% Idee
%\paragraph*{\cite{Processnet_2010}}50 \% Idee
%\paragraph*{\cite{Processnet_2009}}50 \% Idee
%\paragraph*{\cite{f3_2014}} F3 Factory
%\paragraph*{\cite{copiride_2014}}

%\section{Sicherheit}
%\paragraph*{\cite{Rath_2009}} Erkl\"arungen zu \textbf{quantitativen} Risikoanalysen anhand zweier Beispiele. \hfill \newline
%
%\textit{Alles nur zitiert!} \hfill \newline
%
%Im internationalen Anlagenbau wird in zunehmendem Ma\ss{}e die Durchf\"uhrung einer quantitativen Risikoanalyse gefordert. Die Methodik kann nicht nur zum Nachweis der Einhaltung \"ubergeordneter Akzeptanzkriterien dienen, sondern auch als eine qualifizierte Entscheidungsgrundlage z. B. zu Sicherheitsabst\"anden und -barrieren verwendet werden. Dies kann von nicht probabilistischen quantitativen Verfahren (z. B. HAZOP)) nicht geleistet werden. Durch die Identifizierung der Hauptrisikoquellen in der Anlage erm\"oglicht eine quantitative Risikoanalyse (QRA) zudem die Ableitung von Risikominderungsma\ss{}nahmen, deren
%Wirksamkeit sich mit Hilfe von Sensitivit\"atsberechnungen analysieren und bewerten l\"asst. \textcolor{red}{Quantitative Sicherheitsanalyse}

%\paragraph*{\cite{Kockmann_2017}} \textcolor{red}{Noch lesen!}

%\paragraph*{\cite{Bridges_2010}} Probleme bei der Anwendung von LOPA (Layer of Protection Analysis) (vereinfachte quantitative Sicherheitsbetrachtung ausgew\"ahlter Probleme, wenn eine HAZOP o. \"a zur Identifikation risikoreicher Szenarien bereits durchgef\"uhrt wurde) \hfill \newline
%Ein erstes Zusammenh\"angendes Buch zur Anwendung von LOPA ist 2001 erschienen. Die Methode hat vielerlei Anwendung in der Industrie gefunden, wurde jedoch h\"aufig auch zweckentfremdet. Die gemachten Erfahrungen und Probleme wurden gesammelt und 2010 eine neue Richtlinie Richtlinie zur Anwendung der Methode ver\"offentlicht (zum Zeitpunkt dieses Papers stand dieses 2. Buch noch aus).
%\textbf{Absicht von LOPA:} Risikobewertung eines bekannten Szenarios mit Hilfe unabh\"angiger Schutzschichten (independent Protection Layers IPL), welche durch strenge Regeln definiert werden, und Ausl\"osungsereignissen (initiating events IEs). Durch korrekte Anwendung der Methode ist eine vereinfache Risikobewertung eines Ursache-Wirkung Paares (=Szenario) m\"oglich. Das Auffinden von m\"oglicher St\"orungen ist nicht Teil der Methode, nur die Bewertung von bekannten Szenarien! Die Methode eignet sich besser als eine FMEA f\"ur komplexe Probleme, ohne f\"ur simple Probleme viel zu aufwendig zu sein (wie es bei einer Fehlerbaumanalyse der Fall w\"are). 
%Der aktuelle Nutzen einer LOPA liegt in der Bewertung, ob eine SIF notwendig ist und ob sie die richtige Wahl zur Risikoreduktion darstellt (es existieren auch andere Methoden, welche diesen Zweck erf\"ullen). Wird eine SIF als L\"osung gew\"ahlt, so kann LOPA das notwendige SIL liefern. \textbf{Vorteile LOPA}\begin{itemize}
%\item Konsistente Definition von Schutzschichten, was die Filterung der entscheidenden Schutzeinrichtungen vereinfacht und somit ein umfassendes Sicherheitsmanagement vereinfacht. \item Die Detailbetrachtung mit LOPA kann \"uberfl\"ussige Schutzeinrichtungen identifizieren
%\item Durch die anhand klarer Regeln definierten Schutzschichten kann ein gefordertes SIL besser auf Erf\"ullung \"uberpr\"uft werden, eine \"Ubererf\"ullung durch zu viele SIS wird dadurch weniger wahrscheinlich
%\item LOPA braucht weniger Aufwand als eine QRA, wodurch insbesondere komplexe, schwerwiegende Risikoszenarien schneller quantifiziert werden k\"onnen (Arbeitsaufwand von Stunden statt Tagen)
%\item Durch die konsistenten Bewertungsregeln f\"ur Risiken und das vereinfachende Vorgehen k\"onnen durch verschiedene Expertengruppen gewonnene Analyseergebnisse komplexer Risikoszenarien besser verglichen werden 
%\item LOPA erm\"oglicht das Festlegen eines geeigneten Vorgehens, wenn Schutzschichten z. B. wegen Wartung deaktiviert werden m\"ussen.
%\end{itemize} \textbf{Nachteile/Probleme LOPA} \begin{itemize}
%\item Die Regeln der LOPA werden missachtet. Beispiele \begin{itemize}
%  \item Es wird nicht gepr\"uft, dass Schutzschichten wirklich unabh\"angig von einander sind (Ein Anlagenfahrer darf beispielsweise nur in max. einer Schicht vorkommen!)
%  \item Die Werte von Ausfallraten und anderen statistischen Gr\"o\ss{}en werden ungefiltert aus der Literatur \"ubernommen und nicht das konkrete Umfeld angepasst (z. B. konkrete Betriebsbedingungen)
%  \item Die Sicherheitswerte (richtiger Begriff?) von Schutzschichten und IEs werden w\"ahrend dem Betrieb einer Anlage nicht aufrecht erhalten, da Wartungen und Tests nicht ausreichend (Umfang und Frequenz) geplant werden. Ursache ist fehlende Erfahrung und der Mangel eines standardisierten Vorgehens bei der Wartungs-/ Testplanung, um konkrete Zahlenwerte von IPLs zu erreichen und zu halten. Die Ergebnisse von Tests/Wartung werden nicht ausreichend dokumentiert, insbesondere wird bei nicht-erreichen und fast-nicht-erreichen geforderter IPLs nicht ausreichend weiterverfolgt, wie dies zustande kam. Solche Untersuchungen sind aber notwendig, um statistische Verf\"ugbarkeit genauer mit Zahlenwerten belegen zu k\"onnen. 
%  \item Die durch IPLs verhinderten Auswirkungen werden zu ungenau spezifiziert. Dadurch kommt es zu \"Uber- und Untersch\"atzen von Risiken. Die Erfahrung hat gezeigt, dass Risiken eher \"Ubersch\"atzt werden, wodurch unn\"otig viel Geld f\"ur Schutzma\ss{}nahmen ausgegeben wird.  
%  \item \"Uberverwendung von LOPA. Angedacht ist die Methode f\"ur eine einzige Person im Anschluss an eine HAZOP f\"ur 1-5\% der gefundenen Szenarien. Die Person sollte Teil des Risikobewertungsteams sein, oder mit diesem einfach kommunizieren k\"onnen. LOPA wurde teilweise mit dem gesamten Analyseteam im Rahmen der Risikoanalyse gemacht. Daf\"ur ist die Methode nicht ausgelegt. Der Brainstormingprozess des Teams wird durch das analytische Vorgehen einer LOPA gest\"ort. M\"ogliche Risikoszenarien werden dadurch leicht \"ubersehen. Qualitative und quantitative Betrachtungen sollten zeitlich getrennt ablaufen. Weiterhin ist die Bestimmung der Notwendigkeit und des Grades eines SIL durch das Expertenteam zul\"assig f\"ur SIL-1 und SIL-2.  Nur f\"ur Szenarien, welche f\"ur das Expertenteam zu komplex sind, sollte eine LOPA unter SIL-3 angewandt werden. LOPA wird aber teilweise prinzipiell zur Bestimmung der Notwendigkeit/ des Grades von SIL genutzt. Insbesondere die Entscheidung \"uber die Notwendigkeit eines SIL sollte aber dem Expertenteam im Rahmen der HAZOP \"uberlassen werden.
%  \item \"Uberverwendung von Software: LOPA soll ein Szenario im Detail erkl\"aren, eine IPL definieren/ das Szenario einer IPL zuordnen und die Aufrechterhaltung einer IPL belegen. Dies geschieht in Textform und Software kann daher die Arbeit nur geringf\"ugig unterst\"utzen. 
%\end{itemize}
%\end{itemize}
%Abschluss: LOPA ist ne dolle Sache zur quantitativen Betrachtung. Aktuelle Richtlinie: \cite{ChemicalProcessSafety_2015}
%\textcolor{green}{fertig}

%\paragraph*{\cite{Savkovic_2010}} Reliability and safety analysis of the process plant
% 
%\paragraph*{\cite{Li_2013}} Risk identification and assessment of modular construction utilizing fuzzy analytic hierarchy process (AHP) and simulation

%\paragraph*{\cite{Baig_2013}} Literatur\"ubersicht zum Thema \glqq Fault Tree Analysis\grqq { }. \hfill \newline
%Fehlerbaumanalyse entstammt der Luftfahrtbranche und wurde zun\"achst auch zur Bewertung der Sicherheit von Atomkraftwerken verwendet. Durch die guten Erfahrungen in Bereich der Stromerzeugung ist dieser Ansatz auch in der chemischen Industrie sehr beliebt geworden. Es handelt sich um eine Top-Down Analyse, bei welcher die Ursachen f\"ur einen Fehler ermittelt werden. Insbesondere die Einwirkung von Ger\"ateversagen, menschlichem Versagen und externen Einfl\"ussen wird betrachtet. Ausgehend von einem unerw\"unschten Ereignis wie beispielsweise einem Unfall werden Ereignisse mit Hilfe logischer Gatter verkn\"upft, um die Ursache f\"r eben dieses Unfall zu ermitteln. Die Ursachen-Ereignisse werden dabei Stufenweise in Sublevel unterteilt und erhalten je nach Leveltiefe eine andere Symbolik. Eine Fehlerbaumanalyse l\"asst sich in mehrere Schritte einteilen. Siehe dazu beispielsweise \cite{Ayyub_2014}. \begin{itemize}
%\item Vorteile: 
%  \begin{itemize}
%  \item sehr effektiv bei der Risikobewertung von System moderater Gr\"o\ss{}e
%  \item die m\"oglichen Ursachen eines vom Nutzer vorgegebenen Ereignisses lassen sich detailliert ermitteln und darstellen
%  \item Eine Fehlerbaum kann mit Hilfe von Software erstellt und ausgewertet werden
%  \item sind empirische Daten vorhanden, so kann eine quantitative Aussage \"uber die Eintrittswahrscheinlichkeit f\"ur ein Ereignis gemacht werden
%  \end{itemize}
%\item Nachteile: 
%  \begin{itemize}
%  \item Bei gr\ss{}en Systemen ist die Herleitung des Fehlerbaumes sehr zeitaufwendig. 
%  \item Vollst\"andigkeit kann nicht garantiert werden
%  \item keine Beachtung von Teilausf\"allen m\"oglich. Ein System ist entweder komplett funktionsf\"ahig oder garnicht. 
%  \item die konkrete Struktur einer Fehlerbaumes h\"angt vom Entwickler und dessen Erfahrung/ Vorlieben ab. Das Untersuchungsergebnis eines Systems ist also nicht generisch.  
%  \item Die Eintrittwahrscheinlichkeit eines Ereignisses einer h\"oheren Ebene ist nur m\"oglich, wenn die Eintrittswahrscheinlichkeiten aller Elemente der Subebene verf\"ugbar sind, welche einen Pfad zum Ereignis bilden. Diese Unterwahrscheinlichkeiten sind oft nicht konkret bekannt. Dies ist das mit Abstand gr\"o\ss{}te Problem dieser Methode.
%  \end{itemize}
%\end{itemize}
%Die Arbeit zeigt einige konkrete Anwendungsf\"alle der Fehlerbaumanalyse auf. Genannte  Anwendungsgebiete sind Nuklearreaktoren, schienengebundene Verladestationen f\"ur chemische Stoffe (Analyse der Gefahren beim Be- und Entladen), Verhinderungsma\ss{}nahmen von Suiziden im Bahnverkehr und Analysen zur Verhinderung von Arbeitsunf\"allen durch z. B. Ausrutschen. \hfill \newline
%Weiterhin listet die Arbeit einige Ans\"atze auf, mit Hilfe derer die Nachteile der FTA kompensiert werden sollen. Ziel ist zumeist eine vereinfachte Erstellung des Fehlerbaumes durch  gezielte Betrachtung von Subproblemen und die Computer gest\"utzte Auswertung. Dadurch wird beispielsweise die Betrachtung der dynamischen Entwicklung der Eintrittswahrscheinlichkeit einer St\"orung durch dynamische \"Anderung der Eintrittswahrscheinlichkeiten der Ursachen m\"oglich (z. B. durch Alterung \"andert sich die Ausfallwahrscheinlichkeit eines Ger\"ates, durch steigende Erfahrung eines Anlagenbedieners sinkt dessen Fehleranf\"alligkeit, durch geeignete Wartungsintervalle sinken Ausfallwahrscheinlichkeiten; diese Auswirkungen k\"onnen bezogen auf eine Zeitskala ber\"ucksichtigt werden). Weiterhin wird beschrieben, wie die Wirkung menschlicher Fehler und deren psychologische Ursachen untersucht werden k\"onnen. Es wird weiterhin auf Arbeiten verwiesen, welche den Umgang mit einer bekannten Schwankungsbreite f\"ur eine Ausfallwahrscheinlichkeit darlegen (Nutzung von Fuzzy-Logik).  \hfill \newline
%\textcolor{green}{Fertig.} 

%\paragraph*{\cite{Chung_2002}}
%Die Verwendung der aus der Computerwissenschaft bekannten Methode des \glqq model checking\grqq { }wird verwendet, um die Anlagensicherheit eines Crackers zu bewerten. \hfill \newline
%Traditionelle Methoden zur Sicherheitsbetrachtung wie HAZOP betrachten die verwendeten Sicherheitseinrichtungen nicht explizit. Um dieses Problem zu l\"osen wurde die graphenbasierte Darstellung einer Anlage durch \glqq Process Control Event Diagrams = PCED\grqq { }eingef\"uhrt. Diese stellt den Informationsfluss zwischen Komponenten eines Systems dar (z. B. Anlagenfahrer, Sensor und Regeleinrichtung). Mit Hilfe dieser Beschreibungsform und ein geeigneten Beschreibung der Regellogik der Anlage kann eine Sicherheitsanalyse im Stile einer HAZOP durchgef\"uhrt werden. Dieser Prozess ist jedoch sehr zeitaufwendig. Die Methode des \glqq model checking\grqq { }soll nun genutzt werden, um durch Modellverifikation diesen Analyseprozess automatisierbar zu machen. Das System muss dazu als Zustandsgraph mit Transitionen beschrieben werden. Ein solches Modell kann dann durch geeignete Software (z. B. Symbolic Model Verifier = SMV) mit Hilfe symbolischer Operationen  automatisch untersucht werden. Verwendet man PCEDs, so kann die Modellstruktur teil-automatisiert in ein von SMV lesbares Format umgewandelt werden. Es existieren aber alternative Ans\"atze, um eine Anlage in ein von SMV lesbares Format zu bringen, beziehungsweise komplett andere Formalismen (Condition/Event Systems). Weiterhin existieren Methoden, welche statt symbolischer Operationen mit Hilfe mathematischer Programmierung eine Modellbeschreibung vornehmen. Die Analyse geschieht dann mit Hilfe von \glqq Integer Programming\grqq { }. 
%Diese Arbeit zeichnet sich durch die fr\"uhe Anwendbarkeit im Entwicklungsprozess aus. 
%PCED haben 5 Schichten zur Beschreibung des Informationsflusses. Die PCEDs werden im Detail definiert und die Symbolik wird erl\"autert. 
%Notwendige Grundlage zur Verwendung ist das ausgearbeitete Flie\ss{}bild der Anlage. Mit dessen Hilfe kann der Entwurf einer Sicherheitsfunktion auf Erreichen der gew\"unschten Wirkung untersucht werden. Die referenzierte Beschreibungssprache von SMV ist modular aufgebaut. Sie kann die Wechselwirkung zwischen Modulen abbilden. Es existiert eine Bibliothek zur Beschreibung von PLT Einrichtungen/ Funktionen in SMV. Ein Modulverhalten wird durch diskrete Zustandsvariablen beschrieben. Ein Modul im Sinne von SMV ist aber ziemlich low-level! Beispielsweise beschreibt ein Modul das Verhalten eines Sensor. Das Modul kann Sensordefekte, korrekte Messung und Unter-/\"Uberschreiten von Grenzwerten modellieren. Es ist f\"ur Sicherheitsbetrachtungen also durchaus geeignet. Es wird auf weitere Module der Bibliothek eingegangen (Aktoren, Regler/Controller. Die Wechselwirkung zwischen Modulen wird in einem Main-Modul beschrieben. Als \glqq sicher\grqq { }angesehene Zust\"ande k\"onnen gezielt definiert werden (durch SPEC). Die Anwendung der Methode wird anhand der Temperaturregelung eines Crackers dargelegt. F\"ur diesen existiert angeblich eine ver\"offentlichte Sicherheitsbetrachtung! Verschiedene Szenarien k\"onnen getestet werden, indem man Modulen konkrete Zust\"ande zuweist. Ob dies automatisch gemacht werden kann wird nicht beleuchtet. Es scheint, als ob Fehlerszenarien manuell vorgegeben werden m\"ussen.  \hfill \newline
%\textcolor{green}{fertig. Wenn SMV verwendet wird, so kann die Modellbildung entsprechend dieser Arbeit hier geschehen.}
%\textcolor{red}{Die auf Seite 3 zitierten Arbeiten mal anschauen! Die Modellierung mit SMV im Detail pr\"ufen. }

%\paragraph*{\cite{Fleischer_2015}} Die wichtigste Arbeit bisher. \hfill \newline Sicherheitstechnische Aspekte bei Planung und Bau modularer Produktionsanlagen

%\paragraph*{\cite{Pfeffer_2015}} Vorstellung verschiedener Architekturen, wie Sicherheitsfunktionen (SIF) erf\"ullt werden k\"onnen. Im Hinblick auf Modularisierung ist es von gro\ss{}er Wichtigkeit, wie SIF implementiert werden. Sind SIF in jedem Modul einzeln implementiert, so muss bei Wechsel eines Moduls in erster Linie das Modul selbst validiert sein/werden. Das gleich gilt bei Einf\"uhrung oder \"Anderung einer SIF. Werden SIF durch \"ubergeordnete Sicherheitsregler implementiert, so muss f\"ur eine neue/ge\"anderte SIF nicht nur das Modul selbst, sondern auch eben diese \"ubergeordnete Einheit erneut gepr\"uft werden. Der Aufwand steigt also ma\ss{}geblich an, wenn SIF nicht direkt im Modul implementiert sind.   
%
%\paragraph*{\cite{Graf_2000}} Ein modellbasierter Ansatz zur rechnergest{\"u}tzten Sicherheitsbetrachtung von Chemieanlagen w{\"a}hrend der Planungsphase
%
%\paragraph*{\cite{Graf_2000a}} Early hazard identification of chemical plants with statechart modelling techniques

%\paragraph*{\cite{Lu_2007}} {SDG}-based hazop and fault diagnosis analysis to the inversion of synthetic ammonia

%\paragraph*{\cite{Wang_2009}} {SDG}-based {HAZOP} analysis of operating mistakes for {PVC} process

%\paragraph*{\cite{Florea_2014}} Risk and Hazard Control the new process control paradigm
%
%\paragraph*{\cite{Conti_2013}} A preliminary study of thermal hydraulic models for virtual hazard and operability analysis and model-based design of rotating machine packages
%
%\paragraph*{\cite{Zhou_2012}} Hazard rate models for early detection of reliability problems using information from warranty databases and upstream supply chain
%
%\paragraph*{\cite{Aldemir_1994}} This book contains the proceedings of a NATO Advanced Research Workshop on the Reliability and Safety Analysis of Dynamic Process Systems. \hfill \newline
%\textcolor{red}{Ziemlich alt, trotzdem mal reinschauen.}

%\paragraph*{\cite{Ayyub_2014}}Scheinbar ein Grundlagenbuch zum Thema Risikoanalyse und Reduktion von Risiken \hfill \newline
%Wird bereits im Text zitiert. \hfill \newline
%\textcolor{red}{Auf jeden Fall noch mal reinschauen!.}
%
%\paragraph*{\cite{ChemicalProcessSafety_2007a}} \textcolor{red}{Buch, nachlesen}
%\paragraph*{\cite{ChemicalProcessSafety_2007}} \textcolor{red}{Buch, nachlesen}
%\paragraph*{\cite{ChemicalProcessSafety_2008}} \textcolor{red}{Buch, nachlesen}
%\paragraph*{\cite{ChemicalProcessSafety_2008a}} \textcolor{red}{Buch, nachlesen}
%\paragraph*{\cite{ChemicalProcessSafety_2008b}} \textcolor{red}{Buch, nachlesen}
%\paragraph*{\cite{ChemicalProcessSafety_2009}} \textcolor{red}{Buch, nachlesen}
%\paragraph*{\cite{ChemicalProcessSafety_2009a}} \textcolor{red}{Buch, nachlesen}
%\paragraph*{\cite{ChemicalProcessSafety_2010}} \textcolor{red}{Buch, nachlesen}
%\paragraph*{\cite{ChemicalProcessSafety_2012}} \textcolor{red}{Buch, nachlesen}
%\paragraph*{\cite{ChemicalProcessSafety_2013}} \textcolor{red}{Buch, nachlesen}
%\paragraph*{\cite{ChemicalProcessSafety_2015}} \textcolor{red}{Buch, nachlesen}

%\paragraph*{\cite{Nolan_2014}} Safety and Security Review for the Process Industries: Application of Hazop, Pha, What-If and Sva Reviews

%\paragraph*{\cite{Cochard_2015}}
%Mit Hilfe von Automaten wird durch Anwendung von Erreichbarkeitsanalyse gezeigt, wie durch eine automatisch ermittelte, sichere Folge an Prozessschritten ein gew\"unschter Modellzustand erreicht werden kann. Fokus dieser und \"ahnlicher Arbeiten liegt auf Batchprozessen.  \hfill \newline
%Im Rahmen sicherheitstechnisch kritischer Prozesse muss sicher gestellt werden, dass ein geplantes Prozedere zur Beeinflussung einer Prozessvariable keine Sicherheitsbestimmungen verletzt - es muss verifiziert werden. Konkret muss sichergestellt werden, dass physikalische Gr\"o\ss{}en innerhalb definierter Grenzen verbleiben, maximale Aktorstellgr\"o\ss{}en nicht \"uberschritten werden und dass Einschr\"ankungen bez\"uglich der Verf\"ugbarkeit von ben\"otigten Subsystemen nicht verletzt werden(es werden keine Subsysteme f\"ur das Prozedere angefordert, welche nicht verf\"ugbar sind). Folgende Begriffe sind in dieser Arbeit wichtig: \textbf{Aktionssequenz:} F\"uhrt ein System von einer Istsituation in eine Sollsituation. Die Sollsituation muss durch die Aktionssequenz erreicht werden. Sie besteht aus mehreren, einzelnen Aktionen, welche manuell oder automatisch ausgef\"uhrt werden k\"onnen. Aktionen ver\"andern den Zustand von Equipment und ver\"andern dadurch physikalische Kennwerte.  \textbf{Situation:} Zustand des Systems und Verf\"ugbarkeit von Komponenten. Diese Arbeit will automatisch generierte Aktionssquenzen auf Sicherheit \"uberpr\"ufen. Dazu wird ein Systemmodell entsprechend \textbf{ISA88-Standard} als Kommunikations-Automat entworfen. Dieses Modell wird mit einer Erreichbarkeitsanalyse (mit Hilfe von Modell-\"uberpr\"ufungssoftware) untersucht. An einem Beispiel wird gezeigt, wie auf Basis eines R\& I Flie\ss{}bildes und anhand von zus\"atzlichen \"Uberlegungen (z.B. Verriegelungen und andere sicherheitsrelevante Funktionen) ein Modell in der geforderten Form entworfen werden kann. Der dazu notwendige Prozess ist nur teilautomatisiert. Das entworfenen Modell kann dann auf Erreichbarkeit von Zust\"anden gepr\"uft werden. F\"ur erreichbare Zust\"ande kann dann eine Aktionssequenz generiert werden. Die Aktionssequenzen f\"uhren zu gew\"unschten Situationen und halten (die vorher von Hand definierten) Sicherheitsbestimmungen ein. Sie sind jedoch nicht optimal (sinnloses \"offnen/ schlie\ss{}en von Ventilen m\"oglich).  Weiterhin ist insbesondere f\"ur komplexe Systeme der notwendige Automat riesig (die Anzahl notwendiger Zust\"ande w\"achst exponentiell mit der Anzahl der untersuchten Objekte). Es soll in Zukunft untersucht werden, ob Modelle abstrakter formuliert werden k\"onnen, welche sich genauso wie die Detailmodelle verhalten. Ziel ist die Reduktion notwendiger Zust\"ande. \hfill \newline
%Die Arbeit verweist auf andere Ans\"atze, welche automatisiert eine sichere Aktionssequenz zum erreichen eines Zustandes ermitteln sollen. Die zitierten Arbeiten sind jedoch alle schon recht alt. Es werden unter anderem grafische Methoden, Petrinetze und Statecharts genannt. Die Methoden liefern jedoch entweder nur nicht-ideale Sequenzen, k\"onnen Hierarchien nicht gut abbilden, oder sind nicht in der Lage ein Equipment gleichzeitig mehreren Ebenen einer Hierarchie/ mehreren Funktionen zuordnen.   \textcolor{green}{Fertig. Der Ansatz hat mit meiner Arbeit nicht viel zu tun, zeigt aber die Verwendung von Graphen/Automaten/Zust\"anden zur Sicherheitsbetrachtung von Systemen. Die Suche nach aktuellen Arbeiten zu diesem Thema k\"onnte noch etwas bringen.} \textcolor{red}{die zitierten Ans\"atze zur Modellbildung k\"onnten noch interessant sein. Diese werden am Ende markiert.}  

%\paragraph*{\cite{Schetinin_2013}} Why do verification approaches in automation rarely use {HIL}-test?
%
%\paragraph*{\cite{Herrmann_2000}} A Tool for Hazard Detection in Hybrid Systems
%
%\paragraph*{\cite{Christiansen_2015}} Wissensgest{\"u}tztes Diagnosekonzept durch Kombination von Anlagenstruktur und Prozessmodell
%\textcolor{red}{Dissertation. Sp\"ater noch mal anschauen}

%\paragraph*{\cite{Paltrinieri_2016}}The focus of this book is the emerging topic of dynamic risk analysis, as opposed to traditional risk analysis. Research on how to dynamically assess risk has been carried out in several chemical and petroleum companies, but no real implementation has been attempted. This book is not aimed to be an exhaustive review of dynamic risk analysis; it is rather a concrete support for the application of new risk analysis techniques.

%\paragraph*{\cite{Weber_2015}} Grundlagenbuch zur Sicherheitsbetrachtung von Chemischen Anlagen

%\section{HAZOP}
%\paragraph*{\cite{Dunjo_2010}} HAZOP Literatur\"ubersicht \hfill \newline
%Die Arbeit liefert eine \"Ubersicht zur Entwicklung der HAZOP von den ersten Arbeiten, welche die Methode konkret beschrieben haben %\cite{•}
%1974 bis hin zu aktuellen Arbeiten (2007). Die Ver\"offentlichungen werden in verschiedene Themengebiete eingeteilt. Diese sind: \begin{itemize}
%\item Einf\"uhrung in verschiedene Themen zur Risikoanalyse
%\item Einf\"uhrung in HAZOP; konkrete Guidelines und Erfahrungsberichte zur Anwendung der Methode
%\item Vergleich von HAZOP mit anderen Methoden zur Risikoanalyse; jeweils St\"arken und Schw\"achen der Methoden
%\item Erweiterungen der HAZOP (Quantifizierung, Einfluss des Menschen, Erfahrungsaustausch mit gemachten HAZOPs welcher immerhin 18\% der untersuchten Arbeiten zum Thema hat...)
%  \begin{itemize}
%  \item Erweiterung durch Kombination mit FMEA oder LOPA soll die Qualit\"at der HAZOP verbessern, eine Kombination mit FTA liefert zus\"atzliche, quantitative Aussagen
%  \end{itemize}
%\item HAZOP f\"ur programmierbare Steuerungen, SIL - Zuweisung (ungef\"ahr 22\% der untersuchten Arbeiten)
%\item Automatisierung der HAZOP (durch Software); 40\% der untersuchten Arbeiten (insgesamt etwas mehr als 160 St\"uck) zum Thema HAZOP besch\"aftigen sich mit dieser Thematik
%\item HAZOP und dynamische Simulationen
%\end{itemize}
%Siehe f\"ur noch aktuellere Entwicklungen \cite{Pasman_2016}.
%\textcolor{red}{Insbesondere der Abschnitt 3.5 verweist auf Arbeiten zur Automatisierbarkeit der HAZOP und ist daher f\"ur mich relevant.}
%\textcolor{red}{Die Arbeit noch mal lesen und ausf\"uhrlicher wiedergeben.}

%\paragraph*{\cite{Pasman_2016}} Aktueller Untersuchungen zur Automatisierbarkeit von HAZOP. Wird bereits im Text zitiert
%
%\paragraph*{\cite{Ramzan_2007}} Methodology for the generation and evaluation of safety system alternatives based on extended Hazop
%
%\paragraph*{\cite{Eizenberg_2006}} Combining {HAZOP} with dynamic simulation {\textemdash} Applications for safety education
%
%\paragraph*{\cite{Labovsk__2007}} Model-based {HAZOP} study of a real {MTBE} plant
%
%\paragraph*{\cite{Denti_2010}} New trends for conducting hazard \& operability (HAZOP) studies in continuous chemical processes

%\section{automatisierte HAZOP}
%\paragraph*{\cite{Venkatasubramanian1994}} Vorstellung von {HAZOPExpert}
%\paragraph*{\cite{Vaidhyanathan_1995}} Spezielles Verfahren zur HAZOP als Teil von {HAZOPExpert} Verwendung von \ac{sdg}. Auf dieser Arbeit baue ich meine Implementierung auf. 
%
%\paragraph*{\cite{Vaidhyanathan_1996}} A semi-quantitative reasoning methodology for filtering and ranking {HAZOP} results in {HAZOPExpert}
%
%\paragraph*{\cite{Venkatasubramanian_2000}} {HAZOPExpert}

%\paragraph*{\cite{Catino_1995}} Model-based approach to automated hazard identification of chemical plants

%\paragraph*{\cite{Khan_1997}} {OptHAZOP }{\textemdash} an effective and optimum approach for {HAZOP} study
%
%\paragraph*{\cite{Khan_1997a}} {TOPHAZOP}: a knowledge-based software tool for conducting {HAZOP} in a rapid, efficient yet inexpensive manner

%\paragraph*{\cite{McCoy_1999}} {HAZID}, A Computer Aid for Hazard Identification

%\paragraph*{\cite{Khan_2000}} Towards automation of {HAZOP} with a new tool {EXPERTOP}

%\paragraph*{\cite{Batres_2004}} Entwurf einer Ontologie um die Informationen, welche im Rahmen einer \ac{hazop} gewonnen werden, in einer Art darzustellen, dass sie von einem PC leicht weiter verarbeitet oder wieder verwendet werden k\"onnen. 

%\paragraph*{\cite{Zhao_2005}} {PHASuite}: An Automated {HAZOP} Analysis Tool for Chemical Processes Part I
%
%\paragraph{\cite{Zhao_2005a}} {PHASuite}: An Automated {HAZOP} Analysis Tool for Chemical Processes Part II

%\paragraph*{\cite{Rahman_2009}} {ExpHAZOP}$\mathplus$: Knowledge-based expert system to conduct automated {HAZOP} analysis

%\paragraph*{\cite{Palmer_2009}} An automated system for batch hazard and operability studies

%\paragraph*{\cite{Rossing_2010}} A functional {HAZOP} methodology


%\paragraph*{\cite{Wang_2012}} A new intelligent assistant system for {HAZOP} analysis of complex process plant
%
%\paragraph*{\cite{Boonthum_2014}} A systematic formulation for {HAZOP} analysis based on structural model
%
%\paragraph*{\cite{Mechhoud_2016}}

%\section{Fehlerfortpflanzung}
%\subsection{modellbasiert, quantitativ}
%\subsection{modellbasiert, qualitativ}
%\subsubsection{\cite{Parmar_1987}} Eine Computer basierte Methode zur Identifizierung von Gefahren in kontinuierlich betriebenen Anlagen wird vorgestellt. Das Vorgehen der Gefahridentifizierung orientiert sich dabei stark am Vorgehen in einer \ac{hazop}. Die Methode soll bei einer \ac{hazop} als Erg\"anzung einsetzbar sein. 
%
%Als Basis der Gefahridentifikation dient die Menge alle Prozessvariablen -- erg\"anzt um die Menge aller Guide-W\"orter (mehr, weniger, zu viel, zu wenig...), die im Rahmen einer \ac{hazop} angewandt werden. Die Gesamtanlage wird in Basis Einheiten zerlegt. Eine Basiseinheiten ist durch ihr Verhalten bez\"uglich der Weiterleitung von Gefahren beschrieben. Gefahren k\"onnen entstehen, verst\"arkt, ged\"ampft, weitergeleitet oder aufgehoben werden. Physisch werden Basis Einheiten in Leitungselemente (\glqq lines\grqq { }) (Pumpen, Rohre Ventile...) und Gef\"a\ss{}e (\glqq vessels\grqq { }) (Tanks, Reaktoren, Kolonnen...) unterschieden. Ob ein Fehler durch eine Einheit weitergereicht wird, oder nicht, wird durch definierte Regeln beschrieben. Die Aufstellung dieser Regeln wird im Detail genannt. Au\ss{}erdem wird auf die Funktion des erstellten Computer Programms eingegangen. 
%
%Die Anwendbarkeit der Methode wird an einem konkreten Beispiel gezeigt: \cite{Parmar_1987a}
%
%\textcolor{red}{Methode ist als Grundlage sinnvoll. Weiterentwicklung der Methode suchen!}

%\subsubsection{\cite{Yang_2010}} Ziel ist die Fortpflanzung von Fehler zu ermitteln. Einerseits um w\"ahrend der Entwurfsphase die Auswirkungen eines Fehlers besser absch\"atzen und in Folge dessen Gegenma\ss{}nahmen in geeignetem Umfang entwerfen zu k\"onnen und andererseits um im Betrieb die genau Ursache bzw. die Verkettung von Ursachen ermitteln zu k\"onnen, welche zu einem Alarm gef\"uhrt haben. Dazu werden gerichtete Graphen (\ac{sdg}) verwendet.
%
%Gerichtete Graphen bestehen aus Knoten und gerichteten Kanten, welche die Beziehung zwischen den Knoten darstellen. Die Knoten umfassen bspw. Prozessgr\"o\ss{}en, Arbeitspunkte, Stellgr\"o\ss{}en und bekannte St\"orungen.  
%Die Erstellung eines solchen Graphen ist auf Textbasis mit Hilfe von Grapheditoren/ Werkzeugen wie Graphviz (http://www.graphviz.org/) m\"oglich. Dies erm\"oglicht eine Automatisierung.
%
%Die historische Entwicklung von \ac{sdg} wird kurz benannt. Der Aufbau von \ac{sdg} kann auf Basis von \acp{dae} oder auf Grundlage von Wissen \"uber den Prozess geschehen. F\"ur Standardkomponenten oder Module kann bereits vorab ein \ac{sdg} erstellt werden. Die Informatio n der Kopplung kann aus dem \ac{pid} gewonnen werden. Wird die Struktur des \ac{pid} z. B. als XML oder als Verbindungs- und Adjazenzmatrix abgebildet, so k\"onnen Modul-\acp{sdg} korrekt gekoppelt werden.
%\begin{itemize}
%\item Vorteile einer wissensbasierten Erstellung von \acp{sdg}: 
%  \begin{itemize}
%  \item teilautomatisiert durch Softwareunterst\"utzung m\"oglich
%  \end{itemize}
%\item Nachteile: 
%  \begin{itemize}
%  \item Graph wird nicht plausibilisiert, kein Test auf Kausalit\"at
%  \item Kanten werden nicht gewichtet -> extrem unwahrscheinliche Fehlerfortpflanzungen sind nicht von wahrscheinlichen zu unterscheiden
%  \item Wechselwirkungen zwischen mehreren Eingangsparametern werden nicht abgebildet
%  \item Zeithorizont wird nicht abgebildet, in Folge dessen sind transiente Vorg\"ange nicht von Station\"aren zu unterscheiden
%  \end{itemize}   
%\end{itemize}
%Die Arbeit empfiehlt daher einen \"Ubergang zu datenbasierter Graphenentwicklung. \textcolor{red}{Dadurch wird das schon wieder Mist.}. Durch die Verwendung von Messdaten und den Einsatz gezielter Verz\"ogerungen kann die Korrelation zwischen Gr\"o\ss{}en bestimmt und der Graph konstruiert werden. Die Korrelation muss dabei mit Hilfe geeigneter Verfahren auf Signifikanz untersucht werden. 
%\begin{itemize}
%\item Vorteile einer datenbasierten Erstellung von \acp{sdg}: 
%  \begin{itemize}
%  \item Ergebnis ist kausal und enth\"alt Zeitbezug (Verz\"ogerungen)
%  \end{itemize}
%\item Nachteile: 
%  \begin{itemize}
%  \item nur messbare Gr\"o\ss{}en werden untersucht
%  \item Verfahren zum Signifikantest von Korrelationen sind nicht eindeutig
%  \end{itemize}   
%\end{itemize}
%Fazit: Kombination der beiden Methoden. Wissensbasierter Ansatz schafft komplettes Netz, datenbasierter Ansatz verfeinert den Graphen. Anhand einer Fallstudie wird der Erfolg der Methode gezeigt. 
%
%\textcolor{green}{Fazit: Ansatz der Graphen sieht vielversprechend aus, die Verfeinerung und Verifizierung des \ac{sdg} mit Messdaten wird aber nicht m\"oglich sein. \textbf{Weitere Suche in die Richtung lohnt sich.}}

%\subsection{datenbasiert}
%\subsubsection{\cite{Zhang_2017}} 
%Als datenbasierte Risikoanalyse wird Deep Learning vorgeschlagen. Genau genommen wird die Verwendung eines \ac{dbn} vorgeschlagen. Dessen Schichten bestehen aus \ac{rbn}. Ziel ist eine verbesserte Erkennung von Fehlern.
%
%\"Ubersicht \"uber aktuelle Methoden der datenbasierten Risikoanalyse (als Alternative zu quantitativen oder qualitativen modellbasierten Methoden): \begin{enumerate}
%\item statistische Methoden
%  \begin{itemize}
%  \item principal component analysis (PCA)
%  \item partial least squares (PLS)
%  \item independent component analysis (ICA)
%  \item fisher discriminant analysis (FDA)
%  \item subspace aided approach (SAP)
%  \item correspondence analysis (CA)
%  \item $\rightarrow$ Vergleich der Methoden anhand des \ac{teb} durch  (Yin, et al., 2012)
%  \item Bayesian diagnosis 
%  \end{itemize}
%\item Methods based  on pattern classification
%  \begin{itemize}
%  \item artificial neural network (ANN)
%  \item k-nearest neighbor (KNN) 
%  \item self-organizing map (SOM)
%  \item support vector machine (SVM) 
%  \item artificial immune system (AIS)
%  \end{itemize}
%\end{enumerate}
%Weitere \"Ubersichtsarbeit zum Thema Datenbasierte Fehlerfortpflanzung findet sich in \cite{Thornhill_2006}. 
%
%Die vorgestellte Methode erweist sich bei Anwendung auf den \ac{teb} als sehr erfolgreich. Jedoch wird zum Training der Struktur (wie auch bei allen anderen Risikoanalysemethoden, welche auf Daten basieren) eine gro\ss{}e Mengen an Messdaten erfordert. Insbesondere eine gro\ss{}e Anzahl an Daten zu Fehlfunktionen in der Anlage ist erforderlich um die Fortpflanzung von Fehlern im Netz beschreiben zu k\"onnen.
%
%\textcolor{green}{Fazit: Methode ist wenig Erfolg versprechend f\"ur den Einsatz in modularen Anlagen, bevor die Gesamtanlage bestehen aus Modulen in Betrieb genommen wurde.}

%\subsubsection{\cite{Wang2012}} Verwendung von \ac{pca} in Verbindung mit einer angepassten Variante von \ac{dtw} um insbesondere in der schwierigen Phase des Starts einer verfahrenstechnischen Anlage Fehler erkennen zu k\"onnen. Die Einleitung liefert zahlreiche Verweise zur Verwendung von \ac{pca}. Es kommt die Software \glqq SymCure\grqq { } der Firma \glqq GenSym\grqq { }zum Einsatz, mit Hilfe derer umfangreiche Entwicklungen zum Thema Fehlererkennung und -behebung m\"oglich sind. Die vorgestellte Methode vergleicht online den aktuellen Signalverlauf mit bekannten Verl\"aufen von Fehlerzust\"anden und Normverl\"aufen. Auf diese Weise wird das Auftreten bekannter Fehler diagnostiziert. Neue Fehler werden durch die eingesetzte Software \glqq SymCure\grqq { }erkannt, deren Ursache bestimmt und der Verlauf anschlie\ss{}end der Signaldatenbank hinzugef\"ugt. Die Methode zeichnet sich vor allem durch hohen Recheneffizienz aus.
%
%\textcolor{green}{Der Verweis auf viele Arbeiten zu \ac{pca} und auf die verwendete Software ist hilfreich. Au\ss{}erdem kann die Arbeit als Beispiel verwendet werden.}

%\subsubsection{\cite{Mallick_2013}} Entwicklung einer echtzeitf\"ahigen Methode zum Erkennen des Vorliegen eines Fehlers und Identifikation der Grundursache. Verwendung von \ac{pca} in Kombination mit \ac{bbn}. Aus Daten zum Normalbetrieb wird das \ac{pca} aufgebaut. Im laufenden Betrieb kann dann damit das Vorliegen eines Fehlers erkannt werden. Mit Hilfe historischer Daten zur Prozessdynamik oder durch Formulierung von Differentialgleichungen oder durch Expertenwissen wird das \ac{bbn} aufgestellt. Auf Basis des \ac{bbn} kann bei vorliegen eines Fehlers dessen Ursache ermittelt werden. Das initiale \ac{bbn} wird mit fortschreitender Betriebszeit durch Informationen aus der \ac{pca} verbessert um die Ursachenanalyse genauer zu machen.  
%
%\textcolor{green}{Sch\"one Methode, welche statische Verfahren (\ac{pca}) mit Prozesswissen (\ac{bbn}) vereint, die ben\"otigten Daten zum Gesamtprozess, welche f\"ur \ac{pca} notwendig sind, liegen bei modularen Anlagen aber nicht vor $\mapsto$ bringt nichts}

%\subsubsection{\cite{Liang_2017}}Der Einfluss von Alterungserscheinungen in Mehrkomponentensystemen wird untersucht. Besteht ein Systemen aus sicherheitsrelevanten und nicht sicherheitsrelevanten Systemen, so kann die Alterung/ der Verschlei\ss{} der nicht sicherheitsrelevanten Systemen die Alterung der sicherheitsrelevanten Systeme durch Fortpflanzung der Alterungserscheinung beschleunigen/ beeinflussen. Setzt sich durch Alterung beispielsweise ein Filter langsam zu, so muss eine Pumpe immer mehr Druck auf eine Fl\"ussigkeiten aufbauen, um einen konstanten Volumenstrom hinter dem Filter zu gew\"ahrleisten. Die Pumpe verschlei\ss{}t dadurch schneller. Eine solche Wechselwirkung kann geeignet durch Verwendung einer multi-layered vector-valued continuous-time Markov chain abgebildet werden. Dazu werden statistische Daten ben\"otigt.
%
%\textcolor{green}{Fazit: Die Fortpflanzung \glqq normaler\grqq { }Fehler in Sinne der \ac{hazop} kann dadurch nicht ohne aufwendige Umformungen/ Adaptionen abgebildet werden.} 

%\subsubsection{\cite{Bauer_2008}}
%Die vorgestellte Methode verwendet historische Daten um die Fehlerfortpflanzung zu untersuchen.  Mit Hilfe von Kreuzkorrelation wird die Zeitverz\"ogerung zwischen Prozessgr\"o\ss{}en berechnet und daraus eine Struktur der Abh\"angigkeiten abgeleitet. Aus der Kreuzkorrelation wird eine kausale Abfolge von Ereignissen konstruiert, dass die Fortpflanzung eines Fehlers in einem Graphen dargestellt werden kann. 
%
%\textcolor{green}{Die Methode ben\"otigt Messdaten der Gesamtanlage und kann daher nicht verwendet werden.}

%\subsubsection{\cite{Wang_2016}} Die Methode basiert auf der Auswertung von Residuen und einer \glqq contribution plot analysis\grqq { }. Die Residuen werden durch Identifikation von Parit\"ats- und Unterr\"aume berechnet. Die optimalen Residuen werden dann genutzt, um ein Fehleridentifikationsschema zu erstellen. Auf Basis des Durchschnitts aktueller und historischer Residuen wird ein Index definiert. Die Fehlerfortpflanzung kann dann anhand dieses Index ermittelt werden. Anhand eines \ac{teb} wird die Wirksamkeit des Verfahrens demonstriert.
%
%Die Einleitung gibt einen umfassenden \"Uberblick \"uber datenbasierte Methoden und referenziert zahlreiche Quellen.  
%
%\textcolor{green}{Die Methode ist f\"ur meine Zwecke nicht geeignet, das Einleitungskapitel lohnt sich aber wegen der vielen datenbasierten Verfahren noch mal anzusehen.}

%\subsubsection{\cite{Hajihosseini2014}} Datenbasierter Ansatz welcher ein qualitatives Modell mit Hilfe statistischer Entropien liefert.  Das Verfahren wurde erfolgreich eingesetzt, um einen gerichteten Ursache-Wirkungsgraphen zu erstellen (Quellen werden genannt). Das Verfahren wird derart erweitert, dass einzeln auftretende Fehler, welche eine erkannte St\"orung verursacht haben, identifiziert werden k\"onnen.

%\subsubsection{\cite{Li_2016}} Komplettes Framework zum Erkennen von Fehlern und Identifikation der zugrunde liegenden Ursachen. 
%
%\textcolor{green}{Das erste Kapitel verweist auf viele Arbeiten zum Thema datenbasierte Analyse und benennt bekannte Probleme.}

%\subsection{hybride Methoden}
%\subsubsection{\cite{Hu_2015}} 
%\textcolor{green}{modellbasiert qualitativ und datenbasiert} \linebreak
%Fehlerfortpflanzung mit Hilfe von Dynamischen Bayesschen Netzen. Grundlage bietet die \ac{hazop} der Gesamtanlage. Ziel ist es w\"ahrend dem Betrieb einer Anlage bei Auftreten eines Fehlers (eines Alarms) dessen wahrscheinlichste Ursache oder Ursachen schnell zu ermitteln und den Pfad von der Ursache oder den Ursachen zum Fehler/Alarm einfach lesbar (grafisch) darstellbar zu machen. Dem Operator wird so die Wahl, welche Aktion zum Beheben des Fehlers notwendig ist, ma\ss{}beglich vereinfacht.
%
%Grundlage des Netzes sind eine Struktur, welche die Entwicklung von Fehlern abbildet und eine Menge von Messdaten, welche das Netz verfeinert. \linebreak
% Die statischen Knoten des Netzes werden aus einer voran gegangen \ac{hazop} ermittelt. Jeder Knoten entspricht dabei einer in der \ac{hazop} untersuchten Prozessvariable. Alternative Verfahren dazu sind die Durchf\"uhrung einer \ac{fta} oder die Anwendung der \textbf{Bow Tie Methode}. Insbesondere eine \ac{fta} w\"urde die Fortpflanzung von Fehlern genauer beschreiben und w\"are daher zu bevorzugen. Die Informationen, welche aus einer \ac{hazop} entstehen, werden aber als ausreichend betrachtet. Die statischen Knoten werden um dynamische Knoten erg\"anzt. Sie repr\"asentieren (versteckte) Fehler-Modi des Systems. Ihre genauen Werte werden aus Messdaten aus dem Live-Betrieb ermittelt.  \linebreak
%Neben der Festlegung der Knoten muss die Struktur bestimmt werden. Der L\"osungsaufwand dieses Problems w\"achst exponentiell mit der Anzahl an Knoten. Mit dem Wissen aus der \ac{hazop} (dem Wissen, wie sich ein Fehler fortpflanzt) kann der L\"osungsraum bei der Strukturberechnung des Netzes aber stark eingeschr\"ankt werden. Zu beachten ist dabei, dass im Rahmen der \ac{hazop} in der Regel nur direkt benachbarte Bauteile (Knoten, definierte Untersuchungseinheiten) auf Wechselwirkung hin untersucht werden. Weiterf\"uhrende Fortpflanzungen werden nicht betrachtet.  Die Berechnung der optimalen Struktur eines Dynamischen Bayesschen Netzes kann durch verschiedene Algorithmen erfolgen. Eine Variante ist der \textbf{K2 Algorithmus} von Cooper und Hersovits, 1992. Dazu werden jedoch Messdaten ben\"otigt, welche die zeitliche Entwicklung eines Fehlers abbilden (aus Simulation gewinnbar?). Die Aufstellung der Conditional probability tables (CPT) erfordert weiterhin gro\ss{}e Mengen an Messdaten. Das erhaltene Netz kann dann aber die Entwicklung von Fehlern \"uber mehrerer Ebenen hinweg ermitteln / untersuchen. Eine Darstellung der Netzentstehung findet sich in \figureref{fig:Hu_2015_netzentstehung}. 
%
%\textcolor{green}{Fazit: Methode sieht schick aus, Datenbasis ist bei neuer Anlage aber nicht gegeben. Nur eine bestehende (Komplett-!) Anlage mit historischen Messdaten kann mit dieser Methode tiefgreifend analysiert werden. Eine Sicherheitsanalyse kann dann damit verbessert werden.} 
%
%\begin{figure}[h!tb]
%\begin{center}
%\includegraphics[width=0.5\textwidth]{bilder/Hu_2015_Netzentstehung.png}
%\caption{Erstellung eines Dynamischen Bayesschen Netzes}
%\label{fig:Hu_2015_netzentstehung}
%\end{center}
%\end{figure}
%
%\subsubsection{\cite{Cai_2015}} Resiliente Systeme, Verwendung eines \ac{hfpm} als Erweiterung der \ac{irml}. Ziel des Modells ist die Beschreibung des dynamischen Verhaltens und die dynamische Fortpflanzung von Fehlern. Des neue Modell hat zwei Teile: statische Analyse der Struktur und dynamische Analyse f\"ur Fehlerfortpflanzung. Untersucht werden Fehler durch Abweichungen von Stoffparametern und Ger\"ateausf\"alle. Der Fokus liegt auf der Analyse der Resilienz (F\"ahigkeit des Systems trotz Fehler in definierten Zustand zu gelangen). \linebreak
%
%Wird eine \ac{hazop} um ein Strukturmodell erg\"anzt, so kann eine Fehleranalyse systematisch durchgef\"uhrt werden ( Boonthum et al., 2014 ). Eine \ac{hazop} erfasst aber nur die Abh\"angigkeit von Prozessgr\"o\ss{}en und nicht die der Komponenten eines Systems, eine makroskopische Betrachtung von Fehlerfortpflanzungen ist daher nicht m\"oglich. Die folgenden Methoden zur makroskopischen Betrachtung werden genannt: 
%\begin{itemize}
%\item Utne et al. (2011): cascade diagram
%\item Johansson and Hassel (2010): agent based approach $\mapsto$ ben\"otigt detaillierte Prozesskenngr\"o\ss{}en und Messwerte, quantitativ
%\item  Kj{\o}lle et al. (2012) across-sector approach $\mapsto$ ben\"otigt vielseitiges Wissen und umfangreiche Daten \"uber den Prozess, diese sind oft nicht vorhanden
%\item Haimes and Jiang (2011) Framework basierend auf Leontief’s input-output model $\mapsto$ ungeeignet f\"ur die Analyse von Verkn\"upfungen auf Komponentenebene
%\item SDG \textcolor{green}{insbesondere in Verbindung mit \ac{hazop}}
%\item Functional Resonance Analysis Method (FRAM) $\mapsto$ emphasizes the functional relationship between components and provides a way to describe the fault evolution
%\item fault semantic network (FSN) using genetic programming (GP) and neural networks (NN) $\mapsto$ braucht viele Daten und Prozesswissen
%\item Petrinetze
%\item hierarchical Bayesian model (HBM)
%\item Markov network combined with Bayesian theory
%\end{itemize}
%Vorteile \ac{hfpm}: \begin{enumerate}
%\item Fehleranalyse auf Systemebene (statt Komponentenebene)
%\item durch Simulation k\"onnen im vorweg geeignete Sicherheitsma\ss{}nahmen ermittelt werden, um gew\"unschte Sicherheitslevel zu erreichen
%\end{enumerate}
%Systemaufbau: \begin{itemize}
%\item basierend auf \ac{irml}
%\item ein System wird in Subsysteme gegliedert, die Struktur durch ein hierarchisches Framework dargestellt, die Gliederung in mehrere Ebenen von Subsystemen stellt die Prozessstruktur und Fehlerszenarien dar
%\item statische Analyse: zeigt den Grad der Abh\"angigkeit von Subsystemen auf, das System wird auf die Teile mit den st\"arksten Wechselwirkungen reduziert und diese Teilsysteme werden im Detail weiter untersucht
%\item dynamische Analyse: Anwendung von Testf\"allen; durch Wenn--dann Analyse werden die Fortpflanzung eines bekannten Fehlers und die Operationen des Systems, welche notwendig sind, um in einen stabilen Zustand zur\"uck zu kehren, im Modell so eingestellt, dass das Verhalten dem Testfall entspricht, mit Hilfe geeigneter Parameter, welche vor allem das zeitliche Verhalten widerspiegeln, wird das Verhalten gekennzeichnet und darauf aufbauen ein Zustandsgraph ermittelt
%\end{itemize}
%
%\textcolor{green}{Fazit: statische Analyse ist hilfreich, bietet aber keinen besonderen Mehrwert, dynamische Analyse erforderte historische Messdaten, Aufteilung auf Module ist fragw\"urdig, Analyse der anf\"alligsten Subsysteme ist hilfreich, Fokus liegt auf Resilienz und Berechnungen zum Sicherheitslevel insbesondere w\"ahrend ein Fehler sich im System verbreitet und das System darauf reagiert, f\"ur meine Arbeit eher nicht zu gebrauchen}
%\textcolor{red}{die verwiesenen Methoden zur makroskopischen Betrachtung k\"onnten hilfreich sein.}

%\subsection{Rezensionen}
%\begin{itemize}
%\item Einf\"uhrung in \cite{Zhang_2017}: \textcolor{green}{\"Ubersicht datenbasierte Methoden}
%\item Einf\"uhrung in \cite{Yang_2010}: \textcolor{green}{Entwicklung von Methoden zu \ac{sdg}}
%\item Einf\"uhring in \cite{Mallick_2013}: \textcolor{green}{Verweise auf mehrere Methoden/Anwendungen von \ac{sdg}}
%\item Einf\"uhrung in \cite{Cai_2015} \textcolor{red}{Methoden zur makroskopischen Fehlerfortpflanzung $\mapsto$ genau das will ich eigentlich }
%\item Einf\"uhrung in \cite{Wang_2016}: Datenbasierte Fehlerfortpflanzung
%\item \cite{Thornhill_2006}: Datenbasierte Fehlerfortpflanzung
%\item \cite{Yin_2014}: Datenbasierte Fehlerfortpflanzung
%\item \cite{Varga_2013}: Datenbasierte Fehlerfortpflanzung
%\item \cite{Hwang_2010}: Modellbasierte und stochastische Methoden der Fehlerfortpflanzung; Reglerneueinstellung nach Fehlererkennung
%\item \cite{Ng_2010}
%\item \cite{Zhang2008} \textcolor{green}{Umfassende \"Ubersicht zum Thema Fehlerfortpflanzung und Fehleridentifikation}
%\item \cite{Yang2012}
%\item \cite{Duan2014}: \textcolor{green}{Ursachenanalyse von anlagenweiten, schwingenden Fehlern}
%\item Einf\"uhrung in \cite{Li_2016}: \textcolor{green}{\"Ubersicht zu datenbasierten Methoden}
%\item \cite{Venkatasubramanian_2003}: \textcolor{green}{quantitative modellbasierte Methoden}
%\item \cite{Venkatasubramanian_2003a}: \textcolor{green}{qualitative modellbasierte Methoden}
%\item \cite{Venkatasubramanian_2003b}: \textcolor{green}{datenbasierte Methoden}

%\end{itemize}


%\subsection{unsortiert}
%\paragraph*{\cite{Bartolozzi_2000}} Qualitative models of equipment units and their use in automatic {HAZOP} analysis

%\subsubsection{\cite{Kavcic_2001}}
%Verfahren zur Fehlerbetrachtung werden in off-line und on-line Modelle unterschieden. Off-line sind bspw. \ac{hazop}, \ac{fta}, \ac{fmea}. On-line Verfahren basieren auf statistischen Methoden oder Mustererkennung.  

%\section{Modularisierung}

%\paragraph*{\cite{Bramsiepe_2012}} Sichtweisen auf die Modularisierung chem. Produktionsanlagen. Anforderungen an Module. Grundlegende Gedanken und notwendige Schritte zur Verwendung von Modulen. \hfill \newline
%Verk\"urzte Lebenszyklen chemischer Produkte und st\"arker schwankende Absatzm\"arkte erfordern k\"urzere Entwicklungszeiten neuer Produkte. Modularisierte Anlagenteile wurden als m\"ogliches Mittel identifiziert, um bereits gewonnenes Wissen wiederverwenden zu k\"onnen und so Entwicklungszyklen zu beschleunigen. Aus wirtschaftlicher Sicht konnte die Sinnhaftigkeit kleinskaliger Anlagen bereits gezeigt werden (Quellen sie diese Arbeit). Diese erlauben hohe Flexibilit\"at und eine schnelle Anpassung der Produktionskapazit\"at an Marktver\"anderungen. Mit Hilfe kleinskaliger Anlagen k\"onnen insbesondere Zwischenprodukte einzeln und \"ortlich verteilt hergestellt werden. Die Weiterverarbeitung zu einem Endprodukt erfordert dann nur noch den Transport dieser Zwischenprodukte und nicht mehr den Transport aller Ausgangsstoffe. Dadurch sinkt das zur Endproduktgewinnung notwendige Transportvolumen und damit die Kosten. Ein weiterer Vorteil der Modularisierung ist die M\"oglichkeit einen Gro\ss{}teil der Anlagenmontage an einem beliebigen Ort unter optimalen Bedingungen vornehmen zu k\"onnen. Am Aufstellungsort der Anlage m\"ussen die Module dann nur noch verbunden werden. Dies ist insbesondere bei klimatisch anspruchsvollen Anlagenstandorten sehr hilfreich. Ein Modul muss derart definiert werden, dass es einen hohen Grad an Wiederverwendbarkeit besitzt und l\"osgel\"ost von einer Gesamtanlage getestet werden kann. Module sollten nach ihrem Detaillierungsgrad unterschieden werden. Die Aufteilung in Planungsmodule und Variantenmodule wird als sinnvoll erachtet. Variantenmodule werden in 2D und 3D Varianten unterteilt. Ein 2D Variantenmodul soll Informationen enthalten, welche am Ende des Basic Engineering vorhanden sind. Dies umfasst alle Informationen, welche zum Entwurf eines R\& I Flie\ss{}bildes f\"ur ein Modul notwendig sind. Ein 3D Variantemodul ist um Auslegungsgr\"o\ss{}en derart erweitert, dass die Modulfertigung m\"oglich ist. Weiterhin sind der Planungs- und Entwicklungsprozess geeignet dokumentiert. Die konkrete Auslegung der 3D Variantenmodule kann direkt durch einen Equipmentlieferanten durchgef\"uhrt werden, wenn dieser detaillierte Modulanforderungen und Raumforderungen erh\"alt. Insbesondere eine genau Definition der Schnittstellen ist hierbei notwendig.  
%Ein Planungsmodul bietet Ans\"atze zu Auswahl, Funktionsumfang, Auslegung und Dimensionierung von Variantenmodulen. Planungsmodule stellen also in erster Linie einen Wissensspeicher dar und dienen der Darstellung der Vielfalt von Variantenmodulen. Wird durch einen Wissens-/ Erfahrungsgewinn ein bestehendes Variantenmodul weiterentwickelt, so m\"ussen die Planungsmodule erweitert und angepasst werden.
%Im Planungsprozess hat der Detaillierungsgrad der verwendeten Variantenmodule ma\ss{}geblichen Einfluss. 2D Module erleichtern die Erzeugung von Flie\ss{}bildern einer Gesamtanlage. Insbesondere erm\"oglichen sie den einfachen Vergleich verschiedener Anlagenstrukturen. Es wird auf Literatur verwiesen, welche die zur Erlangung von 2D und 3D Modulen notwendigen Arbeitsschritte darlegen. Mit Hilfe von Simulationen k\"onnen in Kombination mit Planungsmodulen geeignete 3D Module f\"ur einen Prozess ausgew\"ahlt werden. Da diese nur in definierten Gr\"o\ss{}en zur Verf\"ugung stehen wird der Prozess aber wahrscheinlich nicht optimal umgesetzt. Dies muss bei der Planung beachtet werden, z. B. indem der Verfahrensprozess an vorhandene Modulgr\"o\ss{}en angepasst wird. Auf ein solches Vorgehen wird verwiesen (\cite{Seifert_2012}). Bei der Entwicklung von Regelungs- und Sicherheitskonzepten muss betrachtet werden, welche Aufgaben ein einzelnes Modul losgel\"ost vom Gesamtsystem erf\"ullen kann und welche Aufgaben nur im Zusammenspiel mehrerer Module gel\"ost werden k\"onnen. Die implementierten F\"ahigkeiten des Module bestimmen also ma\ss{}geblich den Entwicklungsaufwand neuer Sicherheitsfunktionen einer Gesamtanlage. 
%Die einzelnen Schritte der Projektplanung \"uerlappen sich bei der Verwendung von Modulen zwangsl\"aufig. Um eine effiziente und effektive Projektplanung sicherzustellen ist daher die Entwicklung eines geeigneten Datenformates notwendig. Nur so kann eine st\"andige Weitergabe und Verf\"ugbarkeit notwendiger Information gew\"ahrleistet werden.  
%Notwendige Forschungsarbeiten, um Module verwenden zu k\"onnen: \begin{itemize}
%\item Systematischer Entwurf von 2D, 3D Variantemodulen und Planungsmodulen(Systematik des Modulentwurfs definieren)
%\item Berechnungsmodelle zum Scale-Up von Modulen
%\item Simulationsmodelle von Modulen, um deren Variantenauswahl und konkrete Auslegung durchf\"uhren zu k\"onnen
%\item (Weiter-) Entwicklung von Ans\"atzen, wie Module konkret in den Planungsprozess integriert werden k\"onnen
%\item Entwicklung eines Datenmodell um Datenaustausch und Datenanreichung zu erm\"oglichen
%\end{itemize}
%\hfill \newline
%\textcolor{green}{Inhaltlich fertig.} \textcolor{red}{Arbeiten, welche diesen Artikel zitieren, sind wahrscheinlich hilfreich}.

%\paragraph*{\cite{Kampczyk_2003}}
%Vorstellung eines CAE-Tools zur Anlagenplanung, insbesondere zur Wahl der Positionierung von Anlagenkomponenten, Entwicklung der Vorrohrung und dem Stahlbau. Durch Vorgabe von Nutzerw\"unschen wird auf Basis einer Wissensdatenbank die optimale Positionierung und Leitungsf\"uhrung berechnet, indem alle Varianten (oder eine definierbare Anzahl an Iterationen) an m\"oglichen Kombinationen automatisch durch einen Optimierer berechnet und bewertet wird. Durch Definition von Modulen wird die Anzahl m\"oglicher Varianten reduziert und der Rechenaufwand reduziert. Au\ss{}erdem k\"onnen best-practice L\"osung direkt vorgegeben werden. \hfill \newline
%Modul $=$ Ausr\"ustungen, welche Teil des gleichen Prozessschrittes sind. \textcolor{red}{Diese Arbeit hilft die verschiedenen Verst\"andnisse vom Modulbegriff darzulegen}.

%\paragraph*{\cite{Hady_2012}}
%Das vorgestellte Konzept umfasst die Definition und Identifizierung von Modulen, deren dreidimensionales Design, die Ablage und Know-how-Sicherung zwecks der Wiederverwendung des Engineering und der Ausrüstungen sowie die Planung und Kostenschätzung mit wiederverwendbaren Modulen.\textcolor{red}{Arbeiten, welche diesen Artikel zitieren, sind wahrscheinlich hilfreich}.

%\paragraph*{\cite{Urbas_2012}} \textcolor{red}{\"Ubersichtsbeitrag Modularisierung, offene Forschungsfragen}

%\paragraph*{\cite{Lier_2016}} Relevanz modularisierter Anlagen
%
%\paragraph*{\cite{Obst_2013}} Relevanz modularisierter Anlagen
%
%\paragraph*{\cite{Ohle_2014}} Konkretes Beispiel einer Modularen Anlage


 
% \paragraph*{\cite{Uzuner_2012}, \cite{Uzuner_2013}} Ans\"atze im Paper, Details in der Dissertation \hfill \newline
%Die Entwicklung des R{\&}I-Flie{\ss}bild ist ein wichtiger Schritt eines Anlagenprojektes. In der Arbeit wird gezeigt, wie dieser Prozess durch Unterteilung einer Gesamtanlage in wiederverwendbare Funktionsgruppen (=Module bzw. EQM) und die Verwendung einer wissensbasierten Software geeignet beschleunigt werden kann. Module sollen derart definiert werden, dass sie prozesstechnisch sinnvoll sind und einen m\"oglichst hohen Grad an Wiederverwendbarkeit aufweisen. Ein solches Modul wird als Equipment-Modul (EQM) bezeichnet; gemeint sind damit Standard-Prozesseinheiten wie Pumpen, Verdichter, W\"arme\"ubertrager, Beh\"alter, Reaktoren, Kolonnen. Ein EQM erf\"ullt dann eine prozesstechnische Funktion und umfasst einen Apparat/Eine Maschine und weiterhin die notwendigen Elemente der Sicherheitstechnik, Regelungstechnik, Nahverrohrung und Instrumentierung. F\"ur solche EQM wurden R{\&}I-Flie{\ss}bilder entworfen. Ein EQM kann dabei in Subsysteme zerlegt werden, welche derart definiert werden, dass sie anschlussfertige Elemente f\"ur EQM bilden (z. B. Teilmodul, Baugruppe, Unterbaugruppe). Diese Verschachtelung erfordert eine Erweiterung der \textcolor{red}{NE33 - Datei besorgen}. Der Zusammenbau eines EQM zur Erf\"ullung einer bestimmten Funktion kann durch Kombination verschiedener Subsystemen geschehen (verschiedene Kombinationen bilden verschiedene EQM zur Erf\"ullung der gleichen Aufgabe). Dadurch k\"onnen neue oder abge\"anderte Prozesse flexibel realisiert und Variationen eines EQM bez\"uglich der Kosten verglichen werden. Bereits entworfene EQM bilden die Basis der wissensbasierten SW. Neue EQM und deren Subsysteme erweitern die SW. Neue Anlagen werden dann aus diesem Datenhaushalt - den bereits geplanten Modulen - entworfen, oder die Datenbasis wird zuerst erweitert. Bereits geleistete Entwicklungsarbeit kann so wiederverwendet werden. Durch die Verwendung von Standardbausteinen sinkt die Entwicklungsdauer, wird die Qualit\"at der Planungsarbeit erh\"oht, erlangtes Wissen nachvollziehbar gespeichert und die Dokumentation von Designentscheidungen erleichtert. Die Struktur der SW und die Handhabung/ der Nutzen der SW im Workflow werden erl\"autert. Der Planungsgrad ist auf 2-D Level. Die r\"aumliche Anordnung von Komponenten eines EQM wird nicht betrachtet/ modelliert. \textcolor{green}{fertig}

%\paragraph*{\cite{Lang_2012}} Die aus Standard-Modulen aufgebaute Anlage hat sich bisher noch nicht im gro\ss{}en Ziel durchgesetzt. Im Bereich der Kleinanlagen ist die Entwicklung von Containermodulen jedoch weit voran geschritten. Es existiert bereits wenigstens ein kommerzielles Produkt - der \textit{Evotrainer}. \textcolor{green}{fertig}

%\paragraph*{\cite{Rottke_2012}} Am Beispiel einer Anlage zur Hochleistungsfl\"ussigkeitschromatographie (welche in der Literatur detailliert beschrieben ist) wird gezeigt, wie der Ansatz der modularisierten Anlage praktisch genutzt werden kann. Der Prozess wird in Module zerlegt, welche in verschiedenen Skalierungsvarianten entworfen werden. Die zus\"atzlichen Kosten, wenn ein Modul in Zwischengr\"o\ss{}e ben\"otigt wird, wird dargelegt. Es wird ein Paket an Werkzeugen vorgestellt, mit Hilfe derer die ben\"otigte Dimensionierung von Modulen f\"ur einen Prozess durch Simulation berechnet wird und eine kostenbasierte Empfehlung abgegeben, welche Modulvariante aus einem Katalog definierten Gr\"o\ss{}en gew\"ahlt werden sollte. \hfill \newline
%Ein Ansatz der deutschen Anlagenbauer und -betreiber ist die Entwicklung von High--Tech--Produkten um im globalen Wettbewerb erfolgreich zu bleiben. Dazu ist unter anderem eine stark verk\"urzte Projektzeit bei der Entwicklung neuer Produkte notwendig. Die Verwendung von Modulen bei der Prozessentwicklung ist ein m\"oglicher Ansatz. Im ersten Schritt ist allerdings ein erh\"ohter Engineering Aufwand zur Definition und Erstellung/Beschreibung von Modulen notwendig. Dieser Mehraufwand ist gerechtfertigt, wenn ein entworfenes Modul mehrfach verwendet werden kann. Die erbrachte Engineering--Leistung kann damit wiederverwendet werden. Insbesondere die Dokumentation, wie ein Modul entstanden ist (die konkreten Designentscheidungen) ist daf\"ur notwendig.  \hfill \newline
%Das R{\&}I-Flie{\ss}bild wird so erstellt, dass einzelne Komponenten (die Module) frei ausgetauscht werden k\"onnen. Module m\"ussen also \"uber definierte Schnittstellen verf\"ugen und klar definierte Verfahrenstechnische Prozessschritte erf\"ullen. Weiterhin sollten Module einzeln testbar und skalierbar sein. Wurde eine Anlage so geplant, dass Module austauschbar sind, so kann ein erh\"ohter Bedarf an Produktionsmenge durch Skalierung der Module befriedigt werden. Module sollten daher in definierten Gr\"o\ss{}en entworfen werden. Je nach Anwendungsfall wird dann die am Besten passende Gr\"o\ss{}e ausgew\"ahlt. Der Prozess wird dadurch nicht mehr optimal erf\"ullt (das Modul wird immer etwas \"uberdimensioniert sein, je nach dem wie weit die n\"achte Modulgr\"o\\ss{}e von der eigentlich erforderlichen Gr\"o\ss{} weg ist), daf\"ur l\"auft die Planung aber deutlich schneller ab. Am Beispiel der Hochleistungsfl\"ussigkeitschromatographie wird gezeigt, wie ein Verfahren in Module zerlegt werden kann. Die Module werden f\"ur verschiedene Produktionsmengen entworfen (also Stufenweise). Die Auswirkung auf die Kosten einer Anlage mit nicht-optimalen Modulen bzgl. der Gr\"o\ss{} wird dargelegt. Nachteil einer modularen Bauweise ist, dass sonderw\"unsche von Kunden nur kompliziert erf\"ullt werden k\"onnen. Im Fall von Defekten ist ein Austausch von Ger\"aten daf\"ur wesentlich preiswerter, da eben keine Sonderanfertigungen gemacht werden. Ein Anlagenhersteller gelangt durch die wiederholte Verwendung eines Moduls au\ss{}erdem zu mehr Erfahren/ Detailwissen und kann damit Bedienbesonderheiten und Anwendungspotential eines Moduls besser ausssch\"opfen (und dieses Wissen an den Anlagenbetreiber weiter geben). Insbesondere Schulungen zum Betrieb der Anlage sind fr\"uzeitig im Entwicklungsprozess m\"oglich. Der Prozess wird daher sicherer (lernen aus Erfahrung einfacher, vor allem bzgl. Bedienung und Designfehlern). Zus\"atzlich werden Module vom Layout her bez\"uglich Kosten und Funktion optimiert. Man erkauft sich also durch eine \"Uberdimensionierung der Anlage bez\"uglich der notwendigen Gr\"o\ss{}e (da Gr\"o\ss{}e nicht optimiert wird) durch eine schnelle Planung und Module, welche ihre Funktion optimal erf\"ullen, ein optimales Layout haben und durch mehr Erfahrung besonders sicher sind. Liegt eine erforderliche Modulgr\"o\ss{}e zu weit weg von bereits designten Modulgr\"o\ss{}en, so wird das bekannte Modul sinnvollerweise neu skaliert oder der Kunde lebt mit nicht-optimalem Prozess. Bei Neu-Skalierung kann vorhandenes Wissen \"uber Layout des Moduls aber wieder verwendet werden -> ist also nicht sooo schlimm. 

%\paragraph*{\cite{Obst_2013b}} Automatisierung im Life Cycle modularer Anlagen
%
%\paragraph*{\cite{Urbas_2012a}} Automatisierung von Prozessmodulen

%\paragraph*{\cite{Fleischer_2016}} Planungsansatz f{\"u}r modulare Anlagen in der chemischen Industrie. Viele Literaturreferenze, sehr aktueller Text zur Problematik der Verwendung von modularen Anlagen in der Industrie

%\paragraph{\cite{Lier_2016a}} Transformable Production Concepts: Flexible, Mobile, Decentralized, Modular, Fast: Bereits verwendete Konzepte werden vorgestellt (F3, Evotrainer, Copiride), die Notwendigkeit modularer Anlagen wird dargelegt, Arbeiten zur Wirtschaftlichkeit werden ausgewertet mit dem Schluss, dass sich besonders bei kurzen Produktlebenszyklen der Einsatz modularer Anlagen lohnt.  

%\paragraph*{\cite{Hohmann_2017}} Modules in process industry - A life cycle definition

%\subsection{Moduldarstellung}

%\paragraph*{\cite{Wassilew_2017}}

%\paragraph*{\cite{Obst_2015}} Beschreibung von Prozessmodulen
%
%
%\paragraph*{\cite{Wassilew_2016}}
%Zur Darstellung von Modulen wird ein MTP erstellt. Dies entspricht dem in NAMUR vereinbarten Anforderungen? Der Informationsgehalt des MTP kann automatisiert in das Format von OPC UA transformiert werden. Wie das geht, steht in dieser Arbeit.


%\paragraph*{\cite{Obst_2015a}} Semantic description of process modules

%\section{Mini- und Milliplanttechnik}
%
%\paragraph*{\cite{Grundemann_2012}} Relevanz von Mikroreaktoren und deren Beziehung zur 50 Prozent These anhand zweier Beispiele.
%
%\paragraph*{\cite{Brodhagen_2012}} Kostenvorteil durch Modularisierung von Anlagen, insbesondere bei Verwendung von Mikroreaktoren anhand einer Fallstudie. \hfill \newline
%Es gibt einen Trend zu kundenspezifischen Produkten, welche die Massenprodukte zunehmend ersetzen. Der Umfang umfasst nur bis zu wenigen tausend Tonnen pro Jahr. Die Lebenszyklen solcher Produkte betragen au\ss{}erdem nur wenige Jahre. Eine kurze Entwicklungszeit ist daher von hoher Wichtigkeit, um wettbewersbf\"ahig zu bleiben. Weiterhin ist die Flexibilit\"at einer Anlage ausschlaggebend. 
%Batchprozesse haben folgenden Nachteile: \begin{itemize}
%\item Scale-Up ist schwierig, da insbesondere die W\"armeabfuhr bei gro\ss{}en Reaktoren wesentlich schlechter als im Laborma\ss{}stab ist. Dies f\"uhrt bei stark exothermen Reaktionen zu Sicherheitsproblemen und vergr\"o\ss{}ertem Planungsaufwand
%\item Batchanlagen sind nicht sehr flexibel. Durch feste Chargengr\"o\ss{}en m\"ussen bei geringen Absatzmengen Zielstoffe und Ausgangsstoffe in gro\ss{}en Mengen zwischengelagert werden, um verschiedene Produkte herstellen zu k\"onnen und die Anlage voll auszulasten.
%\end{itemize}
%Milli-Reaktoren bieten eine Alternative. Sie sind gekennzeichnet durch einen kontinuierlichen Betrieb, Str\"omungskan\"ale mit Durchmessern im Millimeterbereich und sehr hohen W\"armeaustauschfl\"achen pro Volumeneinheit (ca. Faktor 100 zu klassischen Anlagen), wodurch die W\"arme bei stark exothermen Reaktionen gut abgef\"uhrt werden kann, was eine inh\"arente Sicherheit zur Folge hat. 
%Um ein einfaches Scale-Up von neuen Prozessen zu erm\"oglichen m\"ussen Anlagen im Produktionsma\ss{}stab m\"oglichst die gleichen Eigenschaften wie solche in Laborgr\"o\ss{}e aufweisen. Entwickelte Prozesse m\"ussen dann nicht aufwendig angepasst werden und der Entwicklungsprozess wird stark beschleunigt. Milli-Reaktoren bieten diese Eigenschaft. Au\ss{}erdem k\"onnen mit ihnen hohe Raum-Zeit Ausbeuten erreicht werden. Milli-Reaktoren k\"onnen flexibel eingesetzt werden und sind besonders f\"ur einen modularen Aufbau geeignet. Durch Parallelisierung mehrerer Milli-Reaktoren k\"onnen Produktionskapazit\"aten flexibel angepasst werden und damit auf Marktver\"anderungen effizient reagiert werden. Mit Hilfe einer Kapitalwertanalyse wird an einem Fallbeispiel das Potential von Milli-Reaktoren nachgewiesen. Es wird ein linear wachsender Mark f\"ur ein neu entwickeltes Produkt angenommen.  Es wird gezeigt, dass eine Aufteilung von Produktionskapazit\"aten durch die Verwendung von Modulen insbesondere bei kurzen Produktlebenszeiten (bis 10 Jahre) sinnvoll ist. Die h\"oheren Betriebskosten mehrere Module werden dabei durch optimierte Produktionsmengen ausgeglichen. Der Bau von Gro\ss{}anlagen rentiert sich erst bei l\"angeren Produktlebenszyklen oder bei extrem schnell wachsender Produktnachfrage. Weiterhin wird gezeigt, dass eine Zeiteinsparung bei der Produktentwicklung h\"ohere Entwicklungs- und Investitionskosten durch die Verwendung von Milli-Reaktoren rechtfertigt und zu einem wirtschaftlich besseren Ergebnis als die Verwendung von Batch-Prozessen f\"uhrt. 
% \textcolor{green}{fertig}


%\paragraph*{\cite{Kockmann_2012}} Hochskalieren von Anlagen mit Hilfe modularer Konzepte und Mikroreaktoren. \hfill \newline
%Basierend auf Kennzahlen für die Reaktionskinetik und - enthalpie wird eine \textcolor{red}{Korrelation} abgeleitet und mit der konvektiven Wärmeübertragung als ein Kriterium des \textcolor{red}{sicheren Reaktorbetriebs} gekoppelt.

%\paragraph*{\cite{Sell_2013}}
%\textit{Alles nur zitiert!} \hfill \newline
%Die Ergebnisse bisheriger vergleichender Kostenanalysen zwischen diskontinuierlich betriebenen chemischen Verfahren und deren mikroverfahrenstechnischem Pendant haben gezeigt, dass letztere nicht pauschal als kosteneffizientere Alternative gesehen werden können. Es bedarf, wie auch bei Batch-Prozessen, einer umfassenden Prozessoptimierung und Effizienzmaximierung. Dann können mikroverfahrenstechnische Prozesse jedoch trotz teilweise höherer Investitionskosten aufgrund geringerer laufender Kosten und einem schnelleren Zugang zum Markt die ökonomisch günstigere Alternative darstellen. \textcolor{red}{Relevanz von Mikroanlagen}

%\paragraph*{\cite{Hessel_2012}}Potenzialanalyse von Milli- und Mikroprozesstechniken
%
%\paragraph*{\cite{Behr_2012}} Vorteile der Miniplant-Technik im Hinblick auf eine effiziente Verfahrensentwicklung anhand von Beispielen homogen katalysierter Verfahren. Literaturverweise auf alle genannten Vorteile und Eigenschaften von Miniplants. \hfill \newline
%Experimentelle Untersuchungen in unterschiedlichen Ma\ss{}st\"aben nehmen den gr\"o\ss{}ten Teil der Zeit w\"ahrend der Prozessentwicklung ein. Zum Beginn der Prozessentwicklung wird meist mit Batchprozessen gearbeitet, um die Einfl\"usse von Z. B. Katalysatoren, Aktivatoren, Druck, Temperatur usw. untersuchen zu k\"onnen. Eine Vielzahl industrieller Prozesse wird jedoch kontinuierlich durchgef\"uhrt. Ein reines Scale-Up einer Batchanlage reicht daher nicht aus. Statt dessen muss im Laufe der Prozessentwicklung eine kontinuierliche Pilotanlage entworfen werden und der Prozess teilweise neu untersucht werden. Statt diesem Entwicklungsprozess kann eine hochflexible Miniplant im Laborma\ss{}stab mit Hilfe von standardisiertem Laborequipment in kurzer Zeit und mit geringen Kosten entworfen werden. Diese wird bereits kontinuierlich betrieben und an ihr kann der Prozess entwickelt und in Echtzeit untersucht werden. Miniplants existieren mit Scale-Up Faktoren von bis zu 10000 und mehr. Apparate Volumina liegen im Bereich weniger Liter, Durchsatzmengen bis zu $1 \frac{kg}{h}$. Die Prozessentwicklung mit Hilfe kontinuierlicher Reaktoren erlaubt insbesondere eine genaue Betrachtung der Einfl\"usse verschiedener Prozessparameter, da station\"are Arbeitspunkte und Schwankungen um diese gezielt untersucht werden k\"onnen. Au\ss{}erdem kann die wirtschaftlich h\"ochst relevante R\"uckgewinnung von Katalysatoren und die R\"uckf\"uhrung von Stoffen - z. B. die R\"uckf\"uhrung von Nebenprodukten zur Steigerung der Ausbeute gew\"unschter Stoffe - geeignet untersucht werden. Auf geeignete Literatur zu dieser Thematik wird in der Arbeit verwiesen. Weiterhin sind Miniplants zur Personalschulung und der \"Uberpr\"ufung der Einhaltung von Industriestandards nutzbar. Die Korrektur von Fehlern im Prozess ist im Rahmen einer Miniplant wesentlich preiswerter als bei Verwendung einer Pilotanlage. \cite{Krekel_1985} Mit Hilfe komplexer Simulationen ist ein Scale-Up der Miniplant auf Produktionsniveau bei vielen Prozessen direkt m\"oglich. Nur noch kritische Prozesseinheiten m\"ussen im Pilotma\ss{}stab gesondert untersucht werden. Die Umgehung einer kompletten Pilotanlage spart mehrere Jahre Entwicklungszeit. Au\ss{}erdem werden hohe Kosten bei Planung und Betrieb einer Pilotanlage eingespart. Die erfolgreiche Untersuchung und Entwicklung von Prozessen mit Hilfe einer Miniplant wird an vier Beispielen ausgef\"uhrt. Der Fokus liegt dabei auf der R\"uckgewinnung eingesetzter Katalysatoren. 
%\hfill \newline
% \textcolor{green}{fertig}
% 
% \paragraph*{\cite{Helling_2012}} Die Verwendung von Mikroprozessen ist eine relativ neue und weiterhin wachsende Technologie. Ihren Durchbruch hatte diese Technologie in den fr\"uhen und mittleren neunziger Jahren. Die Forschung konzentriert sich derzeit auf die Umwandlung von Batch-Prozessen in Kontinuierliche durch Verwendung modularer Mikro- und Minireaktoren, mit Hilfe derer die Entwicklungszeit neuer Anlagen drastisch gesenkt werden soll. Vorteile sind ein erh\"ohter Stoffaustausch und eine verbesserte W\"arme\"ubertragung. Durch die geringen Stoffvolumina in Mini- bzw. Mikroreaktoren steigt automatisch die Sicherheit besonders von exothermen Reaktionen. Dadurch werden Verfahren m\"oglich, die in konventionellen Reaktoren nur durch hohen Aufwand oder garnicht durchf\"uhrbar w\"aren. Mini- und Mikroanlagen mit gleicher Funktion k\"onnen prinzipiell aus verschiedenen Werkstoffen hergestellt werden. Typisch sind Glas, Keramiken und Edelstahl. Einige Typen sind W\"armetauscher (welche die ersten Mikroanlagen waren), Fallfilmreaktoren, m\"aanderf\"ormige Reaktoren und Photoreaktoren. \hfill \newline
%Bestehende Probleme sind das Design und die konkrete innere Geometrie von Mini- Mikroanlagen. Derzeit werden diese durch Versuch/ Fehler Verfahren in Kombination mit Erfahrungen aus bereits entwickelten Anlagen ermittelt. Eine exakte Reproduzierbarkeit der inneren Strukturen ist schwierig. Module vom gleichen Typ sind daher nicht unbedingt identisch im Verhalten. Soll durch die Verwendung mehrerer Module ein Hochskalieren von Stoffmengen umgesetzt werden, so ist dies daher kompliziert und oft nur bis zu gewissen Grenzen realisierbar. \hfill \newline
%Zum Zeitpunkt dieser Arbeit war insbesondere die Stofftrennung mit Hilfe von Mikroanlagen ein noch nicht komplett gel\"ostes Problem und daher Gegenstand aktueller Forschung. Die aktuelle Arbeit \cite{Yang_2017} gibt eine umfangreiche \"Ubersicht \"uber Entwicklungen zum Trennen von Stoffen durch Destillation mit Hilfe von Mikroanlagen. 
%
%\paragraph*{\cite{Kockmann_2012a}} Sicherheitsaspekte bei der Prozessentwicklung und Kleinmengenproduktion mit Mikroreaktoren



%\paragraph*{\cite{Kockmann_2014}} Micro Process Engineering: Fundamentals, Devices, Fabrication, and Applications

%\paragraph*{\cite{Sundberg_2014}} Micro-scale Distillation and Microplants in Process Development

%\paragraph*{\cite{Hugo_2009}}
%\textit{Alles nur zitiert!} \hfill \newline
%Reaktions- und sicherheitstechnische Aspekte der Umwandlung eines diskontinuierlichen
%chemischen Prozesses in ein kontinuierliches Verfahren unter Verwendung von Mikroreaktoren werden untersucht. Betrachtet man Mikroreaktoren als Module, so k\"onnte diese Arbeit interessant werden $\mapsto$ \textcolor{red}{das sollte ich daher fragen}. 

%\section{Weitere}

%\paragraph*{\cite{Seifert_2012}} Small scale, modular and continuous: A new approach in plant design. Wird bisher nur zitiert. 

%\paragraph*{\cite{Meier_2012}} Aus einer Reduktion der Planungszeit (welche Ziel der Verwendung von Modulen ist) kann nicht, wie man intuitiv meinen k\"onnte, allgemein auf eine Reduktion des Projektrisikos geschlossen werden. \textcolor{green}{fertig}.

%\paragraph*{\cite{Yang_2017}} wird im Text direkt zitiert, keine weitere Analyse notwendig.

%\paragraph*{\cite{Krekel_1985}} \textcolor{green}{wird nur zitiert}
%
%\paragraph*{\cite{Perkins_1992}} \textcolor{red}{hierin eine Quelle ab S. 215 bzw. 199}
%
%\paragraph*{\cite{Oppelt_2015}}
%Notwendigkeit von Simulation w\"ahrend dem Lebenszyklus einer Anlage

%\paragraph*{\cite{Strauch_2008}} Modulare Kostensch{\"a}tzung in der chemischen Industrie - Konzept eines integrierten Systems zur Absch{\"a}tzung und Bewertung des Kapitalbedarfes f{\"u}r die Errichtung einer chemischen Anlage

%\section{unsortiert}
\chapter{Wichtige Begriffe}
\paragraph*{Package Unit}
aus Wikipedia: \hfill \newline
Eine Package Unit (aus dem Englischen package und unit entlehnt; wörtlich Paketeinheit[1][2] oder [ab]gepackte sinngemäß auch abgegrenzte Einheit ist eine von einem Fremdunternehmen geplante und gefertigte Anlage. Die Anforderungen und Voraussetzungen für eine Package Unit sind in einem Lastenheft genannt. Spezielle Anforderungen an eine Package unit sind z. B. Leistungsparameter, Abmessungen und der Steuerungsumfang.
\paragraph*{SIF}
Safety Integrated Function: Ein Zusammenschluss von Komponenten um das Risiko durch eine bestimmte Gefahrenquelle (Hazard) zu reduzieren. 
\paragraph*{SIL}
Der Safety Integrity Level bzw. Sicherheitsintegritätslevel, kurz SIL, ist eine Ma\ss{}einheit zur Quantifizierung von Risikoreduzierung im Bereich von 1 bis 4. Je gr\"o\ss{}er die Zahl ist, desto mehr muss ein erkanntes Risiko reduziert werden. 
\paragraph*{IPL}
An independent protection layer (IPL) is a device, system, or action that is capable
of preventing a scenario from proceeding to its undesired consequence independent
of the initiating event or the action of any other layer of protection associated with
the scenario.
\paragraph*{QRA} Quantitative Risk Analysis: 
\paragraph*{IEs} initiating events: Zu einem risikobehafteten Zustand f\"uhrende Ursachen/ Ereignisse

\paragraph*{PHA} Process Hazard Analysis: Untersuchung von Prozessrisiken. 

\paragraph*{Entropy}Entropies are an information theoretical concept to
characterize the amount of information needed to predict the next measurement with a certain precision.

%
% Anhang (Bibliographie darf im deutschen nicht in den Anhang!)
%\bibliography{bib/quellen}
\clearpage
\phantomsection
\pagenumbering{roman} %neue Seitenzahlen
\stepcounter{chapter}
%\addcontentsline{toc}{chapter}{\protect\numberline{\thechapter}{Literaturverzeichnis}}
\addcontentsline{toc}{chapter}{Literaturverzeichnis}
\nocite{*}%alle Elemente im Literaturverzeichnis werden aufgelistet, auch die nicht-zitierten
\printbibliography[title={Literaturverzeichnis}]
\cleardoublepage
%
%% Anhang
\backmatter
\appendix
\chapter{Anhang von Bildern}
\label{cha:anhang_bilder}

\chapter{Anhang von Tabellen}
\label{cha:anhang_tabellen}

%
%\IfDefined{printindex}{\printindex}
%\IfDefined{printnomenclature}{\printnomenclature}
%\cleardoublepage
%\leerseite{}
%\cleardoublepage


\end{document}

